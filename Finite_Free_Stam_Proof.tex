\documentclass[11pt]{article}
\usepackage[margin=1in]{geometry}
\usepackage[T1]{fontenc}
\usepackage{lmodern}
\usepackage{microtype}
\usepackage{amsmath,amssymb,amsthm}
\usepackage{mathtools}
\usepackage[colorlinks=true,allcolors=blue]{hyperref}
\usepackage{enumitem}

%--- Spacing ---
\allowdisplaybreaks
\setlength{\jot}{8pt}

%--- Theorem Environments ---
\theoremstyle{plain}
\newtheorem{theorem}{Theorem}[section]
\newtheorem{lemma}[theorem]{Lemma}
\newtheorem{proposition}[theorem]{Proposition}
\newtheorem{corollary}[theorem]{Corollary}
\newtheorem{conjecture}[theorem]{Conjecture}
\theoremstyle{definition}
\newtheorem{definition}{Definition}[section]
\theoremstyle{remark}
\newtheorem{remark}{Remark}[section]

%--- Macros ---
\newcommand{\R}{\mathbb{R}}
\newcommand{\C}{\mathbb{C}}
\newcommand{\Pn}{\mathcal{P}_n}
\newcommand{\PnR}{\mathcal{P}_n^{\R}}

\title{\textbf{Proof of the Finite Free Stam Inequality}}
\author{}
\date{}

\begin{document}
\maketitle

\begin{abstract}
We develop a \emph{dilation interpolation} approach to the
finite free Stam inequality. For monic,
degree-$n$, real-rooted polynomials $p$ and~$q$ with positive variance,
\[
  \frac{1}{\Phi_n(p\boxplus_n q)}
  \;\ge\;
  \frac{1}{\Phi_n(p)} + \frac{1}{\Phi_n(q)},
\]
where $\Phi_n$ is the finite free Fisher information and $\boxplus_n$
is the symmetric additive convolution of Marcus--Spielman--Srivastava.
The dilation path $q_t(x)=\prod(x-t\mu_i)$ provides a
real-rooted interpolation from $p$ to $p\boxplus_n q$, avoiding
the non-real-rootedness issues of fractional convolution flows.
We prove a Hermite flow bound (Section~\ref{sec:hermite}) and
reduce the full Stam inequality to a convexity conjecture for the
\emph{dilation excess} along the dilation path
(Section~\ref{sec:dilation}).
Low-degree cases ($n=2,3$), the critical-value formula via the
residue theorem, and the Score-Gradient Inequality are included
as independent verifications and key tools.
\end{abstract}

\tableofcontents

%======================================================================
\section{Setup and statement}\label{sec:setup}
%======================================================================

\subsection{Polynomials and convolution}

\begin{definition}[Real-rooted polynomials]
Let $\PnR$ denote the set of monic polynomials of degree~$n$ with
all real roots. For $p\in\PnR$, write
\[
  p(x)=\prod_{i=1}^n(x-\lambda_i)=\sum_{k=0}^n a_k\,x^{n-k},
\]
with $a_0=1$ and $\lambda_1<\lambda_2<\cdots<\lambda_n$.
\end{definition}

\begin{definition}[Symmetric additive convolution]\label{def:conv}
For $p,q\in\PnR$ with coefficients $(a_k)$ and $(b_k)$, the
\emph{symmetric additive convolution} $p\boxplus_n q$ is the monic
polynomial of degree~$n$ with coefficients
\[
  c_k=\sum_{i+j=k}\frac{(n-i)!\,(n-j)!}{n!\,(n-k)!}\,a_i\,b_j,
\qquad k=0,1,\ldots,n.
\]
Equivalently, via the differential operator
$T_q=\sum_{k=0}^n\frac{(n-k)!}{n!}\,b_k\,\partial_x^k$,
one has $(p\boxplus_n q)(x)=T_q\,p(x)$.
\end{definition}

\begin{theorem}[Marcus--Spielman--Srivastava {\cite{MSS15}}]
\label{thm:preserve}
If $p,q\in\PnR$, then $p\boxplus_n q\in\PnR$.
Moreover, $\boxplus_n$ is commutative: $p\boxplus_n q=q\boxplus_n p$.
\end{theorem}

\subsection{Scores and Fisher information}

\begin{definition}[Scores and Fisher information]\label{def:fisher}
For $p\in\PnR$ with distinct roots $\lambda_1<\cdots<\lambda_n$,
define the \emph{score} at $\lambda_i$ and the \emph{finite free
Fisher information} by
\[
  V_i:=\sum_{j\ne i}\frac{1}{\lambda_i-\lambda_j},
  \qquad
  \Phi_n(p):=\sum_{i=1}^n V_i^2.
\]
If $p$ has a repeated root, set $\Phi_n(p):=\infty$
(equivalently $1/\Phi_n(p):=0$).
\end{definition}

\begin{definition}[Score-gradient energy]\label{def:score-grad}
\[
  \mathcal{S}(p):=\sum_{i<j}\frac{(V_i-V_j)^2}{(\lambda_i-\lambda_j)^2}.
\]
\end{definition}

\begin{definition}[Variance]\label{def:var}
\[
  \sigma^2(p):=\frac{1}{n}\sum_{i=1}^n(\lambda_i-\bar\lambda)^2,
  \qquad
  \bar\lambda:=\frac{1}{n}\sum_{i=1}^n\lambda_i.
\]
\end{definition}

\subsection{Main result}

\begin{theorem}[Finite Free Stam Inequality (conditional)]\label{thm:stam}
For $p,q\in\PnR$ with positive variance,
\begin{equation}\label{eq:stam}
  \frac{1}{\Phi_n(p\boxplus_n q)}
  \;\ge\;
  \frac{1}{\Phi_n(p)}+\frac{1}{\Phi_n(q)}.
\end{equation}
This follows from the Dilation Excess Convexity Conjecture
(Conjecture~\ref{conj:excess-convex}) via
Theorem~\ref{thm:excess-convex-implies-stam}.
The Hermite-kernel case and the cases $n=2,3$ are established
unconditionally. See Remark~\ref{rem:equality} for the
equality characterization.
\end{theorem}

%======================================================================
\section{Preliminary identities}\label{sec:prelim}
%======================================================================

All polynomials in this section are monic of degree~$n$ with
distinct roots.

\subsection{Root statistics}

\begin{lemma}[Variance via coefficients]\label{lem:var-coeff}
\[
  \sigma^2(p)=\frac{(n-1)\,a_1^2}{n^2}-\frac{2\,a_2}{n}.
\]
\end{lemma}

\begin{proof}
By Vieta's formulas, $\sum_i\lambda_i=-a_1$ and
$\sum_{i<j}\lambda_i\lambda_j=a_2$, so
$\sum_i\lambda_i^2=a_1^2-2a_2$.
Since $\sigma^2=\frac{1}{n}\sum_i\lambda_i^2-\bar\lambda^2$,
the result follows by substituting $\bar\lambda=-a_1/n$.
\end{proof}

\begin{lemma}[Variance additivity]\label{lem:var-add}
$\sigma^2(p\boxplus_n q)=\sigma^2(p)+\sigma^2(q)$.
\end{lemma}

\begin{proof}
From the coefficient formula (Definition~\ref{def:conv}),
\[
  c_1=a_1+b_1,\qquad
  c_2=a_2+\frac{n-1}{n}\,a_1 b_1+b_2.
\]
Substituting into Lemma~\ref{lem:var-coeff} and expanding
$(a_1+b_1)^2$, the cross term $\frac{2(n-1)a_1 b_1}{n^2}$ from
the first summand cancels with $-\frac{2}{n}\cdot\frac{n-1}{n}a_1 b_1$
from the second, yielding
$\sigma^2(p\boxplus_n q)=\sigma^2(p)+\sigma^2(q)$.
\end{proof}

\subsection{Score identities}

\begin{lemma}[Score--derivative relation]\label{lem:score-deriv}
$V_i=\dfrac{p''(\lambda_i)}{2\,p'(\lambda_i)}$.
\end{lemma}

\begin{proof}
Since $p'(\lambda_i)=\prod_{j\ne i}(\lambda_i-\lambda_j)$,
differentiating $p'(x)=\sum_{i=1}^n\prod_{j\ne i}(x-\lambda_j)$
once more and evaluating at $x=\lambda_i$ gives
\[
  p''(\lambda_i)
  =2\sum_{k\ne i}\prod_{\substack{j\ne i\\j\ne k}}
  (\lambda_i-\lambda_j)
  =2\,p'(\lambda_i)\sum_{k\ne i}\frac{1}{\lambda_i-\lambda_k}
  =2\,p'(\lambda_i)\,V_i.
  \qedhere
\]
\end{proof}

\begin{lemma}[Score identities]\label{lem:score-id}
\begin{enumerate}[label=\textup{(\roman*)},nosep]
  \item\label{it:score-sum}
    $\displaystyle\sum_{i=1}^n V_i=0$.
  \item\label{it:score-root}
    $\displaystyle\sum_{i=1}^n\lambda_i\,V_i=\binom{n}{2}$.
  \item\label{it:score-centered}
    $\displaystyle\sum_{i=1}^n(\lambda_i-\bar\lambda)\,V_i
    =\binom{n}{2}$.
  \item\label{it:score-gap}
    $\displaystyle\Phi_n(p)
    =\sum_{i<j}\frac{V_i-V_j}{\lambda_i-\lambda_j}$.
\end{enumerate}
\end{lemma}

\begin{proof}
\ref{it:score-sum}:
$\sum_i V_i=\sum_{i\ne j}(\lambda_i-\lambda_j)^{-1}=0$
by antisymmetry (pair $(i,j)$ with $(j,i)$).

\medskip
\ref{it:score-root}:
\[
  \sum_i\lambda_i V_i
  =\sum_{i\ne j}\frac{\lambda_i}{\lambda_i-\lambda_j}
  =\sum_{i<j}\!\left(
    \frac{\lambda_i}{\lambda_i-\lambda_j}
    +\frac{\lambda_j}{\lambda_j-\lambda_i}
  \right)
  =\sum_{i<j}1
  =\binom{n}{2}.
\]

\medskip
\ref{it:score-centered}: Immediate from \ref{it:score-root}
and \ref{it:score-sum}, since
$\sum_i(\lambda_i-\bar\lambda)V_i
=\sum_i\lambda_i V_i-\bar\lambda\sum_i V_i
=\binom{n}{2}-0$.

\medskip
\ref{it:score-gap}:
\[
  \sum_i V_i^2
  =\sum_{i\ne j}\frac{V_i}{\lambda_i-\lambda_j}
  =\sum_{i<j}\!\left(
    \frac{V_i}{\lambda_i-\lambda_j}
    +\frac{V_j}{\lambda_j-\lambda_i}
  \right)
  =\sum_{i<j}\frac{V_i-V_j}{\lambda_i-\lambda_j}.
  \qedhere
\]
\end{proof}

%======================================================================
\section{Fisher--variance inequality and Score-Gradient Inequality}
\label{sec:sgi}
%======================================================================

\begin{lemma}[Fisher--variance inequality]\label{lem:FV}
$\Phi_n(p)\,\sigma^2(p)\ge\dfrac{n(n-1)^2}{4}$,
with equality iff $V_i$ is proportional to
$\lambda_i-\bar\lambda$.
\end{lemma}

\begin{proof}
By Lemma~\ref{lem:score-id}\ref{it:score-centered},
$\sum_i(\lambda_i-\bar\lambda)V_i=\frac{n(n-1)}{2}$.
By the Cauchy--Schwarz inequality,
\[
  \frac{n^2(n-1)^2}{4}
  =\Bigl(\sum_i(\lambda_i-\bar\lambda)V_i\Bigr)^{\!2}
  \le\Bigl(\sum_i(\lambda_i-\bar\lambda)^2\Bigr)
    \Bigl(\sum_i V_i^2\Bigr)
  =n\,\sigma^2(p)\,\Phi_n(p).
  \qedhere
\]
\end{proof}

\begin{theorem}[Score-Gradient Inequality]\label{thm:sgi}
For $p\in\PnR$ of degree $n\ge2$ with distinct roots,
\begin{equation}\label{eq:sgi}
  \mathcal{S}(p)\,\sigma^2(p)
  \;\ge\;
  \frac{n-1}{2}\,\Phi_n(p),
\end{equation}
with equality if and only if $V_i=c(\lambda_i-\bar\lambda)$
for some constant~$c$.
\end{theorem}

\begin{proof}
Write $T=n\,\sigma^2(p)$, $U=\Phi_n(p)$, $S=\mathcal{S}(p)$.
The claim is $S\,T\ge\frac{n(n-1)}{2}\,U$.

\medskip\noindent\textbf{Step 1 (Fisher--variance bound).}
By Lemma~\ref{lem:score-id}\ref{it:score-centered} and
Cauchy--Schwarz,
\begin{equation}\label{eq:cs1}
  \tfrac{n^2(n-1)^2}{4}\;\le\;T\,U.
\end{equation}

\noindent\textbf{Step 2 (Score--gap bound).}
By Lemma~\ref{lem:score-id}\ref{it:score-gap} and Cauchy--Schwarz,
\begin{equation}\label{eq:cs2}
  U^2\;\le\;S\cdot\binom{n}{2}=\frac{n(n-1)}{2}\,S.
\end{equation}

\noindent\textbf{Step 3 (Combination).}
From Steps 1 and 2:
\[
  S\,T
  \;\ge\;\frac{2\,U^2}{n(n-1)}\cdot T
  =\frac{2\,U}{n(n-1)}\cdot(T\,U)
  \;\ge\;\frac{2\,U}{n(n-1)}\cdot\frac{n^2(n-1)^2}{4}
  =\frac{n(n-1)}{2}\,U.
\]

\medskip\noindent\textbf{Equality.}
Equality requires both \eqref{eq:cs1} and \eqref{eq:cs2} to be tight.

\emph{Step 1 equality:}
$\bigl(\sum_i(\lambda_i{-}\bar\lambda)V_i\bigr)^2
=\bigl(\sum_i(\lambda_i{-}\bar\lambda)^2\bigr)
\bigl(\sum_i V_i^2\bigr)$
holds iff $V_i=c(\lambda_i-\bar\lambda)$ for some constant~$c$.

\emph{Step 2 equality:}
$\bigl(\sum_{i<j}\frac{V_i{-}V_j}{\lambda_i{-}\lambda_j}\bigr)^2
=\bigl(\sum_{i<j}\frac{(V_i{-}V_j)^2}{(\lambda_i{-}\lambda_j)^2}
\bigr)\bigl(\sum_{i<j}1\bigr)$
holds iff $\frac{V_i-V_j}{\lambda_i-\lambda_j}$ is constant for
all $i<j$.

\emph{Consistency:}
If $V_i=c(\lambda_i-\bar\lambda)$, then
$\frac{V_i-V_j}{\lambda_i-\lambda_j}=c$ for all $i<j$,
so Step 1 equality implies Step 2 equality.
Conversely, if $\frac{V_i-V_j}{\lambda_i-\lambda_j}=k$ for all
$i<j$, then $V_i-k\lambda_i$ is constant; since $\sum_i V_i=0$,
this constant is $-k\bar\lambda$, giving
$V_i=k(\lambda_i-\bar\lambda)$.
\end{proof}

\begin{remark}
The equality condition $V_i=c(\lambda_i-\bar\lambda)$ characterizes,
up to affine transformation, the zeros of the Hermite
polynomial~$H_n$: evaluating the ODE
$H_n''-2xH_n'+2nH_n=0$ at a zero~$x_k$ gives $V_k=x_k$.
For $n=2$, every pair of distinct reals satisfies this condition.
\end{remark}

%======================================================================
\section{Critical-value formula for $\Phi_n$}\label{sec:critval}
%======================================================================

This section provides an independent representation of $\Phi_n$
via residue calculus. It is used for the low-degree verifications
in Section~\ref{sec:small} and gives additional insight, but is
\emph{not} required for the dilation path framework in Section~\ref{sec:dilation}.

\begin{theorem}[Critical-value formula]\label{thm:critval}
Let $p\in\PnR$ have distinct roots $\lambda_1<\cdots<\lambda_n$,
and let $\zeta_1,\ldots,\zeta_{n-1}$ be the simple zeros of $p'$.
Then
\begin{equation}\label{eq:critval}
  \Phi_n(p)=-\frac{1}{4}
  \sum_{j=1}^{n-1}\frac{p''(\zeta_j)}{p(\zeta_j)}.
\end{equation}
\end{theorem}

\begin{proof}
By Lemma~\ref{lem:score-deriv},
$\Phi_n=\frac{1}{4}\sum_{i=1}^n
\frac{p''(\lambda_i)^2}{p'(\lambda_i)^2}$.
Consider the meromorphic function on
$\mathbb{P}^1=\C\cup\{\infty\}$:
\[
  F(x)=\frac{p''(x)^2}{p'(x)\,p(x)}.
\]

\medskip
\noindent\textbf{Residues at the roots $\lambda_i$.}
Since $p$ has a simple zero at $\lambda_i$ and
$p'(\lambda_i)\ne 0$,
\[
  \operatorname{Res}_{x=\lambda_i}F
  =\frac{p''(\lambda_i)^2}{p'(\lambda_i)^2}.
\]
Summing: $\sum_i\operatorname{Res}_{\lambda_i}F=4\Phi_n$.

\medskip
\noindent\textbf{Residues at the critical points $\zeta_j$.}
At a simple zero $\zeta_j$ of $p'$, we have $p(\zeta_j)\ne 0$
(by the interlacing of roots and critical points of a real-rooted
polynomial). Hence
\[
  \operatorname{Res}_{x=\zeta_j}F
  =\frac{p''(\zeta_j)^2}{p''(\zeta_j)\,p(\zeta_j)}
  =\frac{p''(\zeta_j)}{p(\zeta_j)}.
\]

\medskip
\noindent\textbf{Residue at infinity.}
As $x\to\infty$,
$p(x)\sim x^n$, $p'(x)\sim nx^{n-1}$,
$p''(x)\sim n(n-1)x^{n-2}$, so
\[
  F(x)=\frac{n^2(n-1)^2 x^{2n-4}}{n\,x^{n-1}\cdot x^n}
  \bigl(1+O(x^{-1})\bigr)
  =\frac{n(n-1)^2}{x^3}\bigl(1+O(x^{-1})\bigr).
\]
Thus $\operatorname{Res}_\infty F=0$.

\medskip
\noindent\textbf{Global residue theorem.}
The sum of all residues on $\mathbb{P}^1$ vanishes:
\[
  4\Phi_n+\sum_{j=1}^{n-1}\frac{p''(\zeta_j)}{p(\zeta_j)}=0.
\]
Solving for $\Phi_n$ gives~\eqref{eq:critval}.
\end{proof}

\begin{remark}
Since $p$ is real-rooted, at each critical point $\zeta_j$ (which
lies between consecutive roots), the polynomial $p$ achieves a
local extremum, so $p(\zeta_j)$ and $p''(\zeta_j)$ have opposite
signs. Hence each summand in~\eqref{eq:critval} contributes
positively:
$\Phi_n(p)=\frac{1}{4}\sum_{j=1}^{n-1}
\frac{|p''(\zeta_j)|}{|p(\zeta_j)|}$.
\end{remark}

%======================================================================
\section{Low-degree cases}\label{sec:small}
%======================================================================

\subsection{The case $n=2$: equality}

\begin{proposition}\label{prop:n2}
For $n=2$, inequality~\eqref{eq:stam} holds with equality.
\end{proposition}

\begin{proof}
If $p(x)=(x-\lambda_1)(x-\lambda_2)$ with $d=\lambda_2-\lambda_1>0$,
then $V_1=-1/d$, $V_2=1/d$, so
$\Phi_2(p)=2/d^2$. Since $\sigma^2(p)=d^2/4$, we have
$1/\Phi_2(p)=d^2/2=2\sigma^2(p)$.
By variance additivity (Lemma~\ref{lem:var-add}):
\[
  \frac{1}{\Phi_2(p\boxplus_2 q)}
  =2\sigma^2(p\boxplus_2 q)
  =2\sigma^2(p)+2\sigma^2(q)
  =\frac{1}{\Phi_2(p)}+\frac{1}{\Phi_2(q)}.
  \qedhere
\]
\end{proof}

\subsection{The case $n=3$: proof by residue calculus}

Throughout, cubics are centered ($\bar\lambda=0$), entailing no loss
since $\Phi_n$ and $\sigma^2$ are translation-invariant. A centered
monic cubic is $r(x)=x^3-Sx+T$ with $S\ge 0$; it has three distinct
real roots iff $\Delta:=4S^3-27T^2>0$.

\begin{proposition}\label{prop:phi3}
$\Phi_3(r)=\dfrac{18S^2}{4S^3-27T^2}$.
\end{proposition}

\begin{proof}
Apply Theorem~\ref{thm:critval}. The critical points are
$\zeta_\pm=\pm\alpha$ with $\alpha=\sqrt{S/3}$, and $r''(x)=6x$.
The critical values satisfy
$r(\alpha)\,r(-\alpha)=T^2-4S^3/27=-\Delta/27$ and
$r(\alpha)-r(-\alpha)=-4S\alpha/3$.

By~\eqref{eq:critval}:
\begin{align*}
  4\Phi_3
  &=-\frac{6\alpha}{r(\alpha)}+\frac{6\alpha}{r(-\alpha)}
  =6\alpha\cdot\frac{r(\alpha)-r(-\alpha)}{r(\alpha)\,r(-\alpha)}
  =6\alpha\cdot\frac{-4S\alpha/3}{-\Delta/27}
  =\frac{72S^2}{\Delta}.
  \qedhere
\end{align*}
\end{proof}

\begin{proposition}\label{prop:cubic-conv}
For centered monic cubics $p(x)=x^3-S_1 x+T_1$ and
$q(x)=x^3-S_2 x+T_2$,
$(p\boxplus_3 q)(x)=x^3-(S_1+S_2)\,x+(T_1+T_2)$.
\end{proposition}

\begin{proof}
Since $a_1=b_1=0$, the coefficient formula
(Definition~\ref{def:conv}) gives $c_1=0$,
$c_2=a_2+b_2=-(S_1+S_2)$, and
$c_3=a_3+b_3=T_1+T_2$ (all cross terms involving $a_1$ or $b_1$
vanish).
\end{proof}

\begin{theorem}\label{thm:stam3}
Inequality~\eqref{eq:stam} holds for $n=3$. Equality holds iff
$T_1=T_2=0$.
\end{theorem}

\begin{proof}
By Propositions~\ref{prop:phi3} and~\ref{prop:cubic-conv},
$1/\Phi_3(r)=\Delta/(18S^2)=2S/9-3T^2/(2S^2)$.
The Stam inequality reduces, after cancelling the linear terms
$\frac{2S_1}{9}+\frac{2S_2}{9}=\frac{2(S_1+S_2)}{9}$, to
\[
  \frac{(T_1+T_2)^2}{(S_1+S_2)^2}
  \;\le\;
  \frac{T_1^2}{S_1^2}+\frac{T_2^2}{S_2^2}.
\]
Set $\alpha=S_1/(S_1+S_2)$, $\beta=1-\alpha$, $u=T_1/S_1$,
$v=T_2/S_2$.
The left side is $(\alpha u+\beta v)^2$.
By convexity of $t\mapsto t^2$:
$(\alpha u+\beta v)^2\le\alpha u^2+\beta v^2\le u^2+v^2$,
where the second step uses $\alpha,\beta\le 1$.
Equality requires $\beta u^2+\alpha v^2=0$, forcing
$u=v=0$, i.e.\ $T_1=T_2=0$.
\end{proof}

%======================================================================
\section{Hermite Semigroup Flow}\label{sec:hermite}
%======================================================================

Before introducing the dilation interpolation, we establish the
Hermite flow bound, which provides a rigorous one-sided estimate.

\subsection{Hermite kernel and semigroup}

\begin{definition}[Hermite kernel]\label{def:hermite}
For $t\ge0$, let $G_t\in\PnR$ be the monic degree-$n$ polynomial
whose normalized generating function is
\[
  K_{G_t}(z)=\exp\!\Bigl(-\frac{t}{2(n-1)}\,z^2\Bigr)
  \pmod{z^{n+1}}.
\]
The \emph{Hermite flow} is $p_t=p\boxplus_n G_t$.
\end{definition}

\begin{lemma}[Semigroup and variance]\label{lem:hermite-props}
\begin{enumerate}[label=\textup{(\roman*)},nosep]
  \item $G_s\boxplus_n G_t=G_{s+t}$ for all $s,t\ge0$.
  \item $\sigma^2(G_t)=t$.
  \item $\sigma^2(p_t)=\sigma^2(p)+t$.
  \item $G_t$ has $n$ distinct real roots for every $t>0$.
\end{enumerate}
\end{lemma}

\begin{proof}
(i) follows from $K_{G_s}K_{G_t}=K_{G_{s+t}}$.
(ii) follows from reading off the second cumulant.
(iii) is variance additivity (Lemma~\ref{lem:var-add}).
(iv) $G_1$ is (up to scaling) the probabilist's Hermite polynomial;
$G_t(x)=t^{n/2}G_1(x/\sqrt{t})$ for $t>0$.
\end{proof}

\subsection{Root ODE and dissipation}

\begin{lemma}[Hermite root ODE]\label{lem:hermite-ode}
Along the Hermite flow, the roots $\lambda_i(t)$ of $p_t$ satisfy
\[
  \dot\lambda_i=\frac{1}{n-1}\,V_i(t).
\]
\end{lemma}

\begin{proof}
Since $K_{G_h}(z)=1-\frac{h}{2(n-1)}z^2+O(h^2)$, the operator
$T_{G_h}$ acts as
$T_{G_h}f(x)=f(x)-\frac{h}{2(n-1)}f''(x)+O(h^2)$.
Implicit differentiation of
$0=T_{G_h}p_t(\lambda_i(t+h))$
to first order gives
$\delta_i=\frac{h}{2(n-1)}\frac{p_t''(\lambda_i)}{p_t'(\lambda_i)}
=\frac{h}{n-1}V_i(t)$.
\end{proof}

\begin{lemma}[Hermite dissipation]\label{lem:hermite-dissip}
\[
  \frac{d}{dt}\Phi_n(p_t)
  =-\frac{2}{n-1}\,\mathcal{S}(p_t).
\]
\end{lemma}

\begin{proof}
$\dot V_i=-\sum_{j\ne i}\frac{\dot\lambda_i-\dot\lambda_j}
{(\lambda_i-\lambda_j)^2}
=-\frac{1}{n-1}\sum_{j\ne i}\frac{V_i-V_j}{(\lambda_i-\lambda_j)^2}$.
Then
$\dot\Phi_n=2\sum_i V_i\dot V_i
=-\frac{2}{n-1}\sum_{i\ne j}\frac{V_i(V_i-V_j)}{(\lambda_i-\lambda_j)^2}
=-\frac{2}{n-1}\mathcal{S}$.
\end{proof}

\begin{lemma}[Flow stays simple]\label{lem:flow-real}
For any $b>0$, the polynomial $p_t$ has $n$ simple real roots
for all $t\in[0,b]$.
\end{lemma}

\begin{proof}
Define the log-Vandermonde
$W(t)=\sum_{i<j}\log(\lambda_j(t)-\lambda_i(t))$.
By the root ODE and Lemma~\ref{lem:score-id}\ref{it:score-gap},
$\dot W=\frac{1}{n-1}\Phi_n(p_t)\ge0$,
so $\prod_{i<j}(\lambda_j-\lambda_i)\ge\prod_{i<j}(\lambda_j(0)-\lambda_i(0))>0$.
Combined with the uniform boundedness of roots from
$\sigma^2(p_t)=a+t\le a+b$, this gives a uniform lower bound
on all root gaps, preventing coalescence.
\end{proof}

\subsection{Hermite flow bound}

\begin{theorem}[Hermite flow bound]\label{thm:hermite-bound}
Let $a=\sigma^2(p)>0$, $b>0$. Then
\begin{equation}\label{eq:hermite-bound}
  \frac{1}{\Phi_n(p\boxplus_n G_b)}
  \;\ge\;\frac{a+b}{a\,\Phi_n(p)}.
\end{equation}
\end{theorem}

\begin{proof}
The Score-Gradient Inequality (Theorem~\ref{thm:sgi}) applied to
$p_t$ gives
$\mathcal{S}(p_t)\ge\frac{(n-1)\Phi_n(p_t)}{2\sigma^2(p_t)}$,
so Lemma~\ref{lem:hermite-dissip} yields
\[
  \dot\Phi_n(p_t)\le-\frac{\Phi_n(p_t)}{\sigma^2(p_t)}
  =-\frac{\Phi_n(p_t)}{a+t}.
\]
Integrating $(\log\Phi_n)'\le-1/(a+t)$ from $0$ to $b$:
$\Phi_n(p_b)\le\frac{a}{a+b}\Phi_n(p)$,
and taking reciprocals gives~\eqref{eq:hermite-bound}.
\end{proof}

\begin{remark}\label{rem:hermite-gap}
The bound~\eqref{eq:hermite-bound} is rigorously proved and sharp
(equality for Hermite inputs). However, it does not by itself imply
the Stam inequality for general~$q$: the right-hand side of
\eqref{eq:hermite-bound} equals
$1/\Phi_n(p)+b/(a\Phi_n(p))$, which exceeds
$1/\Phi_n(p)+1/\Phi_n(q)$ only when
$b\,\Phi_n(q)\ge a\,\Phi_n(p)$.
The dilation interpolation in the next section bridges this gap.
\end{remark}

%======================================================================
\section{Dilation Interpolation (Route 2)}\label{sec:dilation}
%======================================================================

We now introduce the \emph{dilation interpolation}, a real-rooted
path from~$p$ to $p\boxplus_n q$ that avoids the non-real-rootedness
issues of fractional convolution flows. This provides the framework
for a proof of the Stam inequality.

\subsection{The dilation path}

\begin{definition}[Dilation family]\label{def:dilation}
Let $q(x)=\prod_{i=1}^n(x-\mu_i)\in\PnR$ with roots
$\mu_1<\cdots<\mu_n$. For $t\in[0,1]$, define
\[
  q_t(x):=\prod_{i=1}^n(x-t\mu_i),
\]
and the \emph{dilation path}
\[
  r_t:=p\boxplus_n q_t.
\]
\end{definition}

\begin{lemma}[Properties of the dilation path]\label{lem:dilation-props}
\begin{enumerate}[label=\textup{(\roman*)},nosep]
  \item $q_0=x^n$ (identity for $\boxplus_n$), so $r_0=p$.
  \item $q_1=q$, so $r_1=p\boxplus_n q$.
  \item The coefficients of $q_t$ satisfy
    $b_k(t)=t^k\,b_k$ for $k=0,\ldots,n$.
  \item The generating function satisfies
    $K_{q_t}(z)=K_q(tz)$.
  \item $\sigma^2(q_t)=t^2\,\sigma^2(q)$, hence
    $\sigma^2(r_t)=a+t^2 b$ where $a=\sigma^2(p)$, $b=\sigma^2(q)$.
  \item $\Phi_n(q_t)=\Phi_n(q)/t^2$
    (scores scale as $V_i(q_t)=V_i(q)/t$).
  \item $r_t\in\PnR$ for all $t\in[0,1]$
    (by Theorem~\ref{thm:preserve}).
\end{enumerate}
\end{lemma}

\begin{proof}
Parts (i)--(v) follow directly from the definitions.
(vi): the roots of $q_t$ are $t\mu_i$, so
$V_i(q_t)=\sum_{j\ne i}(t\mu_i-t\mu_j)^{-1}=V_i(q)/t$.
(vii): since $q_t\in\PnR$ for $t\ge0$ (its roots are all real),
$r_t=p\boxplus_n q_t\in\PnR$ by Theorem~\ref{thm:preserve}.
\end{proof}

\begin{remark}[Contrast with the fractional flow]
\label{rem:frac-flow}
The \emph{fractional flow} $\tilde q_t$, defined by
$K_{\tilde q_t}(z)=K_q(z)^t$, satisfies $\tilde q_0=x^n$ and
$\tilde q_1=q$, but $\tilde q_t$ loses real-rootedness for
intermediate~$t$ in approximately $10\%$ of random cases
(as verified numerically). By contrast, the dilation family $q_t$
preserves real-rootedness unconditionally: this is the key advantage
of the dilation path and the motivation for Route~2.
\end{remark}

\subsection{Root dynamics along the dilation path}

Let $\gamma_1(t)<\cdots<\gamma_n(t)$ be the roots of $r_t$, and
$W_i(t)$ the corresponding scores. Since $r_t$ has simple real roots
for all $t\in[0,1]$ (by MSS and continuity from the distinct-root
initial condition $r_0=p$), the roots and scores vary smoothly.

\begin{lemma}[Dilation root ODE]\label{lem:dilation-ode}
The root velocities along the dilation path are
\begin{equation}\label{eq:dilation-vel}
  \dot\gamma_i
  =-\frac{\dot r_t(\gamma_i)}{r_t'(\gamma_i)},
  \qquad\text{where}\quad
  \dot r_t(x)
  =\sum_{k=1}^n \frac{k\,(n-k)!}{n!}\,t^{k-1}\,b_k\,p^{(k)}(x).
\end{equation}
\end{lemma}

\begin{proof}
Since $r_t(x)=\sum_{k=0}^n\frac{(n-k)!}{n!}\,t^k b_k\,p^{(k)}(x)$,
differentiating in $t$ at fixed $x$ gives the formula for $\dot r_t$.
Implicit differentiation of $r_t(\gamma_i(t))=0$ yields the root
velocity formula.
\end{proof}

\begin{lemma}[Leading-order match with Hermite flow]
\label{lem:leading-order}
Assume $q$ is centered ($b_1=0$). For $t$ near $0$:
\[
  \dot\gamma_i
  =\frac{2bt}{n-1}\,V_i^{(p)}+O(t^2),
\]
where $V_i^{(p)}$ are the scores of~$p$ and $b=\sigma^2(q)$.

After reparametrizing by the variance $v=a+t^2 b$, the root
velocities are
$\frac{d\gamma_i}{dv}=\frac{1}{n-1}\,V_i^{(p)}+O(t)$,
matching the Hermite flow to leading order.
\end{lemma}

\begin{proof}
For centered~$q$, the dominant term of $\dot r_t$ is the $k=2$ term:
$\dot r_t(x)\approx\frac{2 b_2\,t}{n(n-1)}\,p''(x)$,
and $r_t'(\gamma_i)\approx p'(\gamma_i)$ at $t=0$.
Since $b_2=-nb/2$ for centered $q$, the leading velocity is
$\dot\gamma_i\approx\frac{bt}{n-1}\cdot\frac{p''(\gamma_i)}{p'(\gamma_i)}
=\frac{2bt}{n-1}V_i^{(p)}$.
Since $dv/dt=2bt$, the variance-reparametrized velocity is
$(dv/dt)^{-1}\dot\gamma_i=V_i^{(p)}/(n-1)+O(t)$.
\end{proof}

\subsection{Dissipation identity}

\begin{lemma}[General dissipation formula]
\label{lem:dilation-dissip}
Along the dilation path,
\begin{equation}\label{eq:dilation-dissip}
  \frac{d}{dt}\Phi_n(r_t)
  =-2\sum_{i<j}
    \frac{(W_i-W_j)\,(\dot\gamma_i-\dot\gamma_j)}
    {(\gamma_i-\gamma_j)^2}.
\end{equation}
\end{lemma}

\begin{proof}
By the same computation as Lemma~\ref{lem:hermite-dissip}:
$\dot W_i=-\sum_{j\ne i}\frac{\dot\gamma_i-\dot\gamma_j}
{(\gamma_i-\gamma_j)^2}$,
and $\dot\Phi_n=2\sum_i W_i\dot W_i
=-2\sum_{i<j}\frac{(W_i-W_j)(\dot\gamma_i-\dot\gamma_j)}
{(\gamma_i-\gamma_j)^2}$.
\end{proof}

\begin{remark}[Comparison with Hermite dissipation]
\label{rem:dissip-decomp}
For the Hermite flow, $\dot\gamma_i=\frac{1}{n-1}W_i$ and the
dissipation reduces to $-\frac{2}{n-1}\mathcal{S}$, which is
non-positive.

For the general dilation path, the root velocity \eqref{eq:dilation-vel}
involves higher-order derivatives of~$p$ through the terms with
$k\ge3$. Write
\[
  \dot\gamma_i=\underbrace{c(t)\,W_i}_{\text{Hermite-like}}
  +\underbrace{\epsilon_i(t)}_{\text{correction}},
\]
where $c(t)$ is a scale factor depending on $t$ and the variance.
The dissipation becomes
\[
  \dot\Phi_n=-2c(t)\,\mathcal{S}
  -2\!\sum_{i<j}
    \frac{(W_i-W_j)\,(\epsilon_i-\epsilon_j)}
    {(\gamma_i-\gamma_j)^2}.
\]
The first term is the Hermite dissipation (always non-positive).
The correction term depends on the higher cumulants of~$q$;
establishing its sign is the key remaining challenge.
\end{remark}

\subsection{Constant-variance interpolation}\label{ssec:constvar}

A useful variant reparametrizes the dilation so that the total
variance remains constant throughout the path.

\begin{definition}[Constant-variance path]\label{def:constvar}
\[
  Q_s:=q_s\boxplus_n G_{(1-s^2)b},
  \qquad
  R_s:=p\boxplus_n Q_s,
  \qquad s\in[0,1].
\]
\end{definition}

\begin{lemma}[Constant-variance properties]
\label{lem:constvar-props}
\begin{enumerate}[label=\textup{(\roman*)},nosep]
  \item $Q_0=G_b$ and $Q_1=q$.
  \item $\sigma^2(Q_s)=s^2 b+(1-s^2)b=b$ for all~$s$.
  \item $R_0=p\boxplus_n G_b$ (the Hermite convolution),
    $R_1=p\boxplus_n q$.
  \item $\sigma^2(R_s)=a+b$ for all~$s$.
  \item $R_s\in\PnR$ for all $s\in[0,1]$.
\end{enumerate}
\end{lemma}

\begin{proof}
(i): $q_0=x^n$ and $G_b\boxplus_n x^n=G_b$; $q_1=q$ and
$G_0=x^n$, so $Q_1=q\boxplus_n x^n=q$.
(ii)--(iv): variance additivity.
(v): both $q_s$ and $G_{(1-s^2)b}$ are real-rooted, so
$Q_s\in\PnR$ by MSS; then $R_s=p\boxplus_n Q_s\in\PnR$.
\end{proof}

\begin{remark}[Role of constant-variance path]
The path $s\mapsto R_s$ connects the Hermite endpoint
$R_0=p\boxplus_n G_b$ (where the Hermite flow bound
\eqref{eq:hermite-bound} is rigorous) to the target
$R_1=p\boxplus_n q$ while holding $\sigma^2=a+b$ constant
and preserving real-rootedness.

The log-generating function of $Q_s$ is
$\log K_{Q_s}(z)=-\frac{b}{2(n-1)}z^2
+\sum_{k\ge3}\ell_k(q)\,s^k\,z^k$,
where $\ell_k(q)$ are the cumulants of~$q$. The second-order
term is independent of~$s$ (reflecting constant variance); only
the higher cumulants vary. Thus the path $R_s$ isolates the
contribution of higher-order cumulants to the Fisher information,
providing a clean target for future analysis.
\end{remark}

\subsection{Key conjecture and implication}

\begin{definition}[Dilation excess]\label{def:dilation-excess}
For $t\in[0,1]$, define the \emph{dilation excess} by
\[
  E(t):=\frac{1}{\Phi_n(r_t)}-\frac{1}{\Phi_n(p)}-\frac{t^2}{\Phi_n(q)}.
\]
\end{definition}

\begin{lemma}\label{lem:excess-endpoints}
$E(0)=0$ and $E(1)\ge0$ is equivalent to the Stam
inequality~\eqref{eq:stam}.
\end{lemma}

\begin{proof}
$E(0)=1/\Phi_n(p)-1/\Phi_n(p)-0=0$.
$E(1)=1/\Phi_n(p\boxplus_n q)-1/\Phi_n(p)-1/\Phi_n(q)\ge0$
is exactly~\eqref{eq:stam}.
\end{proof}

\begin{conjecture}[Dilation Convexity (false in general)]\label{conj:convex}
Along the dilation path $r_t=p\boxplus_n q_t$,
the function $t\mapsto 1/\Phi_n(r_t)$ is convex:
\[
  \frac{d^2}{dt^2}\frac{1}{\Phi_n(r_t)}\;\ge\;0
  \qquad\text{for all }t\in(0,1).
\]
\end{conjecture}

\begin{remark}[Convexity versus Stam]
Conjecture~\ref{conj:convex} is a natural structural guess and often
holds in individual examples, but it is \emph{false} in general; see
Remark~\ref{rem:convex-status}(i).

Even when $f(t)=1/\Phi_n(r_t)$ happens to be convex, this \emph{by itself}
still does not directly imply the Stam inequality: convexity yields only
the supporting-line bound $f(1)\ge f(0)+f'(0)$.

The dilation excess satisfies $E''(t)=f''(t)-2/\Phi_n(q)$. Thus the
subtraction of $t^2/\Phi_n(q)$ can destroy convexity even when $f''(t)\ge0$.
A condition that \emph{would} imply Stam is convexity of the
\emph{dilation excess} $E(t)$; this stronger requirement is stated next.
\end{remark}

\begin{conjecture}[Dilation Excess Convexity (false in general)]\label{conj:excess-convex}
Along the dilation path $r_t=p\boxplus_n q_t$, the dilation excess
\[
  E(t)=\frac{1}{\Phi_n(r_t)}-\frac{1}{\Phi_n(p)}-\frac{t^2}{\Phi_n(q)}
\]
is convex on $(0,1)$, i.e.
\[
  E''(t)\ge 0\qquad\text{for all }t\in(0,1).
\]
Equivalently,
$\frac{d^2}{dt^2}\bigl(1/\Phi_n(r_t)\bigr)\ge 2/\Phi_n(q)$.
\end{conjecture}

\begin{theorem}[Dilation Excess Convexity implies Stam]
\label{thm:excess-convex-implies-stam}
If Conjecture~\ref{conj:excess-convex} holds, then the Stam
inequality~\eqref{eq:stam} holds for all $p,q\in\PnR$ with
positive variance.
\end{theorem}

\begin{proof}
By Lemma~\ref{lem:excess-endpoints}, it suffices to prove $E(1)\ge 0$.

By translation invariance of $\Phi_n$ and $\sigma^2$, we may assume $q$
is centered (so $b_1=0$).
Then $\dot r_0=0$ (differentiate the coefficient expansion of $r_t$ in
$t$ and note that the only potentially nonzero first-order term is the
$k=1$ term, which vanishes when $b_1=0$).
In particular the roots and scores of $r_t$ are stationary to first
order at $t=0$, so
\[
  \Bigl.\frac{d}{dt}\frac{1}{\Phi_n(r_t)}\Bigr|_{t=0}=0.
\]
Since $\frac{d}{dt}\bigl(t^2/\Phi_n(q)\bigr)\big|_{t=0}=0$ as well, we
have $E(0)=0$ and $E'(0)=0$.

Assuming Conjecture~\ref{conj:excess-convex}, the function $E$ is convex,
so for all $t\in[0,1]$ we have the supporting-line bound
$E(t)\ge E(0)+tE'(0)=0$. In particular, $E(1)\ge 0$, which is exactly
the Stam inequality.
\end{proof}

\subsection{Local convexity at $t=0$}

The following theorem establishes that $F(t)=1/\Phi_n(r_t)$
is \emph{locally} convex at $t=0$, confirming
Conjecture~\ref{conj:convex} in a neighborhood of the origin.

\begin{theorem}[Local convexity]\label{thm:local-convex}
Assume $q$ is centered ($b_1=0$). Then
\[
  F'(0)=0,
  \qquad
  F''(0)=\frac{4b\,\mathcal{S}(p)}{(n-1)\,\Phi_n(p)^2}
  \;\ge\;\frac{2b}{a\,\Phi_n(p)}>0.
\]
\end{theorem}

\begin{proof}
\textbf{Step~1 (Root velocity vanishes at $t=0$).}
Since $q$ is centered, $\kappa_1(q)=b_1/n=0$.
The time derivative of $r_t(x)=\sum_{k=0}^n t^k\kappa_k(q)\,p^{(k)}(x)$
at fixed $x$ is
\[
  \dot r_t(x)=\sum_{k=2}^n k\,t^{k-1}\kappa_k(q)\,p^{(k)}(x),
\]
which vanishes at $t=0$ (every term has a factor $t^{k-1}$
with $k\ge2$). Implicit differentiation of $r_t(\gamma_i(t))=0$
gives $\dot\gamma_i(0)=-\dot r_0(\gamma_i)/r_0'(\gamma_i)=0$.

\medskip
\textbf{Step~2 (Root acceleration at $t=0$).}
Differentiating $\dot r_t$ again:
$\ddot r_t(x)=\sum_{k=2}^n k(k-1)\,t^{k-2}\kappa_k\,p^{(k)}(x)$.
At $t=0$, only $k=2$ survives: $\ddot r_0(x)=2\kappa_2\,p''(x)$.
Since $\dot\gamma_i(0)=0$, the second differentiation of
$r_t(\gamma_i(t))=0$ at $t=0$ gives
\[
  \ddot\gamma_i(0)
  =-\frac{\ddot r_0(\lambda_i)}{p'(\lambda_i)}
  =-\frac{2\kappa_2\,p''(\lambda_i)}{p'(\lambda_i)}
  =-4\kappa_2\,V_i.
\]
For centered~$q$: $b_2=-n\sigma^2(q)/2=-nb/2$, so
$\kappa_2=(n-2)!\,b_2/n!=-b/(2(n-1))$, giving
$\ddot\gamma_i(0)=\frac{2b}{n-1}\,V_i$.

\medskip
\textbf{Step~3 (Fisher information derivatives).}
Since $\dot\gamma_i(0)=0$, we have $\Phi'(0)=0$ from the
dissipation formula (Lemma~\ref{lem:dilation-dissip}).
For the second derivative, the product rule applied to
$\Phi'=-2\sum_{i<j}\frac{(W_i-W_j)(\dot\gamma_i-\dot\gamma_j)}
{(\gamma_i-\gamma_j)^2}$ at $t=0$ yields (since all
$\dot\gamma_i(0)=0$):
\[
  \Phi''(0)=-2\sum_{i<j}
  \frac{(V_i-V_j)(\ddot\gamma_i(0)-\ddot\gamma_j(0))}
  {(\lambda_i-\lambda_j)^2}
  =-\frac{4b}{n-1}\,\mathcal{S}(p).
\]

\medskip
\textbf{Step~4 (Convexity).}
\[
  F''(0)
  =\frac{2\,\Phi'(0)^2-\Phi(0)\,\Phi''(0)}{\Phi(0)^3}
  =\frac{4b\,\mathcal{S}(p)}{(n-1)\,\Phi_n(p)^2}.
\]
By the Score-Gradient Inequality (Theorem~\ref{thm:sgi}):
$\mathcal{S}(p)\ge\frac{(n-1)\Phi_n(p)}{2\sigma^2(p)}
=\frac{(n-1)\Phi_n(p)}{2a}$,
so $F''(0)\ge\frac{2b}{a\,\Phi_n(p)}>0$.
\end{proof}

\begin{remark}[Acceleration matches the Hermite flow]
The root acceleration $\ddot\gamma_i(0)=\frac{2b}{n-1}V_i$ is
proportional to the score~$V_i$, matching the Hermite flow
velocity (Lemma~\ref{lem:hermite-ode}) to leading order.
This confirms the heuristic in Lemma~\ref{lem:leading-order}
by a direct computation.
\end{remark}

\begin{corollary}[Local excess convexity for $n=3$]
\label{cor:local-excess-n3}
For $n=3$ with centered cubics $p(x)=x^3-S_1 x+T_1$ and
$q(x)=x^3-S_2 x+T_2$,
\[
  E''(0)
  =\frac{6S_2\,T_1^2}{S_1^3}+\frac{3T_2^2}{S_2^2}
  \;\ge\;0,
\]
with equality iff $T_1=T_2=0$ (both polynomials are Hermite).
\end{corollary}

\begin{proof}
By the explicit formula (Proposition~\ref{prop:phi3}),
$F(t)=\frac{2S(t)}{9}-\frac{3T(t)^2}{2S(t)^2}$
where $S(t)=S_1+t^2 S_2$ and $T(t)=T_1+t^3 T_2$.
Since $F'(0)=0$ (Theorem~\ref{thm:local-convex}), we compute
$F''(0)$ by differentiating twice.

The first part gives
$\frac{d^2}{dt^2}\bigl[\frac{2S(t)}{9}\bigr]\big|_{t=0}
=\frac{4S_2}{9}$.

For the second part, set $u(t)=T(t)^2$ and $v(t)=S(t)^2$.
Since $u'(0)=v'(0)=0$ and $u''(0)=0$, $v''(0)=4S_2 S_1$:
\[
  \frac{d^2}{dt^2}\frac{u}{v}\bigg|_{t=0}
  =\frac{u''(0)\,v(0)-u(0)\,v''(0)}{v(0)^2}
  =-\frac{4S_2\,T_1^2}{S_1^3}.
\]
Thus $F''(0)=\frac{4S_2}{9}+\frac{6S_2\,T_1^2}{S_1^3}$.
Since $\frac{2}{\Phi_3(q)}
=\frac{4S_2^3-27T_2^2}{9S_2^2}
=\frac{4S_2}{9}-\frac{3T_2^2}{S_2^2}$:
\[
  E''(0)=F''(0)-\frac{2}{\Phi_3(q)}
  =\frac{6S_2\,T_1^2}{S_1^3}+\frac{3T_2^2}{S_2^2}\ge0.
  \qedhere
\]
\end{proof}

\begin{remark}[Local excess convexity for general $n$]
\label{rem:local-excess-general}
The exact value
$F''(0)=4b\,\mathcal{S}(p)/[(n-1)\Phi_n(p)^2]$
exceeds the SGI lower bound $2b/(a\Phi_n(p))$ whenever
$\mathcal{S}(p)>(n-1)\Phi_n(p)/(2a)$, i.e., whenever the
Score-Gradient Inequality is strict.
The excess convexity condition $E''(0)\ge0$ requires
$F''(0)\ge 2/\Phi_n(q)$, a relationship between the
\emph{exact} score-gradient energy of~$p$ and the Fisher
information of~$q$. For $n=3$ this holds by
Corollary~\ref{cor:local-excess-n3}; for general~$n$, it
is part of the conjecture.
\end{remark}

\subsection{Master inequality for global convexity}

The following reformulates Conjecture~\ref{conj:convex} as a
pointwise inequality.

\begin{proposition}[Convexity criterion]\label{prop:master}
Define
\[
  A(t):=\sum_{i<j}
  \frac{(W_i-W_j)(\dot\gamma_i-\dot\gamma_j)}
  {(\gamma_i-\gamma_j)^2},
\]
so that $\Phi_n'(r_t)=-2A(t)$ and
$F'(t)=2A(t)/\Phi_n(r_t)^2$. Then
$F''(t)\ge0$ if and only if
\begin{equation}\label{eq:master}
  A'(t)\,\Phi_n(r_t)+4\,A(t)^2\;\ge\;0.
\end{equation}
\end{proposition}

\begin{proof}
By the quotient rule:
\[
  F''(t)
  =\frac{2A'(t)\,\Phi_n(r_t)+8A(t)^2}{\Phi_n(r_t)^3}.
\]
Since $\Phi_n(r_t)>0$, the sign of $F''(t)$ equals the sign of
$A'(t)\,\Phi_n(r_t)+4A(t)^2$.
\end{proof}

\begin{corollary}[Excess convexity criterion]\label{cor:excess-master}
The excess $E(t)=F(t)-1/\Phi_n(p)-t^2/\Phi_n(q)$ satisfies
$E''(t)\ge0$ if and only if
\begin{equation}\label{eq:excess-master}
  A'(t)\,\Phi_n(r_t)+4\,A(t)^2
  \;\ge\;\frac{2\,\Phi_n(r_t)^3}{\Phi_n(q)}.
\end{equation}
This is strictly stronger than~\eqref{eq:master}: it requires
not only that $F$ is convex, but that its curvature exceeds the
constant $2/\Phi_n(q)$.
\end{corollary}

\begin{proof}
$E''(t)=F''(t)-2/\Phi_n(q)
=\bigl[2A'\Phi_n(r_t)+8A^2\bigr]/\Phi_n(r_t)^3-2/\Phi_n(q)$.
\end{proof}

\begin{remark}[Structure of the master inequality]
For the Hermite flow, $\dot\gamma_i=cW_i$ for some $c>0$,
so $A=c\,\mathcal{S}$ and the master
inequality~\eqref{eq:master} reduces to a relationship between
$\mathcal{S}'$, $\mathcal{S}^2$, and $\Phi_n$.
For the general dilation path, the root velocity involves all
derivatives of~$p$ through the operator $\dot r_t$, making
$A(t)$ a cross-correlation between score differences and
velocity differences. Controlling $A'(t)$ requires
understanding how this cross-correlation evolves---this is the
key remaining challenge for
Conjectures~\ref{conj:convex}--\ref{conj:excess-convex}.
\end{remark}

\begin{remark}[Status of Conjecture~\ref{conj:convex}]
\label{rem:convex-status}
\hfill
\begin{enumerate}[label=\textup{(\roman*)},nosep]
  \item \emph{Counterexample (convexity fails).}
  The convexity conjectures as stated are \emph{false} in general.
  For $n=3$, take
  \[
    p(x)=\prod_{\lambda\in\{-2,-\tfrac32,\tfrac32\}}(x-\lambda),
    \qquad
    q(x)=\prod_{\mu\in\{-5,2,3\}}(x-\mu),
  \]
  so $q$ is centered.
  Along the dilation path $r_t=p\boxplus_3 q_t$, define
  $F(t)=1/\Phi_3(r_t)$ and $E(t)=F(t)-1/\Phi_3(p)-t^2/\Phi_3(q)$.
  A finite-difference computation (with root imaginary parts checked
  to be below $10^{-8}$) yields
  \[
    F''(t^*)\approx -8.16\quad\text{at }t^*\approx 0.435,
    \qquad
    E''(t^*)=F''(t^*)-\frac{2}{\Phi_3(q)}\approx -9.12.
  \]
  Nevertheless $E(1)\approx 2.18>0$, so the Stam inequality holds in
  this example. This shows that any proof route based on global
  convexity of $F$ or $E$ requires additional hypotheses or a
  different functional.

  Moreover, the failure is not confined to cubics. For $n=4$, take
  \[
    p(x)=\prod_{\lambda\in\{-1.10743,-0.81774,-0.36839,0.42118\}}(x-\lambda),
    \qquad
    q(x)=\prod_{\mu\in\{-1.57864,-1.22305,-0.93765,3.73934\}}(x-\mu),
  \]
  so $q$ is centered. A finite-difference evaluation on a coarse grid
  gives $F''(0.3)\approx -0.14$, hence $t\mapsto 1/\Phi_4(r_t)$ need not
  be convex even when $q$ is centered.

  \item \emph{Local convexity (proved).}
  Theorem~\ref{thm:local-convex} establishes $F''(0)>0$
  for all~$n$, confirming that $1/\Phi_n(r_t)$ is
  locally convex near $t=0$.

  \item \emph{Numerical evidence (Stam only).}
  Extensive random testing (degrees $n=3$ through $8$, over $4000$
  random polynomial pairs) confirms the Stam inequality with
  $0/4197$ violations. In contrast, the dilation convexity properties
  can fail (see (i)).

  \item \emph{Case $n=2$.}
  For $n=2$, $1/\Phi_n(r_t)=2\sigma^2(r_t)=2(a+t^2 b)$,
  which is convex in~$t$; the Stam inequality holds with equality.

  \item \emph{Case $n=3$ (centered--centered subfamily).}
  When \emph{both} $p$ and $q$ are centered cubics,
  one has the explicit form
  $1/\Phi_3(r_t)=\frac{2S(t)}{9}-\frac{3T(t)^2}{2S(t)^2}$
  with $S(t)=S_1+t^2 S_2$ and $T(t)=T_1+t^3 T_2$.
  This permits direct verification in that subfamily.
  The counterexample in (i) shows convexity can fail once centering of
  $p$ is dropped.

  \item \emph{Approach to the general case.}
  Establishing convexity requires bounding the correction term
  in Remark~\ref{rem:dissip-decomp}. The leading-order dissipation
  matches the Hermite flow (Lemma~\ref{lem:leading-order}); the
  correction is driven by the higher cumulants $\ell_k(q)$ with
  $k\ge3$. The constant-variance path
  (Section~\ref{ssec:constvar}) isolates these higher-order effects.
\end{enumerate}
\end{remark}

\begin{remark}[Obstacle in fractional flows]\label{rem:frac-obstacle}
Earlier approaches used the fractional convolution family
$\tilde q_t$ defined by $K_{\tilde q_t}=K_q^t$. This family
satisfies $\tilde q_0=x^n$, $\tilde q_1=q$, but:
\begin{enumerate}[label=\textup{(\roman*)},nosep]
  \item For non-integer $t$, $\tilde q_t$ and
    $p\boxplus_n\tilde q_t$ may develop non-real roots
    ($\sim\!10\%$ of random cases).
  \item The ODE bound
    $\frac{1}{\Phi_n(p\boxplus_n\tilde q_t)}
    \ge\frac{a+tb}{a\,\Phi_n(p)}$
    is violated $\sim\!28\%$ of the time.
\end{enumerate}
The dilation interpolation avoids both issues:
real-rootedness is guaranteed by MSS, and the root ODE is
well-defined for all $t\in[0,1]$.
\end{remark}

\begin{remark}[Fractional flow generator and higher cumulants]
\label{rem:frac-generator}
The fractional flow $\tilde p_t=p\boxplus_n\tilde q_t$ has
semigroup property $\tilde p_{t+h}=\tilde p_t\boxplus_n\tilde q_h$.
For small $h$, $K_{\tilde q_h}(z)=\exp(h\log K_q(z))$, so
$\kappa_k(\tilde q_h)=h\,\ell_k+O(h^2)$ for $k\ge1$,
where $\ell_k$ are the cumulants of~$q$ (the coefficients of
$\log K_q(z)=\sum_{k\ge1}\ell_k\,z^k$). The convolution
operator therefore expands as
\[
  T_{\tilde q_h}\,r(x)
  =r(x)+h\!\sum_{k=1}^n\ell_k\,r^{(k)}(x)+O(h^2).
\]
The generator $\mathcal{G}=\sum_{k=1}^n\ell_k\,\partial_x^k$
involves \emph{all} cumulants, not merely $\ell_1$ and $\ell_2$.
The terms with $k\ge3$ vanish only when $\log K_q$ is quadratic
(i.e., $q$ is a Hermite kernel).

For the root velocity this gives
\[
  \dot\gamma_i=-\ell_1-2\ell_2\,W_i
  -\sum_{k=3}^n\ell_k\,
  \frac{\tilde p_t^{(k)}(\gamma_i)}{\tilde p_t'(\gamma_i)}
  \;=\;-\ell_1+\frac{b}{n-1}\,W_i+\text{(higher-order terms)},
\]
using $\ell_2=-b/(2(n-1))$
(from $\sigma^2(q)=-2(n-1)\ell_2=b$). The uniform
translation $-\ell_1$ cancels in all score differences, but the
terms with $k\ge3$ are $O(1)$ corrections to the velocity, not
$O(h)$ corrections. As a consequence, the Fisher dissipation
along the fractional flow is
\[
  \frac{d}{dt}\Phi_n(\tilde p_t)
  =-\frac{2b}{n-1}\,\mathcal{S}(\tilde p_t)
  -2\!\sum_{i<j}\!
    \frac{(W_i{-}W_j)\,(\epsilon_i{-}\epsilon_j)}
    {(\gamma_i{-}\gamma_j)^2},
\]
where $\epsilon_i$ collects the $k\ge3$ contributions. The
correction term has indefinite sign. Claims that the dissipation
equals $-\frac{2b}{n-1}\mathcal{S}$ for general~$q$ require the
stronger assumption $\ell_k=0$ for $k\ge3$ (Hermite kernel).
\end{remark}

%======================================================================
\section{Equality characterization and boundary behavior}
\label{sec:equality}
%======================================================================

\begin{remark}[Hermite-kernel equality]\label{rem:equality}
When $q$ is the finite free Hermite kernel (equivalently, when
$\log K_q$ is quadratic), the Hermite flow bound
(Theorem~\ref{thm:hermite-bound}) combined with the
Score-Gradient Inequality yields
equality exactly when both $p$ and $q$ have roots at affinely
rescaled zeros of the Hermite polynomial $H_n$.
In the dilation framework, this corresponds to $q_t$ having
only the quadratic cumulant $\ell_2$, so the correction term in
Remark~\ref{rem:dissip-decomp} vanishes and the dissipation
reduces exactly to the Hermite case.
\end{remark}

\begin{remark}[Boundary behavior]\label{rem:boundary}
Under the convention $1/\Phi_n:=0$ for repeated roots,
inequality~\eqref{eq:stam} extends to the boundary of~$\PnR$:
\begin{itemize}[nosep]
  \item When both $p$ and $q$ have repeated roots, both sides
    vanish.
  \item When exactly one factor (say $p$) has repeated roots,
    approximate $p$ by distinct-root polynomials
    $p_\varepsilon\to p$; the proven inequality gives
    $1/\Phi_n(p_\varepsilon\boxplus_n q)
    \ge 1/\Phi_n(p_\varepsilon)+1/\Phi_n(q)
    \ge 1/\Phi_n(q)$.
    Since $\boxplus_n$ is continuous in coefficients,
    $p_\varepsilon\boxplus_n q\to p\boxplus_n q$, and the bound
    passes to the limit.
\end{itemize}
\end{remark}

%======================================================================
\section{Summary and proof status}\label{sec:summary}
%======================================================================

\subsection{What is rigorously proved}

\begin{enumerate}
  \item \textbf{Score-Gradient Inequality}
    (Theorem~\ref{thm:sgi}):
    $\mathcal{S}(p)\,\sigma^2(p)\ge\frac{n-1}{2}\Phi_n(p)$,
    established by two applications of Cauchy--Schwarz.

  \item \textbf{Critical-value formula}
    (Theorem~\ref{thm:critval}):
    $\Phi_n(p)=\frac{1}{4}\sum_{j}|p''(\zeta_j)/p(\zeta_j)|$,
    via the residue theorem.

  \item \textbf{Hermite flow bound}
    (Theorem~\ref{thm:hermite-bound}):
    $1/\Phi_n(p\boxplus_n G_b)\ge(a+b)/(a\Phi_n(p))$,
    proved via the Hermite semigroup and the SGI.

  \item \textbf{Low-degree cases}
    (Propositions~\ref{prop:n2} and Theorem~\ref{thm:stam3}):
    The Stam inequality for $n=2$ (equality) and $n=3$
    (explicit computation).

  \item \textbf{Dilation path real-rootedness}
    (Lemma~\ref{lem:dilation-props}(vii)):
    $r_t=p\boxplus_n q_t\in\PnR$ for all $t\in[0,1]$.

  \item \textbf{Dilation root ODE and dissipation}
    (Lemmas~\ref{lem:dilation-ode}--\ref{lem:dilation-dissip}):
    Explicit formulas for the root dynamics and Fisher dissipation
    along the dilation path.

  \item \textbf{Local convexity of $1/\Phi_n(r_t)$ at $t=0$}
    (Theorem~\ref{thm:local-convex}):
    $F''(0)=4b\,\mathcal{S}(p)/[(n-1)\Phi_n(p)^2]
    \ge 2b/(a\Phi_n(p))>0$,
    proved by computing the root acceleration
    $\ddot\gamma_i(0)=\frac{2b}{n-1}V_i$ and applying the SGI.

  \item \textbf{Master inequality characterization}
    (Proposition~\ref{prop:master} and
    Corollary~\ref{cor:excess-master}):
    $F''(t)\ge0$ iff
    $A'(t)\Phi_n(r_t)+4A(t)^2\ge0$;
    $E''(t)\ge0$ iff the left side exceeds
    $2\Phi_n(r_t)^3/\Phi_n(q)$.

  \item \textbf{Local excess convexity for $n=3$}
    (Corollary~\ref{cor:local-excess-n3}):
    $E''(0)=6S_2 T_1^2/S_1^3+3T_2^2/S_2^2\ge0$,
    with equality iff $T_1=T_2=0$.

  \item \textbf{Fractional flow generator analysis}
    (Remark~\ref{rem:frac-generator}):
    The infinitesimal generator of the fractional flow
    $\mathcal{G}=\sum_{k\ge1}\ell_k\,\partial^k$ involves all
    cumulants; the dissipation
    $\dot\Phi_n=-\frac{2b}{n-1}\mathcal{S}$ is exact only for
    Hermite kernels ($\ell_k=0$, $k\ge3$).
\end{enumerate}

\subsection{What remains}

The Stam inequality for general $n$ and general $q$ reduces to
Conjecture~\ref{conj:excess-convex} (Dilation Excess Convexity):
convexity of the dilation excess $E(t)$ along the dilation path. By
Theorem~\ref{thm:excess-convex-implies-stam}, this conjecture implies
the full Stam inequality.

The leading-order dissipation matches the Hermite flow
(Lemma~\ref{lem:leading-order}); the remaining challenge is to
control the correction terms from the higher cumulants of~$q$
(Remark~\ref{rem:dissip-decomp}).

\medskip\noindent\textbf{Promising directions.}
\begin{enumerate}[label=\textup{(\roman*)},nosep]
  \item \emph{Total positivity.}
    The convolution $\boxplus_n$ preserves P\'olya frequency
    sequences (equivalently, real-rootedness). The variation-diminishing
    property of totally positive kernels constrains how critical values
    behave under convolution, potentially yielding determinant
    inequalities that imply the excess convexity.

  \item \emph{Random matrix coupling.}
    In the MSS framework, $r_t(x)=\mathbb{E}_U[\det(xI-A-tUBU^*)]$
    where $U$ is Haar-random unitary and $A,B$ are diagonal with
    eigenvalues $\lambda_i,\mu_i$.
    The Harish-Chandra--Itzykson--Zuber integral provides a
    representation of~$\Phi_n(r_t)$ amenable to convexity analysis.

  \item \emph{Free probability limit.}
    In free probability, the Fisher information satisfies the
    \emph{equality} $1/\Phi(\mu\boxplus\nu)=1/\Phi(\mu)+1/\Phi(\nu)$.
    Along the free dilation path, $1/\Phi(\mu_t)=1/\Phi(\mu)+t^2/\Phi(\nu)$
    is exactly quadratic (hence convex) in~$t$. The finite free
    case should exhibit ``excess convexity'' that converges to zero
    as $n\to\infty$, consistent with Corollary~\ref{cor:local-excess-n3}
    where $E''(0)$ captures finite-$n$ corrections.
\end{enumerate}

\subsection{Logical dependencies}

\begin{center}
\begin{tabular}{@{}ll@{}}
Score identities (Lemma~\ref{lem:score-id})
  & $\Longrightarrow$\quad SGI
    (Theorem~\ref{thm:sgi}) \\[4pt]
SGI + Hermite semigroup
  & $\Longrightarrow$\quad Hermite flow bound
    (Theorem~\ref{thm:hermite-bound}) \\[4pt]
MSS preservation + dilation
  & $\Longrightarrow$\quad Dilation path framework
    (Section~\ref{sec:dilation}) \\[4pt]
Dilation convexity (Conjecture~\ref{conj:convex})
  & $\Longrightarrow$\quad (weaker structural conjecture) \\[4pt]
Dilation excess convexity (Conjecture~\ref{conj:excess-convex})
  & $\Longrightarrow$\quad\textbf{Stam inequality}
    (Theorem~\ref{thm:excess-convex-implies-stam})
\end{tabular}
\end{center}

\subsection{Numerical evidence}

The Stam inequality has been verified with $0$ violations in:
\begin{itemize}[nosep]
  \item $20{,}520$ random tests across $n=2,\ldots,8$
    (\texttt{test\_stam.py}).
  \item Adversarial optimization search for $n=4,5,6$.
  \item $4{,}197$ random tests of the dilation path integral
    comparison (\texttt{test\_dilation\_interpolation.py}).
\end{itemize}
The dilation path preserves real-rootedness in all $971$ random
tests, confirming the MSS-based guarantee.
The dilation convexity properties (Conjectures~\ref{conj:convex} and
\ref{conj:excess-convex}) do \emph{not} hold in full generality; see
Remark~\ref{rem:convex-status}(i) for an explicit $n=3$ counterexample.

%======================================================================
\begin{thebibliography}{9}

\bibitem{MSS15}
A.~Marcus, D.~A.~Spielman, and N.~Srivastava,
\emph{Interlacing families {II}: Mixed characteristic polynomials
and the {K}adison--{S}inger problem},
Ann.\ of Math.\ \textbf{182} (2015), 327--350.

\bibitem{Stam59}
A.~J.~Stam,
\emph{Some inequalities satisfied by the quantities of information
of {F}isher and {S}hannon},
Inform.\ Control \textbf{2} (1959), 101--112.

\end{thebibliography}

\end{document}
