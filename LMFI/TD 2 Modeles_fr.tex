\documentclass[11pt, reqno]{amsart}
\usepackage[utf8]{inputenc}
\usepackage[T1]{fontenc}
\usepackage[french]{babel}
\usepackage[inner=2.0cm,outer=2.0cm,top=2.5cm,bottom=2.5cm]{geometry}
\usepackage{setspace}
\usepackage{float}
\usepackage{amsmath}
\usepackage{amssymb}
\usepackage{nomencl}
\usepackage[makeroom]{cancel}
\usepackage{algorithm}
\usepackage{algpseudocode}
\usepackage{cite}
\usepackage{multirow}
\usepackage{fullpage} 
\usepackage{fancyvrb}
\usepackage{tikz-cd}
\usepackage{epsfig}
\usepackage{fancyhdr}
\usepackage{amssymb}
\usepackage{pifont}
\usepackage{amsmath}
\usepackage{amssymb}
\usepackage{dsfont}
\usepackage{enumerate}
\usepackage{mathtools}
\usepackage{bm}
\usepackage{listings}
\usepackage{setspace}
\usepackage{amsfonts}
\usepackage[document]{ragged2e}
\usepackage{mathtools}
\usepackage{longtable}
\usepackage{verbatim}
\usepackage{subcaption}
\usepackage{amsgen,amsmath,amstext,amsbsy,amsopn,amssymb}
%\usetikzlibrary{through,backgrounds}

%\usetikzlibrary{shadows}
\usepackage{booktabs}
\input{macros.tex}
\newcommand{\op}[1]{ \operatorname{#1} }
\newcommand{\LL}{\mathcal L}
\newcommand{\MM}{\mathcal M}
\newcommand{\N}{\mathbb N}
\newcommand{\Z}{\mathbb Z}
\newcommand{\R}{\mathbb R}
\newcommand{\Q}{\mathbb Q}
\newcommand{\F}{\mathbb F}
\newcommand{\C}{\mathbb C}
\newcommand{\NN}{\mathcal N}
\newcommand{\UU}{\mathcal U}
\doublespacing
\begin{document}
\homework{Théorie des modèles TD2}{Date: 04/10/2020}{T. Servi}{}{Juan Ignacio Padilla}{M2 LMFI}
\justify
\textbf{Exercice 0.1} Considérer les classes suivantes. Pour chaque classe, décider si elle est élémentaire. Si oui, trouver un langage approprié et une axiomatisation. La classe est-elle finiment axiomatisable ? La théorie de la classe est-elle complète ?
\begin{enumerate}
    \item La classe des ensembles totalement ordonnés.
    \item La classe des ensembles bien ordonnés.
    \item La classe des groupes abéliens finis.
    \item La classe des corps algébriquement clos.
    \item La classe des graphes non orientés connexes sans boucles.
          \\
\end{enumerate}
\textbf{Solution 0.1}
\begin{enumerate}
    \item La classe des ensembles totalement ordonnés. Prendre $\LL= \{ < \}$. Les axiomes sont
          \begin{itemize}
              \item $\forall x \forall y \forall z (x<y \land y< z \Rightarrow x < z)$
              \item $\forall x x \not< x$.
              \item $\forall x \forall y (x=y \lor x<y \lor y<x)$.
                    Cette théorie n'est pas complète, $(\N,<)$ et $(\R,<)$ sont des ordres totaux mais le second satisfait la densité :
                    $$\forall x \forall y (x<y \Rightarrow \exists z \ x<z<y)$$
          \end{itemize}
    \item La classe des ensembles bien ordonnés. Elle n'est pas axiomatisable : comme au TD1, prendre $(\N , < )$ et $\N^\UU$ pour un certain ultrafiltre $\UU$ sur $\N$. La première structure est un bon ordre alors que la seconde ne l'est pas. Considérons l'ensemble $I$ des éléments infinis dans $\N^\UU$, il n'est pas vide mais n'a pas de minimum, car sinon on aurait pour $x= \min I$,
          $$\underbrace{\op{pred}(x)}_{fini} < \underbrace{x}_{infini}$$
          ce qui impliquerait que $\N$ a un élément maximum. $\op{pred}(x)$ est défini par
          $$y = \op{pred}(x) \iff y<x \land \forall z (z<x \implies z \leq y).$$
    \item La classe des groupes abéliens finis. Les groupes abéliens peuvent être axiomatisés par les axiomes de groupe et $\forall x,y \  x+y=y+x$ dans le langage $\LL = \{0, + \}$. Les groupes abéliens finis, cependant, ne le peuvent pas. Supposons qu'une théorie $T$ les axiomatise. Prenons
          $$T' = T_{gr.ab} \cup T \cup \{\phi_k \}_{k \in \N}$$
          Où $\phi_k$ est une formule énonçant « a au moins $k$ éléments ». Pour tout
          $$\Delta \subseteq T_{gr.ab} \cup T \cup \{\phi_k \}_{k < N}$$
          fini, on peut donner un modèle : $(\Z / N\Z , +)$. Par compacité, $T'$ est satisfaisable. Un modèle de $T'$ est un groupe abélien fini avec plus de $k$ éléments pour tout $k$, et c'est impossible.
    \item La classe des corps algébriquement clos. On peut prendre $\LL$ comme le langage des anneaux $\{0,1,+,\cdot,- \}$ et $T_{corps}$ comme les axiomes des corps. On ajoute à $T$, pour chaque $n$, la formule
          $$\varphi_n = \forall a_1, \forall a_2 \dots \forall a_{n-1} \exists x \  x^n + a_{n-1}x^{n-1} + \dots + a_1x_1 + a_0 = 0.$$
          Cette théorie n'est pas complète : $\C$ et $\bar{\mathbb F_p}^{alg}$ sont tous deux algébriquement clos mais ont des caractéristiques différentes. Cette théorie n'est pas non plus finiment axiomatisable : supposons qu'elle le soit, alors il existe une formule $\phi$ qui l'axiomatise. Par un lemme vu en cours, il existe un sous-ensemble fini $\Delta \subseteq T \cup \{ \varphi_n\}_{n \in \N}$ tel que $\Delta \models \phi$. Puisque $\Delta$ est fini, pour un certain $n$, $\varphi_n \not\in \Delta$. On va montrer que $\Delta \cup \{\neg \varphi_n  \}$ est satisfaisable.\\
          Considérons $\Q$ et soit $\Q_n$ le corps obtenu en adjoignant toutes les racines de tous les polynômes de degré strictement inférieur à $n$. Soit $p>n$ un nombre premier, alors le polynôme $x^p = 1$ n'a pas de racines dans $\Q_n$ (car pour tout $\alpha \in \Q_n$, le degré de son polynôme minimal sur $\Q$ est au plus $n$). On a que
          $$\Q_n \models \Delta  \cup \{\neg \varphi_n  \} \models \phi$$
          de sorte que $\Q_n$ serait algébriquement clos, une contradiction.
    \item La classe des graphes non orientés connexes sans boucles. Considérons $\LL = \{ R\}$ où $R$ est un symbole de prédicat binaire. Un graphe est une $\LL$-structure dont le domaine est considéré comme un ensemble de sommets et on considère $x R y$ comme $x,y$ étant reliés par une arête. Les axiomes pour les graphes non orientés sans boucles sont
          \begin{itemize}
              \item $\forall x \ \neg  x  R x$
              \item $\forall x, y \ xRy \Rightarrow y R x$
          \end{itemize}
\end{enumerate}
La connexité n'est pas axiomatisable, considérons la formule
$$\varphi_n(x,y) = \neg  \exists z_1,\dots,z_n \left(z_1 = x , z_n = y , \bigwedge_{k=1}^{n-1}z_k R z_{k+1} \right)$$
Qui dit « il n'y a pas de chemin de longueur $n$ reliant $x$ et $y$ ».\\
Supposons par contradiction qu'une théorie $T$ axiomatise les graphes connexes. Alors, considérons pour $c_1,c_2$ de nouvelles constantes la théorie
$$T' = T \cup \{ \phi_k(c_1,c_2\}_{k \in \N}.$$
Pour un
$$\Delta \subseteq T \cup \{ \phi_k(c_1,c_2\}_{k < N}$$
fini, on construit le graphe avec des sommets étiquetés $\{0,1,\dots,N+1\}$, tel que pour $0 \leq k \leq N$, $k R k+1$. Ce graphe satisfait $\phi_k(0,N+1)$ pour tout $k<N$ donc c'est un modèle de $\Delta$. Par le théorème de compacité, il existe un graphe connexe avec deux sommets qui ne sont reliés par aucun chemin fini, c'est impossible.\\




\textbf{Exercice 0.2} Soit $\LL$ un langage et $I$ un ensemble infini. Soient $\{ \MM\}_{i \in I}$, et $\{ \NN\}_{i \in I}$ deux familles de $\LL$-structures. Montrer que les conditions suivantes sont équivalentes :
\begin{enumerate}
    \item pour tout ultrafiltre non principal $\UU$ sur $I$, on a $\prod_{i \in I} \MM_i / \UU \equiv \prod_{i \in I} \NN_i / \UU$.
    \item Pour toute $\LL$-phrase $\varphi$, $\MM_i \models \varphi \iff \NN_i \models \varphi$ est vraie pour tous les $i \in I$ sauf un nombre fini.
\end{enumerate}
\textbf{Solution 0.2}\\
D'abord, supposons (1). Par le théorème de Łos,
\begin{align*}
    \{ i, \MM_i \models \varphi \} \in \UU & \iff \prod_{i \in I} \MM_i / \UU \models \varphi                        \\
                                           & \iff \prod_{i \in I} \NN_i / \UU \models \varphi \ \text{par hypothèse} \\
                                           & \iff \{ i, \NN_i \models \varphi \} \in \UU
\end{align*}
Remarquons que $C= \{i, \MM_i \models \varphi , \NN_i \not\models \varphi \}$ est contenu dans les deux ensembles $A=\{i, \MM_i \models \varphi\}$ et $B=\{i, \NN_i \not\models \varphi  \}$, donc si $A \in \UU$ alors $  B \not\in \UU$ par ce qui précède, et $C \not \in \UU$, sinon, si $A \not\in \UU$ alors à nouveau $C \not\in \UU$. On peut répéter cet argument pour montrer que $C'=\{i, \MM_i \not\models \varphi , \NN_i \models \varphi \} \not\in \UU$ et conclure que $$C \cup C' = \{ i, \neg( \MM_i \models \varphi \iff \NN_i \models \varphi) \} \not\in \UU.$$ Cela montre que pour tout ultrafiltre $\UU$, $\{ i,  \MM_i \models \varphi \iff \NN_i \models \varphi \} \in \UU$. Puisque cela est vrai pour tout choix de $\UU$, cela est vrai pour tout $i \in\cap_{ \ \UU \text{ultrafiltre} \ } \UU \subseteq Frechet$, donc cela est vrai pour un ensemble cofini d'indices.\\
On suppose maintenant (2). Soit $\UU$ un ultrafiltre quelconque sur $I$ et supposons par contradiction que $\{i, \MM_i \models \varphi \} \in \UU$ et $\{i, \NN_i \models \varphi \} \not\in \UU$. Alors leur intersection, $ \{i, \MM_i \models \varphi , \NN_i \not\models \varphi \} \in \UU$ est contenue dans $$\{ i, \neg( \MM_i \models \varphi \iff \NN_i \models \varphi) \},$$
mais cet ensemble ne peut pas être dans $\UU$ car cela contredirait notre hypothèse (si une condition est vraie pour un ensemble cofini d'indices alors cet ensemble appartient au filtre de Fréchet qui est contenu dans $\UU$). On conclut de cela que $\{i , \MM_i \models \varphi \} \in \UU$ si et seulement si $\{i , \N_i \models \varphi \} \in \UU$. On peut maintenant appliquer le théorème de Łos :
\begin{align*}
    \prod_{i \in I} \MM_i / \UU \models \varphi & \iff  \{ i, \MM_i \models \varphi \} \in \UU                        \\
                                                & \iff \{ i, \NN_i \models \varphi \} \in \UU  \ \text{par hypothèse} \\
                                                & \iff   \prod_{i \in I} \NN_i / \UU \models \varphi
\end{align*}


\newpage

\textbf{Exercice 1.} Montrer qu'un groupe abélien est ordonnable si et seulement si tous ses sous-groupes de type fini le sont.\\
\textbf{Solution 1.} Soit $G$ un groupe abélien. Une direction est claire, si $(G,<)$ est ordonnable alors pour tout sous-groupe $H$, la restriction de $<$ à $H$ l'ordonne et respecte la structure de groupe. Maintenant, supposons que tout sous-groupe de type fini de $G$ est ordonnable. Soit $\LL$ le langage des groupes et considérons le langage $\LL(G)$. Définissons $\Psi$ comme la théorie contenant les $\LL(G)$-phrases suivantes pour tous $a,b,c \in G$ :
\begin{enumerate}
    \item $a \not < a$.
    \item $a<b \lor a=b \lor a>b$.
    \item $a<b \land b<c \Rightarrow a<c$.
    \item $a<b \land a+c <b+c$.
\end{enumerate}
Considérons aussi l'ensemble
$$D(G) = \{ \varphi \text{ une $\LL(G)$-phrase}  , G^* \models \varphi\}$$
où $G^*$ est l'expansion de $G$ obtenue en interprétant chaque symbole de constante comme lui-même dans $G$. On considère maintenant la $\LL(G)$-théorie
$$T' =  \Psi \cup D(G).$$
Cette théorie est finiment satisfaisable : si $\Delta \subseteq T'$ est fini, alors il y a un nombre fini $a_1,\dots,a_n \in G$ apparaissant comme symboles dans $T'$, et puisque $H=\langle a_1,\dots a_n \rangle$ (le groupe engendré par les $a_i$) est une sous-structure de $G^*$, il satisfait toute phrase sans quantificateur qui les « mentionne » et qui est vraie dans $G^*$, donc ils satisfont toute partie finie de $D(G)$ qui peut être dans $\Delta$. Par notre hypothèse, $H$ peut être ordonné, ce qui signifie $H \models \Psi$. Par le théorème de compacité, il existe $G'$, un modèle de $T'$. On doit montrer que $G$ est contenu dans $G'$, donc considérons l'application $f:G \to G' $, qui envoie $a \mapsto a^{G'}$ (envoie un élément de $G$ sur l'interprétation de son symbole dans $G'$). C'est un plongement puisque clairement $f(0)=0$ et si, pour $a,b,c \in G$, $a+b=c$ alors $G' \models a+b=c$ ou de manière équivalente $G' \models a^{G'} + b^{G'} = c^{G'}$ de sorte que $f(c) = f(a)+f(b)$. De même on prouve que $f(-a) = -f(a)$. Finalement, on peut conclure que $G$ est isomorphe à un sous-groupe de $G' \restriction{\LL}$, et puisque $G'$ est ordonnable, alors $G$ l'est.\\

\pagebreak
\textbf{Exercice 2.} Soit $\MM$ une $\LL$-structure, et soit $\NN \equiv \MM$.
\begin{enumerate}
    \item Montrer que $|M| = |N|$.
    \item Soit $k \in \N$. Montrer qu'il existe des formules finies $\varphi_1,\dots,\varphi_N$ avec $k$ variables libres telles que pour toute $\varphi \in \mathcal F(\LL)$, et tout $\bar{b} \in M^k$, pour un certain $\varphi_i$
          $$\MM \models \varphi(\bar{b}) \leftrightarrow \varphi_i(b).$$
    \item Soit $k = |M|$ et soient $\varphi_1,\dots,\varphi_N$ comme dans la question précédente. Écrire $M = \{ a_1,\dots,a_n\}$. Montrer qu'il existe une bijection $\sigma: M \to N$ telle que pour tout $i$,
          $$\MM \models \varphi_i(a_1,\dots,a_n) \iff \NN \models  \varphi_i(\sigma(a_1),\dots,\sigma(a_n)). $$
    \item Montrer que $\MM \simeq \NN$.
\end{enumerate}
\textbf{Solution 2.}
\begin{enumerate}
    \item Pour les structures finies, on sait déjà que leur cardinalité peut être exprimée par une phrase du premier ordre. Puisque $\NN \equiv \MM$, alors elles satisfont toutes deux cette phrase.
    \item Soit $M = \{a_1 , \dots , a_n\}$. On va étiqueter toutes les variables libres qu'on va utiliser comme $x_1,\dots,x_n,\dots$. Pour chaque symbole dans le langage, on considère l'ensemble des formules composé de
          \begin{itemize}
              \item Si $c$ est un symbole de constante et $c^\MM = a_i$, alors ajouter la formule $x_i=c$. Sinon, ajouter sa négation.
              \item Si $f$ est un symbole de fonction $p$-aire, et si $f(a_{i_1},\dots,a_{i_p})=a_{i_j}$, alors ajouter la formule $f(x_{i_1},\dots,x_{i_p})=x_{i_j}$. Sinon, ajouter sa négation.
              \item Si $R$ est un symbole de prédicat $p$-aire, et si $R(a_{i_1},\dots,a_{i_p})$, alors ajouter la formule $R(x_{i_1},\dots,x_{i_p})$. Sinon, ajouter sa négation.
          \end{itemize}
          Définir $F$ comme l'ensemble formé par les formules de la forme
          $$\phi(x_1,\dots,x_n) = \bigvee_{S_1}\bigwedge_{S_2} \varphi(x_1,\dots,x_n)$$
          et leurs négations, où $\varphi$ est l'une des formules ci-dessus et $M_1,M_2$ sont des sous-ensembles finis quelconques de $\{1,2,\dots,n\}$.
          On va montrer par récurrence que toute autre formule $\phi(x_1,\dots,x_k)$ est équivalente à l'une d'elles (peut-être en ajoutant au plus un nombre fini de formules supplémentaires). Remarquons que notre ensemble fini de formules ont toutes leurs variables libres parmi $x_1,\dots,x_n$, cela n'a pas beaucoup d'importance car on compare deux formules avec un nombre différent de variables libres, on peut ajouter à l'une quelque chose de la forme $x_{n+1} = x_{n+1}$ pour la faire dépendre de plus de variables. On supposera alors $k=n$.\\
          Soit $t(\bar{x})$ un terme, alors si $t^\MM(\bar{a})= a_i$ pour un certain $i$, il existe une formule $\varphi(\bar{x})$ dans $F$ telle que $t^\MM(\bar{a})= a_i$ ssi $\MM \models \varphi(\bar{a})$, on le montre par récurrence : si $t$ est une constante $c$, alors $t^\MM(\bar{a})= a_i$ ssi $c = a_i$, et on a une formule pour exprimer cela. Si $t$ est $x_j$ alors $t^\MM(\bar{a})= a_i$ ssi $a_j=a_i$, donc on peut ajouter $x_i = x_j$ à $F$. Et si $t(\bar{x}) = f(u_1(\bar{x}),\dots,u_p(\bar{x})$, alors $t^\MM(\bar{a})= a_i$ ssi pour $i=1,2,\dots p$ et pour certains $a_{j_1},\dots a_{j_p} \in M$, $u_i(\bar{a})=a_{j_i}$ et $t(a_{j_1},\dots a_{j_p} ) = a_i$, mais par hypothèse de récurrence on a une formule dans $F$ pour chaque $u_i$ et aussi la formule $t(x_{j_1},\dots x_{j_p} ) = x_i$.\\
          On peut maintenant appliquer la récurrence sur les formules. Si $\phi(\bar{x})$ a la forme $t_1(\bar{x}) = t_2(\bar{x})$, alors $t_1^\MM(\bar{a})=t_2^\MM(\bar{a})=a$ si et seulement si $\bigvee_{a\in M} t_1^\MM(\bar{a})=a \land t_2^\MM(\bar{a})=a$, on peut alors trouver une formule dans $F$ équivalente à chacune des $t_1^\MM(\bar{a})=a , t_2^\MM(\bar{a})=a$ les connecter par $\land$ et distribuer pour en obtenir une dans $F$. Si $\phi(\bar{x})$ a la forme $R(t_1(\bar{x}),\dots,t_p(\bar{x}))$, alors $R^\MM(t_1^\MM(\bar{a}),\dots,t_p^\MM(\bar{a}))$ si et seulement si pour chaque $1\leq i \leq p$ il existe $b_i$ dans $M$ tel que $t_i(\bar{a}) = b_i$ et $  R^\MM(b_1,\dots,b_p)$, et on a une formule dans $F$ pour chacune d'elles.\\
          Pour le cas booléen, si $\phi(\bar{x}) = \psi(\bar{x}) \land \theta(\bar{x})$, alors on peut trouver une formule dans $F$ pour chaque $\psi$ et $\theta$ et considérer leur conjonction, puis distribuer pour en obtenir une dans $F$. Pour le cas de la négation, c'est une preuve similaire (rappelons que par construction, $F$ est close sous $\neg$).\\
          Si $\phi(\bar{x})=\exists y \psi(y,\bar{x})$ alors on trouve une formule dans $F$ équivalente à $\psi(y,\bar{x})$ et on remarque que puisque $M$ est fini, $\phi(\bar{b})$ est équivalent à $\bigvee_{a \in M}\psi(a,\bar{b})$.
    \item Soit $\bar{a} = \{a_1,\dots, a_n \}$, pour chaque $i = 1,\dots,N$, soit $\MM \models \varphi_i(\bar{a})$ soit non, donc on définit $\varphi'(\bar{x})$ comme $\varphi(\bar{x})$ si c'est le cas, et comme $\neg \varphi(\bar{x})$ sinon. Considérons
          $$\Phi = \exists x_1 ,\dots x_n \bigwedge_{i=1}^{N} \varphi'_i(\bar{x})$$
          Par construction $\MM \models \Phi$, donc $\NN \models \Phi$ par équivalence élémentaire. Soient $b_1,\dots,b_n$ le $n$-uplet qui satisfait cette formule. Alors on peut prendre $a_i \mapsto b_i$, cette application vérifie ce qu'on veut.
    \item On vient de trouver une bijection qui respecte chaque $\varphi_i$. Puisque ces formules (par construction) expriment effectivement ce que sont les constantes, prédicats et fonctions, cette bijection est un plongement. Par conséquent $\MM \simeq \NN$.
\end{enumerate}



\pagebreak
\textbf{Exercice 3.} Soit $\mathcal K$ une classe de $\LL$-structures finies et soit $T=\op{Th}(\mathcal K)$.
\begin{enumerate}
    \item Donner une condition nécessaire et suffisante pour que $T$ ait un modèle infini.
    \item Supposons que $T$ a des modèles infinis. Trouver une axiomatisation $T^{\infty}$ de la classe de tous les modèles infinis de $T$.
    \item Montrer que pour toute phrase $\varphi$, $\varphi$ est une conséquence de $T^{\infty}$ si et seulement s'il existe $n \in \N$ tel que $\varphi$ est vraie dans tout $\MM \in \mathcal K$ tel que $|M|>n$.
\end{enumerate}
\textbf{Solution 3. }
\begin{enumerate}
    \item $T$ a un modèle infini ssi il a des modèles finis arbitrairement grands.\\
          \textbf{Preuve :} Supposons d'abord que $T$ a des modèles arbitrairement grands, et considérons la théorie
          $$T'  = T \cup \{\phi_k \}_{k \in \N }$$
          Où $\phi_k$ est l'énoncé « a plus de $k$ éléments ». $T'$ est finiment satisfaisable par hypothèse, donc par le théorème de compacité il existe un modèle de $T'$, ce modèle est infini. Supposons que $T$ n'a pas de modèle infini, alors la théorie $T'$ n'est pas satisfaisable, donc par compacité il existe un fragment fini de $T'$ qui n'est pas satisfaisable, donc pour un certain $N \in \N$,
          $$T \cup \{\phi_k \}_{k \leq N}$$
          n'a pas de modèle. Cela signifie qu'il n'y a pas de membre de $\mathcal K$ avec $N$ éléments ou plus.
    \item Prendre $T^\infty = T'$ comme ci-dessus.
    \item Supposons $T^{\infty} \models \varphi$. Alors par compacité, pour un certain $\Delta \subseteq T^{\infty}$ fini, $\Delta \models \varphi$. Alors pour un certain $N \in \N$, $T \cup \{\phi_k \}_{k \leq N} \models \phi$, mais tout $\MM \models T \cup \{\phi_k \}_{k \leq N}$ est un membre de $\mathcal K$ avec $|M| > n$. Réciproquement, si $T^\infty \not \models \varphi$, cela signifie que $T^\infty \cup \{ \neg \varphi \}$ est satisfaisable, donc cela signifie qu'il existe des modèles infinis de $T$ qui ne satisfont pas $\varphi$, et par (1), il existe des membres de $\mathcal K$ arbitrairement grands où $\varphi$ est fausse.
\end{enumerate} \pagebreak
\textbf{Exercice 4.} Soit $G$ un groupe qui a des éléments d'ordre fini arbitrairement grand et soit $T= \op{Th}(G)$. Donner deux preuves de ce qui suit (l'une utilisant les ultraproduits et l'autre utilisant la compacité).
\begin{enumerate}
    \item Il existe $H\models T$ tel que $H$ a une infinité d'éléments d'ordre infini.
    \item Il n'existe pas de $\LL_G$-formule qui définit l'ensemble des éléments d'ordre fini.
\end{enumerate}
\textbf{Solution 4.}
\begin{enumerate}
    \item \textit{Ultraproduits : } Soit $\UU$ un ultrafiltre non principal sur $\N$ et considérons l'ultrapuissance $G^\UU$. Soit $\{ g_1,g_2,\dots \} \subseteq G $ où l'ordre de $g_i$ est $i$. Considérons $x= [g_i]_\UU$, cet élément est d'ordre infini puisque pour tout $k$, $\{i , g_k^i \neq 1 \} $ est fini, donc $x^k \neq 1$. Par conséquent on peut considérer l'ensemble des éléments d'ordre infini $\{x,x^2,x^3,\dots \}$\\
          \textit{Compacité : } Soit $c$ un nouveau symbole de constante et considérons la théorie
          $$T' = T \cup \{c^n \neq 1 \}_{n \in \N}.$$
          Pour tout fragment fini $T_0 \subseteq T$, il existe $N$ tel que
          $$T_0 \subseteq T \cup \{c^n \neq 1 \}_{n < N}.$$
          Donc $G$ peut modéliser $T_0$ en interprétant $c$ comme $g_{N+1}$. Par compacité, on peut trouver $\mathcal H  \models T'$, où $x=c^{\mathcal  H}$ est un élément d'ordre infini. Par conséquent on peut considérer l'ensemble des éléments d'ordre infini $\{x,x^2,x^3,\dots \}$.

    \item Supposons qu'une telle formule existe, appelons-la $\phi(y)$.\\
          \textit{Ultraproduits : } Dans la même construction que ci-dessus, on a que $G^\UU \models \phi(g_i)$ pour tout $i$, mais $G^\UU \not\models \phi(x)$, ce qui est une contradiction.\\
          \textit{Compacité : } Considérons la théorie
          $$T'' = T' \cup \{\phi(c) \}.$$
          Elle est satisfaisable par le même argument ci-dessus et un modèle de $T''$ a un élément $c$ d'ordre fini plus grand que tout naturel, ce qui est impossible. \newpage
\end{enumerate}
\textbf{Exercice 5. } Une $\LL$-structure infinie $\MM$ est dite \textit{minimale} si tout $A\subseteq M$ définissable est soit fini soit cofini. Supposons que $\LL$ contient un prédicat unaire $P$. Montrer que la classe des $\LL$-structures minimales n'est pas élémentaire. Qu'en est-il pour un langage arbitraire ?\\
\textbf{Solution : } On va montrer que cette classe n'est pas close sous les ultraproduits. Considérons la famille de $\LL$-structures indexée par $\N$ donnée par $\mathcal N _i = \langle \N , P \rangle$, où $P^{\NN_i} = \{0, \dots,i-1 \}$. Remarquons que $\NN_i$ est minimale pour tout $i$, car être définissable dans $\NN_i$ revient à être définissable dans $\N$ avec le langage vide \footnote{On a montré au TD1 que dans le langage vide toute structure est minimale}, puisque $P$ peut être défini comme
$$P(x) \iff \bigvee_{k<n} k=x.$$
Considérons un ultrafiltre non principal $\UU$ sur $\N$, et définissons l'ultraproduit $$\NN = (\prod_{i \in \N} \NN_i)/ \UU.$$ $P^\NN$ est infini et cofini : observons que pour tout $n$, $[n,n,\dots] \in P^\NN$ et $[n,n+1,n+2,\dots ] \not\in P^\NN$. On conclut que $\NN$ n'est pas minimale.\\
Pour les langages arbitraires, on peut considérer les cas :
\begin{itemize}
    \item Si $P$ est $k$-aire alors on peut définir un prédicat unaire $P'(x) \iff P(x,\dots,x)$, et répéter ce qui précède.
    \item Si $f$ est un symbole de fonction $k$-aire, on peut définir un prédicat $k+1$-aire $\op{Graph}(f)$ et répéter ce qui précède.
    \item Les constantes n'affectent pas la définissabilité puisqu'on peut utiliser des paramètres.
\end{itemize}

\end{document}
