\documentclass[11pt, reqno]{amsart}
\usepackage[utf8]{inputenc}
\usepackage[T1]{fontenc}
% Set target color model to RGB
\usepackage[inner=2.0cm,outer=2.0cm,top=2.5cm,bottom=2.5cm]{geometry}
\usepackage{setspace}
\usepackage{amsmath}
\usepackage{amssymb}
\usepackage{nomencl}
\usepackage[makeroom]{cancel}
\usepackage{algorithm}
\usepackage{algpseudocode}
\usepackage{cite}
\usepackage{fullpage} 
\usepackage{fancyvrb}
\usepackage{tikz-cd}
\usepackage{epsfig}
\usepackage{fancyhdr}
\usepackage{amssymb}
\usepackage{pifont}
\usepackage{amsmath}
\usepackage{amssymb}
\usepackage{dsfont}
\usepackage{mathtools}
\usepackage{bm}
\usepackage{listings}
\usepackage{amsfonts}
\usepackage[document]{ragged2e}
\usepackage{mathtools}
\usepackage{longtable}
\usepackage{verbatim}
\usepackage{lmodern}
\usepackage{subcaption}
\usepackage{amsgen,amsmath,amstext,amsbsy,amsopn,amssymb}
%\usetikzlibrary{through,backgrounds}

%\usetikzlibrary{shadows}
\usepackage{booktabs}
\input{macros.tex}
\newcommand{\op}[1]{ \operatorname{#1} }
\newcommand{\LL}{\mathcal L}
\newcommand{\MM}{\mathcal M}
\newcommand{\N}{\mathbb N}
\newcommand{\R}{\mathbb R}
\newcommand{\NN}{\mathcal N}
\newcommand{\FF}{\mathcal F}
\begin{document}
\homework{Computability and Incompleteness TD1}{Date: 23/09/2020}{P. Rozière}{}{Juan Ignacio Padilla}{M2 LMFI}
\justify
\textbf{Exercise 1.} Show that the set of primitive recursive functions is countable.\\
\textbf{Solution: } We can define by induction
\begin{align*}
    \FF_0     & = \{\lambda x . 0 , \lambda x. s(x)  \} \cup \{ p_k^i , 1 \leq i \leq k \}_{k \in \N} \\
    \FF_{n+1} & = \{ f \in \N^{\N ^ k} , \exists g, h \in \FF_n ,  f = \op{Rec}(g,h)\}_{n\in\N}       \\  \cup  &\{  f \in \N^{\N ^ k} , \exists g_1,\dots,g_m , h \in \FF_n , f \equiv h (g_1,\dots,g_m) \}_{k \in \N}
\end{align*}
Each $\FF_n$ is countable since the operations $\op{Rec}$ and composition only require finitely many arguments. We have that the set of primitive recursive functions is
$\FF = \bigcup_n \FF_n$, and consequently it is countable.\\

\textbf{Exercise 2. (examples, special cases of the primitive recursion scheme).}\begin{enumerate}
    \item Show that constant functions are primitive recursive.
          By induction: the function $\lambda x . 1$ equals the composition of $s(x)$ and the zero function. Now, if $f(x) =\lambda x. k$ is primitive recursive, then $\lambda x . k+1 = s(f(x))$, which is primitive recursive by the composition scheme.
    \item Show that $x \mapsto x+2$, $x \mapsto 2x$ and $x \mapsto 2x + 1$ are primitive recursive.
          $f(x) = \lambda x. x+2 = s(s(p_1^1(x)))$, the doubling function is defined by primitive recursion as $g(0) = 0$ and $g(x+1) = f(p_2^1(x,g(x))$. Finally $h(x) = s(g(x))$.
    \item Show that addition, multiplication and exponentiation are primitive recursive functions.
          \begin{align*}
              +(x,0)        & = x = p_1^1(x)                                      \\
              +(x,y+1)      & = s(p_3^3(x,y,+(x,y)))                              \\
              \times(x,0)   & = 0                                                 \\
              \times(x,y+1) & = +(p_3^3(x,y,\times(x,y)),p_3^1(x,y,\times(x,y))   \\
              \exp(x,0)     & = 1                                                 \\
              \exp(x,y+1)   & = \times (p_3^3(x,y,\exp(x,y)),p_3^1(x,y,\exp(x,y))
          \end{align*}

    \item Show that the function $\op{sg}$ which maps $0$ to $0$ and all other integers to $1$, as well as the function $\bar{\op{sg}}$ which maps $0$ to $1$ and all other integers to $0$, are primitive recursive.
          \begin{align*}
              \op{sg}(0)   & = 0                              \\
              \op{sg}(x+1) & = \lambda x y . 1 (x,\op{sg}(0))
          \end{align*}
          The other case is the same.
    \item Show that the set of primitive functions is closed under the iteration definition scheme, which associates to a function $g$ from $\N^p \to \N$ and to a function $h:\N^{p+1} \to \N$ the function $f:\N^p \to \N$ defined by:
          \begin{align*}
              f(a_1,\dots,a_p,0)   & = g(a_1,\dots,a_p)                      \\
              f(a_1,\dots,a_p,x+1) & = h(a_1,\dots,a_p,f(a_1,\dots,a_p,x))).
          \end{align*}
          We can write
          \begin{align*}
              f(a_1,\dots,a_p,0)   & = g(a_1,\dots,a_p)                                                                                                    \\
              f(a_1,\dots,a_p,x+1) & = h(p_{p+2}^1(\bar{a},x,f(\bar{a},x)),\dots,p_{p+2}^p(\bar{a},x,f(\bar{a},x)), p_{p+2}^{p+2}(\bar{a},x,f(\bar{a},x)))
          \end{align*}
          to express $f$ in primitive recursive form.
          Then show that the functions introduced so far in this exercise can be defined from the base functions and the iteration scheme.
          We have
          \begin{align*}
              +(x,0)        & = x                   \\
              +(x,y+1)      & = s(x,+(x,y))         \\
              \times(x,0)   & = x                   \\
              \times(x,y+1) & = +(x,\times(x,y))    \\
              \exp(x,0)     & = x                   \\
              \exp(x,y+1)   & = \times(x,\exp(x,y)) \\
          \end{align*}
    \item Show that the set of primitive recursive functions is closed \textit{by case definition} on a primitive recursive predicate: if $g$ and $h$ are primitive recursive functions from $\N^p$ to $\N$, and $P$ is a primitive recursive predicate on $\N^p$, then the function $f$ from $\N^p$ to $\N$ defined below is primitive recursive:
          $$f(a_1,\dots,a_p) = \begin{cases}
                  g(a_1,\dots,a_p) & \text{ if } P(a_1\dots,a_n) \\
                  h(a_1,\dots,a_n) & \text{ otherwise}
              \end{cases}$$
          We have $f(\bar{a}) = g(\bar{a})\chi_P(\bar{a}) + h(\bar{a})\chi_P(\bar{a})$.
\end{enumerate}
\textbf{Exercise 3 (bounded sum and product).} Show that if $f:\N^{p+1} \to \N$ is primitive recursive, the functions $g$ and $h$ defined by
$$g(\bar{a},x) = \sum_{i=0}^{x} f(\bar{a},i) \text{ and }  h(\bar{a},x) = \prod_{i=0}^{x} f(\bar{a},i)$$
are primitive recursive.\\
\textbf{Solution: } We have
\begin{align*}
    g(\bar{a},0)   & = f(\bar{a},0)                       \\
    g(\bar{a},x+1) & = g(\bar{a},x) + f(\bar{a},x+1)      \\
    h(\bar{a},0)   & = f(\bar{a},0)                       \\
    h(\bar{a},x+1) & = h(\bar{a},x) \times f(\bar{a},x+1) \\
\end{align*}

\textbf{Exercise 4 (predecessor, comparison)}
\begin{enumerate}
    \item Show that the function $\op{pred} \N \to \N$ which equals $0$ at $0$ and $n-1$ at $n >0$ is primitive recursive.
          \begin{align*}
              \op{pred}(0)   & = 0 \\
              \op{pred}(n+1) & = n
          \end{align*}
    \item Show that $ x \overset{.}{-} y = x-y $ if $x \geq y$ and $0$ otherwise, as well as the function $x,y \mapsto |x-y|$ are primitive recursive.
          \begin{align*}
              x \overset{.}{-} 0     & = x                             \\
              x \overset{.}{-} (y+1) & = \op{pred}(x \overset{.}{-} y)
          \end{align*}
    \item Show that the comparison predicates $\leq , \geq ,< , < , = , \neq$ are primitive recursive.
          We have $\chi_{\leq}(x,y) = \bar{\op{sg}}(x \overset{.}{-} y)$ , $\chi_{\geq}(x,y) = \bar{\op{sg}}(y \overset{.}{-} x)$ , $\chi_{=}(x,y) = \chi_{\leq}(x,y) \chi_{\geq}(x,y)$ , $\chi_{\neq}(x,y) = \bar{\op{sg}}(\chi_{=}(x,y)) $ , $\chi_{<}(x,y) = \chi_{\leq}(x,y)\chi_{\neq}(x,y)$, $\chi_{>}(x,y) =\chi_{\geq}(x,y)\chi_{\neq}(x,y)$.
\end{enumerate}

\textbf{Exercise 5 (Primitive recursive predicates, boolean operations)}
\begin{enumerate}
    \item Show that the set of primitive recursive predicates of any arity is closed under boolean operations.
    \item Deduce that the set of primitive recursive sets is closed under union, intersection and complement.
\end{enumerate}
\textbf{Solution: (1) and (2)} If $P[\bar{x},\bar{y}]$ and $Q[\bar{x}',\bar{y}]$ are primitive recursive predicates, we have
\begin{align*}
    \chi_{P\land Q}(\bar{x},\bar{x}',\bar{y}) & = \chi_P(\bar{x},\bar{y}) \chi_Q (\bar{x}',\bar{y})               \\
    \chi_{P\lor Q}(\bar{x},\bar{x}',\bar{y})  & = \op{sg} ( \chi_P(\bar{x},\bar{y})  + \chi_Q (\bar{x}',\bar{y})) \\
    \chi_{\neg P} (\bar{x},\bar{y})           & = \overline{\op{sg}}(\chi_P(\bar{x},\bar{y}))
\end{align*}
The same applies to primitive recursive sets in $\N^p$. \\

\textbf{Exercise 6.} Show that finite and cofinite subsets of $\N^p$ are primitive recursive.\\
\textbf{Solution: } If $p=0$, $\varnothing$ has the zero function as its characteristic function. If $p>0$ and $A \subseteq \N^p$ is finite, we have $A = \{\bar{a}_1,\dots,\bar{a}_n\}$. We can write $$\chi_A(\bar{x}) = \begin{cases}
        1 & \text{ if } \bigvee_{i=0}^n \bar{x} = \bar{a_i} \\
        0 & \text{ otherwise}
    \end{cases}$$
By \textit{case definition}, $A$ is primitive recursive. Note that the predicate $P[\bar{x}] : \bar{x} =  \bar{a}$ has as its characteristic function $\chi_P(\bar{x}) = \chi_{=}(\bar{x},\bar{a})$, so it is primitive recursive. If $A$ is cofinite, we have
$\chi_A(\bar{x}) = \overline{\op{sg}}(\chi_{\N^p \setminus A}(\bar{x})).$\\

\textbf{Exercise 7 (bounded minimization)} The \textit{bounded minimization} scheme associates to a primitive recursive predicate $B \subseteq \N^{p+1}$ the function $f:\N^{p+1} \to \N$ defined by:
\begin{alignat*}{2}
    f(a_1,\dots,a_p,x) & = \text{ the smallest integer $t \leq x$ such that $B(\bar{a},t)$} \  &  & \text{ if such an integer exists}       \\
    f(a_1,\dots,a_p,x) & = 0 \                                                                &  & \text{ if no such integer exists}
\end{alignat*}
We write $f(\bar{a},x) = \mu t \leq x B(\bar{a},t)$.
\begin{enumerate}
    \item Given a primitive recursive predicate $B \subseteq \N^{p+1}$, show that the function $b:\N^{p+1} \to \N$ is primitive recursive, where $b$ is defined by:
          \begin{align*}
              b(a_1,\dots,a_p,x) & = 0 \text{ if there exists an integer $t \leq x$ such that $B(\bar{a},t)$} \\
              b(a_1,\dots,a_p,x) & = 1  \text{ if no such integer exists}
          \end{align*}
          We can write
          $$b(\bar{a},x) = \overline{\op{sg}}\left(\sum_{t=0}^x \chi_{B}(\bar{a},t)\right)$$
    \item Deduce that the set of primitive recursive functions is closed under the bounded minimization scheme.
          We can write, using the helper function $b$,
          $$f(\bar{a},x) = \sum_{t=0}^{x} b(\bar{a},x).$$
\end{enumerate}
\textbf{Exercise 8 (bounded quantification).} Show that the set of primitive recursive predicates is closed under bounded existential and universal quantification.\\
\textbf{Solution:} If $P$ is a primitive recursive predicate, and we define
\begin{align*}
    P_e[\bar{x},y] & = \exists z \leq y P[\bar{x},z] \\
    P_q[\bar{x},y] & = \forall z \leq y P[\bar{x},z]
\end{align*}
Then,
\begin{align*}
    \chi_{P_e}(\bar{x},y) = \op{sg} & \left(\sum_{t=0}^y \chi_{P}(\bar{x},t) \right)  \\
    \chi_{P_q}(\bar{x},y) =         & \left(\prod_{t=0}^y \chi_{P}(\bar{x},t) \right)
\end{align*}
\textbf{Exercise 9 (Euclidean division).} Show that the functions $q:\NN^2 \to \N$ and $r:\N^2 \to \N$ where $q(n,p)$ is the quotient and $r(n,p)$ the remainder of the division of $n$ by $p$ are primitive recursive functions. Deduce that the binary predicate $a|b$ is primitive recursive.\\
\textbf{Solution: }
\begin{align*}
    q(n,p) & = \mu t \leq n ( pt \leq n \land p(t+1) > n ) \\
    r(n,p) & = n \overset{.}{-} (p \times q(n,p))
\end{align*}
And we have that
$$\chi_{n|p}(n,p) = \begin{cases}
        1 & \text{ if } r(n,p) =0 \\
        0 & \text{ otherwise}
    \end{cases}$$\\

\textbf{Exercise 10 (prime numbers).} Let $p:\N \to \N$ be the function such that $p(n)$ is the $(n+1)$-th prime number.
\begin{enumerate}
    \item Show that the predicate ``being prime'' is primitive recursive.
          $$p \text{ is prime iff } p>1 \land \forall x \leq p ( \neg x |p \lor x = 1 \lor x = p)$$
    \item Show that $p(n+1) \leq p(n)! +1$ and that the factorial function is primitive recursive.\\
          Let $q$ be prime such that $q | p(n)! + 1$, we know that $q \notin \{ p(0) ,\dots , p(n) \} $ (otherwise it would imply the absurdity $q|1$), this implies that $p(n+1)  \leq q \leq p(n)!+1$. We also have
          \begin{align*}
              0!     & = 1        \\
              (n+1)! & = (n+1) n!
          \end{align*}
          This shows that $n!$ is primitive recursive.
    \item Show that the function $p$ is primitive recursive.\\
          Consider the primitive recursive function
          $$p'(n,y_1,y_2) = \mu t \leq y_1 ( t \text{ is prime} \land y_2 \leq t). $$
          Then,
          \begin{align*}
              p(0)   & = 2                  \\
              p(n+1) & = p'(n,p(n)!+1,p(n))
          \end{align*}
          This shows that $p(n)$ is primitive recursive.
\end{enumerate}
\textbf{Exercise 11 (encoding of pairs and $k$-tuples).} Let $\alpha$ be the Cantor bijection from $\N \times \N$ to $\N$, defined by
$$\alpha(n,p) = \left( \sum_{i=0}^{n+p}i\right) + p .$$
\begin{enumerate}
    \item Verify that $\alpha$ is indeed bijective and primitive recursive. Verify that $\alpha$ is increasing on each of its two components.
          If $ m = n+n'$ we have, for all $p$
          $$ \alpha(m,p) = \left(\sum_{i=0}^{m+p} i\right) + p = \left(\sum_{i=0}^{n+p} i\right) + \left(\sum_{i=n+p+1}^{n+n'+p} i\right) + p \geq \left( \sum_{i=0}^{n+p}i\right)+p = \alpha(n,p).$$
          If $p \leq q$ it is evident that for all $n$
          $$\left( \sum_{i=0}^{n+p}i\right)+p \leq \left( \sum_{i=0}^{n+q}i\right)+q.$$
          From ex 3. it is clear that $\alpha$ is primitive recursive. We show injectivity, denote $\left( \sum_{i=0}^{n}i\right) = \Delta(n)$\\
          Let $(n,p) \neq (m,q)$, if $n+p = m+q$ then $\Delta(n+p) = \Delta(m+q)$, and if we assume $\alpha(n,p) = \alpha(m,q)$ this would imply $p=q$ and hence $n=m$, a contradiction.
          For surjectivity, let $m \in \N$, take the smallest $x$ such that $\Delta(x) \leq m \leq \Delta(x+1)$ and take $r = m - x$. Note that $r \leq x$, otherwise we would have $r > x \Rightarrow m= \Delta(x)+r \geq \Delta(x+1)$ which contradicts the minimality of $x$. Then, $x= r+m$ and $m=\Delta(r+m)+r = \alpha(m,r)$
    \item Define in primitive recursive fashion the two associated projections $\pi_2^1$ and $\pi_2^2$ satisfying
          $$\alpha(\pi_2^1(c),\pi_2^2(c))= c \ , \ \pi_2^1(\alpha(n,p))=n \ , \ \pi_2^2(\alpha(n,p))=p.$$
          It is evident that $n,p \leq \alpha(n,p)$, we can write
          \begin{align*}
              \pi_2^1(c) & = (\mu z \leq c)(\exists t \leq c) (\alpha(z,t) = c) \\
              \pi_2^2(c) & = (\mu z \leq c)(\exists t \leq c) (\alpha(t,z) = c)
          \end{align*}
    \item We define by induction on $k \leq 1$ the functions $\alpha_k: \N^p \to \N$ by:
          \begin{align*}\alpha_1(n)                     & = n                                      \\
              \alpha_{k+1}(n_1,\dots,n_{k+1}) & = \alpha(n_1,\alpha_k(n_2\dots,n_{k+1}))\end{align*}
          Show that, for all $k \leq 1$, $\alpha_k$ is a primitive recursive bijection and define recursively the associated projections $\pi_k^i :\N \to \N$. Verify that $\alpha_k$ is increasing on each of its components. We also write $\langle x_1,\dots,x_k \rangle$ for $\alpha_k(x_1,\dots,x_k)$.\\
          The fact that $\alpha_k$ is bijective and primitive recursive is easily shown by induction, because $\alpha_k$ is a composition of primitive recursive functions. By induction, if $\pi_k^i$ are defined for $i =1,\dots,k$, we define
          \begin{align*}
              \pi_{k+1}^1(n_1,\dots, n_{k+1}) & = \pi_2^1( \alpha(n_1,\alpha_k(n_2,\dots, n_{k+1})))                                                     \\
              \pi_{k+1}^i(n_1,\dots, n_{k+1}) & = \pi_k^{i-1}(\pi_2^2( \alpha(n_1,\alpha_k(n_2,\dots, n_{k+1}))))  \ \text{ for } i \in \{2,\dots,k+1 \} \\
          \end{align*}
\end{enumerate}
\textbf{Exercise 12 (Definitions by mutual recursion).} Use the function $\alpha_k$ to show that if the functions $g_1,\dots,g_k : \N^n \to \N$ and $h_1,\dots,h_k: \N^{n+k+1}$ are primitive recursive, then the functions $f_1,\dots,f_k$ defined below are primitive recursive
\begin{align*}
     & f_1(\bar{a},0) =g_1(\bar{a})                                          \\
     & \vdots                                                                \\
     & f_k(\bar{a},0) =g_k(\bar{a})                                          \\
     & f_1(\bar{a},x+1) = h_1(\bar{a},x,f_1(\bar{a},x),\dots,f_k(\bar{a},x)) \\
     & \vdots                                                                \\
     & f_k(\bar{a},x+1) = h_k(\bar{a},x,f_1(\bar{a},x),\dots,f_k(\bar{a},x))
\end{align*}
We can simply write for $i \leq i \leq k$
\begin{align*}
    f_i(\bar{a},0)   & = \pi_k^i (\alpha_k (g_1(\bar{a}),\dots,g_k(\bar{a}))                                                                                                 \\
    f_i(\bar{a},x+1) & = \pi_k^i \bigg(\alpha_k \bigg(h_1(\bar{a},x,f_1(\bar{a},x),\dots,f_k(\bar{a},x)),\dots,h_k(\bar{a},x,f_1(\bar{a},x),\dots,f_k(\bar{a},x)\bigg)\bigg) \\
\end{align*}
By the composition scheme, $f_i$ is primitive recursive.\\
\textbf{Exercise 13 (A bijective encoding of finite sequences).} We obtain the function $::$
$$x::y = 1+ \alpha_2(x,y)$$
We thus obtain a bijective primitive recursive function $\N^2 \to \N^*$. We call $\op{hd}$ and $\op{tl}$ the functions satisfying
\begin{alignat*}{2}
     & \op{hd}(0)= 0 \quad    &  & \op{tl}(0) = 0    \\
     & \op{hd}(x::y)= x \quad &  & \op{tl}(x::y) = y \\
\end{alignat*}
We define a function $\op{list}$ from the set $\mathcal S$ of finite sequences of integers to $\N$ as follows (we write $[a_0 ; \dots ; a_n] = \op
    {list}(a_0,\dots,a_n)$)
\begin{align*}
    [ \ ]               & = 0                           \\
    [a_0 ; \dots ; a_n] & =  a_0 :: [a_1 ; \dots ; a_n]
\end{align*}
Show that the function $\op{list}$ is bijective, and that the functions $\op{hd}$ and $\op{tl}$ are primitive recursive.
We have
\begin{align*}
    \op{hd}(c) & = \pi_2^1(c \overset{.}{-} 1) \\
    \op{tl}(c) & = \pi_2^2(c \overset{.}{-} 1) \\
\end{align*}
To show that $\op{list}$ is injective, let $[a_0;\dots;a_n) = [b_0;\dots;b_{n+k}]$ for some $k \geq 0$, then
\begin{align*}
                & a_0::[a_1;\dots;a_n] = b_0::[b_1;\dots;b_{n+k}]        \\
    \Rightarrow & a_0 = b_0  \land [a_1;\dots;a_n] = [b_1;\dots;b_{n+k}]
\end{align*}
We can repeat this argument starting from $[a_1;\dots;a_n] = [b_1;\dots;b_{n+k}]$ and arrive at
$$\bigwedge_{i=0}^n a_i = b_i \land [ \ ] = [b_1,;\dots;b_{k+1}]$$
This shows that $k=0$ and $(a_0,\dots,a_n) = (b_0,\dots,b_n)$. To show surjectivity, simply note that for all $m \in \N$, there is $k \in \N$ such that $\op{tl}^k(m) = 0$ (because the sequence $\{ \op{tl}^k(m)\}_{k \in \N}$ is strictly decreasing), then
$$m = [\op{hd}(m) ; \op{hd}(\op{tl}(m));\dots ;  \op{hd}(\op{tl}^k(m)) ] = [\op{nth}(m,0) ; \dots ; \op{nth}(m,k)].$$

\textbf{Exercise 14 (recursion on the sequence of values).}
\begin{enumerate}
    \item Prove that the set of recursive functions is closed by the following scheme of recursion on the sequence of values: if $g:\N^{p} \to \N$ and $h:\N^{p+2} \to \N$ are primitive recursive,
          then $f : \N^{p+1} \to \N$ defined by
          \begin{align*}
              f(a_1,\dots,a_p,0)   & = g(a_1,\dots,a_p)                                      \\
              f(a_1,\dots,a_p,x+1) & = h(\bar{a},x, [f(\bar{a},x) ; \dots ; f(\bar{a},0 )]).
          \end{align*}
          It suffices to prove that the function $F(\bar{a},x) =  [f(\bar{a},x) ; \dots ; f(\bar{a},0 ]$ is primitive recursive. We have
          \begin{align*}
              F(\bar{a},0)   & = [f(\bar{a},0)] = g(\bar{a}))::0                                         \\
              F(\bar{a},x+1) & = f(\bar{a},x+1):: F(\bar{a},x) = h(\bar{a},x,F(\bar{a},x))::F(\bar{a},x)
          \end{align*}
          We thus have $f(\bar{a},x) = \op{hd}(F(\bar{a},x)$
    \item Show that the function $\op{nthl}(l,i)$ which returns the sequence encoded by $l$ starting from the $(i+1)$-th element ($0$ otherwise), and the function $\op{nth}(l,i)$ which returns the $(i+1)$-th element of the sequence encoded by $l$, are primitive recursive.
          \begin{alignat*}{2}
               & \op{nthl}(l,0) = l   \quad                       &  & \op{nth}(l,0) =  \op{hd}(l)               \\
               & \op{nthl}(l,i+1) = \op{tl}(\op{nthl}(l,i)) \quad &  & \op{nth}(l,i+1) = \op{hd}(\op{nthl}(l,i))
          \end{alignat*}
    \item Show that if $g:\N^p \to \N$, $h:\N^{p+k+1} \to \N$ are primitive recursive, and if $p_1,\dots,p_k : \N \to \N$ are primitive recursive functions each satisfying
          $$\forall x \in \N p_i(x) \leq x$$
          then $f : \N^{p+1} \to \N$ defined by
          \begin{align*}
              f(a_1,\dots,a_p,0)   & = g(a_1,\dots,a_p)                                           \\
              f(a_1,\dots,a_p,x+1) & = h(\bar{a},x,f(\bar{a},p_1(x)) , \dots , f(\bar{a},p_k(x) )
          \end{align*}
          is primitive recursive.
          We can write
          \begin{align*}
              f(\bar{a},x+1) = h\bigg(\bar{a},x,\op{nth}([f(\bar{a},0);\dots;f(\bar{a},x)],x-p_1(x)), \\ \dots ,\op{nth}([f(\bar{a},0);\dots;f(\bar{a},x)],x-p_k(x))\bigg)\end{align*}
\end{enumerate}
\textbf{Exercise 15 (recursion on lists).}
\begin{enumerate}
    \item Show that $f$ is primitive recursive
          \begin{align*}
              f(\bar{a},[])   & = g(\bar{a})                   \\
              f(\bar{a},x::l) & = h(\bar{a},x,l,f(\bar{a},l)).
          \end{align*}
          We can write
          $$f(\bar{a},y) = h(\bar{a},\op{hd}(y),\op{tl}(y),f(\bar{a},\op{tl}(y)))$$
          $f$ is well-defined since the list function is bijective.\\
          mem
          \begin{align*}
              \op{mem}(a,[])   & = 0                         \\
              \op{mem}(a,x::l) & = \chi_=(x,a) \op{mem}(a,l)
          \end{align*}
          @
          \begin{align*}
              \op{@}(l',[])   & = l'        \\
              \op{@}(l',x::l) & = x::(l@l') \\
          \end{align*}
          length
          \begin{align*}
              \op{lg}([])   & = 0            \\
              \op{lg}(x::l) & = \op{lg}(l)+1 \\
          \end{align*}
    \item Show that if $f$ is pr, then the function $\op{map}(f)$ which maps $l=[\bar{u}]$ to $[f(\bar{a},u_1);\dots;f(\bar{a},u_p)]$
          \begin{align*}
              \op{map}_f([ ])  & = 0 ;                           \\
              \op{map}_f(x::l) & = f(\bar{a},x) :: \op{map}_f(l)
          \end{align*}
    \item
          concat
          \begin{align*}
              \op{concat}([ \ ]) & = [ \ ];                                                \\
              \op{concat}(x::l)  & = x :: [\op{nth}(l,0);\dots,\op{nth}(l,\op{length}(l))]
          \end{align*}
          subst
          \begin{align*}
              \op{subst}([ \ ],k,v) & = [ \ ];                                                         \\
              \op{subst}(x::l,k,v)  & = \begin{cases}
                                            x::\op{subst}(l)              & \text{ if } x \neq v \\
                                            \op{concat}([k,\op{subst}(l)] & \text{ if } x = v
                                        \end{cases}
          \end{align*}
\end{enumerate}
\textbf{Extra Exercise (Encoding lists by prime number decomposition)} Let $\mathcal S$ denote the set of finite sequences of integers. The list encoding function $\op{seq}:\mathcal S \to \N$ associates to each sequence $(x_1,\dots,x_k)$ the following value
$$\op{seq}(x_1,\dots,x_k) = p_0^k p_1^{x_1}\cdots p_k^{x_k}$$
sending the empty sequence to 1.
\begin{enumerate}
    \item Show that this encoding is injective but not surjective.
          Injectivity is clear by the fundamental theorem of arithmetic. There is no sequence sent to $3$, for example.
    \item Show that the function which maps $(x,n)$ to the exponent of $p_n$ in the prime factorization of $x$ is primitive recursive.
          $$\exp(x,n) = (\mu k \leq x)(p_n^{k+1} \nmid x )$$
    \item Deduce that
          \begin{enumerate}
              \item There exists a primitive recursive function that computes the $n$-th element of a sequence represented by $x$, when $x$ represents a sequence of length greater than or equal to $n$.\\
                    Take $\exp(x,n)$
              \item There exists a pr function that computes the length of the sequence encoded by $x$.\\
                    Take $l(x) = \exp(n,0)$.
              \item The characteristic function of the set $C$ of sequence codes is primitive recursive.\\
                    We have $x \in A$ iff $x \neq 0 $ and ($x=1 \lor 2 | x$).
          \end{enumerate}
    \item Show that there exists a primitive recursive function which, given two integers $n=\op{seq}(x_1,\dots,x_k)$ and $m=\op{seq}(y_1,\dots,y_h)$ encoding sequences, returns the number representing the concatenation of the two lists $\op{seq}(\bar{x},\bar{y})$.\\
          Take $\op{concat}(n,m) = \op{seq}(\exp(n,1)\dots,\exp(n,k),\exp(m,1),\dots,\exp(m,h))$
\end{enumerate}
\textbf{Exercise 16 (recursion with parameter substitution).} This is the scheme
\begin{align*}
    f(a,0)   & = g(a)                   \\
    f(a,x+1) & = h(a,x,f(\gamma(a),x)).
\end{align*}
\begin{enumerate}
    \item Show that the function $F$ is PR
          \begin{align*}
              F(p,a,0)   & = g(\gamma^p(a))                       \\
              F(p,a,x+1) & = h(\gamma^{p - (x+1)}(a),x,F(p,a,x)).
          \end{align*}
          We see that
          $$F(p,a,x+1) = h(\op{nth}([a;\gamma(a),\dots,\gamma^p(a),p-(x+1)]),x,F(p,a,x))$$
    \item Show that
          $$\forall x,a,p \in \N (x \leq p \Rightarrow F(p,a,x) = f(\gamma^{p-x}(a),x))$$
          and deduce that $f$ is primitive recursive.\\
          By induction on $x$ (we assume $x \leq p$ always)
          \begin{align*}
              F(p,a,0)   & = g(\gamma^p(a))                                 \\
              F(p,a,x+1) & = h(\gamma^{p-(x+1)}(a),x,F(p,a,x))              \\
                         & =  h(\gamma^{p-(x+1)}(a),x,f(\gamma^{p-x}(a),x)) \\
                         & = f(\gamma^{p-(x+1)}(a),x+1)
          \end{align*}
          We can deduce that $f(a,x) = F(x,a,x)$.
    \item Application: show that the function $\op{inc}:\N^2 \to \N$ which maps $i$ and $l = [a_0 ; \dots ; a_i ; \dots ; a_n]$ to  $[a_0 ; \dots ; a_i+1 ; \dots ; a_n]$, is primitive recursive.\\
          We can write
          \begin{align*}
              f(l,0)   & =(\op{hd}(l)+1)::\op{tl}(l)              \\
              f(l,i+1) & = \begin{cases}
                               f(\op{tl}(l),i) & \text{ if } i \leq n \\
                               l               & \text{ otherwise}
                           \end{cases}
          \end{align*}
\end{enumerate}
\textbf{Exercise 17 (double recursion without nesting).} Show that the function $f$ defined by
\begin{align*}
    f(0,y)     & = a                       \\
    f(x+1,0)   & = b                       \\
    f(x+1,y+1) & = h(x,y,f(x,y),f(x+1,y)).
\end{align*}
is primitive recursive.\\
We can use the encoding of pairs ($t = \langle x , y \rangle$), and the scheme of recursion on the sequence of values to write $f$ as follows
\begin{align*}
    f(t) = \begin{cases}
               b                                                                                                                         & \text{ if } \pi_2^1(t) = 0            \\
               a                                                                                                                         & \text{ otherwise and } \pi_2^2(t) = 0 \\
               h \Bigg( \pi_2^1(t) -1 ,  \pi_2^2(t)-1 ,                                                                                                                     \\ \op{nth} \bigg([f(0);f(1);\dots ;f(\alpha( \pi_2^1(t) , \pi_2^2(t)-1)], \alpha(\pi_2^1(t) -1 ,  \pi_2^2(t)-1 )\bigg) , \\
               \op{nth} \bigg([f(0);f(1);\dots ;f(\alpha( \pi_2^1(t) , \pi_2^2(t)-1)], \alpha(\pi_2^1(t) ,  \pi_2^2(t)-1 )\bigg)\Bigg) , & \text{ otherwise}                    \\
           \end{cases}
\end{align*}
\textbf{Ackermann Function}
\begin{align*}
    \op{Ack}(0,x)     & = x+2                        \\
    \op{Ack}(1,0)     & = 0                          \\
    \op{Ack}(n+2,0)   & = 1                          \\
    \op{Ack}(n+1,x+1) & =\op{Ack}(n,\op{Ack}(n+1,x))
\end{align*}
\textbf{Exercise 18. } Show that each function $\op{Ack}_n(x) = \op{Ack}(n,m)$ is primitive recursive and strictly increasing. Make explicit $\op{Ack}_n$, for $n =1,2,3$.
We proceed by induction on $n$. If $n=0,1$, it is evident that $\op{Ack}_n$ is primitive recursive. We assume that $\op{Ack}_n$ is primitive recursive for $n \geq 2$. We note that
\begin{align*}
    \op{Ack}_{n+1}(0)   & = 1                             \\
    \op{Ack}_{n+1}(x+1) & = \op{Ack}_n(\op{Ack}_{n+1}(x)) \\
\end{align*}
By the induction hypothesis, $\op{Ack}_{n+1} = \op{Rec}(1,\op{Ack}_n\circ \pi_2^2) \Rightarrow \op{Ack}_{n+1}$ is primitive recursive. We also have
\begin{align*}
    \op{Ack}_1(x) & = 2x                                                                                 \\
    \op{Ack}_2(x) & = 2^x                                                                                \\
    \op{Ack}_3(x) & = \underbrace{2 \  \hat{ \ } \cdots \hat{ \ } \  2 }_{x - \text{times } }\hat { \ } x
\end{align*}
The fact that the function is strictly increasing follows from an immediate application of induction on $n$.

\textbf{Exercise 19 (Ackermann function)}
\begin{enumerate}
    \item Verify that there exists exactly one function from $\N^2 \to \N$ satisfying the Ackermann function equations.
          Consider the recurrence
          $$ \op{Ack}_{n+1}(x+1) = \op{Ack}_n(\op{Ack}_{n+1}(x)),$$
          and note that, if $>_{lex}$ denotes the lexicographic order on $\N^2$, we have
          \begin{align*}
              (n+1,x+1) & >_{lex} (n+1,x)               \\
              (n+1,x+1) & >_{lex} (n,\op{Ack}_{n+1}(x))\end{align*}
          We can deduce that, to compute the value $\op{Ack}_{n+1}(x+1)$, we need the values that $\op{Ack}$ takes at pairs strictly smaller than $(x+1,n+1)$ (according to $>_{lex}$). Since $(\N^2, >_{lex})$ is a well-ordered set, it follows that the set of pairs needed to compute $\op{Ack}_{n+1}(x+1)$ is finite. Therefore, $\op{Ack}$ is well-defined on $\N^2$, and it is ``intuitively computable''.
    \item Show that
          $$\forall n \in \N \ \forall x > 0 \op{Ack}_{n+1}(x) = \op{Ack}_{n}^x(\op{Ack}_{n+1}(0))) $$
          and verify the expressions of the functions $\op{Ack}_1$, $\op{Ack}_2$, $\op{Ack}_3$.
          By induction on $x$. If $x=0$, it is trivial. For the case $x+1$, we have
          \begin{align*}
              \op{Ack}_{n+1}(x+1) & = \op{Ack}_{n}(\op{Ack}_{n+1}(x)) \text{ by definition }      \\
                                  & =  \op{Ack}_{n} (\op{Ack}_{n}^x(\op{Ack}_{n+1}(0))) \text{ IH} \\
                                  & =\op{Ack}_{n}^{x+1}(\op{Ack}_{n+1}(0)))
          \end{align*}
    \item Verify that each of the functions $\op{Ack}_n$ has a definition using exactly $n$ instances of the iteration definition scheme. The explicit forms of $\op{Ack}_1,\op{Ack}_2,\op{Ack}_3$ are in exercise 18.
          This follows directly from the definition, and by induction on $n$, noting that
          \begin{align*}
              \op{Ack}_{n+1}(0)   & = 1                             \\
              \op{Ack}_{n+1}(x+1) & = \op{Ack}_n(\op{Ack}_{n+1}(x)) \\
          \end{align*}
          Then, if $\op{Ack}_n \in \mathcal C_n$,  $\op{Ack}_{n+1} \in \mathcal C_{n+1}.$
    \item Show that  $\op{Ack}_n(x) > x$.\\
          By induction on $n$. If $x > 0$,
          \begin{align*}
              \op{Ack}_0(x) & = x+2 >x \\
              \op{Ack}_1(x) & = 2x >x
          \end{align*}
          If $n \geq 2$ and we assume that for all $x > 0$, $\op{Ack}_n(x)>x$,
          \begin{align*}
              \op{Ack}_{n+1}(1) & = \op{Ack}_n( \op{Ack}_{n+1}(0)) =  \op{Ack}_n(1) > 1 \\
          \end{align*}
          If $x > 1$, since $\op{Ack}$ is strictly increasing, $\op{Ack}_{n+1}(x) > 1 \neq 0$, and we can apply induction on $x$,
          \begin{align*}
              \op{Ack}_{n+1}(x+1) & = \op{Ack}_n( \op{Ack}_{n+1}(x))      \\
                                  & \geq\op{Ack}_{n+1}(x)+1 \text{ (IH1)} \\
                                  & >x+1 \text{ (IH2)}
          \end{align*}
    \item Deduce that for all integers $m$,  $\op{Ack}_m$ is strictly increasing.\\
          This has already been demonstrated in the previous exercise.
    \item Deduce from question $4$, that, from $2$ onwards, $\op{Ack}$ is non-decreasing on its first argument, the second being fixed:
          $$ \forall x \geq 2 \ \forall n \in \N \ \op{Ack}(n,x) \leq \op{Ack}(n+1,x).$$
          We have
          $$\op{Ack}_{n+1}(x) = \op{Ack}_n(\underbrace{\op{Ack_{n+1}}(x-1)}_{\geq x}) \geq \op{Ack}_{n}(x)$$
    \item Show that $\forall k , n \in \N \op{Ack}_n^k \in \mathcal C _n$.\\
          This is clear in view of exercise 19.3 and since $\mathcal C_n$ is closed under composition.
    \item Show that  $\forall k , n \in \N \op{Ack}_n^k(x) \leq \op{Ack}_{n+1}(x+k)$.
          By induction on $k$, the case $k=0$ is trivial, then
          \begin{align*}
              \op{Ack}_n^{k+1}(x) & = \op{Ack}_n(\op{Ack}_n^k(x))                    \\
                                  & \leq \op{Ack}_n(\op{Ack}_{n+1}(x+k)) \text{ IH } \\
                                  & = \op{Ack}_{n+1}(x+k+1) \text{ def }
          \end{align*}
    \item Show by induction on the definition of the set of primitive recursive functions that if $f \in \mathcal C_n$, then $\exists k \op{Ack}_n^k$ dominates $f$. \\
          It is easy to see that the base functions are dominated by $\op{Ack}_3(x)$.\newline If $h,g_1,\dots,g_m \in \mathcal C _n$, $h(\bar{x}) \leq \op{Ack}_n^{k}\sup(\bar{x},K)$ and  $g_i(\bar{x}) \leq \op{Ack}_n^{k_i}\sup(\bar{x},K_i)$, we set $M = \sup (K_1,\dots,K_m,K)$, $l = \sup_i k_i$, and $M(\bar{x}) = \sup( \bar{x} , M)$.
          \begin{align*}
              h(g_1(\bar{x}),\dots,g_m(\bar{x})) & \leq \op{Ack}_n^{k}\sup_i(g_i(\bar{x}),K)               \\ &\leq \op{Ack}_n^{k}\sup_i(\op{Ack}_n^{k_i}\sup(\bar{x},K_i),K) \\
                                                 & \leq \op{Ack}_n^{k}\sup_i(\op{Ack}_n^{k_i}(M(\bar{x}))) \\
                                                 & = \op{Ack}_n^{k}(\op{Ack}_n^{l}(M(\bar{x})))            \\
                                                 & = \op{Ack}_n^{k+l}\sup (\bar{x},M)
          \end{align*}
          Now, if $g(\bar{x}) \leq \op{Ack}_n^{k_1}\sup(\bar{x},N_1)$, and $h(\bar{x},y,z) \leq \op{Ack}_n^{k_2}\sup(\bar{x},y,z,N_2)$, the function obtained by primitive recursion $f \in \mathcal C_{n+1}$ satisfies
          $$f(\bar{x},y) \leq \op{Ack}_n^{k_1+k_2y}(\sup(\bar{x},y,N_1,N_2)$$
          We prove by induction on $y$,
          \begin{align*}
              f(\bar{x},0)   & = g(\bar{x}) \leq \op{Ack}_n^{k_1}\sup(\bar{x},N_1)                                    \\
              f(\bar{x},y+1) & = h(\bar{x},y,f(\bar{x},y))                                                            \\
                             & \leq \op{Ack}_n^{k_2}(\sup(\bar{x},y,f(\bar{x},y),N_2)                                 \\
                             & \leq \op{Ack}_n^{k_2}(\sup(\bar{x},y,\op{Ack}_n^{k_1+k_2y}\sup(\bar{x},y,N_1,N_2),N_2) \\
                             & =\op{Ack}_n^{k_2}(\op{Ack}_n^{k_1+k_2y}\sup(\bar{x},y,N_1,N_2))                        \\
                             & =  \op{Ack}_n^{k_1+k_2(y+1)}\sup(\bar{x},y,N_1,N_2)                                    \\
                             & \leq \op{Ack}_{n+1}(\sup(\bar{x},y,N_1,N_2) + k_1+k_2y)
          \end{align*}
          This last function is a composition of $C_{n+1}$ functions and therefore is dominated by some $\op{Ack}_{n+1}^l$.
    \item Show that $\op{Ack}_n^k$ is dominated by $\op{Ack}_{n+1}$.\\
          Note that if $y>0$, $\op{Ack}_{n+1}(y) \geq \op{Ack}_{1}(y) =2y$, and we can deduce that if  $x>2k$, \\ $\op{Ack}_{n+1}(x-k) \geq 2x - 2k > x$. Then, for all $x > 2k$,
          \begin{alignat*}{2}
                          &  & \op{Ack}_{n+1}(x-k)                 & > x                                     \\
              \Rightarrow &  & \op{Ack}_{n}^k(\op{Ack}_{n+1}(x-k)) & >\op{Ack}_{n}^k(x)                      \\
              \Rightarrow &  & \op{Ack}_{n+1}(x)                   & >  \op{Ack}_{n}^k(x)  \text{ (ex 19.2)}
          \end{alignat*}
          This shows that $\op{Ack}_n^k$ is dominated by $\op{Ack}_{n+1}$.
    \item Deduce that if $f \in \mathcal C_n$, then $\op{Ack}_{n+1}$ dominates $f$.\\
          If $f\in \mathcal C_n$, $\exists k$ such that $f$ is dominated by $\op{Ack}_n^k$, and by the previous exercise, $\op{Ack}_n^k$ is dominated by $\op{Ack}_{n+1}$. Moreover, $\op{Ack}_{n+1} \notin \mathcal C_n$.
    \item Deduce that the Ackermann function is not primitive recursive. Show that the diagonal function $\op{Ack}(n,n)$ dominates all primitive recursive functions. \\
          If $\op{Ack}(n,n) \in \mathcal C_k$,
          $$\exists N  \ \forall n  > N \ \op{Ack}(n,n) \leq \op{Ack}_k(n),$$
          which is impossible if $n > N,k$.
          If $f$ is primitive recursive, $f \in \mathcal C_n$ for some $n$, so using the previous exercises, except for finitely many values of $\bar{x}$
          $$f(\bar{x}) \leq \op{Ack}_{n}^k(\sup(\bar{x})) \leq \op{Ack}_{n+1}(\sup(\bar{x})) \leq \op{Ack}_{\sup(\bar{x})}(\sup(\bar{x}))$$
\end{enumerate}
\end{document}
