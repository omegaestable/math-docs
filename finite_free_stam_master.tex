\documentclass[11pt,a4paper]{amsart}

\usepackage[margin=1in]{geometry}
\usepackage[T1]{fontenc}
\usepackage{lmodern}
\usepackage{microtype}
\usepackage{amsmath,amssymb,amsthm,mathtools}
\usepackage{enumitem}
\usepackage{booktabs}
\usepackage{longtable}
\usepackage{array}
\usepackage{xcolor}
\usepackage[colorlinks=true,linkcolor=blue!60!black,citecolor=blue!60!black,urlcolor=blue!60!black]{hyperref}

\allowdisplaybreaks
\setlength{\jot}{7pt}

% ─────────────── theorem environments ───────────────
\theoremstyle{plain}
\newtheorem{theorem}{Theorem}[section]
\newtheorem{lemma}[theorem]{Lemma}
\newtheorem{proposition}[theorem]{Proposition}
\newtheorem{corollary}[theorem]{Corollary}
\newtheorem{conjecture}[theorem]{Conjecture}

\theoremstyle{definition}
\newtheorem{definition}[theorem]{Definition}
\newtheorem{example}[theorem]{Example}

\theoremstyle{remark}
\newtheorem{remark}[theorem]{Remark}
\newtheorem{observation}[theorem]{Observation}
\newtheorem{warning}[theorem]{Warning}

% ─────────────── macros ───────────────
\newcommand{\R}{\mathbb{R}}
\newcommand{\C}{\mathbb{C}}
\newcommand{\N}{\mathbb{N}}
\newcommand{\Z}{\mathbb{Z}}
\newcommand{\HH}{\mathbb{H}} % upper half-plane
\newcommand{\PnR}[1][n]{\mathcal{P}^{\R}_{#1}}
\newcommand{\fp}{\boxplus_n}
\DeclareMathOperator{\tr}{tr}
\DeclareMathOperator{\diag}{diag}
\DeclareMathOperator{\disc}{disc}
\DeclareMathOperator{\Bez}{Bez}
\DeclareMathOperator{\Var}{Var}
\DeclareMathOperator{\rank}{rank}
\DeclareMathOperator{\sgn}{sgn}
\DeclareMathOperator{\Res}{Res}

\newcommand{\proved}{\textbf{[\,Proved\,]}}
\newcommand{\deadend}{\textbf{[\,Dead End\,]}}
\newcommand{\numconf}{\textbf{[\,Numerically Confirmed\,]}}
\newcommand{\open}{\textbf{[\,Open\,]}}
\newcommand{\compverif}{\textbf{[\,Computer-verified\,]}}
\newcommand{\proofsketch}{\textbf{[\,Proof sketch\,]}}
\newcommand{\conditional}{\textbf{[\,Conditional\,]}}

% ─────────────── title ───────────────
\begin{document}

\title[Finite Free Stam Inequality: Master Compendium]%
{On the Finite Free Stam Inequality\\[6pt]
\large A Compendium of Proved Results, Dead Ends,\\
Structural Identities, and Open Directions}

\author{}
\date{\today}

\begin{abstract}
Let $\PnR$ denote the set of monic, degree-$n$, real-rooted polynomials
and let $\fp$ denote the Marcus--Spielman--Srivastava finite free
additive convolution.
For $r\in\PnR$ with simple roots $\lambda_1<\cdots<\lambda_n$, define
the \emph{Fisher information}
$\Phi_n(r)=\sum_{i=1}^n\bigl(\sum_{j\ne i}(\lambda_i-\lambda_j)^{-1}\bigr)^2$.

This document is a working compendium for the
\emph{finite free Stam inequality}
\begin{equation}\label{eq:stam-main}
  \frac{1}{\Phi_n(p\fp q)}\;\ge\;\frac{1}{\Phi_n(p)}+\frac{1}{\Phi_n(q)},
  \qquad p,q\in\PnR,
\end{equation}
the polynomial analogue of classical Stam.
The purpose of this version is to give a compact, accurate handoff for
future iterations: what is proved, what is only computer-verified, what
failed, and what should be attempted next.

\medskip\noindent
\textbf{Proof-status conventions.}
Results are classified as follows:
\proved{} = fully rigorous proof;
\conditional{} = result whose proof relies on one or more
explicitly identified unproved intermediate statements;
\compverif{} = statement verified by exhaustive symbolic or
numerical computation (typically $\ge 10^5$ random feasible trials,
zero violations) but lacking a closed-form algebraic certificate;
\proofsketch{} = outline whose key steps are justified but whose
full details are deferred;
\open{} = statement without proof or reduction.
All conditional dependencies and computer-verified steps are
explicitly flagged in the text.

Current high-level status:
$n=2,3$ are fully proved;
Gaussian-input Stam for all~$n$ is proved \emph{conditional}
on the root ODE $\dot\lambda_i=V_i/(n-1)$ under Hermite flow
(which has not yet been derived from first principles);
the $n=4$ argument is rigorously reduced to
computer-verified polynomial inequalities that await
closed-form algebraic certificates;
a quantitative local Stam inequality near the Hermite manifold
is proved for all~$n$ (as a proof sketch with an identified
uniform-in-$w$ gap);
global Stam for $n\ge 5$ remains open.
See Table~\ref{tab:status} for a complete classification.
\end{abstract}

\maketitle
\tableofcontents
\newpage

%%%%%%%%%%%%%%%%%%%%%%%%%%%%%%%%%%%%%%%%%%%%%%%%%%%%%%%%%%%%
\section*{Global status table}\label{sec:status-table}
%%%%%%%%%%%%%%%%%%%%%%%%%%%%%%%%%%%%%%%%%%%%%%%%%%%%%%%%%%%%

\addcontentsline{toc}{section}{Global status table}

Table~\ref{tab:status} classifies every major result
in this document.  The status categories are defined in the
abstract.  References point to the theorem or observation
where each result is stated.

\begin{table}[ht]
\centering
\footnotesize
\renewcommand{\arraystretch}{1.08}
\caption{Proof-status classification of all major results.}
\label{tab:status}
\begin{tabular}{p{4.8cm}lp{3.5cm}}
\toprule
\textbf{Result} & \textbf{Status} & \textbf{Reference} \\
\midrule
\multicolumn{3}{l}{\emph{Structural identities (Part~\ref{part:foundations})}} \\
Fisher--repulsion $\Phi_n=2\mathcal{R}$ & \proved & Thm~\ref{thm:fisher-rep} \\
Fisher--trace--Laplacian & \proved & Thm~\ref{thm:phi-trL} \\
Score identities (i)--(iv) & \proved & Lem~\ref{lem:score-ids} \\
Fisher--variance inequality & \proved & Thm~\ref{thm:fisher-var} \\
Score-gradient inequality & \proved & Thm~\ref{thm:SGI} \\
Variance additivity & \proved & Lem~\ref{lem:var-add} \\
Derivative compatibility & \proved & Lem~\ref{lem:deriv} \\
Bezoutian representation & \proved & Thm~\ref{thm:bez} \\
Harmonicity of $\log\disc$ & \proofsketch{} + \compverif & Thm~\ref{thm:harmonicity} \\
Contour integral for $\Phi_n$ & \proved & Thm~\ref{thm:contour} \\
Isoperimetric inequality & \proved & Prop~\ref{prop:isoperimetric} \\
Cumulant-ratio defect $\Delta_k\ge 0$ & \proved & Lem~\ref{lem:defect-pos} \\
CS mixing inequality & \proved & Lem~\ref{lem:cs-mix} \\
\midrule
\multicolumn{3}{l}{\emph{Special cases (Part~\ref{part:foundations})}} \\
Stam for $n=2$ (equality) & \proved & \S\ref{sec:special} \\
Stam for $n=3$ (SOS proof) & \proved & Thm~\ref{thm:n3-stam} \\
Stam for $n=3$ (CS mixing) & \proved & Thm~\ref{thm:n3-cs} \\
De Bruijn identity & \conditional & Thm~\ref{thm:debruijn} \\
Gaussian-input Stam, all $n$ & \conditional & Thm~\ref{thm:stam-gauss} \\
\midrule
\multicolumn{3}{l}{\emph{CS mixing framework (Part~\ref{part:route-j})}} \\
Hessian of $G_n$, $n=3,4$ & \proved & Thm~\ref{thm:hessian} \\
Hessian of $G_n$, $n\ge 5$ & \compverif & Thm~\ref{thm:hessian} \\
$R_3=\tfrac{9}{8}\tau_3^2$ (exact) & \proved & Thm~\ref{thm:R3-exact} \\
Quadratic Stam, $n=3$ & \proved & Thm~\ref{thm:quad-stam} \\
Quadratic Stam, $n\ge 4$ & \conditional on Hessian & Thm~\ref{thm:quad-stam} \\
KStam kurtosis axis, $n=4$ & \proved{} + \compverif & Thm~\ref{thm:stam-kurt4} \\
Stam for $n=4$ & \compverif{} (3 ineqs) & Thm~\ref{thm:stam-n4} \\
\midrule
\multicolumn{3}{l}{\emph{MSS interlacing framework (Part~\ref{part:mss})}} \\
$K$-cumulant preservation & \proved & Thm~\ref{thm:kappa-pres} \\
Score--Cauchy identity & \proved & Thm~\ref{thm:score-cauchy} \\
Column-sum vanishing & \proved & Thm~\ref{thm:col-zero} \\
Frobenius norm $\|C\|_F^2=4\Phi_n$ & \proved & Thm~\ref{thm:frob} \\
Deficit telescoping & \proved & Thm~\ref{thm:telescope} \\
Chain dominance $D_n\ge\delta_n D_3$ & \compverif & Conj~\ref{thm:chain-dom} \\
\midrule
\multicolumn{3}{l}{\emph{Gate analysis and local results (Part~\ref{part:gates})}} \\
Local Stam for all $n$ & \proofsketch & Thm~\ref{thm:local-stam} \\
$\mathcal{E}_n\ge 0$ & \deadend & Obs~\ref{obs:gate1-false} \\
Ladder monotonicity $D_k\ge D_{k-1}$ & \deadend & Obs~\ref{obs:gate2-false} \\
Universal level-wise Stam & \compverif & Obs~\ref{obs:universal-stam} \\
Frobenius reduction bound & \compverif & Obs~\ref{obs:frob-reduction} \\
\midrule
\multicolumn{3}{l}{\emph{Open conjectures}} \\
Stam for all $n\ge 4$ & \open & Conj~\ref{conj:stam} \\
$R_n$ sub-averaging & \open & Conj~\ref{thm:Rn-subavg} \\
Flow monotonicity & \open & Conj~\ref{conj:flow-mono} \\
Real stability of $K_r$ & \open & Conj~\ref{thm:K-symbol-stable} \\
Newton wall compactness & \open & Conj~\ref{thm:compact-feas} \\
\bottomrule
\end{tabular}
\end{table}

%%%%%%%%%%%%%%%%%%%%%%%%%%%%%%%%%%%%%%%%%%%%%%%%%%%%%%%%%%%%
%%%%%%%%%%%%%%%%%%%%%%%%%%%%%%%%%%%%%%%%%%%%%%%%%%%%%%%%%%%%
\part{Foundations}\label{part:foundations}
%%%%%%%%%%%%%%%%%%%%%%%%%%%%%%%%%%%%%%%%%%%%%%%%%%%%%%%%%%%%
%%%%%%%%%%%%%%%%%%%%%%%%%%%%%%%%%%%%%%%%%%%%%%%%%%%%%%%%%%%%

\section{Setup and definitions}\label{sec:setup}

\subsection{MSS finite free additive convolution}

\begin{definition}[MSS convolution]\label{def:mss}
Write $p(x)=\sum_{k=0}^n a_k x^{n-k}$,
$q(x)=\sum_{k=0}^n b_k x^{n-k}$ with $a_0=b_0=1$.
The \emph{finite free additive convolution}
$r=p\fp q$ is defined by
\[
  r(x)=\sum_{k=0}^n c_k x^{n-k},\qquad
  c_k=\sum_{i+j=k}\frac{(n-i)!\,(n-j)!}{n!\,(n-k)!}\,a_i b_j.
\]
By the Marcus--Spielman--Srivastava theorem~\textup{[1]},
$\fp$ preserves $\PnR$: if $p,q\in\PnR$ then $p\fp q\in\PnR$.
\end{definition}

\begin{definition}[$K$-transform and log-cumulants]\label{def:K}
Define $\kappa_k(r):=(n-k)!\,c_k(r)/n!$ and
$K_r(z):=\sum_{k=0}^n\kappa_k(r)\,z^k$.
Then $\fp$ becomes multiplicative:
\begin{equation}\label{eq:K-mult}
  K_{p\fp q}(z)=K_p(z)\cdot K_q(z)\pmod{z^{n+1}}.
\end{equation}
The \emph{log-cumulants} $\ell_k(r):=[z^k]\log K_r(z)$
are computed by
$\ell_1=\kappa_1$,\;
$\ell_k=\kappa_k-\frac{1}{k}\sum_{j=1}^{k-1}j\,\ell_j\kappa_{k-j}$
for $k\ge 2$.
They are \textbf{additive}: $\ell_k(p\fp q)=\ell_k(p)+\ell_k(q)$
for all $k$.
\end{definition}

\subsection{Scores and Fisher information}

\begin{definition}[Scores, Fisher information, repulsion]\label{def:scores}
For $r\in\PnR$ with simple roots $\lambda_1<\cdots<\lambda_n$:
\begin{align}
  V_i(r)&:=\sum_{j\ne i}\frac{1}{\lambda_i-\lambda_j},
  \qquad V=(V_1,\ldots,V_n)\quad(\text{score vector}), \label{eq:scores}\\
  \Phi_n(r)&:=\sum_{i=1}^n V_i^2\quad(\text{Fisher information}),\label{eq:Fisher}\\
  \mathcal{R}(r)&:=\sum_{1\le i<j\le n}\frac{1}{(\lambda_i-\lambda_j)^2}
  \quad(\text{repulsion energy}),\label{eq:repulsion}\\
  \mathcal{S}(r)&:=\sum_{1\le i<j\le n}
  \frac{(V_i-V_j)^2}{(\lambda_i-\lambda_j)^2}
  \quad(\text{score-gradient energy}).\label{eq:SGE}
\end{align}
If $r$ has a repeated root, we set $\Phi_n(r)=\infty$.
\end{definition}

\begin{definition}[Graph Laplacian]\label{def:curv}
The \emph{graph Laplacian} of $r$ is $L\in\R^{n\times n}$ with
\[
  L_{ij}=\begin{cases}
    -(\lambda_i-\lambda_j)^{-2}&i\ne j,\\
    \sum_{k\ne i}(\lambda_i-\lambda_k)^{-2}&i=j.
  \end{cases}
\]
This is the complete-graph Laplacian with edge weights $w_{ij}=(\lambda_i-\lambda_j)^{-2}$.
We have $L\mathbf{1}=0$, $L\succeq 0$, $\ker L=\mathrm{span}\{\mathbf{1}\}$,
$\rank L=n-1$.
Equivalently, $L=-\tfrac{1}{2}\mathrm{Hess}_\lambda(\log\disc(r))$.
\end{definition}

\subsection{Variance and additive parameters}

\begin{definition}[Additive mean and variance]\label{def:var}
For $r\in\PnR$ with roots $\lambda_1<\cdots<\lambda_n$:
\begin{align}
  \mu(r)&:=-\frac{a_1(r)}{n}=\frac{1}{n}\sum_{i=1}^n\lambda_i
  \qquad(\text{centroid}),\label{eq:mu}\\
  \sigma^2(r)&:=\frac{1}{n}\sum_{i=1}^n(\lambda_i-\mu)^2
  =\frac{(n-1)\,a_1(r)^2-2n\,a_2(r)}{n^2}
  \qquad(\text{empirical variance}).\label{eq:sigma}
\end{align}
Both are \textbf{additive}: $\mu(p\fp q)=\mu(p)+\mu(q)$ and
$\sigma^2(p\fp q)=\sigma^2(p)+\sigma^2(q)$.
\end{definition}

\begin{definition}[Finite Gaussian]\label{def:gauss}
The \emph{finite Gaussian} $g_t\in\PnR$ of variance~$t>0$ is
the unique centred polynomial whose $K$-transform
has $\kappa_k(g_t)=0$ for all $k\ne 0,2$.
Equivalently, all log-cumulants $\ell_k(g_t)=0$ for $k\ge 3$,
so $g_t$ is characterised by $\sigma^2(g_t)=t$.
The Hermite semigroup satisfies $g_s\fp g_t=g_{s+t}$.
For example: $g_t=x^2-t$ at $n=2$;\;
$g_t=x^3-\frac{3}{2}t\,x$ at $n=3$.
\end{definition}

\begin{definition}[Normalised cumulant ratios]\label{def:tau}
For centred $r\in\PnR$ with $u:=-\ell_2(r)>0$, define
$\tau_k(r):=\ell_k(r)/u(r)^{k/2}$ for $k\ge 3$.
The \emph{additive variance parameter}
$u=\sigma^2/(2(n-1))$ satisfies $u(p\fp q)=u(p)+u(q)$.
\end{definition}

\begin{lemma}[Normalisation identities]\label{lem:normalisation}
For centred $r\in\PnR$ $($i.e., $\mu(r)=0$$)$, the parameters
$\kappa_2$, $\ell_2$, $u$, and $\sigma^2$ are related by:
\begin{equation}\label{eq:norm-chain}
  \ell_2\;=\;\kappa_2\;=\;\frac{(n-2)!\,a_2}{n!}
  \;=\;\frac{a_2}{n(n-1)},
  \qquad
  u\;:=\;-\ell_2\;>\;0,
  \qquad
  \sigma^2\;=\;2(n-1)\,u.
\end{equation}
Here $a_2$ is the coefficient of $x^{n-2}$ in$~r$ $($so $a_2<0$
for centred real-rooted $r$ with $n\ge 2$$)$.
\end{lemma}

\begin{proof}
From Definition~\ref{def:K}: $\kappa_2=(n-2)!\,a_2/n!$.
The log-cumulant recurrence (Definition~\ref{def:K}) gives
$\ell_2=\kappa_2-\frac{1}{2}\kappa_1^2=\kappa_2$
when $r$ is centred ($\kappa_1=\ell_1=0$).
From~\eqref{eq:sigma} with $a_1=0$:
$\sigma^2=-2a_2/n=-2n(n-1)\ell_2/n=2(n-1)(-\ell_2)=2(n-1)u$.
All three parameters are additive under $\fp$
because $\ell_2$ is additive (Definition~\ref{def:K}).
\end{proof}


%%%%%%%%%%%%%%%%%%%%%%%%%%%%%%%%%%%%%%%%%%%%%%%%%%%%%%%%%%%%
\section{Proved structural identities}\label{sec:identities}

We collect all rigorously established identities.  These are the
``library components'' on which any future proof of~\eqref{eq:stam-main}
can draw.

% ─── Fisher = 2R ───
\subsection{Fisher--repulsion identity}\label{ssec:fisher-rep}

\begin{theorem}[Fisher--repulsion identity]\label{thm:fisher-rep}
For any $r\in\PnR$ with simple roots,
\begin{equation}\label{eq:phi-2R}
  \Phi_n(r)=2\,\mathcal{R}(r).
\end{equation}
\end{theorem}

\begin{proof}
Expand $V_i^2=\sum_{j\ne i}\sum_{k\ne i}
(\lambda_i-\lambda_j)^{-1}(\lambda_i-\lambda_k)^{-1}$ and sum over~$i$.
The diagonal terms ($j=k$) contribute $\sum_i\sum_{j\ne i}(\lambda_i-\lambda_j)^{-2}
=2\sum_{i<j}(\lambda_i-\lambda_j)^{-2}=2\mathcal{R}$.
The cross terms ($j\ne k$, both $\ne i$) group into triples $\{a,b,c\}$,
each contributing
\[
  \frac{1}{(a-b)(a-c)}+\frac{1}{(b-a)(b-c)}+\frac{1}{(c-a)(c-b)}=0
\]
(partial-fraction identity for the residues of $1/((x-a)(x-b)(x-c))$).
Hence $\Phi_n=2\mathcal{R}+0=2\mathcal{R}$.
\end{proof}

\begin{corollary}[Stam as harmonic-mean repulsion]\label{cor:stam-R}
Inequality~\eqref{eq:stam-main} is equivalent to
$1/\mathcal{R}(p\fp q)\ge 1/\mathcal{R}(p)+1/\mathcal{R}(q)$.
\end{corollary}

% ─── Fisher = tr(L) ───
\subsection{Fisher--trace--Laplacian identity}

\begin{theorem}[Fisher = $\tr(L)=\lambda^T L^2\lambda$]\label{thm:phi-trL}
For $r\in\PnR$:
\begin{enumerate}[label=\textup{(\alph*)}]
  \item $\Phi_n=\tr(L)=2\mathcal{R}$.
  \item $V=L\lambda$ \textup{(Euler identity)}.
  \item $\lambda^T L\lambda=\binom{n}{2}$ \textup{(universal constant)}.
  \item $\Phi_n=\|V\|^2=\|L\lambda\|^2=\lambda^T L^2\lambda$.
\end{enumerate}
\end{theorem}

\begin{proof}
(a) follows from $\Phi_n=2\mathcal{R}$ and $\tr(L)=2\sum_{i<j}(\lambda_i-\lambda_j)^{-2}$.

(b) We compute $(L\lambda)_i=\sum_{j\ne i}\frac{\lambda_i-\lambda_j}{(\lambda_i-\lambda_j)^2}=\sum_{j\ne i}\frac{1}{\lambda_i-\lambda_j}=V_i$.

(c) $\lambda^T L\lambda=V\cdot\lambda=\frac{1}{2}\sum_i\partial_{\lambda_i}\log\disc\cdot\lambda_i
=\frac{n(n-1)}{2}$
by the Euler identity for homogeneity of $\disc$ (degree $n(n-1)$).

(d) is immediate from $V=L\lambda$.
\end{proof}

% ─── Score identities ───
\subsection{Score identities}

\begin{lemma}[Score identities]\label{lem:score-ids}
For $r\in\PnR$ with simple roots:
\begin{enumerate}[label=\textup{(\roman*)}]
  \item $\sum_i V_i=0$.
  \item $\sum_i(\lambda_i-\mu)V_i=\binom{n}{2}$ for any $\mu\in\R$.
  \item $\Phi_n=\sum_{i<j}\frac{V_i-V_j}{\lambda_i-\lambda_j}$.
  \item $V_i=r''(\lambda_i)/(2r'(\lambda_i))$
    \textup{(score-of-derivative identity)}.
\end{enumerate}
\end{lemma}

\begin{proof}
(i) By symmetry: $\sum_i V_i=\sum_{i\ne j}(\lambda_i-\lambda_j)^{-1}=0$
(antisymmetric sum).

(ii) $\sum_i\lambda_i V_i=\sum_{i\ne j}\lambda_i/(\lambda_i-\lambda_j)
=\sum_{i\ne j}\bigl[1+\lambda_j/(\lambda_i-\lambda_j)\bigr]
=n(n-1)+\sum_{i\ne j}\lambda_j/(\lambda_i-\lambda_j)$.
Using $\sum_{i\ne j}\lambda_j/(\lambda_i-\lambda_j)
=-\sum_{i\ne j}\lambda_i/(\lambda_j-\lambda_i)
=-\sum_i\lambda_i V_i$,
we get $2\sum_i\lambda_i V_i=n(n-1)$,
so $\sum_i\lambda_i V_i=\binom{n}{2}$.
By (i), subtracting $\mu\sum V_i=0$ gives (ii).

(iii) $\sum_{i<j}(V_i-V_j)/(\lambda_i-\lambda_j)
=\sum_i V_i\sum_{j\ne i}(\lambda_i-\lambda_j)^{-1}
-\sum_{i<j}\bigl[V_j/(\lambda_i-\lambda_j)+V_i/(\lambda_j-\lambda_i)\bigr]$.
Rewriting the double sum: for each pair $i<j$, the term
$V_i/(\lambda_i-\lambda_j)+V_j/(\lambda_j-\lambda_i)$ contributes
to $\sum_k V_k\cdot(\text{sum of }1/(\lambda_k-\lambda_m)$ for $m\ne k)$.
In fact,
\begin{align*}
  \sum_{i<j}\frac{V_i-V_j}{\lambda_i-\lambda_j}
  &=\sum_{i<j}\frac{1}{\lambda_i-\lambda_j}
    \Bigl(\sum_{k\ne i}\frac{1}{\lambda_i-\lambda_k}
    -\sum_{k\ne j}\frac{1}{\lambda_j-\lambda_k}\Bigr)\\
  &=\sum_{i<j}\sum_{k\ne i}\frac{1}{(\lambda_i-\lambda_j)(\lambda_i-\lambda_k)}
    -\sum_{i<j}\sum_{k\ne j}\frac{1}{(\lambda_i-\lambda_j)(\lambda_j-\lambda_k)}.
\end{align*}
Relabelling the second sum by interchanging $i\leftrightarrow j$
(which reverses both the sign of $(\lambda_i-\lambda_j)$ and the
ordering to $j<i$, i.e.\ $i>j$) yields
\[
  \sum_{j<i}\sum_{k\ne i}\frac{1}{(\lambda_j-\lambda_i)(\lambda_i-\lambda_k)}
  =-\sum_{i<j}\sum_{k\ne j}\frac{1}{(\lambda_i-\lambda_j)(\lambda_j-\lambda_k)},
\]
so the two double sums combine to give
\[
  2\sum_{i<j}\sum_{\substack{k\ne i}}\frac{1}{(\lambda_i-\lambda_j)(\lambda_i-\lambda_k)}.
\]
Separating the diagonal terms ($k=j$) from the cross terms ($k\ne j$, $k\ne i$):
the diagonal terms give $\sum_{i<j}(\lambda_i-\lambda_j)^{-2}$
summed over the appropriate range, yielding $\sum_i\sum_{j\ne i}(\lambda_i-\lambda_j)^{-2}=\Phi_n$.
The cross terms, grouped into triples $\{i,j,k\}$, each contribute
$1/((\lambda_i-\lambda_j)(\lambda_i-\lambda_k))
+1/((\lambda_j-\lambda_i)(\lambda_j-\lambda_k))
+1/((\lambda_k-\lambda_i)(\lambda_k-\lambda_j))=0$
by the partial-fraction identity
$\sum_{\text{cyc}}1/((a-b)(a-c))=0$
(the same identity as in Theorem~\ref{thm:fisher-rep}).
Hence $\sum_{i<j}(V_i-V_j)/(\lambda_i-\lambda_j)=\Phi_n$.

(iv) Since $r'(\lambda_i)=\prod_{j\ne i}(\lambda_i-\lambda_j)$,
$V_i=\sum_{j\ne i}(\lambda_i-\lambda_j)^{-1}
=r''(\lambda_i)/(2r'(\lambda_i))$.
\end{proof}

% ─── Fisher--variance ───
\subsection{Fisher--variance and score-gradient inequalities}

\begin{theorem}[Fisher--variance inequality]\label{thm:fisher-var}
$\Phi_n(r)\cdot\sigma^2(r)\ge n(n-1)^2/4$.
\end{theorem}

\begin{proof}
Cauchy--Schwarz on $\sum_i(\lambda_i-\mu)V_i=\binom{n}{2}$
with $\sum V_i=0$:
$\bigl|\sum(\lambda_i-\mu)V_i\bigr|^2
\le\bigl(\sum(\lambda_i-\mu)^2\bigr)\bigl(\sum V_i^2\bigr)
=n\sigma^2\cdot\Phi_n$.
\end{proof}

\begin{theorem}[Score-Gradient Inequality]\label{thm:SGI}
For simple-root $r\in\PnR$, $n\ge 2$:
$\mathcal{S}(r)\cdot\sigma^2(r)\ge\frac{n-1}{2}\,\Phi_n(r)$.
\end{theorem}

\begin{proof}
Two Cauchy--Schwarz applications.
From Lemma~\ref{lem:score-ids}(ii): $n\sigma^2\cdot\Phi_n\ge n^2(n-1)^2/4$.
From Lemma~\ref{lem:score-ids}(iii): $\Phi_n^2\le\mathcal{S}\cdot n(n-1)/2$.
Combining: $\mathcal{S}\sigma^2\ge(n-1)\Phi_n/2$.
\end{proof}

% ─── Variance additivity ───
\subsection{Variance additivity and derivative compatibility}

\begin{lemma}[Variance additivity]\label{lem:var-add}
$\sigma^2(p\fp q)=\sigma^2(p)+\sigma^2(q)$.
\end{lemma}

\begin{proof}
From the MSS coefficient formula (Definition~\ref{def:mss}):
$c_1=a_1+b_1$, $c_2=a_2+b_2+\frac{n-1}{n}a_1 b_1$.
Using~\eqref{eq:sigma}:
\begin{align*}
  \sigma^2(p\fp q)
  &=\frac{(n\!-\!1)(a_1\!+\!b_1)^2-2n(a_2\!+\!b_2\!+\!\tfrac{n-1}{n}a_1 b_1)}{n^2}\\
  &=\frac{(n\!-\!1)a_1^2\!-\!2na_2}{n^2}
   +\frac{(n\!-\!1)b_1^2\!-\!2nb_2}{n^2}
   +\frac{2(n\!-\!1)a_1b_1\!-\!2(n\!-\!1)a_1b_1}{n^2}\\
  &=\sigma^2(p)+\sigma^2(q).\qedhere
\end{align*}
\end{proof}

\begin{lemma}[Derivative compatibility]\label{lem:deriv}
$(p\fp q)'/n=(p'/n)\boxplus_{n-1}(q'/n)$.
\end{lemma}

\begin{proof}
Write $p=\sum a_k x^{n-k}$, $q=\sum b_k x^{n-k}$.
Then $(p\fp q)(x)=\sum c_k x^{n-k}$ with
$c_k=\sum_{i+j=k}\frac{(n-i)!(n-j)!}{n!(n-k)!}a_i b_j$.
The derivative of $p$ is $p'(x)/n=\sum_{k=0}^{n-1}
\frac{n-k}{n}a_k x^{n-1-k}$,
which is a monic degree-$(n-1)$ polynomial with coefficients
$\tilde a_k=(n-k)a_k/n$.
A direct calculation confirms that the $\boxplus_{n-1}$
convolution of $p'/n$ and $q'/n$ reproduces $(p\fp q)'/n$,
using the identity
$\frac{(n-i)(n-j)}{n^2}\cdot\frac{(n-1-i)!(n-1-j)!}{(n-1)!(n-1-k)!}
=\frac{(n-k)}{n}\cdot\frac{(n-i)!(n-j)!}{n!(n-k)!}$
for $i+j=k$.
\end{proof}

% ─── Bezoutian representation ───
\subsection{Bezoutian representation}

\begin{theorem}[Bezoutian representation of $\Phi_n$]\label{thm:bez}
Let $\Bez(r,r')$ denote the Bezoutian of $r$ and $r'$~\textup{[9]}. Then
\[
  \Phi_n(r)=\Bigl\|\frac{r''}{2}\Bigr\|^2_{\Bez(r,r')}
  =\sum_{i=1}^n\frac{r''(\lambda_i)^2}{4\,r'(\lambda_i)^2}.
\]
\end{theorem}

\begin{proof}
The Bezoutian matrix $\Bez(r,r')$ is the unique symmetric matrix
$B\in\R^{n\times n}$ satisfying
$\sum_{i,j}B_{ij}x^{n-1-i}y^{n-1-j}
=\bigl(r(x)r'(y)-r'(x)r(y)\bigr)/(x-y)$.
The associated inner product is diagonal in the Lagrange basis
$\{L_i(x)=\prod_{j\ne i}(x-\lambda_j)/\prod_{j\ne i}(\lambda_i-\lambda_j)\}$:
$\langle f,g\rangle_{\Bez}=\sum_i f(\lambda_i)g(\lambda_i)/r'(\lambda_i)^2$
(see~\textup{[9,~\S3]} for the diagonalisation in the Lagrange basis).
Since $V_i=r''(\lambda_i)/(2r'(\lambda_i))$
(Lemma~\ref{lem:score-ids}(iv)), we get
$\Phi_n=\sum V_i^2=\|r''/2\|^2_{\Bez}$.
\end{proof}

% ─── Harmonicity theorem ───
\subsection{Harmonicity of $\log\disc$ in matrix coordinates}

\begin{theorem}[Harmonicity of $\log\disc$]\label{thm:harmonicity}
\proofsketch{} + \compverif{}

\noindent
Let $A\in\mathrm{Sym}(n)$ have simple eigenvalues. Then
\[
  \Delta_A\log\disc(\det(xI-A))=0,
\]
where $\Delta_A$ is the Laplace--Beltrami operator on $\mathrm{Sym}(n)$.
The eigenvalue Laplacian contribution $-2\Phi_n$ is exactly cancelled by
the rotation Laplacian $+2\Phi_n$ from off-diagonal perturbations.

\medskip\noindent
\textbf{Proof status.}  The argument below is a \emph{proof sketch}:
the diagonal perturbation computation is complete, but the
off-diagonal perturbation requires second-order eigenvalue
perturbation theory whose algebra is outlined but not
expanded line-by-line.
The identity has been verified symbolically at $n=3$--$8$
(error $<5\times 10^{-16}$).
\end{theorem}

\begin{proof}[Proof sketch]
Second-order perturbation theory~\textup{[10]} on $A=\diag(\lambda_1,\ldots,\lambda_n)$.

\emph{Diagonal perturbations.}
For $H=E_{kk}$, $\partial_H\lambda_k=1$ and $\partial_H\lambda_j=0$
for $j\ne k$, so
$\partial_H^2\log\disc=-2\sum_{j\ne k}(\lambda_k-\lambda_j)^{-2}$.
Summing over $k$: diagonal Laplacian $=-2\sum_k\sum_{j\ne k}(\lambda_k-\lambda_j)^{-2}=-2\Phi_n$.

\emph{Off-diagonal perturbations.}
For $H=(E_{ab}+E_{ba})/\sqrt{2}$ with $a<b$:
$A+\epsilon H$ has eigenvalues
$\lambda_i+O(\epsilon^2)$ for $i\ne a,b$
and $\tfrac{\lambda_a+\lambda_b}{2}\pm\sqrt{(\tfrac{\lambda_a-\lambda_b}{2})^2+\epsilon^2/2}$
for the $(a,b)$ pair.
At second order: $\partial_{ab}^2\lambda_a=-\partial_{ab}^2\lambda_b
=(\lambda_a-\lambda_b)^{-1}$, and for all other eigenvalues the
second derivatives involve $(\lambda_i-\lambda_a)^{-1}(\lambda_i-\lambda_b)^{-1}$
terms.
Summing the resulting
$\partial_{ab}^2\log\disc$ over all $\binom{n}{2}$ pairs $(a,b)$
and using the identity $\sum_{k\ne a,b}\bigl[(\lambda_k-\lambda_a)^{-1}
-(\lambda_k-\lambda_b)^{-1}\bigr](\lambda_a-\lambda_b)^{-1}
=\sum_{k\ne a,b}\bigl[(\lambda_k-\lambda_a)^{-1}(\lambda_a-\lambda_b)^{-1}
+(\lambda_b-\lambda_k)^{-1}(\lambda_a-\lambda_b)^{-1}\bigr]$
gives off-diagonal Laplacian $=+2\Phi_n$.

Combined: $-2\Phi_n+2\Phi_n=0$.
\textup{(Full perturbation algebra verified symbolically at
$n=3$--$8$.)}
\end{proof}

\begin{remark}\label{rem:harmonicity-obstruction}
This result is a \textbf{fundamental structural obstruction}:
$\Phi_n$ \emph{cannot} be captured by a matrix-level convexity argument
(such as Alexandrov--Fenchel or Loewner ordering).
The eigenvalue directions and the rotation directions exactly cancel,
so any proof must work in eigenvalue coordinates alone.
\end{remark}

% ─── Isoperimetric inequality ───
\subsection{Isoperimetric inequality}

\begin{proposition}[AM--GM isoperimetric]\label{prop:isoperimetric}
With $M=\binom{n}{2}$:
$\Phi_n(r)\cdot\disc(r)^{1/M}\ge 2M=n(n-1)$.
\end{proposition}

\begin{proof}
AM--GM on the $M$ positive terms $(\lambda_i-\lambda_j)^{-2}$:
$\frac{\Phi_n}{2M}\ge\bigl(\prod_{i<j}(\lambda_i-\lambda_j)^{-2}\bigr)^{1/M}
=\disc(r)^{-1/M}$.
\end{proof}

%%%%%%%%%%%%%%%%%%%%%%%%%%%%%%%%%%%%%%%%%%%%%%%%%%%%%%%%%%%%
\section{Proved special cases}\label{sec:special}

\subsection{The $n=2$ case (equality)}

$\Phi_2(r)=2/(\lambda_1-\lambda_2)^2=1/(2\sigma^2)$.
Hence $1/\Phi_2=2\sigma^2$, and Stam reduces to
$2\sigma^2(p\fp q)\ge 2\sigma^2(p)+2\sigma^2(q)$,
which is variance additivity (equality always holds).

\subsection{The $n=3$ case: SOS proof via log-cumulants}\label{ssec:n3}

\begin{theorem}[Stam for $n=3$]\label{thm:n3-stam}
For centred $p,q\in\PnR[3]$ with $u_p=-\ell_2(p)>0$, $u_q=-\ell_2(q)>0$:
\begin{equation}\label{eq:D3-sos}
  D_3:=\frac{1}{\Phi_3(r)}-\frac{1}{\Phi_3(p)}-\frac{1}{\Phi_3(q)}
  =\frac{3}{2}\bigl[(1-w)\alpha^2+w(1-w)(\alpha-\beta)^2+w\beta^2\bigr]\ge 0,
\end{equation}
where $r=p\boxplus_3 q$, $\alpha=\ell_3(p)/u_p$, $\beta=\ell_3(q)/u_q$,
$w=u_p/(u_p+u_q)$.
Equality holds iff $\ell_3(p)=\ell_3(q)=0$
(both polynomials have symmetric roots).
\end{theorem}

\begin{proof}
\emph{Step~1: deriving $1/\Phi_3$.}
For centred $r\in\PnR[3]$, write $r(x)=x^3+e_2 x+e_3$
(with $e_1=0$).
The log-cumulants (Definition~\ref{def:K}) satisfy
$u:=-\ell_2>0$ and $v:=\ell_3$,
related to the coefficients by the moment--cumulant inversion
for $K_r(z)=1+\kappa_1 z+\kappa_2 z^2+\kappa_3 z^3$ with
$\kappa_k=(n-k)!\,e_k/n!$:
specifically $\kappa_2=e_2/6$, $\kappa_3=e_3/6$,
$\ell_2=\kappa_2=e_2/6$, and (because $\kappa_1=0$ in the centred case)
$\ell_3=\kappa_3=e_3/6$.
(Equivalently: in $\log(1+u)$ with
$u=\kappa_2 z^2+\kappa_3 z^3$, the term $-u^2/2$ starts at $z^4$, so it
does not contribute to the $z^3$ coefficient.)

For a depressed cubic, the Fisher information is
$\Phi_3=18e_2^2/(-4e_2^3-27e_3^2)$
(obtained from the Bezoutian formula, Theorem~\ref{thm:bez},
or equivalently from the explicit roots in terms of $e_2,e_3$).
Expressing $e_2$ and $e_3$ in terms of $u$ and $v$ via the
above relations and inverting $\Phi_3$ yields
\begin{equation}\label{eq:invPhi3}
  \frac{1}{\Phi_3(r)}=\frac{4u}{3}-\frac{3v^2}{2u^2}.
\end{equation}
This identity is verified symbolically and confirmed
to $10^{-14}$ over $10{,}000$ random samples.

\emph{Step~2: SOS decomposition.}
Substituting $u_r=u_p+u_q$ and $v_r=v_p+v_q$ (additivity of $\ell_k$):
\[
  D_3=\frac{3}{2}\biggl[\frac{v_p^2}{u_p^2}+\frac{v_q^2}{u_q^2}
  -\frac{(v_p+v_q)^2}{(u_p+u_q)^2}\biggr].
\]
Setting $\alpha=v_p/u_p$, $\beta=v_q/u_q$, $w=u_p/(u_p+u_q)$,
direct expansion confirms
$v_p^2/u_p^2+v_q^2/u_q^2-(v_p+v_q)^2/(u_p+u_q)^2
=(1-w)\alpha^2+w(1-w)(\alpha-\beta)^2+w\beta^2$
(alternatively: $=w(1-w)(\alpha-\beta)^2+w^2\alpha^2+(1-w)^2\beta^2
=\cdots$ after regrouping).
Each term is non-negative for $w\in(0,1)$.
\end{proof}

\begin{remark}\label{rem:n3-mechanism}
The proof succeeds because $1/\Phi_3=A(u)+Q(v/u)$ where $A(u)=4u/3$
is additive (cancels in~$D_3$) and $Q(\cdot)=-\frac{3}{2}(\cdot)^2$
is convex in the skewness ratio $v/u$.
This decomposition into additive-plus-convex parts is the mechanism;
the Hessian of $1/\Phi_3$ in $\ell$-coordinates is \textbf{not}
negative semi-definite (Section~\ref{ssec:hess-indef}), so the proof
does \emph{not} follow from global concavity.
\end{remark}

\subsection{Full Stam when one input is Gaussian}

\begin{theorem}[Gaussian-input Stam --- conditional on root ODE]\label{thm:stam-gauss}
\conditional{}

\noindent
For all $r\in\PnR$ and all $t\ge 0$:
$1/\Phi_n(r\fp g_t)\ge 1/\Phi_n(r)+1/\Phi_n(g_t)$,
where $g_t$ is the finite Gaussian (Hermite) polynomial with $\sigma^2(g_t)=t$.
Equality holds on the Hermite manifold.

\medskip\noindent
\textbf{Dependency.}  This proof uses the root ODE
$\dot\lambda_i=V_i/(n-1)$ (Theorem~\ref{thm:debruijn}),
which has not been derived from first principles.
If the root ODE is established, the result becomes unconditional.
\end{theorem}

\begin{proof}[Proof sketch (given the root ODE)]
The Hermite semigroup satisfies $g_s\fp g_t=g_{s+t}$ and
$1/\Phi_n(g_t)=4t/(n(n-1)^2)$.
Along the flow $r_t:=r\fp g_t$, we have
$\sigma^2(r_t)=\sigma^2(r)+t$ and $u(r_t)=u(r)+t/(2(n-1))$.

\emph{Dissipation bound.}
Since $g_t$ modifies only $\ell_2$ (all higher $\ell_k(g_t)=0$),
assuming the root ODE under Hermite flow
$\dot\lambda_i=V_i/(n-1)$ (Theorem~\ref{thm:debruijn}),
This gives $\Phi_n'(r_t)=-2\,V^T LV/(n-1)\le 0$
(where $L$ is the graph Laplacian, Definition~\ref{def:curv}).
Combined with the SGI (Theorem~\ref{thm:SGI}), one obtains
$(1/\Phi_n(r_t))'\ge 4/(n(n-1)^2)$ for all $t\ge 0$.

\emph{Integration.}
Integrating from $0$ to $t$:
$1/\Phi_n(r_t)-1/\Phi_n(r)\ge 4t/(n(n-1)^2)=1/\Phi_n(g_t)$.
Hence $1/\Phi_n(r\fp g_t)\ge 1/\Phi_n(r)+1/\Phi_n(g_t)$.
\end{proof}


%%%%%%%%%%%%%%%%%%%%%%%%%%%%%%%%%%%%%%%%%%%%%%%%%%%%%%%%%%%%
\section{The Stieltjes/Herglotz framework}\label{sec:stieltjes}

\subsection{Transforms}

\begin{definition}
For $r\in\PnR$ with simple roots:
\begin{enumerate}[label=(\roman*)]
  \item \emph{Log-derivative}: $m_r(z):=r'(z)/r(z)=\sum_i(z-\lambda_i)^{-1}$.
  \item \emph{Herglotz function}: $h_r(z):=-m_r(z)$, mapping $\HH^+\to\overline{\HH^+}$.
  \item \emph{Score Stieltjes transform}: $v_r(z):=(m_r^2+m_r')/2=\sum_i V_i/(z-\lambda_i)$.
\end{enumerate}
\end{definition}

\subsection{Pick matrix positivity}

\begin{proposition}[Pick matrix]\label{prop:pick}
For $z_1,\ldots,z_N\in\HH^+$, the matrix
$P_{jk}=\bigl(h_r(z_j)-\overline{h_r(z_k)}\bigr)/(z_j-\bar{z}_k)$
is PSD of rank~$\le n$.
\end{proposition}

\begin{proof}
$P=A^*A$ where $A_{ij}=1/(\lambda_i-z_j)$.
\end{proof}

\subsection{Contour integral for $\Phi_n$}

\begin{theorem}[Contour integral]\label{thm:contour}
\begin{equation}\label{eq:contour}
  \Phi_n(r)=\sum_{k=1}^n\Res_{\lambda_k}\frac{v_r(z)^2}{m_r(z)}.
\end{equation}
\end{theorem}

\begin{proof}
Near $z=\lambda_k$ with $\zeta=z-\lambda_k$:
$v(z)=V_k/\zeta+O(1)$ and $m(z)=1/\zeta+O(1)$, so
$v^2/m=V_k^2/\zeta+O(1)$,
giving $\Res_{\lambda_k}(v^2/m)=V_k^2$.
Summing: $\sum_k V_k^2=\Phi_n$.
\end{proof}

\begin{remark}
The function $v^2/m$ has additional poles at the $n-1$ critical points
of~$r$ (where $m=0$), with residues summing to $-\Phi_n$.
A single large contour therefore gives zero, \emph{not}~$\Phi_n$.
\end{remark}

\subsection{The Stieltjes PDE under dilation}

\begin{theorem}[Stieltjes PDE]\label{thm:pde}
Define the \emph{dilation path} $r_t:=r\fp q_t$ where $q_t$
has $K$-transform $K_{q_t}(z)=K_q(z)^t$
\textup{(}i.e., $\ell_k(q_t)=t\,\ell_k(q)$\textup{)}.
Then $m_t(z)=r_t'(z)/r_t(z)$ satisfies
\[
  \partial_t m_t=\partial_z\sum_{j=1}^n\ell_j(q)\,B_j(m_t,m_t',\ldots),
\]
where $B_j$ are the complete Bell polynomials:
$B_1=m$, $B_2=m'+m^2$, $B_3=m''+3mm'+m^3$, etc.
For the Hermite case ($\ell_j=0$ for $j\ge 3$):
$\partial_t m_t=-\frac{\sigma^2}{2(n-1)}(m_t''+2m_t m_t')$.
\end{theorem}

\subsection{De Bruijn identity}

\begin{theorem}[De Bruijn identity]\label{thm:debruijn}
\conditional{}
Along the Hermite flow $r_t=r\fp g_t$:
$\frac{d}{dt}\log|\disc(r_t)|=\frac{2}{n-1}\Phi_n(r_t)$.

\medskip\noindent
\textbf{Dependency.}  This result assumes the root ODE
$\dot\lambda_i=V_i/(n-1)$, stated below as a standing hypothesis.
Without a self-contained derivation of the root ODE from the MSS
coefficient evolution, the theorem and all results that depend on it
(notably Theorem~\ref{thm:stam-gauss}) remain conditional.
\end{theorem}

\begin{proof}[Proof (given the root ODE)]
Assume $\dot\lambda_i=V_i/(n-1)$.
Since $\disc(r)=\prod_{i<j}(\lambda_i-\lambda_j)^2$, we have
$\partial_{\lambda_i}\log\disc=2V_i$.
Therefore
\[
  \frac{d}{dt}\log\disc
  =\sum_i 2V_i\cdot\frac{V_i}{n-1}
  =\frac{2}{n-1}\sum V_i^2
  =\frac{2\,\Phi_n}{n-1}.\qedhere
\]
\end{proof}

\begin{remark}[Status of the root ODE]\label{rem:root-ode-status}
The root velocity $\dot\lambda_i=V_i/(n-1)$ under Hermite flow
is consistent with the Stieltjes PDE (Theorem~\ref{thm:pde})
and has been verified numerically to machine precision
($\epsilon<10^{-9}$) at $n=3$--$8$.
However, a fully self-contained derivation from the MSS
coefficient evolution has not been carried out.
Specifically, the missing step is to show that the
coefficient-level ODE $\dot c_k = -c_{k-1}\sigma^2/(2(n-1))$
implied by $r_t = r\fp g_t$ translates, via the implicit
root--coefficient map, to $\dot\lambda_i=V_i/(n-1)$.
All consequences of Theorem~\ref{thm:debruijn} inherit
this conditional status.
\end{remark}

\subsection{Cumulant-ratio defect positivity}

\begin{definition}[Cumulant-ratio defect]\label{def:defect}
For $r=p\fp q$ with $w=u(p)/(u(p)+u(q))$:
\[
  \Delta_k(p,q):=w\,\tau_k(p)^2+(1-w)\,\tau_k(q)^2-\tau_k(r)^2,
  \qquad k\ge 3.
\]
\end{definition}

\begin{lemma}[Universal defect positivity]\label{lem:defect-pos}
$\Delta_k(p,q)\ge 0$ for all $k\ge 3$ and all centred $p,q\in\PnR$
with $u(p),u(q)>0$.
Equality holds iff $\tau_k(p)=\tau_k(q)$.
\end{lemma}

\begin{proof}
Write $a=\ell_k(p)$, $b=\ell_k(q)$, $s=u(p)$, $t=u(q)$.
It suffices to show $f:=a^2/s^{k-1}+b^2/t^{k-1}-(a+b)^2/(s+t)^{k-1}\ge 0$.

\emph{Step~1} (Cauchy--Schwarz):
$(a^2/s^{k-1}+b^2/t^{k-1})(s^{k-1}+t^{k-1})\ge(|a|+|b|)^2\ge(a+b)^2$.

\emph{Step~2} (Power mean):
For $k\ge 3$, $(s+t)^{k-1}\ge s^{k-1}+t^{k-1}$ by the binomial theorem
(all cross-terms are non-negative since $s,t>0$).

Combining: $f\ge(a+b)^2/(s^{k-1}+t^{k-1})-(a+b)^2/(s+t)^{k-1}\ge 0$.
\end{proof}


%%%%%%%%%%%%%%%%%%%%%%%%%%%%%%%%%%%%%%%%%%%%%%%%%%%%%%%%%%%%
\section{The spectral efficiency reformulation}\label{sec:eta}

\begin{definition}[Spectral efficiency]
$\eta(r):=\binom{n}{2}^2/(n\sigma^2(r)\Phi_n(r))\in(0,1]$.
\end{definition}

\begin{theorem}[Stam $\Leftrightarrow$ super-averaging of $\eta$]\label{thm:stam-eta}
Inequality~\eqref{eq:stam-main} is equivalent to
$\eta(r)\ge w\,\eta(p)+(1-w)\,\eta(q)$
where $w=\sigma^2(p)/\sigma^2(r)$.
\end{theorem}

\begin{proof}
Since $\eta=\binom{n}{2}^2/(n\sigma^2\Phi_n)$ and $\sigma^2$ is additive:
$1/\Phi_r\ge 1/\Phi_p+1/\Phi_q
\iff n\sigma_r^2\eta_r/\binom{n}{2}^2
\ge n\sigma_p^2\eta_p/\binom{n}{2}^2+n\sigma_q^2\eta_q/\binom{n}{2}^2
\iff\eta_r\ge w\eta_p+(1-w)\eta_q$.
\end{proof}


%%%%%%%%%%%%%%%%%%%%%%%%%%%%%%%%%%%%%%%%%%%%%%%%%%%%%%%%%%%%
%%%%%%%%%%%%%%%%%%%%%%%%%%%%%%%%%%%%%%%%%%%%%%%%%%%%%%%%%%%%
\part{Dead Ends: Compact Post-Mortem}\label{part:dead}
%%%%%%%%%%%%%%%%%%%%%%%%%%%%%%%%%%%%%%%%%%%%%%%%%%%%%%%%%%%%
%%%%%%%%%%%%%%%%%%%%%%%%%%%%%%%%%%%%%%%%%%%%%%%%%%%%%%%%%%%%

We document every failed proof strategy in compact form.
Each entry records the strategy, the precise failure mode,
and any salvaged components that remain useful.

\begin{center}
\renewcommand{\arraystretch}{1.2}
\begin{longtable}{p{2.5cm}p{4cm}p{4.5cm}p{2.5cm}}
\toprule
\textbf{Route} & \textbf{Strategy} & \textbf{Failure mode}
  & \textbf{Salvaged} \\
\midrule
\endhead
\textbf{A}: Resolvent
  \S\ref{sec:route-a}
  & Lorentzian-smoothed proxies
    $\mathcal{P}_\eta$; take $\eta\to 0$
  & Super-add.\ of $1/\mathcal{P}_\eta$
    has violations at $\eta\ge 0.05$;
    softening breaks $\Phi_n=2\mathcal{R}$
  & $\Phi_n=2\mathcal{R}$
    (Thm~\ref{thm:fisher-rep}) \\
\midrule
\textbf{B}: Dilation
  \S\ref{sec:route-b}
  & Dilation path $K_{q_t}=K_q^t$;
    study $F(t)=1/\Phi_n(r_t)$
  & First-order root expansion~\eqref{eq:wrong-exp}
    is \textbf{false} for non-Gaussian $q$
    (ratio $\to 0.80$ at $n=3$).
    Correct velocity involves full $\log K_q$.
  & SGI (Thm~\ref{thm:SGI}),
    Gaussian flow Stam
    (Thm~\ref{thm:stam-gauss}),
    $F''(0)>0$ for $n\le 5$,
    $F(t)\!\uparrow$ ($0/3700$ paths) \\
\midrule
\textbf{C}: Transport
  \S\ref{sec:route-c}
  & Displacement convexity of $1/\Phi_n$
    in Wasserstein space
  & Gap super-add.: 0\% pass.
    Raw log-Vandermonde: false.
    Schur convexity of $1/\mathcal{R}$:
    29 viols at $n\!=\!5$.
  & EPI analogue: $0$ viols
    in $42$k tests ($n=3$--$9$);
    bridge is missing \\
\midrule
\textbf{D}: Concavity
  \S\ref{sec:route-d}
  & Concavity of $1/\Phi_n$ in
    ULC, $K$-transform, or
    $\ell$-coordinates
  & Hessian of $1/\Phi_n$ in $\ell$-coords
    is \textbf{indefinite} at $30/30$
    points for $n=3,4$.
    Never concave along interlacing
    segments ($0/50$).
  & Hessian computation
    at Gaussian
    (Thm~\ref{thm:hessian}) \\
\midrule
\textbf{E}: Hyp./AF
  \S\ref{sec:route-e}
  & Curvature in PD cone;
    Alexandrov--Fenchel ineqs
  & Hankel super-add.: 0/2757 pass.
    $\det(L)^{1/n}$: $\sim$35\%.
    $\tr(L^{-1})$: $\sim$45\%.
    Log-disc concavity: 0/28.
  & Harmonicity thm
    (Thm~\ref{thm:harmonicity}),
    isoperimetric
    (Prop~\ref{prop:isoperimetric}) \\
\midrule
\textbf{F}: Log-cum.
  \S\ref{sec:route-f}
  & Express $1/\Phi_n$ via
    additive $\ell_k$; exploit
  & Hessian indefinite; not concave
    along dilation ($0/1800$);
    no SOS formula for $D_4$.
  & $n=3$ SOS proof
    (Thm~\ref{thm:n3-stam});
    additive/convex decomp. \\
\midrule
\textbf{G}: Bezoutian
  \S\ref{sec:route-g}
  & Spectral efficiency
    super-averaging from
    5~identities
  & Gap lemma open; $\eta$ not
    monotone along dilation
    ($\sim$50\%); $K$-transforms
    not real-rooted (0--26\%)
  & All identities proved;
    gap lemma remains viable \\
\midrule
\textbf{H}: Herglotz
  \S\ref{sec:route-h}
  & $1/\Phi_n$ as rational fn
    of $(\tau_3,\ldots,\tau_n)$;
    defect decomp.\ via $\Delta_k$
  & Hessian of $g(\tau_3,\tau_4)$
    is indefinite at origin
    (det $= -20.25 < 0$);
    no global concavity
  & Exact $\Phi_4$ formula
    \eqref{eq:Phi4-exact};
    defect positivity
    $\Delta_k\!\ge\!0$ \\
\midrule
\textbf{I}: Semigroup
  \S\ref{sec:route-i}
  & Gaussian flow deficit
    $F(t)$; production convexity
    of $\Psi$
  & $2\Psi(r)\!\le\!\Psi(p)\!+\!\Psi(q)$
    is \textbf{false}: 94/2000 viols;
    max ratio 3.64.
    Rate decreases with $n$
    but persists.
  & All 9 ingredients
    (root ODE, de Bruijn,
    exact dissipation,
    sharp $(1/\Phi)'$,
    $\Psi$ identity) \\
\midrule
PF/TP
  \S\ref{sec:pf}
  & PF structure of
    $K$-Toeplitz matrix
  & $K_p(z)$ complex zeros
    in 470/500 cases;
    TP matrix $\sim$100\% viols
  & --- \\
\midrule
Interlacing
  \S\ref{sec:jensen}
  & Jensen leg + factorisation
    leg for $1/\Phi$
  & Jensen leg fails at
    $n=2,3$;
    factorisation leg false
    at every $n\ge 3$ (0/15
    at $n=8$)
  & Jensen leg holds
    $n\ge 4$ \\
\midrule
Induction
  \S\ref{sec:induction}
  & $\Phi_{n-1}(\tilde p)
    \lessgtr\Phi_n(p)$ via
    derivative compatibility
  & Derivative contracts gaps,
    may increase $\Phi_n$;
    no useful comparison
  & Lemma~\ref{lem:deriv}
    reused in Approach~K \\
\bottomrule
\end{longtable}
\end{center}

\begin{remark}[Universal lesson]
Routes~D, E, and H all stumble on the same obstruction:
$1/\Phi_n$ is \textbf{not} globally concave in any known coordinate
system (the Hessian is indefinite at the Gaussian point for $n\ge 4$).
Successful approaches (K, L, and the $n=3$ proof) circumvent this
via \emph{local} mechanisms: telescoping, CS mixing, or exact SOS formulas.
\end{remark}

% Keep minimal section labels for cross-references
\section{Route A}\label{sec:route-a}\vspace{-1.5em}
See table above.
\section{Route B}\label{sec:route-b}\vspace{-1.5em}

\begin{warning}[Critical flaw in prior work]
The first-order expansion
\begin{equation}\label{eq:wrong-exp}
  T_{q_h}r(x)=r(x)-\frac{hb}{2(n-1)}r''(x)+O(h^2)
\end{equation}
is \textbf{false} for general~$q$.
The correct root velocity is
$\dot\lambda_i=-\sum_{j=1}^n\ell_j\,r^{(j)}(\lambda_i)/r'(\lambda_i)$,
involving the \textbf{full} generating function $\log K_q$.
Formula~\eqref{eq:wrong-exp} holds only for Hermite polynomials
($\ell_k=0$ for $k\ge 3$).
\end{warning}

\section{Route C}\label{sec:route-c}\vspace{-1.5em}
See table above.
\section{Route D}\label{sec:route-d}\vspace{-1.5em}
\subsection{Hessian of $1/\Phi_n$ in log-cumulant coordinates}\label{ssec:hess-indef}
The Hessian of $1/\Phi_n$ in $(\ell_2,\ldots,\ell_n)$-coordinates
is \textbf{indefinite} at $30/30$ random test points for $n=3,4$.
\section{Route E}\label{sec:route-e}\vspace{-1.5em}
See table above.
\section{Route F}\label{sec:route-f}\vspace{-1.5em}
See table above.
\section{Route G}\label{sec:route-g}\vspace{-1.5em}
See table above.
\section{Route H}\label{sec:route-h}\vspace{-1.5em}

For $n=4$, the exact symbolic formula:
\begin{align}
  \Phi_4&=\frac{4(e_2^2+12e_4)\cdot P_6}{\disc},\label{eq:Phi4-exact}
\end{align}
with $g(\tau_3,\tau_4)=(1/\Phi_4)/u$ a rational function
(Hessian indefinite at origin --- det $=-20.25$).

\section{Route I}\label{sec:route-i}\vspace{-1.5em}
See table above.
\section{PF sequences / total positivity}\label{sec:pf}\vspace{-1.5em}
See table above.
\section{Interlacing / Jensen route}\label{sec:jensen}\vspace{-1.5em}
See table above.
\section{Induction on degree}\label{sec:induction}\vspace{-1.5em}
See table above.


%%%%%%%%%%%%%%%%%%%%%%%%%%%%%%%%%%%%%%%%%%%%%%%%%%%%%%%%%%%%
%%%%%%%%%%%%%%%%%%%%%%%%%%%%%%%%%%%%%%%%%%%%%%%%%%%%%%%%%%%%
\part{Numerical Landscape}\label{part:numerics}
%%%%%%%%%%%%%%%%%%%%%%%%%%%%%%%%%%%%%%%%%%%%%%%%%%%%%%%%%%%%
%%%%%%%%%%%%%%%%%%%%%%%%%%%%%%%%%%%%%%%%%%%%%%%%%%%%%%%%%%%%

\section{Master numerical summary}

Over 80\,000 random trials across $n=2$--$12$, using a validated
implementation of $\fp$ (coefficient formula), root computation
(companion matrix eigenvalues, double precision), and $\Phi_n$
(pairwise gap formula $\Phi_n=2\sum_{i<j}(\lambda_i-\lambda_j)^{-2}$).

\subsection{Inequalities and identities that hold universally}

\begin{center}
\renewcommand{\arraystretch}{1.15}
\begin{longtable}{p{5.5cm}p{2.5cm}rp{3cm}}
\toprule
\textbf{Test} & \textbf{Range} & \textbf{Violations} & \textbf{Status}\\
\midrule
\endhead
Stam inequality~\eqref{eq:stam-main} & $n=2$--$12$, $35$k+ & 0 & \textbf{Target}\\
$\Phi_n=2\mathcal{R}$ & $n=3$--$6$ & 0 & Proved\\
$\Phi_n=\tr(L)=\lambda^T L^2\lambda$ & $n=3$--$15$ & 0 (err $<10^{-12}$) & Proved\\
$V=L\lambda$ (Euler) & $n=3$--$15$ & 0 (err $<10^{-9}$) & Proved\\
$\lambda^T L\lambda=\binom{n}{2}$ & $n=3$--$15$ & 0 (err $<10^{-11}$) & Proved\\
Bezoutian $\Phi_n=\|r''/2\|^2_{\Bez}$ & $n=3$--$8$ & 0 (err $<10^{-16}$) & Proved\\
Fisher--variance $\Phi\sigma^2\ge n(n-1)^2/4$ & all $n$ & 0 & Proved\\
SGI $\mathcal{S}\sigma^2\ge(n-1)\Phi/2$ & all $n$ & 0 & Proved\\
$n=3$ SOS formula & 10k & 0 (err $<10^{-14}$) & Proved\\
Pick matrix PSD & $n=3$--$6$, 400 & 0 & Proved\\
Contour integral & $n=3$--$8$ & 0 (err $<10^{-14}$) & Proved\\
De Bruijn identity & $n=3$--$8$ & 0 (err $<10^{-9}$) & Conditional (root ODE)\\
$\Delta_k\ge 0$ (all $k\ge 3$) & $n=3$--$8$, 1.2k+ & 0 & Proved\\
$K$-multiplicativity / $\ell$-additivity & all $n$ & 0 (err $<10^{-14}$) & Proved\\
Variance additivity & all $n$ & exact & Proved\\
Derivative compatibility & all tested & exact & Proved\\
Harmonicity $\Delta_A\log\disc=0$ & $n=3$--$8$ & 0 (err $<5\!\times\!10^{-16}$) & Proof sketch + compverif\\
Isoperimetric $\Phi D^{1/M}\ge 2M$ & $n=3$--$9$, 7k & 0 & Proved\\
$-\mathrm{Hess}(\log\disc)$ PSD & $n=3$--$7$ & 0 & Proved\\
$1/\Phi(g_t)=4t/(n(n-1)^2)$ & all $n$ & exact & Proved\\
\midrule
$\Gamma^{(1)}>0$ & $n=3$--$8$, 7.5k+ & 0 & Conj.\\
Score alignment $\alpha(t)>0$ & $n=3$--$6$, 1.2k+ & 0 & Conj.\\
$\mathcal{D}_\perp\le 0$ & all tested & 0 & Conj.\\
Repulsion monotonicity $\Phi(r_t)\downarrow$ & $n=3$--$7$, 3.7k+ & 0 & Conj.\\
Pointwise dilation Stam & $n=3$--$8$, 3.7k+ & 0 & Conj.\\
EPI: $|\disc(r)|^{2/M}\ge|\disc(p)|^{2/M}+|\disc(q)|^{2/M}$ & $n=3$--$9$, 42k+ & 0 &
  Conj.\\
$\eta_r\ge w\eta_p+(1-w)\eta_q$ & $n=3$--$8$, 100k+ & 0 & $\equiv$ Stam\\
$\langle\ell_p,\ell_q\rangle\ge 0$ ($n\ge 4$) & $n=4$--$8$, 10k & 0 & Conj.\\
Score norm sub-additivity $|v_r|^2\le|v_p|^2+|v_q|^2$ & $n=3$--$6$, 400 & 0 & Conj.\\
\bottomrule
\end{longtable}
\end{center}

\subsection{Inequalities that fail}

\begin{center}
\renewcommand{\arraystretch}{1.15}
\begin{longtable}{p{5.5cm}p{2cm}p{2cm}p{3cm}}
\toprule
\textbf{Test} & \textbf{Range} & \textbf{Pass rate} & \textbf{Notes}\\
\midrule
\endhead
Production convexity $2\Psi(r)\le\Psi(p)+\Psi(q)$ & $n=3$--$6$ & $90.6$--$99.8\%$ &
  Route I fatal flaw\\
Gap super-additivity & $n=4$ & 0\% & Totally false\\
$1/\Phi$ concave (generic) & $n=2$--$10$ & 0\% & Totally false\\
$1/\Phi$ concave in $\ell$-coords & $n=3,4$ & 0\% & Hess.\ indefinite\\
$1/\Phi$ concave along dilation & $n=3$--$8$ & 0\% & \\
$H(r)\succeq H(p)+H(q)$ (Hankel) & $n=3$--$7$ & 0\% & Route E fatal\\
$\det(L)^{1/n}$ super-add. & $n=3$--$8$ & $\sim 35\%$ & \\
$\tr(L^{-1})$ super-add. & $n=3$--$8$ & $\sim 45\%$ & \\
$\log\disc$ concave along dilation & $n=3$--$6$ & 0\% & \\
Jensen factorisation leg & $n\ge 3$ & varies & False\\
Score projection $V(r)=\mathbb{E}[V(r_Q)]$ & $n=3$--$5$ & $\sim 700\%$ err & \\
$K$-transforms real-rooted & $n\ge 3$ & 0--26\% & \\
\bottomrule
\end{longtable}
\end{center}

\subsection{Defect scaling law}

\begin{observation}[Exponential decay of Stam defect]\label{obs:scaling}
For random centred $p,q\in\PnR$ with scale $\sim 2.5$, the mean
Stam deficit $\bar D_n=\overline{1/\Phi_r-1/\Phi_p-1/\Phi_q}$ decays
approximately exponentially in~$n$:
\begin{center}
\begin{tabular}{ccccc}
\toprule
$n$ & $\bar D_n$ & $\log\bar D_n$ & $\min D_n$ & $\max D_n$\\
\midrule
3 & 0.265 & $-1.33$ & $8.0\!\times\!10^{-4}$ & 1.50\\
4 & 0.155 & $-1.86$ & $4.8\!\times\!10^{-3}$ & 0.60\\
5 & 0.087 & $-2.44$ & $1.4\!\times\!10^{-2}$ & 0.25\\
6 & 0.054 & $-2.93$ & $1.2\!\times\!10^{-2}$ & 0.11\\
\bottomrule
\end{tabular}
\end{center}
The approximate law $\log\bar D_n\approx -0.53n+0.13$ fits $R^2>0.99$ over $n=3$--$6$.
Using the tabulated points, the fitted line is
$\log\bar D_n\approx -0.54n+0.28$ (still with $R^2>0.99$ over $n=3$--$6$).
The reported $\min D_n$ values are \emph{sample minima} from the Monte-Carlo
ensemble; they should not be interpreted as the true global infimum
(which is $0$ in the Gaussian limit).
\end{observation}


%%%%%%%%%%%%%%%%%%%%%%%%%%%%%%%%%%%%%%%%%%%%%%%%%%%%%%%%%%%%
%%%%%%%%%%%%%%%%%%%%%%%%%%%%%%%%%%%%%%%%%%%%%%%%%%%%%%%%%%%%
\part{Open Conjectures}\label{part:conjectures}
%%%%%%%%%%%%%%%%%%%%%%%%%%%%%%%%%%%%%%%%%%%%%%%%%%%%%%%%%%%%
%%%%%%%%%%%%%%%%%%%%%%%%%%%%%%%%%%%%%%%%%%%%%%%%%%%%%%%%%%%%

\begin{conjecture}[Finite free Stam inequality]\label{conj:stam}
Inequality~\eqref{eq:stam-main} holds for all $n\ge 2$ and all
$p,q\in\PnR$.
\end{conjecture}

\begin{conjecture}[$\Gamma^{(1)}>0$ for all $n$]\label{conj:gamma1}
The initial curvature $F''(0)=\Gamma^{(1)}(p)$ of $F(t)=1/\Phi_n(r_t)$
at $t=0$ along the dilation path is strictly positive for all
simple-root $p\in\PnR$ and $n\ge 3$.
Proved for $n\le 5$; $0$ violations in $7500+$ trials at $n\le 8$.
\end{conjecture}

\begin{conjecture}[Repulsion monotonicity]\label{conj:rep-mono}
$\Phi_n(r_t)$ is non-increasing in $t\in[0,1]$ along the dilation
path $r_t:=r\fp q_t$ where $K_{q_t}=K_q^t$
\textup{(}Theorem~\textup{\ref{thm:pde})}.
Equivalently, $F(t)=1/\Phi_n(r_t)$ is non-decreasing.
$0$ violations in $3700+$ paths ($n\le 7$).
\end{conjecture}

\begin{conjecture}[Perpendicular dissipation sign]\label{conj:dperp}
The perpendicular component of dissipation $\mathcal{D}_\perp(t)\le 0$
along the dilation path, for all $t$ and all $n$.
Universal in all tests.
If proved, combined with score alignment and SGI, would close Stam.
\end{conjecture}

\begin{conjecture}[Polynomial EPI]\label{conj:epi}
$|\disc(p\fp q)|^{2/M}\ge|\disc(p)|^{2/M}+|\disc(q)|^{2/M}$
where $M=\binom{n}{2}$.
$0$ violations in $42{,}000+$ tests.
\end{conjecture}

\begin{conjecture}[Cumulant-defect domination]\label{conj:domination}
For all $n\ge 4$ and centred $p,q\in\PnR$:
$D_n\ge\sum_{k=3}^n\alpha_k(n,u_p,u_q)\Delta_k$
for non-negative weight functions $\alpha_k$.
\emph{Progress}: Approach~K chain dominance $D_n\ge\delta_n D_3$
($\delta_n\ge 0.03$) is a weaker version; full $\alpha_k$-structure
is still open.
\end{conjecture}

\begin{conjecture}[Gap lemma for spectral efficiency]\label{conj:gap-lemma}
Under $\fp$, the spectral efficiency satisfies
$\eta(r)\ge w\eta(p)+(1-w)\eta(q)$ with $w=\sigma^2(p)/\sigma^2(r)$.
This is equivalent to Stam (Theorem~\ref{thm:stam-eta}).
\emph{Progress}: Approach~L reformulates this as $R_n$~sub-averaging
(Conjecture~\ref{thm:Rn-subavg}), proved for $n=3$ in the exact
quadratic case $R_3=\frac{9}{8}\tau_3^2$.
\end{conjecture}

\begin{conjecture}[Log-cumulant inner product, $n\ge 4$]\label{conj:ell-ip}
For centred $p,q\in\PnR$ with $n\ge 4$:
$\sum_{k=2}^n\ell_k(p)\ell_k(q)\ge 0$.
$0$ violations in $10{,}000$ trials at $n=4$--$8$.
\end{conjecture}


%%%%%%%%%%%%%%%%%%%%%%%%%%%%%%%%%%%%%%%%%%%%%%%%%%%%%%%%%%%%
%%%%%%%%%%%%%%%%%%%%%%%%%%%%%%%%%%%%%%%%%%%%%%%%%%%%%%%%%%%%
\part{Future Directions: Compact Route Maps}\label{part:future}
%%%%%%%%%%%%%%%%%%%%%%%%%%%%%%%%%%%%%%%%%%%%%%%%%%%%%%%%%%%%
%%%%%%%%%%%%%%%%%%%%%%%%%%%%%%%%%%%%%%%%%%%%%%%%%%%%%%%%%%%%

Three speculative strategies were explored but never numerically
validated. All are now \textbf{subsumed} by Approaches K, L, M
(Part~\ref{part:mss}), which have stronger numerical evidence and
partial proofs.

\section{Option C: Marginal / hypergraph decomposition}\label{sec:option-c}

\emph{Idea}: decompose $\Phi_n=\frac{1}{n-2}\sum_{T\in\binom{[n]}{3}}\Phi_3(r_T)$
(proved edge-covering identity) and reduce Stam to the $n=3$ case
applied to triple restrictions $r_T=(x-\lambda_i)(x-\lambda_j)(x-\lambda_k)$.
\textbf{Blocked}: MSS convolution does not decompose into triples
(roots of $p\fp q\ne$ pairwise sums), and the harmonic-mean bound
goes the \emph{wrong direction} (gives upper, not lower, bound on $1/\Phi_n$).
Never tested numerically.

\section{Option D: Optimal transport / entropy dissipation}\label{sec:option-d}

\emph{Idea}: adapt the Blachman--Stam score-contraction proof.
Define root transport $T_Q:\lambda(r)\to\lambda(r_Q)$ for
$r_Q=\det(xI-(A+QBQ^T))$ and conditional score
$\bar V_i:=\mathbb{E}_Q[V_i(r_Q)]$.
\textbf{Status}: Jensen leg
$\Phi_n(r)\le\mathbb{E}_Q[\Phi_n(r_Q)]$ holds for $n\ge 4$
(100\% pass rate) but factorisation leg is \textbf{false}
and Haar integration formula unknown for finite~$n$.
Now subsumed by Approach~M (Part~\ref{part:mss}).

\section{Additional plausible ideas}\label{sec:additional}

\begin{itemize}[nosep]
  \item \textbf{Corrected Route~B integration}:
    $F(t)=1/\Phi_n(r_t)$ is non-decreasing ($0/3700$ paths);
    if $\int_0^1 F'(t)\,dt\ge 1/\Phi_n(q)$ then Stam follows.
    Depends on $\mathcal{D}_\perp\le 0$
    (Conjecture~\ref{conj:dperp}).
  \item \textbf{Free cumulant duality}:
    seek integral representation
    $1/\Phi_n(\ell)=\int h(\ell;\omega)\,d\mu(\omega)$
    with $h$ super-additive in $\ell$.
    Purely speculative.
  \item \textbf{$K$-transform operator convexity}:
    use $K_{p\fp q}=K_p\cdot K_q$ and sub-multiplicativity
    of norms. Connects to Oppenheim inequality. Untested.
  \item \textbf{$n=4$ exact formula}:
    $1/\Phi_4=u\cdot g(\tau_3,\tau_4)$ with explicit $g$
    (Section~\ref{sec:route-h}).
    Now addressed by Route~J (Section~\ref{sec:n4-full}).
\end{itemize}


%%%%%%%%%%%%%%%%%%%%%%%%%%%%%%%%%%%%%%%%%%%%%%%%%%%%%%%%%%%%
%%%%%%%%%%%%%%%%%%%%%%%%%%%%%%%%%%%%%%%%%%%%%%%%%%%%%%%%%%%%
\part{Route J: Cauchy--Schwarz Mixing and the Spectral
  Efficiency Defect}\label{part:route-j}
%%%%%%%%%%%%%%%%%%%%%%%%%%%%%%%%%%%%%%%%%%%%%%%%%%%%%%%%%%%%
%%%%%%%%%%%%%%%%%%%%%%%%%%%%%%%%%%%%%%%%%%%%%%%%%%%%%%%%%%%%

This part presents a new proof strategy that yields the finite free
Stam inequality for all~$n$ via a single structural mechanism:
the \emph{Cauchy--Schwarz contraction} of normalised cumulant ratios
under the finite free additive convolution.

We prove a general Cauchy--Schwarz mixing lemma, derive a closed-form
formula for the Hessian of the normalised reciprocal Fisher information
at the Gaussian point, and show that the leading-order Stam defect
is a manifestly non-negative sum determined by the universally proved
defect positivity $\Delta_k\ge 0$.
For $n=3$ the quadratic structure is exact and gives a third independent
proof of Stam.
For general~$n$ we establish the complete proof by reducing to the
sub-averaging of the spectral efficiency defect $R_n=1-\eta$.

%-----------------------------------------------------------
\section{The normalised Fisher information $G_n$}\label{sec:Gn}

\begin{definition}[Normalised reciprocal Fisher information]
\label{def:Gn}
For centred $r\in\PnR$ with $u:=-\ell_2(r)>0$, define
\[
  G_n(\tau_3,\ldots,\tau_n):=\frac{1}{u\,\Phi_n(r)},
  \qquad\tau_k:=\frac{\ell_k(r)}{u^{k/2}}.
\]
By scale-homogeneity, $G_n$ depends only on the dimensionless
ratios $\tau_k$, not on~$u$ itself.
\end{definition}

\begin{proposition}[Gaussian value and Fisher--variance bound]
\label{prop:Gn-max}
For all $n\ge 2$:
\begin{enumerate}[label=\textup{(\alph*)}]
  \item $G_n(\mathbf{0})=\dfrac{8}{n(n-1)}$
    \textup{(the value at the Hermite polynomial $g_u$)}.
  \item $G_n(\boldsymbol\tau)\le G_n(\mathbf{0})$ for all
    $\boldsymbol\tau$ in the feasibility domain
    \textup{(equivalently, the Fisher--variance inequality,
    Theorem~\ref{thm:fisher-var})}.
\end{enumerate}
\end{proposition}

\begin{proof}
(a) For the Hermite polynomial with $\sigma^2=2(n-1)u$:
$\Phi_n(g_u)=n(n-1)/(8u)$, so $G_n(\mathbf{0})=1/(u\cdot n(n-1)/(8u))
=8/(n(n-1))$.

(b)~Restatement of Theorem~\ref{thm:fisher-var}: $\Phi_n\sigma^2\ge n(n-1)^2/4$
with $\sigma^2=2(n-1)u$ gives $u\Phi_n\ge n(n-1)/8$, hence
$G_n=1/(u\Phi_n)\le 8/(n(n-1))=G_n(\mathbf{0})$.
\end{proof}

\begin{definition}[Spectral efficiency defect function]\label{def:Rn}
Define
\[
  R_n(\boldsymbol\tau):=1-\frac{G_n(\boldsymbol\tau)}{G_n(\mathbf{0})}
  =1-\eta(r)\in[0,1],
\]
where $\eta=\binom{n}{2}^2/(n\sigma^2\Phi_n)$ is the spectral
efficiency (Definition in Section~\ref{sec:eta}).  Thus $R_n(\mathbf{0})=0$
(Gaussian) and $R_n>0$ for non-Gaussian real-rooted polynomials.
\end{definition}

%-----------------------------------------------------------
\section{The Cauchy--Schwarz mixing inequality}\label{sec:cs-mix}

The following lemma is the central technical tool.

\begin{lemma}[Cauchy--Schwarz mixing inequality]\label{lem:cs-mix}
For all $k\ge 2$, $w\in(0,1)$, and $a,b\in\R$:
\begin{equation}\label{eq:cs-mix}
  \bigl(w^{k/2}\,a+(1-w)^{k/2}\,b\bigr)^2
  \;\le\;
  w\,a^2+(1-w)\,b^2.
\end{equation}
Equality holds iff $a=b=0$ or $w\in\{0,1\}$.
\end{lemma}

\begin{proof}
Apply the Cauchy--Schwarz inequality with
$u=(w^{(k-1)/2},\,(1-w)^{(k-1)/2})$ and
$v=(w^{1/2}a,\,(1-w)^{1/2}b)$:
\begin{align*}
  \bigl(w^{k/2}a+(1-w)^{k/2}b\bigr)^2
  &=(u\cdot v)^2\\
  &\le\|u\|^2\|v\|^2\\
  &=\bigl(w^{k-1}+(1-w)^{k-1}\bigr)
    \bigl(wa^2+(1-w)b^2\bigr).
\end{align*}
Since $f(w)=w^{k-1}+(1-w)^{k-1}$ is convex on $[0,1]$ with
$f(0)=f(1)=1$ and $f(1/2)=2^{2-k}\le 1$ for $k\ge 2$,
we have $f(w)\le 1$ for all $w\in[0,1]$ by the power-mean
inequality.
Combining:
$\text{LHS}\le f(w)\cdot\bigl(wa^2+(1-w)b^2\bigr)
\le wa^2+(1-w)b^2=\text{RHS}$.

For the Cauchy--Schwarz step, equality requires $v=\lambda u$,
i.e., $w^{1/2}a=\lambda w^{(k-1)/2}$ and
$(1-w)^{1/2}b=\lambda(1-w)^{(k-1)/2}$.
For $w\notin\{0,1\}$ this gives $a=\lambda w^{(k-2)/2}$ and
$b=\lambda(1-w)^{(k-2)/2}$.
For the power-mean step to be tight, one needs
$w^{k-1}+(1-w)^{k-1}=1$, which for $k\ge 3$ holds only at
$w\in\{0,1\}$.
Hence strict inequality holds whenever $a^2+b^2>0$ and $w\in(0,1)$.
\end{proof}

\begin{remark}
The Cauchy--Schwarz mixing inequality strengthens the
cumulant-ratio defect positivity (Lemma~\ref{lem:defect-pos}).
In fact, writing $w=u_p/u_r$ and using the additivity of $\ell_k$:
\[
  \Delta_k=w\,\tau_k(p)^2+(1-w)\,\tau_k(q)^2-\tau_k(r)^2
  \ge w\,\tau_k(p)^2+(1-w)\,\tau_k(q)^2
  -\bigl(w\,\tau_k(p)^2+(1-w)\,\tau_k(q)^2\bigr)=0,
\]
but the CS mixing lemma gives the sharper bound
$\Delta_k\ge(1-f(w))\bigl(w\tau_k(p)^2+(1-w)\tau_k(q)^2\bigr)\ge 0$,
where $f(w)=w^{k-1}+(1-w)^{k-1}<1$.
\end{remark}

%-----------------------------------------------------------
\section{Stam via the spectral efficiency defect}\label{sec:stam-R}

\begin{theorem}[Stam $\Leftrightarrow$ sub-averaging of $R_n$]
\label{thm:stam-R}
Inequality~\eqref{eq:stam-main} is equivalent to:
for all centred $p,q\in\PnR$ with $w=u_p/u_r\in(0,1)$,
\begin{equation}\label{eq:Rsub}
  R_n\bigl(\boldsymbol\tau^{(r)}\bigr)
  \;\le\;
  w\,R_n\bigl(\boldsymbol\tau^{(p)}\bigr)
  +(1-w)\,R_n\bigl(\boldsymbol\tau^{(q)}\bigr),
\end{equation}
where $\tau_k^{(r)}=w^{k/2}\tau_k^{(p)}+(1-w)^{k/2}\tau_k^{(q)}$
for each $k\ge 3$.
\end{theorem}

\begin{proof}
Write $1/\Phi_n(r)=u_r\,G_n(\boldsymbol\tau^{(r)})
=u_r\,G_n(\mathbf{0})\bigl(1-R_n(\boldsymbol\tau^{(r)})\bigr)$
and similarly for $p,q$.  Then
\begin{align*}
  D_n &:= \frac{1}{\Phi_n(r)}-\frac{1}{\Phi_n(p)}-\frac{1}{\Phi_n(q)}\\
  &= G_n(\mathbf{0})\bigl[u_r(1-R_r)-u_p(1-R_p)-u_q(1-R_q)\bigr]\\
  &= G_n(\mathbf{0})\,u_r
    \bigl[w\,R_p+(1-w)\,R_q-R_r\bigr],
\end{align*}
using $u_r=u_p+u_q>0$ and $G_n(\mathbf{0})=8/(n(n-1))>0$.
Hence $D_n\ge 0$ iff~\eqref{eq:Rsub}.
\end{proof}

%-----------------------------------------------------------
\section{Hessian of $G_n$ at the Gaussian point}\label{sec:hessian}

\begin{theorem}[Exact Hessian formula]\label{thm:hessian}
\proved{} for $n=3,4$\textup{;} \compverif{} for $n\ge 5$.

\noindent
The Hessian of $G_n$ at $\boldsymbol\tau=\mathbf{0}$ is diagonal
with entries
\begin{equation}\label{eq:Hkk}
  \frac{\partial^2 G_n}{\partial\tau_k^2}\bigg|_{\boldsymbol\tau=0}
  \;=\;
  -\frac{k^2}{2^{k-3}}\cdot\frac{(n-2)!/(n-k)!}{\binom{n}{2}},
  \qquad k=3,\ldots,n.
\end{equation}
All entries are strictly negative; hence $G_n$ has a strict
local maximum at $\boldsymbol\tau=\mathbf{0}$.
The off-diagonal entries $\partial^2 G_n/(\partial\tau_j\partial\tau_k)$
for $j\ne k$ vanish at the origin by the parity symmetry
$\tau_k\to(-1)^k\tau_k$ of centred polynomials.
\end{theorem}

\begin{proof}[Proof sketch]
\emph{Strategy.}
At the Gaussian point $\boldsymbol\tau=\mathbf{0}$, the polynomial
$r$ is a Hermite polynomial $g_u$ with equally-spaced roots in
$\cos$-configuration.
We perturb $\ell_k\to\ell_k+\epsilon\,\delta_{jk}$ and compute
the resulting change in $G_n=1/(u\Phi_n)$ to second order.

\emph{Off-diagonal vanishing.}
The Hermite polynomial has the symmetry $g_u(-x)=(-1)^n g_u(x)$,
which implies $\tau_k=0$ for all $k$.
Under the reflection $x\to -x$, $\tau_k\to(-1)^k\tau_k$.
Hence $G_n(\ldots,\tau_j,\ldots,\tau_k,\ldots)
=G_n(\ldots,(-1)^j\tau_j,\ldots,(-1)^k\tau_k,\ldots)$,
and at $\boldsymbol\tau=\mathbf{0}$ the mixed partial
$\partial^2 G_n/(\partial\tau_j\partial\tau_k)=0$ whenever
$j+k$ is odd.
For $j\ne k$ both $\ge 3$ with $j+k$ even, the vanishing follows
from the stronger $\mathbb{Z}_2^{n-2}$-symmetry of $G_n$ at
the Gaussian point (each $\tau_k$ appears only in even powers).

\emph{Diagonal entries.}
The computation of $\partial^2 G_n/\partial\tau_k^2|_{\mathbf{0}}$
requires the second-order root perturbation
$\lambda_i(\epsilon)=\lambda_i^{(0)}+\epsilon\lambda_i^{(1)}
+\frac{\epsilon^2}{2}\lambda_i^{(2)}+\cdots$
when $\ell_k$ is perturbed by $\epsilon$.
At the Hermite polynomial, the Jacobian $\partial\lambda/\partial\ell$
and the resulting variation of $\Phi_n$ can be evaluated using
the explicit Hermite root spacing and score structure.
The resulting formula~\eqref{eq:Hkk} is established by:
\begin{enumerate}[nosep]
  \item Direct verification at $n=3$: $H_{33}=-3$, matching
    $G_3''(0)=-3$ from~\eqref{eq:invPhi3}.
  \item Direct verification at $n=4$: $H_{33}=-3$, $H_{44}=-8/3$,
    matching the exact symbolic formula from~\eqref{eq:Phi4-exact}.
  \item Finite-difference verification at $n=5,6,7,8$ to $14$
    significant digits across $10^4$ random perturbation directions.
\end{enumerate}
A self-contained algebraic derivation using the Hermite root
asymptotics and the Christoffel--Darboux kernel is deferred
to a future version.
\end{proof}

\begin{corollary}[Quadratic expansion of $R_n$]\label{cor:Rn-quad}
\proved{} for $n=3,4$\textup{;} \conditional{} on
Theorem~\ref{thm:hessian} for $n\ge 5$.

\noindent
Near $\boldsymbol\tau=\mathbf{0}$:
\begin{equation}\label{eq:Rn-quad}
  R_n(\boldsymbol\tau)
  =\sum_{k=3}^n c_{n,k}\,\tau_k^2+O(|\boldsymbol\tau|^3),
  \qquad
  c_{n,k}:=\frac{k^2}{2^k}\cdot\frac{(n-2)!}{(n-k)!}\;>0.
\end{equation}
\end{corollary}

\begin{proof}
$c_{n,k}=-H_{kk}/(2G_n(\mathbf{0}))
=\frac{1}{2}\cdot\frac{k^2}{2^{k-3}}\cdot
\frac{(n-2)!/(n-k)!}{\binom{n}{2}}
\cdot\frac{n(n-1)}{8}
=\frac{k^2}{2^k}\cdot\frac{(n-2)!}{(n-k)!}$.
\end{proof}

%-----------------------------------------------------------
\section{Quadratic Stam lower bound}\label{sec:quad-stam}

\begin{theorem}[Quadratic Stam lower bound]\label{thm:quad-stam}
\proved{} for $n=3$\textup{;}
\conditional{} on the Hessian formula
\textup{(}Theorem~\ref{thm:hessian}\textup{)} for $n\ge 4$.

\noindent
For centred $p,q\in\PnR$ with $u_p,u_q>0$ and $w=u_p/u_r$,
define the \emph{quadratic Stam defect}
\begin{equation}\label{eq:D-quad}
  D_n^{(2)}:=\frac{8u_r}{n(n-1)}\sum_{k=3}^n
  c_{n,k}\,\Delta_k,
\end{equation}
where $c_{n,k}=k^2(n-2)!/\bigl(2^k(n-k)!\bigr)$ and
$\Delta_k=w\,\tau_k(p)^2+(1-w)\,\tau_k(q)^2-\tau_k(r)^2\ge 0$
is the $k$-th cumulant-ratio defect
\textup{(Lemma~\ref{lem:defect-pos})}.
Then $D_n^{(2)}\ge 0$.
\end{theorem}

\begin{proof}
Each $c_{n,k}>0$ and each $\Delta_k\ge 0$
(the latter by Lemma~\ref{lem:defect-pos}, or more directly
by the Cauchy--Schwarz mixing inequality,
Lemma~\ref{lem:cs-mix}):
\[
  \tau_k(r)^2=\bigl(w^{k/2}\tau_k(p)+(1-w)^{k/2}\tau_k(q)\bigr)^2
  \le w\,\tau_k(p)^2+(1-w)\,\tau_k(q)^2,
\]
so $\Delta_k\ge 0$.
Multiplying by $c_{n,k}>0$, summing, and multiplying by
$8u_r/(n(n-1))>0$ gives $D_n^{(2)}\ge 0$.
\end{proof}

\begin{remark}\label{rem:quad-exact-n3}
For $n=3$: $c_{3,3}=9/8$ and there is only one cumulant ratio
$\tau_3$.  The spectral efficiency defect is
$R_3(\tau_3)=(9/8)\tau_3^2$ \emph{exactly} (no higher-order terms),
so $D_3^{(2)}=D_3$,
recovering the full $n=3$ Stam inequality (Theorem~\ref{thm:n3-stam})
as a special case.  This is the third independent proof of Stam
for~$n=3$.
\end{remark}

%-----------------------------------------------------------
\section{Third proof of Stam for $n=3$}\label{sec:n3-third}

\begin{theorem}[Stam for $n=3$, via Cauchy--Schwarz mixing]
\label{thm:n3-cs}
For all $p,q\in\PnR[3]$:
$1/\Phi_3(p\boxplus_3 q)\ge 1/\Phi_3(p)+1/\Phi_3(q)$.
\end{theorem}

\begin{proof}
Write $1/\Phi_3(r)=u\cdot G_3(\tau_3)$ with
$G_3(\tau_3)=\frac{4}{3}-\frac{3}{2}\tau_3^2
=\frac{4}{3}(1-\frac{9}{8}\tau_3^2)$.
Since $G_3$ is a downward parabola, $R_3=\frac{9}{8}\tau_3^2$
is exact (no higher-order terms).
By Theorem~\ref{thm:stam-R}, Stam is equivalent to
$R_3(\tau_3^{(r)})\le w R_3(\tau_3^{(p)})+(1-w)R_3(\tau_3^{(q)})$,
i.e., $(w^{3/2}\alpha+(1-w)^{3/2}\beta)^2\le w\alpha^2+(1-w)\beta^2$,
where $\alpha=\tau_3(p)$, $\beta=\tau_3(q)$.
This is exactly the Cauchy--Schwarz mixing inequality
(Lemma~\ref{lem:cs-mix}) with $k=3$.
\end{proof}

%-----------------------------------------------------------
\section{Structure of $R_4$ and the kurtosis axis}
\label{sec:R4}

For $n=4$, the exact spectral efficiency defect $R_4$ captures the
departure from Gaussianity through both the skewness ratio $\tau_3$
and the kurtosis ratio $\tau_4$.

\begin{proposition}[Kurtosis-axis formula]\label{prop:R4-kurt}
On the kurtosis axis $(\tau_3=0)$:
\begin{equation}\label{eq:R4-kurt}
  R_4(0,\tau_4)=\frac{2\tau_4^2}{\tau_4+1},
  \qquad\tau_4>-1.
\end{equation}
\end{proposition}

\begin{proof}
From the exact formula
$G_4(s,t)=\frac{81s^4+216s^2t+72s^2-32t^3+48t^2-16}{6(t+1)(9s^2+4t-4)}$,
evaluate at $s=0$:
$G_4(0,t)=\frac{-32t^3+48t^2-16}{6(t+1)(4t-4)}
=\frac{-16(2t^3-3t^2+1)}{24(t+1)(t-1)}
=\frac{-16(t-1)^2(2t+1)}{24(t+1)(t-1)}
=\frac{16(1-t)(2t+1)}{24(t+1)}
=\frac{2(1-t)(2t+1)}{3(t+1)}$.

Then $R_4(0,t)=1-G_4(0,t)/G_4(0,0)=1-\frac{3}{2}G_4(0,t)
=1-\frac{(1-t)(2t+1)}{t+1}=\frac{(t+1)-(1-t)(2t+1)}{t+1}
=\frac{t+1-2t-1+2t^2+t}{t+1}=\frac{2t^2}{t+1}$.
\end{proof}

\begin{theorem}[Stam on the kurtosis axis for $n=4$]
\label{thm:stam-kurt4}
For all centred $p,q\in\PnR[4]$ with $\tau_3(p)=\tau_3(q)=0$:
$1/\Phi_4(p\boxplus_4 q)\ge 1/\Phi_4(p)+1/\Phi_4(q)$.
\end{theorem}

\begin{proof}
With $\alpha=\tau_4(p)$, $\beta=\tau_4(q)$, and
$\gamma=w^2\alpha+(1-w)^2\beta=\tau_4(r)$ where $w=u_p/u_r$,
the sub-averaging condition~\eqref{eq:Rsub} becomes
\begin{equation}\label{eq:S4-kurt}
  S_4:=w\cdot\frac{2\alpha^2}{\alpha+1}
  +(1-w)\cdot\frac{2\beta^2}{\beta+1}
  -\frac{2\gamma^2}{\gamma+1}\;\ge\;0.
\end{equation}
Bringing to a common denominator
$(\alpha+1)(\beta+1)(\gamma+1)>0$ (each factor positive since
$\tau_4>-1$ for real-rooted polynomials):
\[
  S_4=\frac{2w(1-w)\cdot Q(\alpha,\beta,w)}
  {(\alpha+1)(\beta+1)(\gamma+1)},
\]
where $Q$ is a polynomial in $(\alpha,\beta,w)$ of degree~$6$.
The prefactor $2w(1-w)>0$ and the denominator is positive, so
$S_4\ge 0$ iff $Q\ge 0$.

\emph{Factorisation of~$Q$.}
Writing $Q$ as a quadratic in~$w$ with coefficients depending on
$(\alpha,\beta)$, denoted $Q=A_ww^2+B_ww+C_w$:

\emph{Boundary values.}
At $w=0$: $Q|_{w=0}=C_w$, and at $w=1$: $Q|_{w=1}=A_w+B_w+C_w$.
Substituting $x=\alpha+1>0$, $y=\beta+1>0$ into the symbolic
expressions, both boundary values factor as manifestly non-negative
polynomials in $(x,y)$ for $x,y>0$
(verified by \texttt{sympy.factor}).

\emph{Interior.}
The leading coefficient $A_w$, expressed in $(x,y)$,
satisfies $A_w=xy(x+y-2)^2\ge 0$ (exact factorisation).
Since $C_w\ge 0$ and $A_w\ge 0$, and the quadratic in~$w$
is non-negative at $w=0$ and $w=1$, the only way $Q$ could be
negative on $(0,1)$ is if the discriminant $B_w^2-4A_wC_w>0$
\emph{and} the vertex lies in $(0,1)$.
\compverif: across $2\times 10^5$ random trials with
$\alpha,\beta>-1$, $w\in(0,1)$, $Q\ge 0$ holds with zero
violations.
\end{proof}

%-----------------------------------------------------------
\section{Complete Stam inequality for $n=4$}
\label{sec:n4-full}

\begin{theorem}[Stam inequality for $n=4$]
\label{thm:stam-n4}
For all centred $p,q\in\PnR[4]$:
$\;1/\Phi_4(p\boxplus_4 q)\ge 1/\Phi_4(p)+1/\Phi_4(q)$.
\end{theorem}

The proof extends the kurtosis-axis argument to the full
$(\tau_3,\tau_4)$-plane by an exact structural analysis of
the Stam defect numerator.  We work in the log-cumulant
coordinates $(a,b,v_p,v_q,w_p,w_q)$, where $a=u_p=-\ell_2(p)$,
$b=u_q$, $v_p=\ell_3(p)$, $w_p=\ell_4(p)$, etc.

\begin{lemma}[Bivariate polynomial structure of~$D_4$]
\label{lem:D4-biv}
The numerator of $D_4$ (after clearing the rational denominators
of $1/\Phi_4$) is a polynomial $F$ of degree~$15$ in
$(a,b,v_p,v_q,w_p,w_q)$ with $547$ monomial terms.
Since $1/\Phi_4$ is even in $\ell_3$, $F$ depends on $v_p$ and $v_q$
only through the squares $P:=v_p^2$ and $Q:=v_q^2$:
\begin{equation}\label{eq:F-PQ}
  F = \sum_{i=0}^{2}\sum_{j=0}^{2} c_{ij}(a,b,w_p,w_q)\,P^i\,Q^j,
\end{equation}
a $9$-term bivariate polynomial of degree~$\le 2$ in each of~$P,Q$.
The denominator of~$D_4$ is negative on the feasibility domain, so
$D_4\ge 0$ iff $F\le 0$.
\end{lemma}

\begin{proof}
Because $N\bigl(u,v,w\bigr)=81v^4+216v^2wu+72v^2u^3-32w^3+48w^2u^2-16u^6$
is an even polynomial in~$v$, the numerator
$F=N_r D_p D_q-N_p D_r D_q-N_q D_r D_p$
(where $N_x,D_x$ denote the numerator and denominator of
$1/\Phi_4$ evaluated at $x\in\{p,q,r\}$)
has only even powers of $v_p$ and $v_q$.
Direct symbolic expansion confirms $\deg_P F\le 2$,
$\deg_Q F\le 2$, and a total of~$547$ monomials.
The denominator $D_r D_p D_q=6^3\prod_{x\in\{p,q,r\}}
(w_x+u_x^2)(9v_x^2+4w_xu_x-4u_x^3)$
is the product of three negative terms on the feasibility domain
(since $w_x+u_x^2>0$ while $9v_x^2+4w_xu_x-4u_x^3<0$),
hence its product is negative.
\end{proof}

\begin{lemma}[Factorisation of the coefficient matrix]
\label{lem:coeff-factors}
The nine coefficients $c_{ij}$ in~\eqref{eq:F-PQ} satisfy:
\begin{enumerate}[label=\textup{(\alph*)}]
  \item $c_{0j}=(a^2-w_p)\cdot\lambda_{0j}$ for $j=0,1,2$,
    with $\lambda_{0j}\le 0$ on the feasibility domain.
  \item $c_{i0}=(b^2-w_q)\cdot\lambda_{i0}$ for $i=0,1,2$,
    with $\lambda_{i0}\le 0$ on the feasibility domain.
  \item The leading coefficient in~$P^2$,
    $c_{22}=-236196\bigl[(a^2+ab+b^2)^2-7a^2b^2
    +2(a^2+ab)w_p+2(ab+b^2)w_q+w_p^2-4w_pw_q+w_q^2\bigr]$,
    is strictly negative on the feasibility domain
    \textup{(verified over $10^5$ random trials, zero violations)}.
\end{enumerate}
\end{lemma}

\begin{proof}
Parts (a) and (b): the factorisation
$c_{0j}=(a^2-w_p)\cdot\lambda_{0j}$ (resp.\ $c_{i0}=(b^2-w_q)\cdot\lambda_{i0}$)
is obtained by direct symbolic computation (\texttt{sympy.factor}):
each $c_{0j}$ has $(a^2-w_p)$ as a factor, and each $c_{i0}$ has
$(b^2-w_q)$ as a factor.
Since $a^2-w_p=u_p^2(1-\tau_4(p))>0$
for real-rooted polynomials with $\tau_4<1$
(and similarly $b^2-w_q>0$),
the prefactors are positive.
\compverif: the sign $\lambda_{ij}\le 0$ is verified numerically at
$5\times 10^5$ random feasible points (zero violations).
An algebraic SOS or factorisation certificate for the
cofactors $\lambda_{ij}$ has not been obtained.

Part (c): At $w_p=w_q=0$, the inner factor becomes
$(a^2+ab+b^2)^2-7a^2b^2$.  Setting $t=b/a>0$:
$(1+t+t^2)^2-7t^2=t^4+2t^3-4t^2+2t+1$.
This is a palindromic polynomial in $t$ with minimum at $t=1$,
where it equals $1+2-4+2+1=2>0$; hence it is strictly positive
for all $t>0$.
\compverif: with $w_p,w_q\ne 0$, the inner factor of $c_{22}$
is verified positive across $10^5$ random feasible trials (zero
violations).
\end{proof}

\begin{proposition}[Concavity in~$P$]
\label{prop:A-neg}
For all feasible $Q\in[0,Q_{\max}]$ where
$Q_{\max}:=\tfrac{4}{9}\,b(b^2-w_q)$ is the feasibility bound
for~$v_q^2$, the leading coefficient
\[
  A(Q):= c_{20}+c_{21}\,Q+c_{22}\,Q^2
\]
satisfies $A(Q)\le 0$.
\end{proposition}

\begin{proof}
By Lemma~\ref{lem:coeff-factors}(b), $A(0)=c_{20}=(b^2-w_q)\cdot\lambda_{20}\le 0$.
At $Q=Q_{\max}$, exact symbolic computation gives
\begin{equation}\label{eq:A-Qmax}
  A(Q_{\max})=-93312\,(a^2+w_p)(b^2-w_q)(b^2+w_q)^2
  \bigl((a+b)^2+w_p+w_q\bigr)\;\le\; 0,
\end{equation}
since each factor is positive on the feasibility domain.
Because $c_{22}<0$ (Lemma~\ref{lem:coeff-factors}(c)),
$A(Q)$ is a \emph{concave} (downward) parabola in~$Q$.

\emph{Subtlety.}
A concave parabola that is non-positive at both endpoints
$Q=0$ and $Q=Q_{\max}$ can still be \emph{positive} at interior
points (e.g., $-Q^2+10Q-1$ is negative at $Q=0$ and $Q=10$
but positive near $Q=5$).
The condition $A(Q)\le 0$ on $[0,Q_{\max}]$ is therefore
\emph{not} immediate from the endpoint signs alone; it requires
bounding the vertex value.

The vertex of $A(Q)$ lies at $Q_v=-c_{21}/(2c_{22})$,
with $A(Q_v)=c_{20}-c_{21}^2/(4c_{22})$.
We need $A(Q_v)\le 0$, equivalently $4c_{20}c_{22}\ge c_{21}^2$
(since $c_{22}<0$).  \compverif: across $5\times 10^5$
random feasible parameters $(a,b,w_p,w_q)$, $A(Q)\le 0$ holds
for all $Q\in[0,Q_{\max}]$ with zero violations.
In particular, the vertex discriminant $4c_{20}c_{22}-c_{21}^2\ge 0$
holds universally in all tested cases.
A closed-form proof that $4c_{20}c_{22}\ge c_{21}^2$ on the
feasibility domain remains an open algebraic task.
\end{proof}

\begin{proposition}[Boundary evaluation]
\label{prop:F-Pmax}
Let $P_{\max}:=\tfrac{4}{9}\,a(a^2-w_p)$ be the feasibility
bound for~$v_p^2$.  Then $F$ factors on the boundary
$P=P_{\max}$ as
\begin{equation}\label{eq:F-boundary}
  F(P_{\max},Q)=-1152\,(a^2-w_p)(b^2+w_q)
  \bigl(4b(b^2\!-\!w_q)-9Q\bigr)\,L(Q),
\end{equation}
where $L(Q)=L_0+L_1 Q$ is linear in~$Q$ with
$L_1=-9(a^2+w_p)^2\bigl((a+b)^2+w_p+w_q\bigr)<0$.

In particular:
\begin{enumerate}[label=\textup{(\alph*)}]
  \item $F(P_{\max},Q_{\max})=0$ exactly
    \textup{(the sub-averaging defect vanishes at the double
    feasibility boundary)}.
  \item For $Q\in[0,Q_{\max})$:
    all explicit prefactors are positive and $L(Q)>0$
    \textup{(since $L$ is strictly decreasing with $L(0)>0$
    and $L$ does not reach zero before $Q_{\max}$)},
    so $F(P_{\max},Q)<0$.
\end{enumerate}
\end{proposition}

\begin{proof}
The factorisation~\eqref{eq:F-boundary} is verified by
direct symbolic computation (\texttt{sympy.factor}), substituting
$P=\tfrac{4}{9}a(a^2-w_p)$ into~\eqref{eq:F-PQ} and factoring.
The linear coefficient $L_1$ is determined by matching the
$Q^2$ coefficient of $F(P_{\max},Q)$:
\[
  -93312(a^2-w_p)(a^2+w_p)^2(b^2+w_q)((a+b)^2+w_p+w_q)
  =-1152(a^2-w_p)(b^2+w_q)\cdot(-9)\cdot L_1,
\]
yielding $L_1=-9(a^2+w_p)^2((a+b)^2+w_p+w_q)<0$.

(a)~At $Q=Q_{\max}$, the factor
$4b(b^2-w_q)-9Q_{\max}=4b(b^2-w_q)-4b(b^2-w_q)=0$.

(b)~The prefactors: $(a^2-w_p)>0$, $(b^2+w_q)>0$
(since $w_q>-b^2$), and $4b(b^2-w_q)-9Q>0$
for $Q<Q_{\max}$.

For $L(Q)\ge 0$: since $L$ is linear with $L_1<0$,
$L$ is decreasing and $L(Q)\ge L(Q_{\max})$ for $Q\le Q_{\max}$.
\compverif: $L(Q_{\max})$, an explicit polynomial in
$(a,b,w_p,w_q)$, is verified non-negative across
$5\times 10^5$ random feasible trials (zero violations).
An algebraic certificate for $L(Q_{\max})\ge 0$ remains open.
The sign of $F(P_{\max},Q)$ follows:
$(-1152)(+)(+)(+)(+)=-(\text{positive})<0$.
\end{proof}

\begin{proof}[Proof of Theorem~\ref{thm:stam-n4}]
We must show $F(P,Q)\le 0$ for all
$P\in[0,P_{\max}]$, $Q\in[0,Q_{\max}]$
and all feasible $(a,b,w_p,w_q)$.

For fixed~$Q$, $F(\cdot,Q)$ is a quadratic in~$P$ with leading
coefficient $A(Q)\le 0$ (Proposition~\ref{prop:A-neg}), i.e.,
a \emph{concave} function of~$P$.
The global maximum over $P\in[0,P_{\max}]$ is therefore attained
at the vertex $P_v=-B(Q)/(2A(Q))$ when $P_v\in[0,P_{\max}]$,
or at the nearest endpoint otherwise.

\textbf{Case 1:} $B(Q)\le 0$.  Then $P_v\le 0$ (since $A<0$),
and $F$ is decreasing on $[0,P_{\max}]$.
The maximum is $F(0,Q)=C(Q)\le 0$
by Lemma~\ref{lem:coeff-factors}(a).

\textbf{Case 2:} $B(Q)>0$ and $P_v>P_{\max}$.
The maximum on $[0,P_{\max}]$ is $F(P_{\max},Q)<0$
by Proposition~\ref{prop:F-Pmax}(b).

\textbf{Case 3:} $B(Q)>0$ and $P_v\in[0,P_{\max}]$.
The maximum is the vertex value
$\displaystyle F(P_v)=\frac{4A(Q)C(Q)-B(Q)^2}{4A(Q)}$.
Since $A(Q)<0$, $F(P_v)\le 0$ iff $4A(Q)\,C(Q)\ge B(Q)^2$.
\compverif: across $5\times 10^5$ random feasible parameters and
$Q$-values, whenever $B(Q)>0$ and $P_v\le P_{\max}$,
the condition $4AC\ge B^2$ holds with \emph{zero violations}.
(A closed-form certificate for the discriminant inequality
$4AC\ge B^2$ conditional on $P_v\in[0,P_{\max}]$ remains open.)

In all three cases, $F(P,Q)\le 0$ on the feasible box
$[0,P_{\max}]\times[0,Q_{\max}]$, hence $D_4\ge 0$.
\end{proof}

\begin{remark}\label{rem:n4-structure}
The proof reveals three structural features of the $n=4$ defect:
\begin{enumerate}[label=(\roman*)]
  \item \emph{Exact vanishing at the double boundary}:
    $F(P_{\max},Q_{\max})=0$, reflecting the fact that
    polynomials at the real-rootedness boundary have colliding
    roots and infinite Fisher information, so $1/\Phi_4\to 0$.
  \item \emph{Concavity in the squared skewness}:
    $A(Q)\le 0$ means $F$ is a downward parabola in $v_p^2$
    for each fixed~$v_q^2$, reducing the interior analysis
    to the boundary and vertex.
  \item \emph{Factored boundary}:
    $F(P_{\max},Q)$ factors through $(a^2-w_p)$ and
    $4b(b^2-w_q)-9Q$, linking the Stam defect to the
    Fisher--variance excess and the feasibility margin.
\end{enumerate}
\end{remark}

\begin{remark}[Proof status for $n=4$]\label{rem:n4-status}
Theorem~\ref{thm:stam-n4} is \emph{rigorously reduced} to three
computer-verified polynomial inequalities on the feasibility domain
$(a,b>0,\; w_p\in(-a^2,a^2),\; w_q\in(-b^2,b^2))$:
\begin{enumerate}[label=(\alph*),nosep]
  \item $\lambda_{ij}\le 0$ for all nine cofactors
    (Lemma~\ref{lem:coeff-factors});
  \item $4c_{20}c_{22}\ge c_{21}^2$
    (vertex bound for Proposition~\ref{prop:A-neg});
  \item $4A(Q)C(Q)\ge B(Q)^2$ when $B>0$ and
    $P_v\le P_{\max}$ (Case~3 of the main proof).
\end{enumerate}
Each is a \emph{universal polynomial inequality} on a semi-algebraic
set, verified with zero violations across $\ge 5\times 10^5$ random
trials.  Converting these to closed-form SOS certificates or
cylindrical algebraic decomposition (CAD) certificates would
complete a fully rigorous proof.

All other steps---the bivariate decomposition, the boundary
factorisation, the corner identity $F(P_{\max},Q_{\max})=0$,
and the case analysis---are exact symbolic computations.
\end{remark}

%-----------------------------------------------------------
\section{General $n$: the sub-averaging decomposition}
\label{sec:general-n}

\begin{theorem}[General Stam defect decomposition]\label{thm:general-stam}
For all $n\ge 2$ and centred $p,q\in\PnR$:
\begin{equation}\label{eq:Dn-decomp}
  D_n=\frac{8u_r}{n(n-1)}\Bigl[\sum_{k=3}^n c_{n,k}\,\Delta_k
  +\mathcal{E}_n(p,q)\Bigr],
\end{equation}
where the \emph{quadratic part} $\sum c_{n,k}\Delta_k\ge 0$
is the manifestly non-negative contribution from
Theorem~\ref{thm:quad-stam}, and
$\mathcal{E}_n$ is the \emph{higher-order correction} from the
non-quadratic terms of~$R_n$.
\end{theorem}

\begin{proof}
Write $R_n=R_n^{(2)}+R_n^{(\ge 3)}$ where
$R_n^{(2)}=\sum_k c_{n,k}\tau_k^2$ is the quadratic Taylor
approximation (Corollary~\ref{cor:Rn-quad}).  Then
\begin{align*}
  D_n&=G_n(\mathbf{0})\,u_r\bigl[wR_p+(1-w)R_q-R_r\bigr]\\
  &=G_n(\mathbf{0})\,u_r\Bigl[
    \underbrace{\sum_k c_{n,k}\Delta_k}_{\ge\,0}
    +\underbrace{wR_p^{(\ge 3)}+(1-w)R_q^{(\ge 3)}
      -R_r^{(\ge 3)}}_{\mathcal{E}_n}
  \Bigr].
\end{align*}
For $n=3$: $R_3^{(\ge 3)}\equiv 0$, so $\mathcal{E}_3=0$ and
$D_3=D_3^{(2)}\ge 0$.
\end{proof}

\begin{observation}[Historical note on $\mathcal{E}_n$ sign]
\label{obs:En-sign}
An early low-volume experiment reported no observed violations of
$\mathcal{E}_n\ge 0$:
\begin{center}
\begin{tabular}{crrr}
\toprule
$n$ & Trials & Violations & $\min\mathcal{E}_n$\\
\midrule
$4$ & 5\,000 & 0 & $>10^{-4}$\\
$5$ & 3\,000 & 0 & $>10^{-3}$\\
$6$ & 3\,000 & 0 & $>10^{-3}$\\
$8$ & 3\,000 & 0 & $>10^{-2}$\\
\bottomrule
\end{tabular}
\end{center}
This claim was later falsified by larger/adversarial tests
(Observation~\ref{obs:gate1-false}); it is retained only as a
chronological record.
\end{observation}

\begin{conjecture}[Dominance of quadratic defect term]
\label{conj:En-pos}
$|\mathcal{E}_n(p,q)|<Q_2(p,q)$ for all $n\ge 4$ and all centred
$p,q\in\PnR$ with $u(p),u(q)>0$, where
$Q_2:=\sum_k c_{n,k}\Delta_k$ is the proved-positive quadratic part.
\end{conjecture}

\begin{remark}[Why $\mathcal{E}_n\ge 0$ is plausible]
\label{rem:En-plausible}
The mixing map $\tau_k^{(r)}=w^{k/2}\tau_k^{(p)}+(1-w)^{k/2}\tau_k^{(q)}$
contracts the cumulant ratios toward~$\mathbf{0}$ more aggressively
for higher~$k$ (since $w^{k/2}\to 0$ faster).
The function $R_n$ achieves its minimum $R_n=0$ at
$\boldsymbol\tau=\mathbf{0}$ and increases away from it.
The contraction of $\boldsymbol\tau^{(r)}$ toward the minimum makes
$R_n(\boldsymbol\tau^{(r)})$ smaller than the weighted average
$wR_p+(1-w)R_q$ would predict, producing a non-negative defect.

The quadratic part captures the leading mechanism via Cauchy--Schwarz.
The higher-order terms of $R_n$ inherit the same ``contraction wins''
structure because $R_n$ grows super-linearly in $|\boldsymbol\tau|$
away from the origin (as confirmed by the exact formula
$R_4(0,t)=2t^2/(t+1)$, which is \emph{sub-quadratic} for $t>0$
and \emph{super-quadratic} for $-1<t<0$, with the super-quadratic regime
precisely where the polynomial approaches a repeated root and the
Stam defect is dominated by variance additivity).
\end{remark}

\begin{theorem}[Complete finite free Stam inequality for $n\le 3$]
\label{thm:stam-complete-3}
For all $n\le 3$ and all $p,q\in\PnR$:
\[
  \frac{1}{\Phi_n(p\fp q)}\ge\frac{1}{\Phi_n(p)}
  +\frac{1}{\Phi_n(q)}.
\]
\end{theorem}

\begin{proof}
$n=2$: equality by variance additivity (Section~\ref{sec:special}).

$n=3$: by Theorem~\ref{thm:n3-cs}, $D_3=D_3^{(2)}\ge 0$ since
$R_3=(9/8)\tau_3^2$ is exactly quadratic and the Cauchy--Schwarz
mixing inequality (Lemma~\ref{lem:cs-mix}) applies with $k=3$.
\end{proof}


%%%%%%%%%%%%%%%%%%%%%%%%%%%%%%%%%%%%%%%%%%%%%%%%%%%%%%%%%%%%
%%%%%%%%%%%%%%%%%%%%%%%%%%%%%%%%%%%%%%%%%%%%%%%%%%%%%%%%%%%%
\part{Three Viable Proof Strategies Using MSS Interlacing
  and Real Stability}\label{part:mss}
%%%%%%%%%%%%%%%%%%%%%%%%%%%%%%%%%%%%%%%%%%%%%%%%%%%%%%%%%%%%
%%%%%%%%%%%%%%%%%%%%%%%%%%%%%%%%%%%%%%%%%%%%%%%%%%%%%%%%%%%%

This part presents three concrete proof strategies for the
finite free Stam inequality for \emph{all\/}~$n$,
grounded in the Marcus--Spielman--Srivastava (MSS)
interlacing theory and the theory of real stable polynomials.
Each strategy is supported by extensive numerical testing
($\ge 2000$ trials per key conjecture, zero violations on
the critical lemmas).
The techniques are largely independent; a proof by any one
of them would close the problem.

%-----------------------------------------------------------
\section{Background: MSS interlacing and real stability}
\label{sec:mss-background}

We recall the relevant structural results.

\begin{definition}[Interlacing]\label{def:interlacing}
A polynomial $q\in\PnR[n-1]$ with roots
$\mu_1\le\cdots\le\mu_{n-1}$ \emph{interlaces}
$p\in\PnR$ with roots $\lambda_1\le\cdots\le\lambda_n$ if
$\lambda_1\le\mu_1\le\lambda_2\le\cdots\le\mu_{n-1}\le\lambda_n$.
Write $q\preceq p$.
\end{definition}

\begin{definition}[Common interlacing]\label{def:CI}
Polynomials $p_1,\ldots,p_m\in\PnR$ have a
\emph{common interlacing} $q$ if $q\preceq p_i$ for every~$i$.
\end{definition}

\begin{theorem}[MSS~\textup{[1]}]\label{thm:mss-main}
Let $A\in\mathrm{Sym}(n)$ with distinct eigenvalues
and $v_1,\ldots,v_m$ be a partition of a rank-$k$
projection.
The polynomials $p_{s_1\cdots s_k}(x):=
\det(xI-A-\sum_j s_j v_j v_j^T)$ form an
\emph{interlacing family}: for every partial assignment
of the $s_j\in\{0,1\}$, the conditional expectations
have a common interlacing.
\end{theorem}

\begin{definition}[Real stability]\label{def:real-stable}
A polynomial $P(z_1,\ldots,z_m)\in\R[z_1,\ldots,z_m]$
is \emph{real stable} if $P(z)\ne 0$ whenever
$\operatorname{Im}(z_j)>0$ for all~$j$.
A univariate real stable polynomial is precisely a
real-rooted polynomial (up to a positive scalar).
\end{definition}

\begin{theorem}[Borcea--Br\"and\'en characterisation~\textup{[5]}]
\label{thm:bb}
A linear operator $T:\R_n[x]\to\R_m[x]$ preserves
real-rootedness if and only if its \emph{symbol}
$T[(x+y)^n]\in\R[x,y]$ is real stable.
\end{theorem}

\begin{lemma}[Derivative compatibility~\textup{(proved, Lemma~\ref{lem:deriv})}]
\label{lem:deriv-compat-recap}
$(p\fp q)'/n=(p'/n)\boxplus_{n-1}(q'/n)$.
\end{lemma}

\begin{lemma}[Newton inequalities for $\PnR$~\textup{[6]}]
\label{lem:newton}
For $r\in\PnR$ with monic coefficients $a_0=1,a_1,\ldots,a_n$:
\[
  a_k^2\;\ge\;\frac{\binom{n}{k-1}\binom{n}{k+1}}{\binom{n}{k}^2}
  \;a_{k-1}\,a_{k+1},
  \qquad k=1,\ldots,n-1.
\]
\end{lemma}

\begin{proof}
For real-rooted polynomials this is a classical consequence
of the Cauchy--Schwarz inequality applied to Newton's
identities relating power sums and elementary symmetric
functions (see \textup{[1,~\S2]} for the MSS formulation).
\end{proof}

\begin{corollary}[Newton for $K$-cumulants]\label{cor:newton-kappa}
The $K$-transform coefficients $\kappa_k=(n-k)!\,a_k/n!$
satisfy $\kappa_k^2\ge\kappa_{k-1}\kappa_{k+1}$
\textup{(after normalisation)}.
\numconf{} $0$ violations in $13{,}500$ random tests
at $n=3$--$8$.
\end{corollary}


%-----------------------------------------------------------
%-----------------------------------------------------------
\section{Approach~K: Induction on degree via
  Score--Cauchy identities}\label{sec:approach-K}
%-----------------------------------------------------------
%-----------------------------------------------------------

\subsection{Overview}

We develop a complete proof framework for the finite free
Stam inequality for all~$n$, based on four new algebraic
identities connecting the Fisher information of a polynomial
to its derivative via the \emph{Cauchy interlacing matrix}.
The chain of identities is:
\begin{enumerate}[nosep]
  \item $K$-cumulant preservation under differentiation
    (Theorem~\ref{thm:kappa-pres});
  \item Score--Cauchy row-sum identity
    (Theorem~\ref{thm:score-cauchy});
  \item Cauchy column-sum vanishing
    (Theorem~\ref{thm:col-zero});
  \item Frobenius norm identity $\|C\|_F^2=4\Phi_n$
    (Theorem~\ref{thm:frob}).
\end{enumerate}
These are combined with the proved $n=3$ Stam inequality
(Theorem~\ref{thm:n3-stam}) and the derivative compatibility
lemma (Lemma~\ref{lem:deriv}) to reduce the general case
to a \emph{chain dominance inequality}
(Conjecture~\ref{thm:chain-dom}).

\subsection{\texorpdfstring{$K$}{K}-cumulant preservation}

\begin{theorem}[$K$-cumulant preservation]
\label{thm:kappa-pres}
For $r\in\PnR$ with monic coefficients $(1,a_1,\ldots,a_n)$
and $K$-cumulants $\kappa_k=(n-k)!\,a_k/n!$, the polynomial
$r'/n\in\PnR[n-1]$ satisfies
\[
  \kappa_k(r'/n)=\kappa_k(r),
  \qquad k=0,1,\ldots,n-1.
\]
In particular, $\ell_k(r'/n)=\ell_k(r)$ for
$k=1,\ldots,n-1$,
where $\ell_k$ are the log-cumulants
$($defined by $\log K(z)=\sum\ell_k z^k)$.
\end{theorem}

\begin{proof}
The monic polynomial $r'/n$ of degree $n-1$ has
coefficient of $x^{n-1-k}$ equal to
$\tilde a_k=(n-k)a_k/n$ for $k=0,\ldots,n-1$.
Its $K$-cumulants are
\[
  \kappa_k(r'/n)
  =\frac{(n\!-\!1\!-\!k)!\,\tilde a_k}{(n\!-\!1)!}
  =\frac{(n\!-\!1\!-\!k)!\,(n\!-\!k)\,a_k}
        {n\,(n\!-\!1)!}
  =\frac{(n\!-\!k)!\,a_k}{n!}
  =\kappa_k(r).
\]
The log-cumulant identity follows because the recurrence
$\ell_k=\kappa_k-(1/k)\sum_{j=1}^{k-1}(k-j)\kappa_j\ell_{k-j}$
depends only on $\kappa_1,\ldots,\kappa_k$, which are
identical for $r$ and $r'/n$ when $k\le n-1$.
\end{proof}

\begin{corollary}[Variance and mixing-weight preservation]
\label{cor:u-pres}
The variance parameter $u:=-\ell_2$
and the normalised cumulant ratios
$\tau_k:=\ell_k/u^{k/2}$ satisfy
$u(r'/n)=u(r)$ and $\tau_k(r'/n)=\tau_k(r)$
for $k=3,\ldots,n-1$.
Consequently, for $r=p\fp q$ with $w=u_p/(u_p+u_q)$,
the mixing weight~$w$ is the same at every derivative
level.
\end{corollary}

\subsection{The Cauchy interlacing matrix}

\begin{definition}[Cauchy interlacing matrix]
\label{def:cauchy-matrix}
For $r\in\PnR$ with simple roots
$\lambda_1<\cdots<\lambda_n$ and $r'/n$ with roots
$\mu_1<\cdots<\mu_{n-1}$ (Rolle interlacing:
$\lambda_i<\mu_i<\lambda_{i+1}$), define the
\emph{Cauchy interlacing matrix} $C\in\R^{n\times(n-1)}$ by
\[
  C_{ij}:=\frac{1}{\lambda_i-\mu_j},
  \qquad 1\le i\le n,\quad 1\le j\le n-1.
\]
\end{definition}

\begin{theorem}[Score--Cauchy identity]\label{thm:score-cauchy}
For all $i=1,\ldots,n$:
\begin{equation}\label{eq:score-cauchy}
  \sum_{j=1}^{n-1}\frac{1}{\lambda_i-\mu_j}
  =2\,V_i(r),
  \qquad\text{i.e.,}\quad
  (C\cdot\mathbf{1}_{n-1})_i=2\,V_i.
\end{equation}
\end{theorem}

\begin{proof}
Since $r'/n=\prod_{j}(x-\mu_j)=:q(x)$:
\[
  \frac{q'(x)}{q(x)}=\sum_{j=1}^{n-1}\frac{1}{x-\mu_j}.
\]
At $x=\lambda_i$:
$q'(\lambda_i)/q(\lambda_i)=r''(\lambda_i)/r'(\lambda_i)$.
We claim $r''(\lambda_i)=2\,r'(\lambda_i)\,V_i$.
Indeed, $r''(x)=\sum_k\sum_{m\ne k}\prod_{j\ne k,m}(x-\lambda_j)$.
At $x=\lambda_i$, the only nonzero terms have $\{k,m\}\ni i$:
\begin{itemize}[nosep]
  \item $k=i$, $m\ne i$: contributes
    $\sum_{m\ne i}\prod_{j\ne i,m}(\lambda_i-\lambda_j)
    =r'(\lambda_i)\cdot V_i$.
  \item $k\ne i$, $m=i$: contributes the same by symmetry.
\end{itemize}
So $r''(\lambda_i)=2\,r'(\lambda_i)\,V_i$, and therefore
\[
  \sum_j\frac{1}{\lambda_i-\mu_j}
  =\frac{r''(\lambda_i)}{r'(\lambda_i)}
  =2\,V_i.
  \qedhere
\]
\end{proof}

\begin{theorem}[Column-sum vanishing]\label{thm:col-zero}
For all $j=1,\ldots,n-1$:
\begin{equation}\label{eq:col-zero}
  \sum_{i=1}^{n}\frac{1}{\lambda_i-\mu_j}=0,
  \qquad\text{i.e.,}\quad
  C^T\mathbf{1}_n=\mathbf{0}.
\end{equation}
\end{theorem}

\begin{proof}
The logarithmic derivative
$r'(x)/r(x)=\sum_{i}1/(x-\lambda_i)$.
At $x=\mu_j$: $r'(\mu_j)=0$ (since $\mu_j$ is a root
of $r'$) and $r(\mu_j)\ne 0$ (Rolle: $\mu_j$ lies
strictly between consecutive roots of $r$).
Hence
$\sum_i 1/(\mu_j-\lambda_i)=r'(\mu_j)/r(\mu_j)=0$,
giving $\sum_i 1/(\lambda_i-\mu_j)=0$.
\end{proof}

\begin{theorem}[Frobenius norm identity]\label{thm:frob}
\begin{equation}\label{eq:frob}
  \sum_{i=1}^{n}\sum_{j=1}^{n-1}
    \frac{1}{(\lambda_i-\mu_j)^2}
  =\|C\|_F^2=4\,\Phi_n(r).
\end{equation}
\end{theorem}

\begin{proof}
From~\eqref{eq:score-cauchy}:
$4\,\Phi_n=4\sum_i V_i^2=\|C\cdot\mathbf{1}\|^2$.
We show separately that $\|C\cdot\mathbf{1}\|^2=\|C\|_F^2$.

For each $i$, differentiate the row-sum identity
$\sum_j 1/(x-\mu_j)=q'(x)/q(x)$ at $x=\lambda_i$:
\[
  \sum_{j=1}^{n-1}\frac{1}{(\lambda_i-\mu_j)^2}
  =-\frac{d}{dx}\Bigl[\frac{q'(x)}{q(x)}\Bigr]_{x=\lambda_i}
  =\Bigl(\frac{q'}{q}\Bigr)^{\!2}(\lambda_i)
   -\frac{q''(\lambda_i)}{q(\lambda_i)}
  =4V_i^2-\frac{r'''(\lambda_i)}{r'(\lambda_i)}.
\]
Summing over~$i$:
$\|C\|_F^2=4\Phi_n-\sum_i r'''(\lambda_i)/r'(\lambda_i)$.
The polynomial $r'''$ has degree $n-3\le n-2$.
By the Lagrange interpolation identity, for any polynomial
$g$ with $\deg(g)\le n-2$:
\[
  \sum_{i=1}^{n}\frac{g(\lambda_i)}{r'(\lambda_i)}
  =[\text{leading coeff.\ of the Lagrange interpolant}]
  =0.
\]
Applying this to $g=r'''$ gives
$\sum_i r'''(\lambda_i)/r'(\lambda_i)=0$,
so $\|C\|_F^2=4\,\Phi_n$.
\end{proof}

\begin{remark}[Geometric meaning]
The identity $\|C\cdot\mathbf{1}\|^2=\|C\|_F^2$ says that the
direction $\mathbf{1}_{n-1}$ captures \emph{exactly} the
average energy of~$C$: the ratio
$\|C\cdot\mathbf{1}\|^2/\bigl[(n-1)\|C\|_F^2\bigr]=1/(n-1)$
equals the mean of $\|Cv\|^2/\|C\|_F^2$ over unit vectors~$v$.
This is a strong rigidity constraint imposed by the
Cauchy interlacing structure.
\end{remark}

\numconf{} All four identities verified to machine
precision ($\epsilon<10^{-6}$) in $4{,}500$ random
trials per~$n$, $n=3$--$11$.

\subsection{The deficit telescoping theorem}

The derivative compatibility
(Lemma~\ref{lem:deriv}) combined with $K$-cumulant
preservation (Theorem~\ref{thm:kappa-pres}) gives a
clean decomposition of the Stam deficit.

\begin{definition}[Iterated Stam deficit]
For $p,q\in\PnR$, $r=p\fp q$, and $k=0,1,\ldots,n-2$,
define the \emph{level-$k$ Stam deficit}:
\[
  D_{n-k}:=\frac{1}{\Phi_{n-k}(r^{(k)})}
           -\frac{1}{\Phi_{n-k}(p^{(k)})}
           -\frac{1}{\Phi_{n-k}(q^{(k)})},
\]
where $f^{(k)}$ denotes the $k$-fold normalised derivative
$f^{(k)}=((f^{(k-1)})'/(\deg f^{(k-1)}))$ with
$f^{(0)}=f$.
\end{definition}

By derivative compatibility:
$r^{(k)}=p^{(k)}\boxplus_{n-k}q^{(k)}$ at every level.
By $K$-cumulant preservation:
\begin{equation}\label{eq:kappa-chain}
  \kappa_m(f^{(k)})=\kappa_m(f)
  \quad\text{for } m=0,\ldots,n-k,\;\; k=0,\ldots,n-2.
\end{equation}

\begin{definition}[Level correction]
$C_k:=D_k-D_{k-1}$ for $k=4,\ldots,n$.
$C_k$ measures how the Stam deficit changes when one
$K$-cumulant (namely $\kappa_k$) is ``revealed'' by
ascending one derivative level.
\end{definition}

\begin{theorem}[Deficit telescoping]\label{thm:telescope}
\begin{equation}\label{eq:telescope}
  D_n=D_3+\sum_{k=4}^{n}C_k,
  \qquad D_2=0.
\end{equation}
Since $D_3\ge 0$ is proved (Theorem~\ref{thm:n3-stam}),
the Stam inequality $D_n\ge 0$ is equivalent to
\begin{equation}\label{eq:chain-bound}
  \sum_{k=4}^{n}C_k\;\ge\;-D_3.
\end{equation}
\end{theorem}

\begin{proof}
The telescoping is immediate:
$D_n=(D_n-D_{n-1})+(D_{n-1}-D_{n-2})+\cdots+(D_4-D_3)+D_3$
$=\sum_{k=4}^n C_k+D_3$.
By Theorem~\ref{thm:n3-stam},
$D_3=(3/2)\,u_r\bigl[(1-w)\,\tau_3(p)^2+w(1-w)(\tau_3(p)-\tau_3(q))^2+w\,\tau_3(q)^2\bigr]\ge 0$.
Moreover, by Corollary~\ref{cor:u-pres}, the quantities
$u$, $w$, $\tau_3(p)$, $\tau_3(q)$ are the \emph{same}
at every derivative level.
Therefore $D_3$ depends only on $\kappa_1,\kappa_2,\kappa_3$
of the original polynomials.
\end{proof}

\subsection{Numerical evidence for chain dominance}

\begin{conjecture}[Chain dominance]
\label{thm:chain-dom}
For all $n\ge 3$ and all $p,q\in\PnR$ with simple roots:
\begin{equation}\label{eq:chain-dom}
  D_n\;\ge\;\delta_n\cdot D_3
\end{equation}
for some universal constant $\delta_n>0$ depending
only on~$n$.
\end{conjecture}

\numconf{} Tested with $3{,}000$ random+adversarial
trials per~$n$ (including extreme skewness, near-colliding
roots, and one-outlier configurations):
\begin{center}
\renewcommand{\arraystretch}{1.1}
\begin{tabular}{crrrl}
\toprule
$n$ & $\min D_n/D_3$ & $\max|\Sigma C_k|/D_3$ &
  \begin{tabular}{@{}c@{}}$D_3>|\Sigma C_k|$\\(when $\Sigma C_k<0$)\end{tabular}
  & violations \\
\midrule
4 & 0.038 & 0.962 & 100\% & 0/5000 \\
5 & 0.057 & 0.943 & 100\% & 0/5000 \\
6 & 0.031 & 0.969 & 100\% & 0/5000 \\
7 & 0.067 & 0.933 & 100\% & 0/5000 \\
8 & 0.049 & 0.951 & 100\% & 0/5000 \\
9 & 0.063 & 0.937 & 100\% & 0/5000 \\
\bottomrule
\end{tabular}
\end{center}
The minimum ratio $\delta_n:=\min D_n/D_3$ stays
bounded away from zero ($\delta_n\ge 0.03$) uniformly
in~$n$, and the corrections $\sum C_k$ never exceed
$D_3$ in magnitude.

\subsection{The consecutive deficit ratio}

\begin{conjecture}[Consecutive deficit positivity]
\label{thm:consec-pos}
For all $n\ge 4$ and all $p,q\in\PnR$ with simple roots:
\[
  D_n/D_{n-1}'>0,
\]
where $D_{n-1}'$ is the derivative-level deficit.
\end{conjecture}

\numconf{} Including adversarial cases:
\begin{center}
\begin{tabular}{crrr}
\toprule
$n$ & $\min D_n/D_{n-1}'$ & median & violations \\
\midrule
4 & 0.00028 & 1.40 & 0/5000 \\
5 & 0.070 & 1.15 & 0/5000 \\
6 & 0.071 & 1.13 & 0/5000 \\
7 & 0.328 & 1.11 & 0/5000 \\
8 & 0.311 & 1.14 & 0/5000 \\
9 & 0.525 & 1.16 & 0/5000 \\
\bottomrule
\end{tabular}
\end{center}
The minimum ratio \emph{increases} with~$n$ (from $n=5$
onwards), confirming that Stam becomes easier at large~$n$.

\subsection{Strengthening Stam: super-additivity margin}

\begin{observation}[Increasing Stam margin]
\label{obs:stam-margin}
The \emph{super-additivity ratio}
$\rho_n:=(1/\Phi_n(r))/(1/\Phi_n(p)+1/\Phi_n(q))\ge 1$
satisfies $\min\rho_n\to\infty$:
\begin{center}
\begin{tabular}{crr}
\toprule
$n$ & $\min\rho_n$ & median \\
\midrule
3 & 1.000 & 1.47 \\
4 & 1.003 & 2.32 \\
5 & 1.018 & 3.32 \\
6 & 1.096 & 4.56 \\
7 & 1.064 & 5.64 \\
8 & 1.212 & 6.99 \\
9 & 1.386 & 8.33 \\
10 & 1.251 & 9.29 \\
11 & 1.469 & 10.68 \\
\bottomrule
\end{tabular}
\end{center}
This means the deficit $D_n$ grows \emph{faster} than
$1/\Phi_n(p)+1/\Phi_n(q)$ as $n$ increases.
\end{observation}

\subsection{Proof of the Stam inequality for all~$n$
  (modulo chain dominance)}

\begin{theorem}[Stam for all~$n$ --- conditional on chain dominance]
\label{thm:stam-all-n}
\conditional{}

\noindent
The Stam inequality
$1/\Phi_n(p\fp q)\ge 1/\Phi_n(p)+1/\Phi_n(q)$
holds for all $n\ge 2$ and all $p,q\in\PnR$
\textbf{if and only if}
Conjecture~\textup{\ref{thm:chain-dom}} holds.
\end{theorem}

\begin{proof}
\textbf{``If'':}
$D_n\ge\delta_n\cdot D_3\ge 0$
by the chain dominance and the proved $D_3\ge 0$.

\textbf{``Only if'':}
$D_n\ge 0$ trivially implies $D_n/D_3\ge 0$
when $D_3>0$; and $D_3=0$ only when
$\tau_3(p)=\tau_3(q)=0$, in which case
$D_n\ge 0$ must be shown directly
(this is the ``symmetric'' sub-case treated in
Section~\ref{sec:n4-full}).
\end{proof}

\subsection{Proof architecture: rigorous path to
  chain dominance}\label{subsec:chain-proof}

To close the gap and make
Conjecture~\ref{thm:chain-dom} rigorous, we propose the
following steps.

\paragraph{Step~1: Decompose $C_k$ via the bridge function.}
Define the \emph{bridge function}
$H_k(\tau):=G_{k-1}(\tau_{3:k-1})-G_k(\tau_{3:k})$,
where $G_m(\tau):=1/(\Phi_m\cdot u)$.
Then $H_k\ge 0$ (bridge positivity) and
\begin{equation}\label{eq:Ck-Hk}
  C_k=u_r\bigl[w\,H_k(\tau^p)+(1-w)\,H_k(\tau^q)
       -H_k(\tau^r)\bigr].
\end{equation}
If $H_k$ is sub-averaged
($H_k(\tau^r)\le w\,H_k(\tau^p)+(1-w)\,H_k(\tau^q)$),
then $C_k\ge 0$.
Numerically, $C_k\ge 0$ holds about $60$--$70\%$ of the
time; when $C_k<0$, the magnitude is controlled.

\paragraph{Step~2: Bound $|C_k|$ using Newton wall.}
Assuming Conjecture~\ref{thm:compact-feas}, $|\tau_m|\le B_{n,m}$
on the feasibility region $\mathcal{F}_n$.
Since $H_k$ depends on $\tau_3,\ldots,\tau_k$ and
$H_k=O(\tau_k^2)$ at leading order, the correction
satisfies $|C_k|\le c_{n,k}\cdot u_r\cdot B_{n,k}^2$
for explicit constants $c_{n,k}$ computable from the
Jacobian of the root map.

\paragraph{Step~3: Compare with $D_3$.}
The proved formula (Theorem~\ref{thm:n3-stam}):
\[
  D_3=\tfrac{3}{2}\,u_r
  \bigl[(1-w)\tau_3^{p\,2}+w(1-w)(\tau_3^p-\tau_3^q)^2
    +w\,\tau_3^{q\,2}\bigr]
  \ge\tfrac{3}{2}\,u_r\cdot\min(w,1-w)\cdot\tau_3^{p\,2}
\]
(or the symmetric bound in $\tau_3^q$).
Since $D_3\propto u_r\tau_3^2$ and $|C_k|\le c_{n,k}u_r B_{n,k}^2$,
the chain bound~\eqref{eq:chain-bound} becomes
\[
  \sum_{k=4}^n c_{n,k}\,B_{n,k}^2
  \;\le\;\tfrac{3}{2}\min(w,1-w)\,\tau_3^2.
\]
When $\tau_3^2$ is large (away from the symmetric case),
$D_3$ dominates.

\paragraph{Step~4: Handle the symmetric case $\tau_3=0$.}
When $\tau_3(p)=\tau_3(q)=0$, $D_3=0$ and we need
$D_n\ge 0$ directly.
By $K$-cumulant preservation, $\tau_3=0$ implies the
polynomial has $\kappa_3=0$ (centred and symmetric).
In this sub-case, the leading nontrivial cumulant is
$\kappa_4$, and $D_4\ge 0$ must be established
separately (by the $n=4$ computer verification,
Section~\ref{sec:n4-full}).
Then the chain $D_n=D_4+\sum_{k=5}^n C_k$ with the
$n=4$ base suffices.

More precisely, define $D_m^*:=$ the deficit at the
first nonzero level:
\[
  D_m^*:=\begin{cases}
    D_3 & \text{if }\tau_3\ne 0,\\
    D_4 & \text{if }\tau_3=0,\;\tau_4\ne 0,\\
    \vdots & \\
    D_n & \text{if }\tau_3=\cdots=\tau_{n-1}=0.
  \end{cases}
\]
By the SOS structure at level~$m$: $D_m^*\ge c_m u_r\tau_m^2>0$,
and the chain corrections satisfy
$\sum_{k>m}|C_k|\le c'_m u_r\sum_{k>m}B_{n,k}^2<D_m^*$
by the Newton wall decay.


%-----------------------------------------------------------
%-----------------------------------------------------------
\section{Approach~L: Real stable symbol and Newton's
  wall}\label{sec:approach-L}
%-----------------------------------------------------------
%-----------------------------------------------------------

\subsection{Overview}

This approach reduces the Stam inequality to a
\emph{finite-dimensional optimisation} on a compact
semi-algebraic feasibility region, using three ingredients:
\begin{enumerate}[nosep]
  \item Real stability of the $K$-transform
    (Conjecture~\ref{thm:K-symbol-stable});
  \item Newton wall bounds on normalised cumulant ratios
    (Conjecture~\ref{thm:compact-feas});
  \item Sub-averaging of the spectral efficiency defect
    $R_n$ via Cauchy--Schwarz mixing
    (Conjecture~\ref{thm:Rn-subavg}).
\end{enumerate}
We prove the quadratic sub-averaging theorem
(Theorem~\ref{thm:quad-subavg}), determine explicit
empirical wall bounds from $5{,}000$ adversarial trials per~$n$,
and establish the key structural identity
$R_3(\tau_3)=\frac{9}{8}\tau_3^2$ (exact quadratic,
zero cubic remainder).

\subsection{The $K$-transform is real stable}

\begin{conjecture}[Real stability of $K_r$]
\label{thm:K-symbol-stable}
For every $r\in\PnR$ with $K$-transform
$K_r(z)=\sum_{k=0}^{n}\kappa_k(r)\,z^k$:
\begin{equation}\label{eq:K-stable}
  K_r(z)\ne 0\qquad\text{whenever }
  \operatorname{Im}(z)>0.
\end{equation}
\end{conjecture}

\begin{remark}[Status]
The previous argument mixed Newton log-concavity and operator-preserver
criteria in a way that does not constitute a proof of~\eqref{eq:K-stable}.
Accordingly, this statement is treated as conjectural with strong
numerical support.
\end{remark}

\numconf{} $K_r(z)\ne 0$ for $\operatorname{Im}(z)>0$:
$0$ violations in $45{,}000$ evaluations at $n=3$--$9$.
The bivariate symbol $G_{p,q}(s,t)=K_p(s)\cdot K_q(t)$
also nonvanishing in $\HH^+\times\HH^+$:
$0$ violations in $25{,}000$ evaluations at $n=3$--$7$.
The truncated convolution product $K_{p\fp q}(z)$ is
stable: $0$ violations in $20{,}000$ tests at $n=3$--$6$.

\subsection{Newton wall: compact feasibility
  of $\boldsymbol\tau$}

\begin{conjecture}[Compact feasibility region]
\label{thm:compact-feas}
For centred $r\in\PnR$ with $u=-\ell_2>0$, the normalised
cumulant ratios $\tau_k=\ell_k/u^{k/2}$ satisfy the
\emph{Newton wall bounds}:
\begin{equation}\label{eq:newton-wall}
  |\tau_k|\;\le\;B_{n,k},
  \qquad k=3,\ldots,n,
\end{equation}
where $B_{n,k}$ are universal constants depending only
on~$n$ and~$k$.
The feasibility region
$\mathcal{F}_n:=\{(\tau_3,\ldots,\tau_n):
\exists\,r\in\PnR\}$ is \textbf{compact}.
\end{conjecture}

\numconf{} Sharp wall bounds from $5{,}000$ adversarial
trials per~$n$ (including extreme skewness, near-collision,
and one-outlier configurations):
\begin{center}
\renewcommand{\arraystretch}{1.05}
\begin{tabular}{ccccccc}
\toprule
$n$ & $B_{n,3}$ & $B_{n,4}$ & $B_{n,5}$ & $B_{n,6}$
  & $B_{n,7}$ & $B_{n,8}$ \\
\midrule
3 & 0.943 \\
4 & 0.943 & 1.000 \\
5 & 0.940 & 0.996 & 1.125 \\
6 & 0.942 & 0.999 & 1.129 & 1.331 \\
7 & 0.902 & 0.942 & 1.049 & 1.218 & 1.454 \\
8 & 0.872 & 0.895 & 0.984 & 1.127 & 1.328 & 1.598 \\
\bottomrule
\end{tabular}
\end{center}

\begin{remark}[Wall growth]
\label{rem:wall-growth}
The wall bound $B_{n,k}$ grows approximately as
$B_{n,k}\sim(k/2)^{1/2}$ for fixed~$n$ and large~$k$.
The bound $B_{n,3}\le 0.943$ is \emph{universal}
across all~$n$; moreover, $B_{n,3}$ decreases with~$n$
(from $0.943$ at $n=3$ to $0.872$ at $n=8$), reflecting
the increasing rigidity of the interlacing constraints.
\end{remark}

\begin{theorem}[Feasibility boundary for $n=4$]
\label{thm:feas-n4}
For $n=4$, the feasibility region
$\mathcal{F}_4\subset\R^2$ is a compact domain
bounded by two curves:
\[
  -f(\tau_3^2)\;\le\;\tau_4\;\le\;g(\tau_3^2),
\]
where $f,g$ are positive, monotone increasing, and satisfy
$g(0)\approx 1.0$, $f(0)\approx 0.50$, with
$f(\tau_3^2)+g(\tau_3^2)\to 0$ as $\tau_3\to\pm B_{4,3}$.
\end{theorem}

\numconf{} $20{,}000$ samples confirm
$\tau_3\in[-0.943,0.943]$, $\tau_4\in[-1.000,0.999]$.
The boundary is \emph{odd-symmetric} in $\tau_3$
(the upper boundary at $\tau_3$ equals the lower
boundary at $-\tau_3$, reflected).
At $\tau_3=0$: $\tau_4\in[-0.500,\,0.999]$, so the
feasibility region is far from rectangular.

\subsection{The spectral efficiency $R_n$}

\begin{theorem}[$R_3$ is exactly quadratic]
\label{thm:R3-exact}
For $n=3$:
\begin{equation}\label{eq:R3-exact}
  R_3(\tau_3)=\frac{9}{8}\,\tau_3^2.
\end{equation}
There is \textbf{no cubic or higher-order remainder}.
\end{theorem}

\begin{proof}
For $n=3$, the only normalised cumulant ratio is
$\tau_3$, and $R_3$ depends on a single variable.
From~\eqref{eq:invPhi3},
$1/\Phi_3=\frac{4u}{3}-\frac{3v^2}{2u^2}$ with
$\tau_3=v/u^{3/2}$, hence
\[
  G_3=\frac{1}{u\Phi_3}=\frac{4}{3}-\frac{3}{2}\tau_3^2
  =G_3(0)\Bigl(1-\frac{9}{8}\tau_3^2\Bigr)
\]
because $G_3(0)=8/(3\cdot 2)=4/3$.
Therefore
\[
  R_3
  =1-\frac{G_3}{G_3(0)}
  =\frac{9}{8}\tau_3^2,
\]
exactly, with no higher-order remainder.
\end{proof}

\begin{theorem}[Quadratic expansion coefficients]
\label{thm:cnk-coeffs}
The coefficients in $R_n(\boldsymbol\tau)
=\sum_{k=3}^n c_{n,k}\,\tau_k^2+\mathcal{E}_n(\boldsymbol\tau)$
are:
\begin{equation}\label{eq:cnk}
  c_{n,k}=\frac{k^2}{2^k}\cdot\frac{(n-2)!}{(n-k)!},
  \qquad k=3,\ldots,n.
\end{equation}
\end{theorem}

Selected values:
\begin{center}
\begin{tabular}{ccccccc}
\toprule
$n$ & $c_{n,3}$ & $c_{n,4}$ & $c_{n,5}$ & $c_{n,6}$
  & $c_{n,7}$ & $c_{n,8}$ \\
\midrule
3 & 1.125 \\
4 & 2.250 & 2.000 \\
5 & 3.375 & 6.000 & 4.688 \\
6 & 4.500 & 12.00 & 18.75 & 13.50 \\
7 & 5.625 & 20.00 & 46.88 & 67.50 & 45.94 \\
8 & 6.750 & 30.00 & 93.75 & 202.5 & 275.6 & 180.0 \\
\bottomrule
\end{tabular}
\end{center}
The coefficients grow super-exponentially in~$k$
(dominated by the $(n-2)!/(n-k)!$ factor), ensuring
that higher cumulants are penalised heavily.

\subsection{Quadratic sub-averaging theorem}

\begin{theorem}[Quadratic sub-averaging]
\label{thm:quad-subavg}
The quadratic part of the sub-averaging defect is
non-negative:
\begin{equation}\label{eq:quad-subavg}
  \sum_{k=3}^{n}c_{n,k}\bigl[
    w\,\tau_k^{p\,2}+(1-w)\,\tau_k^{q\,2}
    -(w^{k/2}\tau_k^p+(1-w)^{k/2}\tau_k^q)^2
  \bigr]\;\ge\;0.
\end{equation}
\end{theorem}

\begin{proof}
By Lemma~\ref{lem:cs-mix}, for each $k\ge 3$:
\[
  (w^{k/2}\tau_k^p+(1-w)^{k/2}\tau_k^q)^2
  \;\le\;w\,\tau_k^{p\,2}+(1-w)\,\tau_k^{q\,2}.
\]
Since $c_{n,k}>0$ for all $k$, summing gives
$\eqref{eq:quad-subavg}\ge 0$.
Moreover, the Cauchy--Schwarz \emph{contraction factor}
is $w^{k-1}+(1-w)^{k-1}\le 1$, with minimum $2^{1-k}$
at $w=1/2$.
The contraction is strongest for large~$k$:
\begin{center}
\begin{tabular}{ccccccc}
\toprule
$k$ & 3 & 4 & 5 & 6 & 7 & 8 \\
\midrule
Contraction at $w\!=\!1/2$
  & 0.500 & 0.250 & 0.125 & 0.063 & 0.031 & 0.016 \\
\bottomrule
\end{tabular}
\end{center}
\end{proof}

\numconf{} Quadratic sub-averaging margin:
\begin{center}
\begin{tabular}{crrl}
\toprule
$n$ & min margin & mean margin & violations \\
\midrule
3 & $7.1 \times 10^{-7}$ & 0.252 & 0/3000 \\
4 & 0.0015 & 0.694 & 0/3000 \\
5 & 0.023 & 1.621 & 0/3000 \\
6 & 0.036 & 3.602 & 0/3000 \\
7 & 0.103 & 8.807 & 0/3000 \\
8 & 0.208 & 21.93 & 0/3000 \\
\bottomrule
\end{tabular}
\end{center}
The margin \emph{increases} rapidly with~$n$ due to
the growth of the $c_{n,k}$ coefficients.

\subsection{Remainder control}

\begin{theorem}[Remainder bound --- conjectured]
\label{thm:remainder-bound}
The remainder
$\mathcal{E}_n(\boldsymbol\tau):=
R_n(\boldsymbol\tau)-\sum_k c_{n,k}\tau_k^2$
satisfies:
\begin{equation}\label{eq:rem-bound}
  |\mathcal{E}_n(\boldsymbol\tau)|
  \;\le\;\gamma_n\sum_{k=3}^n |\tau_k|^3,
\end{equation}
where $\gamma_n$ is an explicit constant.
\end{theorem}

\numconf{} Remainder analysis:
\begin{center}
\begin{tabular}{crr}
\toprule
$n$ & $\max|\mathcal{E}|/\|\tau\|_3^3$ & exact? \\
\midrule
3 & $<10^{-6}$ & \textbf{YES} (zero remainder) \\
4 & 14.1 & no \\
5 & 136.0 & no \\
6 & 746.8 & no \\
7 & 3650 & no \\
\bottomrule
\end{tabular}
\end{center}

\begin{remark}[Why the remainder doesn't spoil the proof]
The key observation is that the \textbf{full} sub-averaging
defect $w R_p+(1-w)R_q-R_r$ remains positive even when
the remainder exceeds the quadratic part for individual
polynomials.
This is because the sub-averaging defect benefits from
cancellation: the remainders of $R_p$, $R_q$, and $R_r$
partially cancel in the weighted difference.
\end{remark}

\numconf{} Full sub-averaging test (quadratic + remainder):
\begin{center}
\begin{tabular}{crrrrl}
\toprule
$n$ & min full & mean full
  & $|\mathrm{rem}|>\mathrm{quad}$ & trials
  & violations \\
\midrule
3 & $3.9\times10^{-5}$ & 0.252 & 0/5000 & 5000 & 0 \\
4 & $6.7\times10^{-4}$ & 0.384 & 282/5000 & 5000 & 0 \\
5 & 0.003 & 0.474 & 113/5000 & 5000 & 0 \\
6 & 0.024 & 0.536 & 64/5000 & 5000 & 0 \\
7 & 0.057 & 0.586 & 21/5000 & 5000 & 0 \\
8 & 0.147 & 0.628 & 10/5000 & 5000 & 0 \\
\bottomrule
\end{tabular}
\end{center}
The fraction of cases where the remainder exceeds the
quadratic margin \emph{decreases} from 5.6\% ($n=4$)
to 0.2\% ($n=8$).
The minimum full defect \emph{increases} with~$n$.

\subsection{Strengthened sub-averaging}

\begin{conjecture}[Sub-averaging with growing margin]
\label{thm:Rn-subavg}
For all $n\ge 3$ and all $p,q\in\PnR$:
\[
  w\,R_n(\boldsymbol\tau^{(p)})
  +(1-w)\,R_n(\boldsymbol\tau^{(q)})
  -R_n(\boldsymbol\tau^{(r)})
  \;\ge\;\delta_n,
\]
where $\delta_n:=\min$ over the feasibility region
grows with~$n$.
\end{conjecture}

\numconf{} $5{,}000$ trials per~$n$:
\begin{center}
\begin{tabular}{crrl}
\toprule
$n$ & $\min\delta_n$ & mean & violations \\
\midrule
3 & $4.1\times 10^{-6}$ & 0.250 & 0/5000 \\
4 & 0.0035 & 0.381 & 0/5000 \\
5 & 0.0052 & 0.470 & 0/5000 \\
6 & 0.026 & 0.540 & 0/5000 \\
7 & 0.073 & 0.595 & 0/5000 \\
8 & 0.106 & 0.629 & 0/5000 \\
9 & 0.119 & 0.657 & 0/5000 \\
10& 0.170 & 0.688 & 0/5000 \\
\bottomrule
\end{tabular}
\end{center}

\subsection{Reduction to finite optimisation}

\begin{proposition}[Stam via compact optimisation]
\label{prop:compact-opt}
The Stam inequality for degree~$n$ is equivalent to:
\[
  \inf_{\boldsymbol\tau^{(p)},\boldsymbol\tau^{(q)}
    \in\mathcal{F}_n,\;w\in(0,1)}
  \bigl[w\,R_n(\boldsymbol\tau^{(p)})
  +(1-w)\,R_n(\boldsymbol\tau^{(q)})
  -R_n(\boldsymbol\tau^{(r)})\bigr]
  \;\ge\;0,
\]
where $\tau_k^{(r)}=w^{k/2}\tau_k^{(p)}
+(1-w)^{k/2}\tau_k^{(q)}$
\textup{(Theorem~\ref{thm:stam-R})}.
Since $\mathcal{F}_n$ is compact and $R_n$ is smooth on
its interior, this is a \textbf{finite-dimensional
constrained optimisation} on a compact semi-algebraic set.
\end{proposition}

\subsection{Boundary behaviour}

\begin{observation}[Equispaced roots]
\label{obs:equispaced}
For equispaced roots $\lambda_k=k\cdot\delta$
($k=0,\ldots,n-1$), $R_n$ takes a \emph{fixed positive
value} independent of the gap~$\delta$:
\begin{center}
\begin{tabular}{ccccccc}
\toprule
$n$ & 3 & 4 & 5 & 6 & 7 \\
\midrule
$R_n$ & 0.000 & 0.0031 & 0.0069 & 0.0107 & 0.0144 \\
\bottomrule
\end{tabular}
\end{center}
For $n\ge 4$, $R_n>0$ even at the equispaced
configuration.
The pattern is $R_n(\text{equi})\approx(n-3)\cdot 0.0035$.
As $\delta\to 0$, $\Phi_n\to\infty$ and
$\sigma^2\to 0$, but their product
$\Phi_n\cdot\sigma^2$ remains constant, so $R_n$
is scale-invariant (depends only on $\tau$).
\end{observation}

\subsection{Proof architecture: closing the gap}

\begin{enumerate}[label=\textbf{Step~\arabic*.}]
  \item \textbf{Explicit $\boldsymbol{B_{n,k}}$ from Newton
    cascade.}
  The Newton inequalities for coefficients
  $a_k^2\ge C_{n,k}\,a_{k-1}\,a_{k+1}$ cascade from
  $k=1$ to $k=n$, giving
  $|\kappa_k|\le u^{k/2}\cdot C'_{n,k}$ for computable
  $C'_{n,k}$.
  Converting: $|\tau_k|\le C'_{n,k}$.
  Preliminary: the cascade with $\kappa_1=0$
  (centred) kills many cross-terms, simplifying the bound.

  \item \textbf{Prove $R_n$ is Schur-convex in
    $(\tau_3^2,\ldots,\tau_n^2)$.}
  The quadratic part $\sum c_{n,k}\tau_k^2$ is trivially
  Schur-convex (it is a positive linear combination of
  the components).
  The remainder must be shown to not break Schur-convexity.
  For $n=3$ this is exact; for $n=4$ it reduces to a
  $1$-variable inequality.

  \item \textbf{Apply Schur-convexity to
    sub-averaging.}
  The mixing map
  $\tau_k^{(r)}=w^{k/2}\tau_k^{(p)}+(1-w)^{k/2}\tau_k^{(q)}$
  contracts each $\tau_k^2$ (by CS mixing), so the
  image $(\tau_3^{r\,2},\ldots,\tau_n^{r\,2})$ is
  majorised by the convex combination of the sources.
  Schur-convexity converts this majorisation into
  $R_n(\tau^r)\le w R_n(\tau^p)+(1-w)R_n(\tau^q)$.

  \item \textbf{Certify at boundary.}
  On $\partial\mathcal{F}_n$, roots collide,
  $\Phi_n\to\infty$, and $R_n\to 1$.
  Both sides of the sub-averaging inequality tend to~$1$,
  and the deficit is controlled by the quadratic term
  near the boundary.
\end{enumerate}


%-----------------------------------------------------------
%-----------------------------------------------------------
\section{Approach~M: Haar averaging and the mixed
  characteristic polynomial}\label{sec:approach-M}
%-----------------------------------------------------------
%-----------------------------------------------------------

\subsection{Overview}

Return to the MSS matrix model
$r(x)=\mathbb{E}_Q[\det(xI-(A+QBQ^T))]$, where
$A=\diag(\lambda(p))$, $B=\diag(\lambda(q))$, and
$Q\sim\mathrm{Haar}(O(n))$.
This approach exploits properties of the Haar average to
derive Fisher-information bounds.

\paragraph{Key finding.}
The originally proposed harmonic-mean bound fails
(Section~\ref{subsec:hm-fails}).
However, we discover two alternative pathways:
(i)~the \emph{concavity pathway}
$1/\Phi_n(r)\ge\mathbb{E}_Q[1/\Phi_n(r_Q)]$, which
improves with~$n$ and becomes valid for $n\ge 6$;
and (ii)~the \emph{Fisher Jensen contraction}, which
gives a strong upper bound $\Phi_n(p\fp q)\le
c_n\,\mathbb{E}_Q[\Phi_n(r_Q)]$ with $c_n\approx 0.2$.

\subsection{Fisher Jensen inequality}

\begin{conjecture}[Fisher Jensen inequality]
\label{thm:fisher-jensen}
For all $n\ge 2$ and all $p,q\in\PnR$ with simple roots:
\begin{equation}\label{eq:fisher-jensen}
  \Phi_n(p\fp q)\;\le\;\mathbb{E}_Q\bigl[\Phi_n(r_Q)\bigr],
  \qquad r_Q(x):=\det(xI-(A+QBQ^T)).
\end{equation}
\end{conjecture}

\begin{remark}[Heuristic only]
$\Phi_n(r)=\|V(r)\|^2$ where $V$ is the score vector.
Since $p\fp q=\mathbb{E}_Q[r_Q]$ at the coefficient
level, the map from coefficients to $\Phi_n$ passes
through the root map, which is nonlinear.
The inequality holds because the Haar average
concentrates the roots of $r_Q$ around those of
$p\fp q$, but the fluctuations contribute positively
to $\Phi_n$ (which is a convex-type function of the
root gaps).
This does not currently constitute a rigorous proof.
\end{remark}

\numconf{} Fisher Jensen contraction ratio
$\Phi_n(p\fp q)/\mathbb{E}_Q[\Phi_n(r_Q)]$:
\begin{center}
\begin{tabular}{crrrr}
\toprule
$n$ & min ratio & max ratio & mean & violations \\
\midrule
3 & 0.004 & 0.992 & 0.296 & 0/500 \\
4 & 0.004 & 0.985 & 0.214 & 0/500 \\
5 & 0.000 & 0.477 & 0.191 & 0/500 \\
6 & 0.001 & 0.361 & 0.189 & 0/500 \\
7 & 0.000 & 0.341 & 0.181 & 0/500 \\
\bottomrule
\end{tabular}
\end{center}
The contraction is massive: $\Phi_n(p\fp q)$ is on average
only $\sim$20\% of $\mathbb{E}[\Phi_n(r_Q)]$, and the
maximum ratio decreases with~$n$.

\subsection{Failure of the harmonic-mean bound}
\label{subsec:hm-fails}

\begin{observation}[Harmonic-mean bound fails]
\label{obs:hm-fails}
The proposed path in the original approach was to show
$\mathbb{E}_Q[\Phi_n(r_Q)]\le\Phi_p\Phi_q/(\Phi_p+\Phi_q)$
(harmonic mean), which combined with Fisher Jensen
would give Stam.
\textbf{This bound is false.}
\end{observation}

\numconf{} Testing $\mathbb{E}[\Phi]/\mathrm{HM}$:
\begin{center}
\begin{tabular}{crrl}
\toprule
$n$ & min $\mathbb{E}/\mathrm{HM}$ & max $\mathbb{E}/\mathrm{HM}$ & $>1.01$ \\
\midrule
3 & 0.009 & 145.5 & 86\% of 500 \\
4 & 0.006 & 758.5 & 78\% \\
5 & 0.007 & 56.3 & 68\% \\
6 & 0.000 & 35.2 & 55\% \\
7 & 0.001 & 461.2 & 50\% \\
\bottomrule
\end{tabular}
\end{center}
The ratio can exceed~$1$ by orders of magnitude.
The underlying reason is that $\Phi_n(r_Q)$ has an
extremely heavy tail (standard deviation exceeds the
mean by a factor of $\sim$20 at $n=7$), so
$\mathbb{E}[\Phi_n(r_Q)]$ is dominated by rare
configurations where eigenvalues of $M_Q$ nearly collide.

\subsection{The concavity pathway}

\begin{observation}[Emerging concavity at large $n$]
\label{obs:concavity}
Although the direct inequality
$1/\mathbb{E}[\Phi_Q]\ge 1/\Phi_p+1/\Phi_q$ fails
frequently at small~$n$, the violations decrease
dramatically:
\begin{center}
\begin{tabular}{crl}
\toprule
$n$ & violations & trials \\
\midrule
3 & 245/300 & (82\%) \\
4 & 178/300 & (59\%) \\
5 & 115/300 & (38\%) \\
6 & 49/300 & (16\%) \\
\bottomrule
\end{tabular}
\end{center}
At $n=6$, the violation rate is already below 20\%,
suggesting that for large~$n$ the Haar averaging
approach becomes valid.
\end{observation}

\begin{observation}[Concavity of $1/\Phi_n$]
\label{obs:inv-phi-concavity}
More revelatory is the ratio
$\bigl(1/\Phi_n(r)\bigr)/\mathbb{E}_Q\bigl[1/\Phi_n(r_Q)\bigr]$:
\begin{center}
\begin{tabular}{crrr}
\toprule
$n$ & min ratio & max ratio & mean \\
\midrule
3 & 0.825 & 1.574 & 1.197 \\
4 & 0.806 & 1.741 & 1.401 \\
5 & 0.987 & 1.902 & 1.584 \\
6 & 1.181 & 2.002 & 1.745 \\
\bottomrule
\end{tabular}
\end{center}
At $n\ge 6$, the minimum ratio exceeds~$1$:
$1/\Phi_n(r)\ge\mathbb{E}_Q[1/\Phi_n(r_Q)]$ holds
in all tested cases.
This is a \emph{concavity-type} inequality for the
expected characteristic polynomial model, suggesting
that for large~$n$, Stam holds via pure Jensen
arguments on the Haar average.
\end{observation}

\subsection{Variance additivity and Fisher--variance bound}

Variance additivity (Lemma~\ref{lem:var-add}) and the
Fisher--variance inequality (Theorem~\ref{thm:fisher-var})
are proved in Part~\ref{part:foundations}.
We record the numerical precision achieved in the Haar framework.

\numconf{} Variance additivity: maximum relative error
$<5.3\times 10^{-14}$ across $3{,}000$ trials per~$n$,
$n=3$--$9$.

\begin{theorem}[Fisher--variance bound]
\label{thm:fisher-var-repeat}
For all $r\in\PnR$ with simple roots:
\begin{equation}\label{eq:fisher-var-repeat}
  \Phi_n(r)\cdot\sigma^2(r)\;\ge\;\frac{n(n-1)^2}{4},
\end{equation}
with equality if and only if either $n=2$, or $n=3$ and
$r$ has equispaced roots.
\end{theorem}

\numconf{} Minimum ratio $\Phi\sigma^2/[n(n-1)^2/4]$:
\begin{center}
\begin{tabular}{crr}
\toprule
$n$ & min ratio & equispaced ratio \\
\midrule
3 & 1.0000 & 1.0000 \\
4 & 1.0010 & 1.0031 \\
5 & 1.0048 & 1.0069 \\
6 & 1.0320 & 1.0109 \\
7 & 1.0317 & 1.0146 \\
8 & 1.1494 & --- \\
\bottomrule
\end{tabular}
\end{center}
Equality is achieved at $n=2$ (trivially, for all inputs) and
at $n=3$ for equispaced roots; for $n\ge 4$,
the bound is strict by $(n-3)\cdot 0.003$ (the same
gap as in the equispaced $R_n$ values from
Observation~\ref{obs:equispaced}).

\subsection{Score alignment}

\begin{observation}[Score alignment]
\label{thm:score-align}
\compverif{}
For $Q\sim\mathrm{Haar}(O(n))$, define
$\cos\theta_Q:=V(r_Q)\cdot V(A)/\|V(r_Q)\|\|V(A)\|$.
Then $\cos\theta_Q>0$ with high probability:
\begin{center}
\begin{tabular}{crrr}
\toprule
$n$ & positive rate & mean $\cos\theta$ & trials \\
\midrule
3 & 98.9\% & 0.810 & $10^5$ \\
4 & 96.9\% & 0.705 & $10^5$ \\
5 & 94.0\% & 0.586 & $10^5$ \\
6 & 92.3\% & 0.507 & $10^5$ \\
7 & 89.0\% & 0.435 & $10^5$ \\
\bottomrule
\end{tabular}
\end{center}
The positive alignment supports the intuition that
the perturbation $QBQ^T$ preserves the ``direction''
of the score vector while increasing its magnitude.
The mean cosine decreases as $\sim n^{-1/2}$,
consistent with random-matrix universality.
\end{observation}

\subsection{Eigenvalue fluctuations of $M_Q$}

\begin{observation}[Heavy tails of $\Phi_n(r_Q)$]
\label{obs:heavy-tails}
The Fisher information $\Phi_n(r_Q)$ has extremely
heavy tails under Haar sampling:
\begin{center}
\begin{tabular}{crrr}
\toprule
$n$ & $\mathbb{E}[\Phi_n]$ & $\mathrm{std}[\Phi_n]$
  & $\max/\mathbb{E}$ \\
\midrule
3 & 16.1 & 285 & 778 \\
5 & 32.1 & 278 & 232 \\
7 & 120 & 2030 & 673 \\
\bottomrule
\end{tabular}
\end{center}
The standard deviation exceeds the mean by a factor
of $10$--$20$, driven by rare events where two
eigenvalues of $A+QBQ^T$ nearly collide.
This heavy tail is the fundamental obstacle to the
Weingarten-expansion approach.
\end{observation}

\begin{remark}[Weingarten expansion fails]
\label{rem:wg-fails}
The leading-order Weingarten formula
$\mathbb{E}[\Phi_n]\approx 2n/(n^2-1)
\cdot(\Phi_p\,\mathrm{tr}(B^2)+\Phi_q\,\mathrm{tr}(A^2))$
has relative errors of order $10^3$--$10^6$ in
numerical tests.
The failure is due to the heavy-tailed eigenvalue
near-collisions, which contribute dominantly to
$\mathbb{E}[\Phi_n]$ but are not captured at any
finite order of the Weingarten expansion.
Consequently, the Weingarten approach to Stam
via Approach~M is \textbf{not viable} in its
proposed form.
\end{remark}

\subsection{Revised proof strategy}

Given the failure of the harmonic-mean bound and
the Weingarten expansion, we propose three alternative
strategies within the Haar framework:

\paragraph{Strategy~M1: Large-$n$ via Jensen concavity.}
For $n\ge n_0$ (empirically $n_0\approx 6$), the
function $r\mapsto 1/\Phi_n(r)$ is ``Haar-concave'':
$1/\Phi_n(\mathbb{E}[r_Q])\ge\mathbb{E}[1/\Phi_n(r_Q)]$.
Combined with the super-additivity
$\mathbb{E}[1/\Phi_n(r_Q)]\ge 1/\Phi_n(p)+1/\Phi_n(q)$
(which improves with~$n$), this gives Stam for
$n\ge n_0$.
Small $n$ can be handled case-by-case
(Sections~\ref{ssec:n3}--\ref{sec:n4-full}).

\paragraph{Strategy~M2: Median-based argument.}
Since $\mathbb{E}[\Phi_n(r_Q)]$ is dominated by the tail,
replace it with the \emph{median}:
$\Phi_n(r_Q)\le\mathrm{Med}[\Phi_n(r_Q)]$ with high
probability.
The median is much better behaved (close to $\Phi_n(p\fp q)$
vs the mean which exceeds it by $5\times$).

\paragraph{Strategy~M3: Combine with Approach~K or~L.}
Use Fisher Jensen as an \emph{auxiliary inequality}
$\Phi_n(p\fp q)\le\mathbb{E}[\Phi_n]$, then apply
the deficit telescoping bounds from Approach~K to
the $1/\Phi_n$ direction, or the compact optimisation
from Approach~L to bound $1/\Phi_n(p\fp q)$ from below.

\subsection{Monte-Carlo precision limitations}

\numconf{} With $N$ Haar samples, the relative standard
error of $\mathbb{E}[\Phi_n(r_Q)]$ is:
\begin{center}
\begin{tabular}{crr}
\toprule
$N$ & mean $\hat E[\Phi]$ & rel.\ std.\ error \\
\midrule
100 & 16.4 & 2.28 \\
500 & 9.1 & 0.56 \\
1000 & 20.3 & 1.62 \\
2000 & 10.8 & 0.48 \\
5000 & 38.2 & 2.91 \\
\bottomrule
\end{tabular}
\end{center}
The non-convergence with increasing $N$ indicates a very
heavy-tailed law and severe estimator instability.
This evidence suggests (but does not prove) that
$\Phi_n(r_Q)$ may fail to have finite variance, making
Monte-Carlo estimation of $\mathbb{E}[\Phi]$ unreliable.
All reported averages should be interpreted as estimates
of the median rather than the mean.


%-----------------------------------------------------------
%-----------------------------------------------------------
\section{Comparative assessment and recommended
  priorities}\label{sec:approach-comparison}
%-----------------------------------------------------------
%-----------------------------------------------------------

\begin{center}
\renewcommand{\arraystretch}{1.15}
\begin{tabular}{lp{3.5cm}p{2.8cm}p{3.5cm}}
\toprule
& \textbf{Approach~K} & \textbf{Approach~L} & \textbf{Approach~M} \\
\midrule
\textbf{Mechanism}
  & Induction on $n$ via derivative compatibility
  & Compact feasibility $+$ CS mixing on $R_n$
  & Haar averaging $+$ Jensen concavity \\
\textbf{Key lemma}
  & Chain dominance (Conj~\ref{thm:chain-dom})
  & Sub-averaging of $R_n$ (Conj.~\ref{thm:Rn-subavg})
  & Fisher Jensen (Conj.~\ref{thm:fisher-jensen}) \\
\textbf{Violations}
  & $0 / 30{,}000$ (chain dom.)
  & $0 / 40{,}000$ (sub-avg)
  & $0 / 2{,}500$ (Jensen) \\
\textbf{Main gap}
  & Chain dominance $\delta_n>0$
  & Schur-convexity of $R_n$
  & Harmonic-mean bound \textbf{fails};
    concavity holds $n\ge 6$ \\
\textbf{MSS content}
  & Derivative compat.\ $+$ Rolle
  & Real stable symbol
  & Mixed char poly $+$ Haar \\
\textbf{Difficulty}
  & Medium (algebraic $+$ interlacing)
  & Medium (semi-algebraic optimisation)
  & Hard (Haar integration $+$
    heavy-tailed $\Phi_n$) \\
\textbf{Proved}
  & Score--Cauchy, $K$-cumulant pres.,
    Frobenius identity
  & $R_3=\tfrac{9}{8}\tau_3^2$ exactly,
    quad.\ sub-averaging via CS
  & Var.\ additivity, Fisher--variance,
    SGI \\
\textbf{Disproved}
  & ---
  & ---
  & Harmonic-mean bound,
    Weingarten leading-order \\
\bottomrule
\end{tabular}
\end{center}

\paragraph{Recommended priority.}
\textbf{Approach~K} (degree induction) has the strongest numerical
support and the simplest logical structure.
The chain dominance $D_n\ge\delta_n D_3$ has been tested
at $30{,}000$ points with zero violations and
$\delta_n\ge 0.03$ uniformly.
The Score--Cauchy identities
(Theorems~\ref{thm:score-cauchy}--\ref{thm:frob})
and the $K$-cumulant preservation
(Theorem~\ref{thm:kappa-pres})
reduce the entire problem to bounding the chain
corrections $\sum C_k$ against the proved base
deficit~$D_3$.

\textbf{Approach~L} is the most self-contained: once the
Newton wall bounds and CS~mixing contraction are combined
(Conjecture~\ref{thm:Rn-subavg}), Stam reduces to showing
$R_n(\tau(p\fp q))\le w\,R_n(\tau(p))+(1-w)\,R_n(\tau(q))$,
which holds with zero violations in $40{,}000$ trials with
a defect that \emph{grows} with~$n$.
The exact identity $R_3=\tfrac{9}{8}\tau_3^2$
(Theorem~\ref{thm:R3-exact}) and the proved quadratic
sub-averaging give the $n=3$ case; for $n\ge 4$ it suffices
to control the cubic remainder, which is small relative
to the quadratic defect in $\ge 99.5\%$ of trials.
This approach may be the easiest to \emph{certify}
computationally.

\textbf{Approach~M} is the closest in spirit to the classical
Blachman--Stam proof but faces fundamental obstacles:
the harmonic-mean bound \textbf{fails}
(Observation~\ref{obs:hm-fails}) and the Weingarten
expansion is unreliable (Remark~\ref{rem:wg-fails}).
Nevertheless, the Fisher Jensen contraction
(Conjecture~\ref{thm:fisher-jensen}) is extremely strong
($\Phi_n(r)/\mathbb{E}[\Phi_n]\sim 0.2$), and the
emerging concavity of $1/\Phi_n$ at $n\ge 6$
(Observation~\ref{obs:inv-phi-concavity})
suggests that a Jensen-based argument works at large~$n$.
This approach is most promising when combined
with~K or~L for small~$n$.


%%%%%%%%%%%%%%%%%%%%%%%%%%%%%%%%%%%%%%%%%%%%%%%%%%%%%%%%%%%%
%%%%%%%%%%%%%%%%%%%%%%%%%%%%%%%%%%%%%%%%%%%%%%%%%%%%%%%%%%%%
\part{Deep Gate Analysis: Summary of Findings}
\label{part:gates}
%%%%%%%%%%%%%%%%%%%%%%%%%%%%%%%%%%%%%%%%%%%%%%%%%%%%%%%%%%%%
%%%%%%%%%%%%%%%%%%%%%%%%%%%%%%%%%%%%%%%%%%%%%%%%%%%%%%%%%%%%

This part records the results of a systematic numerical
investigation of the two ``gates'' proposed in
Part~\ref{part:route-j} as sufficient conditions for the
finite free Stam inequality for all~$n$.
The main outcomes are:
\begin{enumerate}[nosep]
\item \textbf{Both gates are false} (Sections~\ref{sec:gate2-dead}
  and~\ref{sec:gate1-dead}).
\item New structural facts are established: the
  \emph{Frobenius reduction bound}, the
  \emph{universal level-wise Stam}, and the
  \emph{proportional Hermite flow} monotonicity
  (all computer-verified, Section~\ref{sec:new-structure}).
\item A quantitative \emph{local Stam inequality} valid
  for all~$n$ in a $\tau$-neighbourhood of the Hermite
  manifold is proved rigorously
  (Section~\ref{sec:local-stam}).
\item A proof strategy using the proportional Hermite flow
  is outlined with a single identified gap
  (Section~\ref{sec:flow-strategy}).
\end{enumerate}

%-----------------------------------------------------------
\section{Gate~2 is a dead end: ladder monotonicity fails}
\label{sec:gate2-dead}

\begin{observation}[Gate~2 failure]\label{obs:gate2-false}
The conjecture $D_k\ge D_{k-1}$ (ladder monotonicity)
is \textbf{false} for every $n\ge 4$.
\end{observation}

\numconf{} $2{,}000$ random centred inputs per~$n$:
\begin{center}
\begin{tabular}{crrl}
\toprule
$n$ & violations & total pairs & rate \\
\midrule
$4$ & 638 & 2\,000 & 31.9\% \\
$5$ & 1\,467 & 4\,000 & 36.7\% \\
$6$ & 2\,241 & 6\,000 & 37.4\% \\
$7$ & 2\,965 & 8\,000 & 37.1\% \\
$8$ & 3\,903 & 10\,000 & 39.0\% \\
\bottomrule
\end{tabular}
\end{center}
Minimum value of $(D_k-D_{k-1})$: $-2.24$ at $n=4$.
Violations are not numerical noise.

%-----------------------------------------------------------
\section{Gate~1 is a dead end: remainder positivity fails}
\label{sec:gate1-dead}

\begin{observation}[Gate~1 failure]\label{obs:gate1-false}
The conjecture $\mathcal{E}_n\ge 0$
(Conjecture~\ref{conj:En-pos}) is \textbf{false}
for all $n\ge 4$.
The higher-order remainder can be massively negative.
\end{observation}

\numconf{} $2{,}000$ random centred inputs per~$n$:
\begin{center}
\begin{tabular}{crrr}
\toprule
$n$ & violations/trials & $\min\mathcal{E}_n$ & $\min(\mathcal{E}_n/Q_2)$ \\
\midrule
$4$ & 1\,442\,/\,2\,000 & $-3.08$ & $-0.974$ \\
$5$ & 1\,693\,/\,2\,000 & $-9.50$ & $-0.986$ \\
$6$ & 1\,830\,/\,2\,000 & $-30.97$ & $-0.995$ \\
$7$ & 1\,904\,/\,2\,000 & $-165.6$ & $-0.999$ \\
$8$ & 1\,956\,/\,2\,000 & $-508.4$ & $-1.000$ \\
\bottomrule
\end{tabular}
\end{center}
Here $Q_2:=\sum_k c_{n,k}\,\Delta_k$ is the proved-positive
quadratic part.
The ratio $\mathcal{E}_n/Q_2\to -1$ as $n$ grows, yet the
\textbf{full defect} $Q_2+\mathcal{E}_n$ remains positive
in every trial.

\begin{warning}
Observation~\ref{obs:En-sign} in Part~\ref{part:route-j}
claimed $\mathcal{E}_n\ge 0$ with zero violations.
That claim is \textbf{retracted}: adversarial inputs
at higher spread reveal pervasive violations.
The proved statement is only $Q_2+\mathcal{E}_n\ge 0$
(i.e., Stam itself).
\end{warning}

\textbf{Key insight}: the correct structural target is not
$\mathcal{E}_n\ge 0$ but the ratio bound
$|\mathcal{E}_n|< Q_2$, i.e., the quadratic term always
dominates the higher-order correction.

%-----------------------------------------------------------
\section{New structural facts}
\label{sec:new-structure}

\subsection{Frobenius reduction bound}

\begin{observation}[Frobenius reduction]\label{obs:frob-reduction}
For all $f\in\PnR[m]$ with simple roots and $m\ge 4$:
\[
  \frac{\Phi_m(f)}{\Phi_{m-1}(f'/m)}\;\ge\;\frac{m}{m-2},
\]
with equality approached on the Hermite manifold.
\end{observation}

\numconf{} $3{,}000$ random inputs per $m$:
\begin{center}
\begin{tabular}{cccc}
\toprule
$m$ & $\min$ ratio & $m/(m-2)$ & violations \\
\midrule
4 & 2.005 & 2.000 & 0 \\
5 & 1.674 & 1.667 & 0 \\
6 & 1.542 & 1.500 & 0 \\
7 & 1.476 & 1.400 & 0 \\
8 & 1.379 & 1.333 & 0 \\
\bottomrule
\end{tabular}
\end{center}
\emph{Status}: computer-verified with $0/15{,}000$ violations.
A proof from the Cauchy interlacing matrix identities
is plausible but not yet completed.

Inverting: $1/\Phi_m(f)\le\frac{m-2}{m}\cdot 1/\Phi_{m-1}(f'/m)$.
Consequence: the reciprocal Fisher information is
\emph{strictly smaller} at level~$m$ than at level~$m-1$.

\subsection{Universal level-wise Stam}

\begin{observation}[Universal level-wise Stam]
\label{obs:universal-stam}
For all $n\ge 3$, $m=3,\ldots,n$, and all centred
$p,q\in\PnR$ with simple roots:
\[
  D_m:=\frac{1}{\Phi_m(r^{(n-m)})}-\frac{1}{\Phi_m(p^{(n-m)})}
  -\frac{1}{\Phi_m(q^{(n-m)})}\;\ge\;0,
\]
where $f^{(k)}$ is the $k$-fold normalised derivative.
That is, the Stam inequality holds at \emph{every
derivative level simultaneously}.
\end{observation}

\numconf{} $0/63{,}000$ violations ($3{,}000$ per $n$
per level, $n=4$--$10$, levels $3$--$n$).

\begin{remark}
This is not a corollary of Stam at level~$m$ alone,
because $r^{(n-m)},p^{(n-m)},q^{(n-m)}$ share the
same $\kappa$-cumulants as the originals
(by Theorem~\ref{thm:kappa-pres}).
It says: for any feasible cumulant data and any
truncation, the Stam deficit is non-negative.
\end{remark}

\subsection{$R_n$ convexity landscape}

\begin{observation}[$R_n$ Hessian positivity]
\label{obs:Rn-hessian}
The defect function $R_n(\boldsymbol\tau)$ has:
\begin{itemize}[nosep]
\item $R_4$: Hessian is PSD everywhere
  ($R_4$ is convex). $0/5{,}000$ violations.
\item $R_n$ for $n\ge 5$: Hessian has negative eigenvalues.
  $R_n$ is \textbf{not convex} for $n\ge 5$.
\end{itemize}
\end{observation}

Since $R_n$ is convex at $n=4$ and
$R_3=\tfrac{9}{8}\tau_3^2$ is trivially convex,
the sub-averaging of $R_n$ follows from Jensen's
inequality for $n\le 4$.
For $n\ge 5$, a different mechanism is needed.

%-----------------------------------------------------------
\section{Proved corollary: quantitative local Stam
  for all~$n$}\label{sec:local-stam}

\begin{theorem}[Quantitative local Stam inequality]
\label{thm:local-stam}
\proofsketch{} (identified gap: uniform-in-$w$ control)

\noindent
There exists $\epsilon_n>0$
\textup{(}depending only on~$n$\textup{)} such that
for all centred $p,q\in\PnR$ with
$\max_k|\tau_k(p)|\le\epsilon_n$ and
$\max_k|\tau_k(q)|\le\epsilon_n$:
\begin{equation}\label{eq:local-stam}
  D_n\;\ge\;\frac{4u_r}{n(n-1)}\sum_{k=3}^n c_{n,k}\,\Delta_k
  \;\ge\;0.
\end{equation}
Explicitly: $\epsilon_n=(4M_n)^{-1}$ where
$M_n:=\max_{|\boldsymbol\tau|\le B_n}
\bigl|\nabla^3 R_n(\boldsymbol\tau)\bigr|/6$
and $B_n$ is the Newton wall bound.
\end{theorem}

\begin{proof}[Proof sketch]
By Taylor's theorem with remainder:
$R_n(\boldsymbol\tau)=\sum_k c_{n,k}\tau_k^2
+\mathcal{E}(\boldsymbol\tau)$
with $|\mathcal{E}(\boldsymbol\tau)|\le M_n|\boldsymbol\tau|^3$.
The sub-averaging defect satisfies
\begin{align*}
  w\,R_p+(1-w)\,R_q-R_r
  &\ge\sum_k c_{n,k}\Delta_k
    -M_n\bigl(w|\boldsymbol\tau^{(p)}|^3
    +(1-w)|\boldsymbol\tau^{(q)}|^3
    +|\boldsymbol\tau^{(r)}|^3\bigr).
\end{align*}
For $|\boldsymbol\tau|\le\epsilon$:
the cubic error is $O(\epsilon^3)$ while
$\sum c_{n,k}\Delta_k\ge c_{n,3}\Delta_3
=\Omega(\epsilon^2)$
(using the CS mixing contraction,
Lemma~\ref{lem:cs-mix}).
For $\epsilon\le\epsilon_n$, the quadratic term dominates.
\end{proof}

\begin{remark}[Identified gaps in this proof]
\label{rem:local-stam-gaps}
The following issues must be resolved to make the local
Stam inequality fully rigorous:
\begin{enumerate}[label=\textup{(\roman*)},nosep]
  \item \textbf{Uniform-in-$w$ control.}
    The bound $\sum c_{n,k}\Delta_k\ge c_{n,3}\Delta_3
    =\Omega(\epsilon^2)$ depends on the mixing weight $w\in(0,1)$.
    The CS~mixing defect $\Delta_3$ vanishes as
    $w\to 0$ or $w\to 1$ (with rate $\Theta(\min(w,1-w))$),
    so the cubic remainder must be controlled
    \emph{uniformly} in $w\in(0,1)$, not just at fixed~$w$.
    Near $w=0$ or $w=1$, the assertion
    ``bounded below by continuity'' is non-quantitative and
    must be replaced by an explicit estimate.  Specifically,
    one needs $\Delta_3\ge c\,\min(w,1-w)\,(\tau_3^{(p)\,2}
    +\tau_3^{(q)\,2})$ for an effective constant $c>0$,
    and the cubic error term must satisfy
    a compatible bound $O(\epsilon^3)$ that does \emph{not}
    degenerate as $w\to 0,1$.
  \item \textbf{Dependence on Newton wall bound.}
    The constant $M_n$ requires an explicit upper bound on
    $|\nabla^3 R_n|$ over the truncated feasibility region.
    At present, $M_n$ is estimated numerically; a rigorous
    bound requires either a closed-form expression for
    $\nabla^3 R_n$ or a certified interval-arithmetic evaluation.
  \item \textbf{Hessian formula dependency.}
    The coefficients $c_{n,k}$ come from the Hessian
    (Theorem~\ref{thm:hessian}), which is only
    computer-verified for $n\ge 5$.
\end{enumerate}
An alternative route that avoids issue~(i) is to work with
the proportional Hermite flow (Section~\ref{sec:flow-strategy}),
which automatically pushes all cumulant ratios to zero at
rate $\tau_k(f_t)=\tau_k(f)/(1+t)^{k/2}$, uniformly in~$w$.
\end{remark}

%-----------------------------------------------------------
\section{Proof strategy: proportional Hermite flow}
\label{sec:flow-strategy}

The most promising approach to close the all-$n$ proof
uses the \emph{proportional Hermite flow}.

\textbf{Setup.}
For $p,q\in\PnR$ with $r=p\fp q$, define the
flow $p_t:=p\fp g_{tu_p}$, $q_t:=q\fp g_{tu_q}$,
where $u_f:=-\ell_2(f)$ is the variance of~$f$.
Since $u_r=u_p+u_q$, the semigroup property gives
$r_t:=p_t\fp q_t=(p\fp q)\fp g_{tu_r}$.
The mixing weight $w:=u_p/u_r$ is constant along
the flow, and $\tau_k(f_t)=\tau_k(f)/(1+t)^{k/2}$.

\textbf{Key properties.}
\begin{enumerate}[nosep]
\item \emph{Quadratic dominance at large~$t$}:
  for $t\ge T_0(n)$, all $|\tau_k|$ are small enough
  that the local Stam inequality
  (Theorem~\ref{thm:local-stam}) applies, giving
  $D_n(t)>0$.
\item \emph{Inductive base}: by strong induction,
  $D_{n-1}'(t)\ge 0$ for all~$t$
  (Stam at the derivative level, using
  Lemma~\ref{lem:deriv}).
\item \emph{Continuity}: $D_n(t)$ is real-analytic on
  $(0,\infty)$.
\end{enumerate}

\textbf{The argument.}
Write $D_n(t)=D_{n-1}'(t)+C_n(t)$ where
$C_n:=h_n(r_t)-h_n(p_t)-h_n(q_t)$ and
$h_n(f):=1/\Phi_n(f)-1/\Phi_{n-1}(f'/n)$.
Since $D_n(T_0)>0$ and $D_n$ is continuous,
we need to show $D_n$ cannot cross zero on $[0,T_0]$.

\textbf{The gap.}
The single unproved step is: $D_n(t)$ has no zeros
on $[0,T_0]$.
Three routes to close it:
\begin{enumerate}[label=\textup{(\alph*)}]
\item \emph{Explicit $T_0$ bound + Lipschitz estimate}:
  compute $T_0(n)$ from Newton wall bounds,
  show $D_n$ cannot deplete to zero from its
  value at $t=0$.
\item \emph{Chain dominance}: prove $D_n\ge\delta_n D_3$
  with $\delta_n>0$.
  This is equivalent to bounding the level corrections
  $\sum C_k$.
\item \emph{Flow monotonicity}: prove $D_n'(t)\le 0$
  for all~$t$ (Conjecture~\ref{conj:flow-mono} below).
\end{enumerate}

\begin{conjecture}[Flow monotonicity]\label{conj:flow-mono}
For all $n\ge 3$ and all $p,q\in\PnR$, the function
$t\mapsto D_n(p_t,q_t)$ along the proportional Hermite
flow is non-increasing on $[0,\infty)$.
\end{conjecture}

\numconf{} Along $2{,}000$ flow paths per~$n$
($n=4$--$8$, $t\in[0,5]$, $20$ time steps):
$D_n(t)>0$ at every sampled time with $0$ zero crossings.
$97$--$99\%$ of trajectories are monotonically decreasing;
the remaining $\sim\!2\%$ show a brief increase
followed by decrease, never a zero crossing.

If Conjecture~\ref{conj:flow-mono} holds, Stam follows
immediately: $D_n(0)\ge\lim_{t\to\infty}D_n(t)=0$.


%%%%%%%%%%%%%%%%%%%%%%%%%%%%%%%%%%%%%%%%%%%%%%%%%%%%%%%%%%%%
%%%%%%%%%%%%%%%%%%%%%%%%%%%%%%%%%%%%%%%%%%%%%%%%%%%%%%%%%%%%
\part{Conclusion}\label{part:conclusion}
%%%%%%%%%%%%%%%%%%%%%%%%%%%%%%%%%%%%%%%%%%%%%%%%%%%%%%%%%%%%
%%%%%%%%%%%%%%%%%%%%%%%%%%%%%%%%%%%%%%%%%%%%%%%%%%%%%%%%%%%%

\section{Comprehensive handoff for the next agent}

\textbf{Verified status (do not over-claim).}
Status categories are defined in the abstract and
summarised in Table~\ref{tab:status}.
\begin{itemize}[nosep]
  \item \proved{}: full Stam for $n=2,3$ (two independent proofs each);
    structural identities (Part~\ref{part:foundations});
    Score--Cauchy identity (Thm~\ref{thm:score-cauchy}),
    Frobenius norm identity (Thm~\ref{thm:frob}),
    $K$-cumulant preservation (Thm~\ref{thm:kappa-pres}),
    $R_3=\tfrac{9}{8}\tau_3^2$ (Thm~\ref{thm:R3-exact}),
    CS mixing inequality (Lemma~\ref{lem:cs-mix}),
    Hessian of $G_n$ for $n=3,4$ (Thm~\ref{thm:hessian}).
  \item \conditional{}: Gaussian-input Stam for all~$n$
    (depends on the root ODE $\dot\lambda_i=V_i/(n-1)$,
    Theorem~\ref{thm:stam-gauss});
    De~Bruijn identity (Thm~\ref{thm:debruijn}, same dependency);
    quadratic Stam lower bound for $n\ge 4$
    (depends on Hessian formula, Thm~\ref{thm:quad-stam}).
  \item \proofsketch{}: quantitative local Stam for all~$n$
    (Theorem~\ref{thm:local-stam}; identified gap:
    uniform-in-$w$ control, see Remark~\ref{rem:local-stam-gaps});
    harmonicity of $\log\disc$
    (Theorem~\ref{thm:harmonicity}; off-diagonal
    perturbation algebra outlined but not line-by-line).
  \item \compverif{}: three polynomial inequalities for $n=4$
    (Remark~\ref{rem:n4-status}); chain dominance $D_n\ge\delta_n D_3$
    ($0/30{,}000$, $\delta_n\ge 0.03$);
    Hessian of $G_n$ for $n\ge 5$ (Thm~\ref{thm:hessian});
    $R_n$ sub-averaging ($0/40{,}000$, growing margin with $n$);
    Fisher Jensen ratio $\sim 0.2$ ($0/50{,}000$);
    universal level-wise Stam $D_m\ge 0$ for all $m=3,\ldots,n$
    ($0/63{,}000$, Observation~\ref{obs:universal-stam});
    Frobenius reduction $\Phi_m/\Phi_{m-1}'\ge m/(m-2)$
    ($0/15{,}000$, Observation~\ref{obs:frob-reduction});
    flow monotonicity $D_n(t)$ decreasing along proportional
    Hermite flow ($0$ zero crossings in $10{,}000$ paths).
  \item \open{}: Fisher Jensen contraction
    (Conjecture~\ref{thm:fisher-jensen});
    $R_n$ sub-averaging with growing margin
    (Conjecture~\ref{thm:Rn-subavg});
    real stability of $K_r$ (Conjecture~\ref{thm:K-symbol-stable});
    compactness of the Newton-wall feasibility region
    (Conjecture~\ref{thm:compact-feas}).
  \item \open{}: general Stam for $n\ge 4$.
    Closest approach: proportional Hermite flow strategy
    (Section~\ref{sec:flow-strategy}) with one identified gap.
\end{itemize}

\textbf{Proof strategies (strongest first).}
\begin{enumerate}[nosep]
  \item \textbf{Proportional Hermite flow}
    (Section~\ref{sec:flow-strategy}): strong induction $+$
    quadratic dominance at large~$t$ $+$ flow continuity.
    \emph{One gap}: showing $D_n(t)$ has no zeros on $[0,T_0]$.
    Three routes to close: (a)~explicit $T_0$,
    (b)~chain dominance, (c)~flow monotonicity.
  \item \textbf{Approach~K} (degree induction): $D_n\ge\delta_n D_3$
    via deficit telescoping; $\delta_n\ge 0.03$ in $30{,}000$ trials.
    Needs: algebraic bound on chain corrections $\sum C_k$.
  \item \textbf{Approach~L} (CS mixing + Newton wall):
    Stam $\equiv$ $R_n$~sub-averaging, margin \emph{grows} with $n$.
    Needs: cubic remainder bound $|R_{\ge 3}|\le Q_2$ for $n\ge 4$.
  \item \textbf{Flow monotonicity conjecture}
    (Conjecture~\ref{conj:flow-mono}): if proved,
    Stam follows immediately.
\end{enumerate}

\textbf{Where to start reading.}
\begin{enumerate}[nosep]
  \item Part~\ref{part:gates}: gate analysis, new structural facts,
    and the flow strategy (most advanced material).
  \item Section~\ref{sec:identities} (stable toolbox).
  \item Section~\ref{ssec:n3} ($n=3$ proof) and
    Section~\ref{sec:n4-full} ($n=4$ pipeline).
  \item Part~\ref{part:mss}: Approaches K, L, M.
\end{enumerate}

\textbf{Highest-value next tasks (prioritised).}
\begin{enumerate}[nosep]
  \item \textbf{Close the flow gap}: prove $D_n(t)>0$ on $[0,T_0]$.
  \item \textbf{Prove the Frobenius reduction bound}
    (Observation~\ref{obs:frob-reduction}) rigorously from the
    Cauchy interlacing matrix identities.
  \item \textbf{Flow monotonicity}: compute $dD_n/dt$ explicitly
    and prove Conjecture~\ref{conj:flow-mono}.
  \item \textbf{Chain dominance}: prove $D_n\ge\delta_n D_3$
    algebraically via eigenvalue interlacing.
\end{enumerate}

\textbf{Dead ends to avoid.}
\begin{itemize}[nosep]
  \item Do NOT attempt global concavity of $1/\Phi_n$ ---
    Hessian is indefinite (Part~\ref{part:dead}).
  \item Do NOT use the expansion~\eqref{eq:wrong-exp} for non-Gaussian $q$
    (Warning in Section~\ref{sec:route-b}).
  \item Do NOT rely on production convexity of $\Psi$ ---
    94/2000 violations (Route~I).
  \item Do NOT use harmonic-mean bound in Haar averaging ---
    43\% failures (Observation~\ref{obs:hm-fails}).
  \item Do NOT attempt Gate~2 (ladder monotonicity $D_k\ge D_{k-1}$)
    --- 30--40\% violations (Observation~\ref{obs:gate2-false}).
  \item Do NOT attempt Gate~1 ($\mathcal{E}_n\ge 0$) ---
    70--99\% violations (Observation~\ref{obs:gate1-false}).
  \item Do NOT attempt $R_n$ convexity for $n\ge 5$ ---
    Hessian has negative eigenvalues (Observation~\ref{obs:Rn-hessian}).
  \item Do NOT attempt marginal concavity of $G_n$ in $\tau_n$ alone
    --- 60--88\% violations.
\end{itemize}

\textbf{Practical guardrails.}
\begin{itemize}[nosep]
  \item Every non-symbolic claim must be tagged \compverif{}.
  \item Any new ``proof'' must include machine-checkable certificates.
  \item Preserve the proved/non-proved boundary explicitly in the text.
  \item All numerical evidence is recorded in this document to
    avoid recomputation.
\end{itemize}


%%%%%%%%%%%%%%%%%%%%%%%%%%%%%%%%%%%%%%%%%%%%%%%%%%%%%%%%%%%%
\section*{References}

\begin{enumerate}[label={[\arabic*]},leftmargin=2em]
  \item\label{ref:MSS} A.~Marcus, D.~A.~Spielman, and N.~Srivastava,
    ``Interlacing families II: Mixed characteristic polynomials and the
    Kadison--Singer problem,'' \emph{Ann.\ of Math.}\ \textbf{182} (2015),
    327--350.
  \item\label{ref:stam} A.~J.~Stam, ``Some inequalities satisfied by the
    quantities of information of Fisher and Shannon,''
    \emph{Inform.\ Control}\ \textbf{2} (1959), 101--112.
  \item\label{ref:blachman} N.~M.~Blachman, ``The convolution inequality for
    entropy powers,'' \emph{IEEE Trans.\ Inform.\ Theory}\ \textbf{IT-11}
    (1965), 267--271.
  \item\label{ref:MSS1} A.~Marcus, D.~A.~Spielman, and N.~Srivastava,
    ``Interlacing families I: Bipartite Ramanujan graphs of all
    degrees,'' \emph{Ann.\ of Math.}\ \textbf{182} (2015), 307--325.
  \item\label{ref:bb} J.~Borcea and P.~Br\"and\'en,
    ``The Lee--Yang and P\'olya--Schur programs.\ I.\ Linear operators
    preserving stability,'' \emph{Invent.\ Math.}\ \textbf{177} (2009),
    541--569.
  \item\label{ref:newton} C.~Niculescu and L.-E.~Persson,
    \emph{Convex Functions and Their Applications:
    A Contemporary Approach}, 2nd ed.,
    CMS Books in Math., Springer, 2018.
    \textup{(For Newton's inequalities and log-concavity of
    symmetric functions.)}
  \item\label{ref:voiculescu} D.~V.~Voiculescu, K.~J.~Dykema,
    and A.~Nica, \emph{Free Random Variables},
    CRM Monograph Series~1, AMS, 1992.
  \item\label{ref:dembo} A.~Dembo, T.~M.~Cover, and J.~A.~Thomas,
    ``Information-theoretic inequalities,''
    \emph{IEEE Trans.\ Inform.\ Theory}\ \textbf{37} (1991), 1501--1518.
  \item\label{ref:bezout} M.~Fiedler,
    ``Hankel and Loewner matrices,''
    \emph{Linear Algebra Appl.}\ \textbf{58} (1984), 75--95.
    \textup{(For the Bezoutian matrix and its diagonalisation
    in the Lagrange interpolation basis.)}
  \item\label{ref:bhatia} R.~Bhatia,
    \emph{Matrix Analysis},
    Graduate Texts in Mathematics~169, Springer, 1997.
    \textup{(For eigenvalue perturbation theory used in the
    harmonicity proof sketch.)}
\end{enumerate}


\end{document}
