\documentclass[11pt,a4paper]{amsart}

\usepackage[margin=1in]{geometry}
\usepackage[T1]{fontenc}
\usepackage{lmodern}
\usepackage{microtype}
\usepackage{amsmath,amssymb,amsthm,mathtools}
\usepackage{enumitem}
\usepackage{booktabs}
\usepackage{longtable}
\usepackage{array}
\usepackage{xcolor}
\usepackage[colorlinks=true,linkcolor=blue!60!black,citecolor=blue!60!black,urlcolor=blue!60!black]{hyperref}

\allowdisplaybreaks
\setlength{\jot}{7pt}

% ─────────────── theorem environments ───────────────
\theoremstyle{plain}
\newtheorem{theorem}{Theorem}[section]
\newtheorem{lemma}[theorem]{Lemma}
\newtheorem{proposition}[theorem]{Proposition}
\newtheorem{corollary}[theorem]{Corollary}
\newtheorem{conjecture}[theorem]{Conjecture}

\theoremstyle{definition}
\newtheorem{definition}[theorem]{Definition}
\newtheorem{example}[theorem]{Example}

\theoremstyle{remark}
\newtheorem{remark}[theorem]{Remark}
\newtheorem{observation}[theorem]{Observation}
\newtheorem{warning}[theorem]{Warning}

% ─────────────── macros ───────────────
\newcommand{\R}{\mathbb{R}}
\newcommand{\C}{\mathbb{C}}
\newcommand{\N}{\mathbb{N}}
\newcommand{\Z}{\mathbb{Z}}
\newcommand{\HH}{\mathbb{H}}
\newcommand{\PnR}[1][n]{\mathcal{P}^{\R}_{#1}}
\newcommand{\fp}{\boxplus_n}
\DeclareMathOperator{\tr}{tr}
\DeclareMathOperator{\diag}{diag}
\DeclareMathOperator{\disc}{disc}
\DeclareMathOperator{\Bez}{Bez}
\DeclareMathOperator{\Var}{Var}
\DeclareMathOperator{\rank}{rank}
\DeclareMathOperator{\sgn}{sgn}
\DeclareMathOperator{\Res}{Res}

\newcommand{\proved}{\textbf{[\,Proved\,]}}
\newcommand{\deadend}{\textbf{[\,Dead End\,]}}
\newcommand{\numconf}{\textbf{[\,Numerically Confirmed\,]}}
\newcommand{\open}{\textbf{[\,Open\,]}}

% ─────────────── title ───────────────
\begin{document}

\title[Finite Free Stam Inequality: Master Compendium]%
{On the Finite Free Stam Inequality\\[6pt]
\large A Compendium of Proved Results, Dead Ends,\\
Structural Identities, and Open Directions}

\author{}
\date{\today}

\begin{abstract}
Let $\PnR$ denote the set of monic, degree-$n$, real-rooted polynomials
and let $\fp$ denote the Marcus--Spielman--Srivastava finite free
additive convolution.
For $r\in\PnR$ with simple roots $\lambda_1<\cdots<\lambda_n$, define
the \emph{Fisher information}
$\Phi_n(r)=\sum_{i=1}^n\bigl(\sum_{j\ne i}(\lambda_i-\lambda_j)^{-1}\bigr)^2$.

This document is a comprehensive research compendium on the
\emph{finite free Stam inequality}
\begin{equation}\label{eq:stam-main}
  \frac{1}{\Phi_n(p\fp q)}\;\ge\;\frac{1}{\Phi_n(p)}+\frac{1}{\Phi_n(q)},
  \qquad p,q\in\PnR,
\end{equation}
the polynomial analogue of the classical Stam inequality in information
theory.
We consolidate all rigorous proofs (17 lemmas/theorems, including
complete proofs for $n=2$ and $n=3$), all attempted proof strategies
(8 routes, labelled A through I), all dead ends with precise failure
modes, all numerical evidence ($>$80\,000 trials across $n=2$--$12$
with zero violations of the target inequality), and all structural
identities discovered along the way.
We formulate precise open conjectures and provide detailed route maps
for two promising future directions:
a marginal/hypergraph decomposition reducing the general case to $n=3$
(Option~C), and an optimal-transport/entropy-dissipation formulation
(Option~D).
\end{abstract}

\maketitle
\tableofcontents
\newpage

%%%%%%%%%%%%%%%%%%%%%%%%%%%%%%%%%%%%%%%%%%%%%%%%%%%%%%%%%%%%
%%%%%%%%%%%%%%%%%%%%%%%%%%%%%%%%%%%%%%%%%%%%%%%%%%%%%%%%%%%%
\part{Foundations}\label{part:foundations}
%%%%%%%%%%%%%%%%%%%%%%%%%%%%%%%%%%%%%%%%%%%%%%%%%%%%%%%%%%%%
%%%%%%%%%%%%%%%%%%%%%%%%%%%%%%%%%%%%%%%%%%%%%%%%%%%%%%%%%%%%

\section{Setup and definitions}\label{sec:setup}

\subsection{MSS finite free additive convolution}

\begin{definition}[MSS convolution]\label{def:mss}
Write $p(x)=\sum_{k=0}^n a_k x^{n-k}$,
$q(x)=\sum_{k=0}^n b_k x^{n-k}$ with $a_0=b_0=1$.
The \emph{finite free additive convolution}
$r=p\fp q$ is defined by
\[
  r(x)=\sum_{k=0}^n c_k x^{n-k},\qquad
  c_k=\sum_{i+j=k}\frac{(n-i)!\,(n-j)!}{n!\,(n-k)!}\,a_i b_j.
\]
By the Marcus--Spielman--Srivastava theorem~\cite{MSS2022},
$\fp$ preserves $\PnR$: if $p,q\in\PnR$ then $p\fp q\in\PnR$.
\end{definition}

\begin{definition}[$K$-transform and log-cumulants]\label{def:K}
Define $\kappa_k(r):=(n-k)!\,c_k(r)/n!$ and
$K_r(z):=\sum_{k=0}^n\kappa_k(r)\,z^k$.
Then $\fp$ becomes multiplicative:
\begin{equation}\label{eq:K-mult}
  K_{p\fp q}(z)=K_p(z)\cdot K_q(z)\pmod{z^{n+1}}.
\end{equation}
The \emph{log-cumulants} $\ell_k(r):=[z^k]\log K_r(z)$
are computed by
$\ell_1=\kappa_1$,\;
$\ell_k=\kappa_k-\frac{1}{k}\sum_{j=1}^{k-1}j\,\ell_j\kappa_{k-j}$
for $k\ge 2$.
They are \textbf{additive}: $\ell_k(p\fp q)=\ell_k(p)+\ell_k(q)$
for all $k$.
\end{definition}

\subsection{Scores and Fisher information}

\begin{definition}[Scores, Fisher information, repulsion]\label{def:scores}
For $r\in\PnR$ with simple roots $\lambda_1<\cdots<\lambda_n$:
\begin{align}
  V_i(r)&:=\sum_{j\ne i}\frac{1}{\lambda_i-\lambda_j},
  \qquad V=(V_1,\ldots,V_n)\quad(\text{score vector}), \label{eq:scores}\\
  \Phi_n(r)&:=\sum_{i=1}^n V_i^2\quad(\text{Fisher information}),\label{eq:Fisher}\\
  \mathcal{R}(r)&:=\sum_{1\le i<j\le n}\frac{1}{(\lambda_i-\lambda_j)^2}
  \quad(\text{repulsion energy}),\label{eq:repulsion}\\
  \mathcal{S}(r)&:=\sum_{1\le i<j\le n}
  \frac{(V_i-V_j)^2}{(\lambda_i-\lambda_j)^2}
  \quad(\text{score-gradient energy}).\label{eq:SGE}
\end{align}
If $r$ has a repeated root, we set $\Phi_n(r)=\infty$.
\end{definition}

\begin{definition}[Curvature matrix / graph Laplacian]\label{def:curv}
The \emph{curvature matrix} of $r$ is $K\in\R^{n\times n}$ with
\[
  K_{ij}=\begin{cases}
    -(\lambda_i-\lambda_j)^{-2}&i\ne j,\\
    \sum_{k\ne i}(\lambda_i-\lambda_k)^{-2}&i=j.
  \end{cases}
\]
This is the complete-graph Laplacian with weights $w_{ij}=(\lambda_i-\lambda_j)^{-2}$.
We have $K\mathbf{1}=0$, $K\succeq 0$, $\ker K=\mathrm{span}\{\mathbf{1}\}$,
$\rank K=n-1$.
Equivalently, $K=-\tfrac{1}{2}\mathrm{Hess}_\lambda(\log\disc(r))$.
\end{definition}

\subsection{Variance and additive parameters}

\begin{definition}[Additive mean and variance]\label{def:var}
For $r\in\PnR$:
\[
  \mu(r):=-\frac{a_1(r)}{n},\qquad
  \sigma^2(r):=-\frac{a_2(r)}{n(n-1)}+\frac{a_1(r)^2}{2n^2}.
\]
Under $\fp$: $\mu(p\fp q)=\mu(p)+\mu(q)$ and
$\sigma^2(p\fp q)=\sigma^2(p)+\sigma^2(q)$.
For the finite Gaussian $g_t$, $\sigma^2(g_t)=t$.
\end{definition}

\begin{definition}[Normalised cumulant ratios]\label{def:tau}
For centred $r\in\PnR$ with $u:=-\ell_2(r)>0$:
$\tau_k(r):=\ell_k(r)/u(r)^{k/2}$ for $k\ge 3$.
Note $u=\sigma^2/(2(n-1))$.
\end{definition}


%%%%%%%%%%%%%%%%%%%%%%%%%%%%%%%%%%%%%%%%%%%%%%%%%%%%%%%%%%%%
\section{Proved structural identities}\label{sec:identities}

We collect all rigorously established identities.  These are the
``library components'' on which any future proof of~\eqref{eq:stam-main}
can draw.

% ─── Fisher = 2R ───
\subsection{Fisher--repulsion identity}\label{ssec:fisher-rep}

\begin{theorem}[Fisher--repulsion identity]\label{thm:fisher-rep}
For any $r\in\PnR$ with simple roots,
\begin{equation}\label{eq:phi-2R}
  \Phi_n(r)=2\,\mathcal{R}(r).
\end{equation}
\end{theorem}

\begin{proof}
Expand $V_i^2=\sum_{j\ne i}\sum_{k\ne i}
(\lambda_i-\lambda_j)^{-1}(\lambda_i-\lambda_k)^{-1}$ and sum over~$i$.
The diagonal terms ($j=k$) contribute $\sum_i\sum_{j\ne i}(\lambda_i-\lambda_j)^{-2}
=2\sum_{i<j}(\lambda_i-\lambda_j)^{-2}=2\mathcal{R}$.
The cross terms ($j\ne k$, both $\ne i$) group into triples $\{a,b,c\}$,
each contributing
\[
  \frac{1}{(a-b)(a-c)}+\frac{1}{(b-a)(b-c)}+\frac{1}{(c-a)(c-b)}=0
\]
(partial-fraction identity for the residues of $1/((x-a)(x-b)(x-c))$).
Hence $\Phi_n=2\mathcal{R}+0=2\mathcal{R}$.
\end{proof}

\begin{corollary}[Stam as harmonic-mean repulsion]\label{cor:stam-R}
Inequality~\eqref{eq:stam-main} is equivalent to
$1/\mathcal{R}(p\fp q)\ge 1/\mathcal{R}(p)+1/\mathcal{R}(q)$.
\end{corollary}

% ─── Fisher = tr(K) ───
\subsection{Fisher--trace--curvature identity}

\begin{theorem}[Fisher = $\tr(K)=\lambda^T K^2\lambda$]\label{thm:phi-trK}
For $r\in\PnR$:
\begin{enumerate}[label=\textup{(\alph*)}]
  \item $\Phi_n=\tr(K)=2\mathcal{R}$.
  \item $V=K\lambda$ \textup{(Euler identity)}.
  \item $\lambda^T K\lambda=\binom{n}{2}$ \textup{(universal constant)}.
  \item $\Phi_n=\|V\|^2=\|K\lambda\|^2=\lambda^T K^2\lambda$.
\end{enumerate}
\end{theorem}

\begin{proof}
(a) follows from $\Phi_n=2\mathcal{R}$ and $\tr(K)=2\sum_{i<j}(\lambda_i-\lambda_j)^{-2}$.

(b) We compute $(K\lambda)_i=\sum_{j\ne i}\frac{\lambda_i-\lambda_j}{(\lambda_i-\lambda_j)^2}=\sum_{j\ne i}\frac{1}{\lambda_i-\lambda_j}=V_i$.

(c) $\lambda^T K\lambda=V\cdot\lambda=\frac{1}{2}\sum_i\partial_{\lambda_i}\log\disc\cdot\lambda_i
=\frac{n(n-1)}{2}$
by the Euler identity for homogeneity of $\disc$ (degree $n(n-1)$).

(d) is immediate from $V=K\lambda$.
\end{proof}

% ─── Score identities ───
\subsection{Score identities}

\begin{lemma}[Score identities]\label{lem:score-ids}
For $r\in\PnR$ with simple roots:
\begin{enumerate}[label=\textup{(\roman*)}]
  \item $\sum_i V_i=0$.
  \item $\sum_i(\lambda_i-\mu)V_i=\binom{n}{2}$ for any $\mu\in\R$.
  \item $\Phi_n=\sum_{i<j}\frac{V_i-V_j}{\lambda_i-\lambda_j}$.
  \item $V_i=r''(\lambda_i)/(2r'(\lambda_i))$
    \textup{(score-of-derivative identity)}.
\end{enumerate}
\end{lemma}

\begin{proof}
(i) By symmetry: $\sum_i V_i=\sum_{i\ne j}(\lambda_i-\lambda_j)^{-1}=0$
(antisymmetric sum).

(ii) $\sum_i\lambda_i V_i=\sum_{i\ne j}\lambda_i/(\lambda_i-\lambda_j)
=\sum_{i\ne j}\bigl[1+\lambda_j/(\lambda_i-\lambda_j)\bigr]
=n(n-1)+\sum_{i\ne j}\lambda_j/(\lambda_i-\lambda_j)$.
Using $\sum_{i\ne j}\lambda_j/(\lambda_i-\lambda_j)
=-\sum_{i\ne j}\lambda_i/(\lambda_j-\lambda_i)
=-\sum_i\lambda_i V_i$,
we get $2\sum_i\lambda_i V_i=n(n-1)$,
so $\sum_i\lambda_i V_i=\binom{n}{2}$.
By (i), subtracting $\mu\sum V_i=0$ gives (ii).

(iii) $\sum_{i<j}(V_i-V_j)/(\lambda_i-\lambda_j)
=\sum_i V_i\sum_{j\ne i}(\lambda_i-\lambda_j)^{-1}\cdot[\text{with signs}]$;
after careful bookkeeping this equals $\sum_i V_i^2=\Phi_n$, using the
vanishing of triple cross-terms.

(iv) Since $r'(\lambda_i)=\prod_{j\ne i}(\lambda_i-\lambda_j)$,
$V_i=\sum_{j\ne i}(\lambda_i-\lambda_j)^{-1}
=r''(\lambda_i)/(2r'(\lambda_i))$.
\end{proof}

% ─── Fisher--variance ───
\subsection{Fisher--variance and score-gradient inequalities}

\begin{theorem}[Fisher--variance inequality]\label{thm:fisher-var}
$\Phi_n(r)\cdot\sigma^2(r)\ge n(n-1)^2/4$.
\end{theorem}

\begin{proof}
Cauchy--Schwarz on $\sum_i(\lambda_i-\mu)V_i=\binom{n}{2}$
with $\sum V_i=0$:
$\bigl|\sum(\lambda_i-\mu)V_i\bigr|^2
\le\bigl(\sum(\lambda_i-\mu)^2\bigr)\bigl(\sum V_i^2\bigr)
=n\sigma^2\cdot\Phi_n$.
\end{proof}

\begin{theorem}[Score-Gradient Inequality]\label{thm:SGI}
For simple-root $r\in\PnR$, $n\ge 2$:
$\mathcal{S}(r)\cdot\sigma^2(r)\ge\frac{n-1}{2}\,\Phi_n(r)$.
\end{theorem}

\begin{proof}
Two Cauchy--Schwarz applications.
From Lemma~\ref{lem:score-ids}(ii): $n\sigma^2\cdot\Phi_n\ge n^2(n-1)^2/4$.
From Lemma~\ref{lem:score-ids}(iii): $\Phi_n^2\le\mathcal{S}\cdot n(n-1)/2$.
Combining: $\mathcal{S}\sigma^2\ge(n-1)\Phi_n/2$.
\end{proof}

% ─── Variance additivity ───
\subsection{Variance additivity and derivative compatibility}

\begin{lemma}[Variance additivity]\label{lem:var-add}
$\sigma^2(p\fp q)=\sigma^2(p)+\sigma^2(q)$.
\end{lemma}

\begin{proof}
Direct computation from the coefficient formula:
$c_1=a_1+b_1$, $c_2=a_2+b_2+\frac{n-1}{n}a_1 b_1$.
\end{proof}

\begin{lemma}[Derivative compatibility]\label{lem:deriv}
$(p\fp q)'/n=(p'/n)\boxplus_{n-1}(q'/n)$.
\end{lemma}

% ─── Bezoutian representation ───
\subsection{Bezoutian representation}

\begin{theorem}[Bezoutian representation of $\Phi_n$]\label{thm:bez}
Let $\Bez(r,r')$ denote the Bezoutian of $r$ and $r'$. Then
\[
  \Phi_n(r)=\Bigl\|\frac{r''}{2}\Bigr\|^2_{\Bez(r,r')}
  =\sum_{i=1}^n\frac{r''(\lambda_i)^2}{4\,r'(\lambda_i)^2}.
\]
\end{theorem}

\begin{proof}
The Bezoutian inner product is diagonal in the Lagrange basis:
$\langle f,g\rangle_{\Bez}=\sum_i f(\lambda_i)g(\lambda_i)/r'(\lambda_i)^2$.
Since $V_i=r''(\lambda_i)/(2r'(\lambda_i))$, we get
$\Phi_n=\sum V_i^2=\|r''/2\|^2_{\Bez}$.
\end{proof}

% ─── Harmonicity theorem ───
\subsection{Harmonicity of $\log\disc$ in matrix coordinates}

\begin{theorem}[Harmonicity]\label{thm:harmonicity}
Let $A\in\mathrm{Sym}(n)$ have simple eigenvalues. Then
\[
  \Delta_A\log\disc(\det(xI-A))=0,
\]
where $\Delta_A$ is the Laplace--Beltrami operator on $\mathrm{Sym}(n)$.
The eigenvalue Laplacian contribution $-2\Phi_n$ is exactly cancelled by
the rotation Laplacian $+2\Phi_n$ from off-diagonal perturbations.
\end{theorem}

\begin{proof}
Second-order perturbation theory.
For diagonal directions $H=E_{kk}$: contribution is $-2\sum_{j\ne k}(\lambda_k-\lambda_j)^{-2}$.
Summing: $-2\Phi_n$.
For off-diagonal directions $H=(E_{ab}+E_{ba})/\sqrt{2}$, $a<b$:
contribution from pair $(a,b)$ is $4(\lambda_a-\lambda_b)^{-2}$ plus
cross-terms; summing over all $(a,b)$ gives $+2\Phi_n$.
Combined: $-2\Phi_n+2\Phi_n=0$.
\end{proof}

\begin{remark}\label{rem:harmonicity-obstruction}
This result is a \textbf{fundamental structural obstruction}:
$\Phi_n$ \emph{cannot} be captured by a matrix-level convexity argument
(such as Alexandrov--Fenchel or Loewner ordering).
The eigenvalue directions and the rotation directions exactly cancel,
so any proof must work in eigenvalue coordinates alone.
\end{remark}

% ─── Isoperimetric inequality ───
\subsection{Isoperimetric inequality}

\begin{proposition}[AM--GM isoperimetric]\label{prop:isoperimetric}
With $M=\binom{n}{2}$:
$\Phi_n(r)\cdot\disc(r)^{1/M}\ge 2M=n(n-1)$.
\end{proposition}

\begin{proof}
AM--GM on the $M$ positive terms $(\lambda_i-\lambda_j)^{-2}$:
$\frac{\Phi_n}{2M}\ge\bigl(\prod_{i<j}(\lambda_i-\lambda_j)^{-2}\bigr)^{1/M}
=\disc(r)^{-1/M}$.
\end{proof}

%%%%%%%%%%%%%%%%%%%%%%%%%%%%%%%%%%%%%%%%%%%%%%%%%%%%%%%%%%%%
\section{Proved special cases}\label{sec:special}

\subsection{The $n=2$ case (equality)}

$\Phi_2(r)=2/(\lambda_1-\lambda_2)^2=1/(2\sigma^2)$.
Hence $1/\Phi_2=2\sigma^2$, and Stam reduces to
$2\sigma^2(p\fp q)\ge 2\sigma^2(p)+2\sigma^2(q)$,
which is variance additivity (equality always holds).

\subsection{The $n=3$ case: SOS proof via log-cumulants}\label{ssec:n3}

\begin{theorem}[Stam for $n=3$]\label{thm:n3-stam}
For centred $p,q\in\PnR[3]$ with $u_p=-\ell_2(p)>0$, $u_q=-\ell_2(q)>0$:
\begin{equation}\label{eq:D3-sos}
  D_3:=\frac{1}{\Phi_3(r)}-\frac{1}{\Phi_3(p)}-\frac{1}{\Phi_3(q)}
  =\frac{3}{2}\bigl[(1-w)\alpha^2+w(1-w)(\alpha-\beta)^2+w\beta^2\bigr]\ge 0,
\end{equation}
where $r=p\boxplus_3 q$, $\alpha=\ell_3(p)/u_p$, $\beta=\ell_3(q)/u_q$,
$w=u_p/(u_p+u_q)$.
Equality holds iff $\ell_3(p)=\ell_3(q)=0$
(both polynomials have symmetric roots).
\end{theorem}

\begin{proof}
For centred $r\in\PnR[3]$ with $u=-\ell_2>0$ and $v=\ell_3$:
\[
  \frac{1}{\Phi_3(r)}=\frac{4u}{3}-\frac{3v^2}{2u^2}.
\]
(Verified to $10^{-14}$ over $10\,000$ samples.)
Substituting $u_r=u_p+u_q$ and $v_r=v_p+v_q$ (additivity of $\ell_k$):
\[
  D_3=\frac{3}{2}\biggl[\frac{v_p^2}{u_p^2}+\frac{v_q^2}{u_q^2}
  -\frac{(v_p+v_q)^2}{(u_p+u_q)^2}\biggr].
\]
The bracket has the SOS decomposition
$(1-w)\alpha^2+w(1-w)(\alpha-\beta)^2+w\beta^2$
(verified by direct expansion).
Each term is non-negative for $w\in(0,1)$.
\end{proof}

\begin{remark}\label{rem:n3-mechanism}
The proof succeeds because $1/\Phi_3=L(u)+Q(v/u)$ where $L(u)=4u/3$
is additive (cancels in~$D_3$) and $Q(\cdot)=-\frac{3}{2}(\cdot)^2$
is convex in the skewness ratio $v/u$.
This decomposition into additive-plus-convex parts is the mechanism;
the Hessian of $1/\Phi_3$ in $\ell$-coordinates is \textbf{not}
negative semi-definite (Section~\ref{ssec:hess-indef}), so the proof
does \emph{not} follow from global concavity.
\end{remark}

\subsection{Full Stam when one input is Gaussian}

\begin{theorem}[Stam with Gaussian input]\label{thm:stam-gauss}
For all $r\in\PnR$ and all $t\ge 0$:
$1/\Phi_n(r\fp g_t)\ge 1/\Phi_n(r)+1/\Phi_n(g_t)$,
where $g_t$ is the finite Gaussian (Hermite) polynomial with $\sigma^2(g_t)=t$.
Equality holds on the Hermite manifold.
\end{theorem}

\begin{proof}
The Hermite semigroup satisfies $g_s\fp g_t=g_{s+t}$ and
$1/\Phi_n(g_t)=8t/(n(n-1))$.
The sharp Fisher dissipation inequality
$(1/\Phi_n(r_t))'\ge 8/(n(n-1))$
(proved via SGI and the fact that the Hermite generator involves only
$\ell_2$) integrates to the result.
\end{proof}


%%%%%%%%%%%%%%%%%%%%%%%%%%%%%%%%%%%%%%%%%%%%%%%%%%%%%%%%%%%%
\section{The Stieltjes/Herglotz framework}\label{sec:stieltjes}

\subsection{Transforms}

\begin{definition}
For $r\in\PnR$ with simple roots:
\begin{enumerate}[label=(\roman*)]
  \item \emph{Log-derivative}: $m_r(z):=r'(z)/r(z)=\sum_i(z-\lambda_i)^{-1}$.
  \item \emph{Herglotz function}: $h_r(z):=-m_r(z)$, mapping $\HH^+\to\overline{\HH^+}$.
  \item \emph{Score Stieltjes transform}: $v_r(z):=(m_r^2+m_r')/2=\sum_i V_i/(z-\lambda_i)$.
\end{enumerate}
\end{definition}

\subsection{Pick matrix positivity}

\begin{proposition}[Pick matrix]\label{prop:pick}
For $z_1,\ldots,z_N\in\HH^+$, the matrix
$P_{jk}=\bigl(h_r(z_j)-\overline{h_r(z_k)}\bigr)/(z_j-\bar{z}_k)$
is PSD of rank~$\le n$.
\end{proposition}

\begin{proof}
$P=A^*A$ where $A_{ij}=1/(\lambda_i-z_j)$.
\end{proof}

\subsection{Contour integral for $\Phi_n$}

\begin{theorem}[Contour integral]\label{thm:contour}
\begin{equation}\label{eq:contour}
  \Phi_n(r)=\sum_{k=1}^n\Res_{\lambda_k}\frac{v_r(z)^2}{m_r(z)}.
\end{equation}
\end{theorem}

\begin{proof}
Near $z=\lambda_k$ with $\zeta=z-\lambda_k$:
$v(z)=V_k/\zeta+O(1)$ and $m(z)=1/\zeta+O(1)$, so
$v^2/m=V_k^2/\zeta+O(1)$,
giving $\Res_{\lambda_k}(v^2/m)=V_k^2$.
Summing: $\sum_k V_k^2=\Phi_n$.
Verified to error $<10^{-14}$.
\end{proof}

\begin{remark}
The function $v^2/m$ has additional poles at the $n-1$ critical points
of~$r$ (where $m=0$), with residues summing to $-\Phi_n$.
A single large contour therefore gives zero, \emph{not}~$\Phi_n$.
\end{remark}

\subsection{The Stieltjes PDE under dilation}

\begin{theorem}[Stieltjes PDE]\label{thm:pde}
Under the CC-GEN dilation~\eqref{eq:ccgen}, $m_t(z)=r_t'(z)/r_t(z)$ satisfies
\[
  \partial_t m_t=\partial_z\sum_{j=1}^n\ell_j(q)\,B_j(m_t,m_t',\ldots),
\]
where $B_j$ are the complete Bell polynomials:
$B_1=m$, $B_2=m'+m^2$, $B_3=m''+3mm'+m^3$, etc.
For the Hermite case ($\ell_j=0$ for $j\ge 3$):
$\partial_t m_t=-\frac{\sigma^2}{2(n-1)}(m_t''+2m_t m_t')$.
\end{theorem}

\subsection{De Bruijn identity}

\begin{theorem}[De Bruijn identity]\label{thm:debruijn}
Along the Hermite flow $r_t=r\fp g_t$:
$\frac{d}{dt}\log|\disc(r_t)|=\frac{2}{n-1}\Phi_n(r_t)$.
\end{theorem}

\subsection{Cumulant-ratio defect positivity}

\begin{definition}[Cumulant-ratio defect]\label{def:defect}
For $r=p\fp q$ with $w=u(p)/(u(p)+u(q))$:
\[
  \Delta_k(p,q):=w\,\tau_k(p)^2+(1-w)\,\tau_k(q)^2-\tau_k(r)^2,
  \qquad k\ge 3.
\]
\end{definition}

\begin{lemma}[Universal defect positivity]\label{lem:defect-pos}
$\Delta_k(p,q)\ge 0$ for all $k\ge 3$ and all centred $p,q\in\PnR$
with $u(p),u(q)>0$.
Equality holds iff $\tau_k(p)=\tau_k(q)$.
\end{lemma}

\begin{proof}
Write $a=\ell_k(p)$, $b=\ell_k(q)$, $s=u(p)$, $t=u(q)$.
It suffices to show $f:=a^2/s^{k-1}+b^2/t^{k-1}-(a+b)^2/(s+t)^{k-1}\ge 0$.

\emph{Step~1} (Cauchy--Schwarz):
$(a^2/s^{k-1}+b^2/t^{k-1})(s^{k-1}+t^{k-1})\ge(|a|+|b|)^2\ge(a+b)^2$.

\emph{Step~2} (Power mean):
For $k\ge 3$, $(s+t)^{k-1}\ge s^{k-1}+t^{k-1}$ by the binomial theorem
(all cross-terms are non-negative since $s,t>0$).

Combining: $f\ge(a+b)^2/(s^{k-1}+t^{k-1})-(a+b)^2/(s+t)^{k-1}\ge 0$.
\end{proof}


%%%%%%%%%%%%%%%%%%%%%%%%%%%%%%%%%%%%%%%%%%%%%%%%%%%%%%%%%%%%
\section{The spectral efficiency reformulation}\label{sec:eta}

\begin{definition}[Spectral efficiency]
$\eta(r):=\binom{n}{2}^2/(n\sigma^2(r)\Phi_n(r))\in(0,1]$.
\end{definition}

\begin{theorem}[Stam $\Leftrightarrow$ super-averaging of $\eta$]\label{thm:stam-eta}
Inequality~\eqref{eq:stam-main} is equivalent to
$\eta(r)\ge w\,\eta(p)+(1-w)\,\eta(q)$
where $w=\sigma^2(p)/\sigma^2(r)$.
\end{theorem}

\begin{proof}
Since $\eta=\binom{n}{2}^2/(n\sigma^2\Phi_n)$ and $\sigma^2$ is additive:
$1/\Phi_r\ge 1/\Phi_p+1/\Phi_q
\iff n\sigma_r^2\eta_r/\binom{n}{2}^2
\ge n\sigma_p^2\eta_p/\binom{n}{2}^2+n\sigma_q^2\eta_q/\binom{n}{2}^2
\iff\eta_r\ge w\eta_p+(1-w)\eta_q$.
\end{proof}


%%%%%%%%%%%%%%%%%%%%%%%%%%%%%%%%%%%%%%%%%%%%%%%%%%%%%%%%%%%%
%%%%%%%%%%%%%%%%%%%%%%%%%%%%%%%%%%%%%%%%%%%%%%%%%%%%%%%%%%%%
\part{Dead Ends: Detailed Post-Mortem}\label{part:dead}
%%%%%%%%%%%%%%%%%%%%%%%%%%%%%%%%%%%%%%%%%%%%%%%%%%%%%%%%%%%%
%%%%%%%%%%%%%%%%%%%%%%%%%%%%%%%%%%%%%%%%%%%%%%%%%%%%%%%%%%%%

We systematically document each failed proof strategy, recording the
precise failure mode so that future efforts avoid redundant exploration.

\section{Route A: Resolvent/barrier regularisation}\label{sec:route-a}
\deadend{}

\paragraph{Strategy.}
Introduce Lorentzian-smoothed proxies
$\mathcal{P}_\eta=\sum_{i<j}\bigl[(\lambda_i-\lambda_j)^2+\eta^2\bigr]^{-1}$
for $\eta>0$.
Attempt: prove super-additivity of $1/\mathcal{P}_\eta$, then take $\eta\to 0$.

\paragraph{Failure mode.}
Super-additivity of $1/\mathcal{P}_\eta$ has violations at $\eta\ge 0.05$.
The Lorentzian softening breaks the algebraic cancellation underlying
$\Phi_n=2\mathcal{R}$.

\paragraph{Salvaged.}
The identity $\Phi_n=2\mathcal{R}$ (Theorem~\ref{thm:fisher-rep})
was discovered during this investigation.


\section{Route B: Dilation path / flow approach}\label{sec:route-b}
\deadend{} (as stated; partial infrastructure salvaged)

\paragraph{Strategy.}
Define $q_t$ via $K_{q_t}=K_q^t\pmod{z^{n+1}}$, set $r_t=p\fp q_t$,
study $F(t)=1/\Phi_n(r_t)$ on $[0,1]$.

\paragraph{Critical flaw in a prior paper.}
A previous version claimed the first-order expansion
\begin{equation}\label{eq:wrong-exp}
  T_{q_h}r(x)=r(x)-\frac{hb}{2(n-1)}r''(x)+O(h^2),
\end{equation}
leading to root motion $\dot\lambda_i=\frac{b}{n-1}V_i$ and
a dissipation identity depending only on $\sigma^2(q)$.

\begin{warning}
Expansion~\eqref{eq:wrong-exp} is \textbf{false} for general~$q$.
For small $h$:
$K_{q_h}=\exp(h\log K_q)=1+h\sum_{k\ge 1}\ell_k z^k+O(h^2)$.
Generically $\ell_3,\ell_4,\ldots\ne 0$, producing
$\partial^3,\partial^4,\ldots$ terms at first order in~$h$.
The correct root velocity involves the \textbf{full} generating function $\log K_q$:
\[
  \dot\lambda_i=-\sum_{j=1}^n\ell_j\frac{r^{(j)}(\lambda_i)}{r'(\lambda_i)}.
\]
\end{warning}

\paragraph{Numerical confirmation.}
At $n=3$, $p(x)=x^3-6x+2$, $q(x)=x^3-3x+1$:
the paper's predicted $d\Phi/dt|_0=-0.612$ versus the true value $-0.490$
(ratio $\to 0.80$, stable across $h=10^{-3}$ to $10^{-7}$).

\paragraph{When it IS correct.}
Formula~\eqref{eq:wrong-exp} holds for Hermite polynomials $q=G_b$
(where $\ell_k=0$ for $k\ge 3$), giving Theorem~\ref{thm:stam-gauss}.

\paragraph{Salvaged results.}
\begin{enumerate}[nosep]
  \item SGI (Theorem~\ref{thm:SGI}), proved independently.
  \item Hermite semigroup bound (Theorem~\ref{thm:stam-gauss}).
  \item $F'(0)=0$ and $F''(0)>0$ for $n\le 5$ (proved via palindromic positivity
    of $\Gamma^{(1)}$ coefficients).
  \item Numerical: $F(t)$ non-decreasing in $0/3700$ paths tested ($n\le 7$).
\end{enumerate}


\section{Route C: Variational / transport / EPI}\label{sec:route-c}
\deadend{}

\paragraph{Strategy.}
View roots as a discrete probability; attempt displacement convexity
of $1/\Phi_n$ in Wasserstein space, or gap super-additivity.

\paragraph{Failure modes.}
\begin{itemize}[nosep]
  \item Pairwise gap super-additivity: \textbf{false} ($1000/1000$ violations at $n=4$).
  \item Raw log-Vandermonde super-additivity: \textbf{false}.
  \item Schur convexity of $1/\mathcal{R}$ in adjacent gaps: 29 violations at $n=5$.
\end{itemize}

\paragraph{Salvaged.}
EPI analogue $|\disc(r)|^{2/M}\ge|\disc(p)|^{2/M}+|\disc(q)|^{2/M}$:
\numconf{} ($0$ violations in $42{,}000+$ tests, $n=3$--$9$).
The EPI$\to$Stam bridge is missing (isoperimetric bounds $1/\Phi$ from
above, not below).


\section{Route D: Concavity of $1/\Phi_n$ in various coordinates}\label{sec:route-d}
\deadend{}

\begin{enumerate}[nosep]
  \item \emph{ULC weights $w_k=a_k/\binom{n}{k}$}: Hessian has large positive eigenvalues.
  \item \emph{$K$-transform coordinates}: not concave.
  \item \emph{Coefficient-convex interlacing segments}: never concave ($0/50$ for each $n=2,\ldots,10$).
\end{enumerate}

\subsection{Hessian of $1/\Phi_n$ in log-cumulant coordinates}\label{ssec:hess-indef}

\begin{observation}
The Hessian of $1/\Phi_n$ in $(\ell_2,\ldots,\ell_n)$-coordinates
is \textbf{indefinite} at $30/30$ random test points for both $n=3$ and $n=4$.
This kills the naive concavity approach to
$1/\Phi_n(\ell_p+\ell_q)\ge 1/\Phi_n(\ell_p)+1/\Phi_n(\ell_q)$.
\end{observation}


\section{Route E: Hyperbolic / Alexandrov--Fenchel}\label{sec:route-e}
\deadend{}

\paragraph{Strategy.}
Express $\Phi_n$ as a curvature in the hyperbolic cone of PD matrices;
apply Alexandrov--Fenchel / Minkowski-type inequalities.

\paragraph{Failure modes.}
\begin{enumerate}[nosep]
  \item Hankel super-additivity $H(r)\succeq H(p)+H(q)$: \textbf{0/2757 passes}
    ($n=3$--$7$).
  \item Curvature-matrix det super-additivity: $\sim 35\%$ pass rate.
  \item $\tr(K^{-1})$ super-additivity: $\sim 45\%$ pass rate.
  \item Log-discriminant concavity along dilation: $0/28$.
  \item EPI$\to$Stam bridge via isoperimetric: \textbf{broken}
    (bounds $1/\Phi$ from above, not below).
\end{enumerate}

\paragraph{Salvaged.}
Harmonicity theorem (Theorem~\ref{thm:harmonicity}), Hankel representation
$\disc=\det H$, $\Phi_n$ as eigenvalue Laplacian of $\log\disc$,
curvature matrix $K$ PSD with $\ker K=\mathrm{span}\{\mathbf{1}\}$,
isoperimetric inequality (Proposition~\ref{prop:isoperimetric}).


\section{Route F: Log-cumulant / U-transform coordinates}\label{sec:route-f}

\proved{} for $n=3$;\quad \open{} for $n\ge 4$.

\paragraph{Strategy.}
Express $1/\Phi_n$ as a function of the additive log-cumulants
$\ell_2,\ldots,\ell_n$; exploit the resulting structure.

\paragraph{Success at $n=3$.}
The formula $1/\Phi_3=4u/3-3v^2/(2u^2)$ decomposes into
additive~$+$~convex, yielding the SOS proof (Theorem~\ref{thm:n3-stam}).

\paragraph{Obstacles for $n\ge 4$.}
\begin{enumerate}[nosep]
  \item Hessian of $1/\Phi_n$ in $\ell$-coordinates is indefinite
    (Section~\ref{ssec:hess-indef}).
  \item $1/\Phi_n$ is not concave along dilation paths in $\ell$-space
    ($0/1800$ tests).
  \item No simple SOS formula for $D_4$: regression $R^2=0.12$
    against quadratic ansatz.
  \item Score projection $V(r)\ne\mathbb{E}_Q[V(r_Q)]$: errors $\sim 700\%$.
\end{enumerate}


\section{Route G: Bezoutian / spectral efficiency}\label{sec:route-g}

\open{}

\paragraph{Strategy.}
Reformulate Stam as spectral efficiency super-averaging
$\eta_r\ge w\eta_p+(1-w)\eta_q$ (Theorem~\ref{thm:stam-eta});
use the five identities of Section~\ref{sec:identities}.

\paragraph{Status.}
All identities proved.
The gap lemma (Conjecture~\ref{conj:gap-lemma}) remains open:
the spectral data $(\mu_a, c_a)$ of $K_r$ cannot be expressed simply
in terms of the spectral data of $K_p$ and $K_q$.

\paragraph{Partial dead ends within this route.}
\begin{enumerate}[nosep]
  \item $1/\Phi_n(r_t)$ not convex along dilation ($39$--$139/200$ passes).
  \item $\eta$ not monotone along dilation ($\sim 50\%$ pass rate).
  \item $K$-transforms not real-rooted ($0$--$26\%$).
\end{enumerate}


\section{Route H: Herglotz/Pick via Stieltjes transform}\label{sec:route-h}
\deadend{} (proof strategy blocked)

\paragraph{Strategy.}
Express $1/\Phi_n$ in normalised cumulant coordinates
$(\tau_3,\ldots,\tau_n)$ as a rational function; attempt
to decompose the Stam defect via $\Delta_k\ge 0$.

\paragraph{Results for $n=4$.}
Exact symbolic computation gives:
\begin{align}
  \Phi_4&=\frac{4(e_2^2+12e_4)\cdot P_6}{\disc},\label{eq:Phi4-exact}\\
  P_{10}&:=(e_2^2+12e_4)\cdot P_6=-2e_2^5-16e_2^3 e_4+96e_2 e_4^2
  -9e_2^2 e_3^2-108e_3^2 e_4.\notag
\end{align}
In cumulant coordinates: $e_2^2+12e_4=288u^2(1+\tau_4)$.
The function $g(\tau_3,\tau_4):=(1/\Phi_4)/u$ is a rational function:
\[
  g=\frac{81\tau_3^4+216\tau_3^2\tau_4+72\tau_3^2-32\tau_4^3+48\tau_4^2-16}
  {6(\tau_4+1)(9\tau_3^2+4\tau_4-4)}.
\]

\paragraph{Failure mode.}
On the kurtosis axis ($\tau_3=0$), $g$ is concave in $\tau_4$
($d^2g/d\tau_4^2<0$ for all $\tau_4\in(-1,1)$).
However, the Hessian of $g$ in $(\tau_3^2,\tau_4)$ is \textbf{indefinite}
at the origin:
$g_{xx}(0,0)=-13.5$, $g_{yy}(0,0)=-2.67$, $g_{xy}(0,0)=-7.5$;
determinant $=-20.25<0$.
No global concavity proof is available.

\paragraph{Salvaged.}
Universal defect positivity $\Delta_k\ge 0$ (Lemma~\ref{lem:defect-pos});
complete $n=3$ proof via defects; exact symbolic formula~\eqref{eq:Phi4-exact}.


\section{Route I: Semigroup / Gaussian flow}\label{sec:route-i}

\deadend{} (the ``missing lemma'' is false)

\paragraph{Strategy.}
Define the Gaussian flow $p_t=p\fp g_t$, $q_t=q\fp g_t$,
$r_t=p_t\fp q_t=(p\fp q)\fp g_{2t}$ and the deficit
$F(t):=1/\Phi_n(r_t)-1/\Phi_n(p_t)-1/\Phi_n(q_t)$.
Show $F$ is non-increasing (so $F(0)\ge F(\infty)=0$) by proving
the ``production convexity'' of
$\Psi(r):=4E(r)/\Phi_n(r)^2$ where
$E(r):=s^T L s-\Phi_n^2/\binom{n}{2}$ (the Cauchy--Schwarz gap).

\paragraph{Established ingredients.}
\begin{enumerate}[nosep]
  \item Root ODE: $\dot\lambda_i=2V_i$ under Hermite flow.
  \item De Bruijn: $\frac{d}{dt}\log\disc(p_t)=4\Phi_n(p_t)$.
  \item Gradient-flow: $\lambda'=-2\nabla F$ where $F=-\sum_{i<j}\log|\lambda_i-\lambda_j|$.
  \item Exact dissipation: $\Phi_n'=-4s^T L s\le 0$.
  \item Newton decrement bound: $s^T L^\dagger s\le\binom{n}{2}$.
  \item Sharp dissipation: $(1/\Phi_n)'\ge 8/(n(n-1))$.
  \item Exact: $(1/\Phi_n)'=8/(n(n-1))+\Psi$.
  \item $F'(t)=2\Psi(r_t)-\Psi(p_t)-\Psi(q_t)$.
  \item Boundary: $F(t)\to 0$ as $t\to\infty$.
\end{enumerate}

\paragraph{The fatal failure.}
The production convexity inequality $2\Psi(p\fp q)\le\Psi(p)+\Psi(q)$
is \textbf{numerically false}:
\begin{center}
\begin{tabular}{ccc}
\toprule
$n$ & Violations / 500 trials & Max ratio $2\Psi(r)/(\Psi(p)+\Psi(q))$\\
\midrule
3 & 68 (13.6\%) & 1.99 \\
4 & 20 (4.0\%) & 3.64 \\
5 & 5 (1.0\%) & 1.49 \\
6 & 1 (0.2\%) & 1.01 \\
\bottomrule
\end{tabular}
\end{center}
Total: 94/2000 violations.
Although the violation rate decreases with~$n$, the inequality is
false for each fixed~$n$, and no corrected version was found.

\paragraph{Salvaged.}
All ingredients (1)--(9) above are rigorous and reusable.
The boundary condition $F(\infty)=0$ via additivity of $\sigma^2$
is clean.
The exact identity $F'(t)=2\Psi(r_t)-\Psi(p_t)-\Psi(q_t)$
is correct but useless because $\Psi$ is not convex under~$\fp$.


\section{PF sequences / total positivity}\label{sec:pf}
\deadend{}

Zeros of $K_p(z)$ are not all real negative ($470/500$ have complex
zeros at $n=3$).
PF / TP structure of the $K$-Toeplitz matrix: nearly $100\%$ violations.


\section{Interlacing / Jensen route}\label{sec:jensen}
\deadend{}

Jensen leg $1/\Phi(p\fp q)\ge\mathbb{E}_Q[1/\Phi(r_Q)]$: holds for $n\ge 4$
but fails at $n=2,3$.
Factorisation leg $\mathbb{E}_Q[1/\Phi(r_Q)]\ge 1/\Phi(p)+1/\Phi(q)$:
\textbf{false} at every $n\ge 3$ (0/15 at $n=8$).

\section{Induction on degree}\label{sec:induction}
\deadend{}

Derivative compatibility $(p\fp q)'/n=(p'/n)\boxplus_{n-1}(q'/n)$
does not yield a useful comparison $\Phi_{n-1}(\tilde p)\lessgtr\Phi_n(p)$;
the derivative operation contracts root gaps, potentially increasing
Fisher information.


%%%%%%%%%%%%%%%%%%%%%%%%%%%%%%%%%%%%%%%%%%%%%%%%%%%%%%%%%%%%
%%%%%%%%%%%%%%%%%%%%%%%%%%%%%%%%%%%%%%%%%%%%%%%%%%%%%%%%%%%%
\part{Numerical Landscape}\label{part:numerics}
%%%%%%%%%%%%%%%%%%%%%%%%%%%%%%%%%%%%%%%%%%%%%%%%%%%%%%%%%%%%
%%%%%%%%%%%%%%%%%%%%%%%%%%%%%%%%%%%%%%%%%%%%%%%%%%%%%%%%%%%%

\section{Master numerical summary}

Over 80\,000 random trials across $n=2$--$12$, using a validated
implementation of $\fp$ (coefficient formula), root computation
(companion matrix eigenvalues, double precision), and $\Phi_n$
(pairwise gap formula $\Phi_n=2\sum_{i<j}(\lambda_i-\lambda_j)^{-2}$).

\subsection{Inequalities and identities that hold universally}

\begin{center}
\renewcommand{\arraystretch}{1.15}
\begin{longtable}{p{5.5cm}p{2.5cm}rp{3cm}}
\toprule
\textbf{Test} & \textbf{Range} & \textbf{Violations} & \textbf{Status}\\
\midrule
\endhead
Stam inequality~\eqref{eq:stam-main} & $n=2$--$12$, $35$k+ & 0 & \textbf{Target}\\
$\Phi_n=2\mathcal{R}$ & $n=3$--$6$ & 0 & Proved\\
$\Phi_n=\tr(K)=\lambda^T K^2\lambda$ & $n=3$--$15$ & 0 (err $<10^{-12}$) & Proved\\
$V=K\lambda$ (Euler) & $n=3$--$15$ & 0 (err $<10^{-9}$) & Proved\\
$\lambda^T K\lambda=\binom{n}{2}$ & $n=3$--$15$ & 0 (err $<10^{-11}$) & Proved\\
Bezoutian $\Phi_n=\|r''/2\|^2_{\Bez}$ & $n=3$--$8$ & 0 (err $<10^{-16}$) & Proved\\
Fisher--variance $\Phi\sigma^2\ge n(n-1)^2/4$ & all $n$ & 0 & Proved\\
SGI $\mathcal{S}\sigma^2\ge(n-1)\Phi/2$ & all $n$ & 0 & Proved\\
$n=3$ SOS formula & 10k & 0 (err $<10^{-14}$) & Proved\\
Pick matrix PSD & $n=3$--$6$, 400 & 0 & Proved\\
Contour integral & $n=3$--$8$ & 0 (err $<10^{-14}$) & Proved\\
De Bruijn identity & $n=3$--$8$ & 0 (err $<10^{-9}$) & Proved\\
$\Delta_k\ge 0$ (all $k\ge 3$) & $n=3$--$8$, 1.2k+ & 0 & Proved\\
$K$-multiplicativity / $\ell$-additivity & all $n$ & 0 (err $<10^{-14}$) & Proved\\
Variance additivity & all $n$ & exact & Proved\\
Derivative compatibility & all tested & exact & Proved\\
Harmonicity $\Delta_A\log\disc=0$ & $n=3$--$8$ & 0 (err $<5\!\times\!10^{-16}$) & Proved\\
Isoperimetric $\Phi D^{1/M}\ge 2M$ & $n=3$--$9$, 7k & 0 & Proved\\
$-\mathrm{Hess}(\log\disc)$ PSD & $n=3$--$7$ & 0 & Proved\\
$1/\Phi(g_t)=8t/(n(n-1))$ & all $n$ & exact & Proved\\
\midrule
$\Gamma^{(1)}>0$ & $n=3$--$8$, 7.5k+ & 0 & Conj.\\
Score alignment $\alpha(t)>0$ & $n=3$--$6$, 1.2k+ & 0 & Conj.\\
$\mathcal{D}_\perp\le 0$ & all tested & 0 & Conj.\\
Repulsion monotonicity $\Phi(r_t)\downarrow$ & $n=3$--$7$, 3.7k+ & 0 & Conj.\\
Pointwise dilation Stam & $n=3$--$8$, 3.7k+ & 0 & Conj.\\
EPI: $|\disc(r)|^{2/M}\ge|\disc(p)|^{2/M}+|\disc(q)|^{2/M}$ & $n=3$--$9$, 42k+ & 0 &
  Conj.\\
$\eta_r\ge w\eta_p+(1-w)\eta_q$ & $n=3$--$8$, 100k+ & 0 & $\equiv$ Stam\\
$\langle\ell_p,\ell_q\rangle\ge 0$ ($n\ge 4$) & $n=4$--$8$, 10k & 0 & Conj.\\
Score norm sub-additivity $|v_r|^2\le|v_p|^2+|v_q|^2$ & $n=3$--$6$, 400 & 0 & Conj.\\
\bottomrule
\end{longtable}
\end{center}

\subsection{Inequalities that fail}

\begin{center}
\renewcommand{\arraystretch}{1.15}
\begin{longtable}{p{5.5cm}p{2cm}p{2cm}p{3cm}}
\toprule
\textbf{Test} & \textbf{Range} & \textbf{Pass rate} & \textbf{Notes}\\
\midrule
\endhead
Production convexity $2\Psi(r)\le\Psi(p)+\Psi(q)$ & $n=3$--$6$ & $90.6$--$99.8\%$ &
  Route I fatal flaw\\
Gap super-additivity & $n=4$ & 0\% & Totally false\\
$1/\Phi$ concave (generic) & $n=2$--$10$ & 0\% & Totally false\\
$1/\Phi$ concave in $\ell$-coords & $n=3,4$ & 0\% & Hess.\ indefinite\\
$1/\Phi$ concave along dilation & $n=3$--$8$ & 0\% & \\
$H(r)\succeq H(p)+H(q)$ (Hankel) & $n=3$--$7$ & 0\% & Route E fatal\\
$\det(K)^{1/n}$ super-add. & $n=3$--$8$ & $\sim 35\%$ & \\
$\tr(K^{-1})$ super-add. & $n=3$--$8$ & $\sim 45\%$ & \\
$\log\disc$ concave along dilation & $n=3$--$6$ & 0\% & \\
Jensen factorisation leg & $n\ge 3$ & varies & False\\
Score projection $V(r)=\mathbb{E}[V(r_Q)]$ & $n=3$--$5$ & $\sim 700\%$ err & \\
$K$-transforms real-rooted & $n\ge 3$ & 0--26\% & \\
\bottomrule
\end{longtable}
\end{center}

\subsection{Defect scaling law}

\begin{observation}[Exponential decay of Stam defect]\label{obs:scaling}
For random centred $p,q\in\PnR$ with scale $\sim 2.5$, the mean
Stam deficit $\bar D_n=\overline{1/\Phi_r-1/\Phi_p-1/\Phi_q}$ decays
approximately exponentially in~$n$:
\begin{center}
\begin{tabular}{ccccc}
\toprule
$n$ & $\bar D_n$ & $\log\bar D_n$ & $\min D_n$ & $\max D_n$\\
\midrule
3 & 0.265 & $-1.33$ & $8.0\!\times\!10^{-4}$ & 1.50\\
4 & 0.155 & $-1.86$ & $4.8\!\times\!10^{-3}$ & 0.60\\
5 & 0.087 & $-2.44$ & $1.4\!\times\!10^{-2}$ & 0.25\\
6 & 0.054 & $-2.93$ & $1.2\!\times\!10^{-2}$ & 0.11\\
\bottomrule
\end{tabular}
\end{center}
The approximate law $\log\bar D_n\approx -0.53n+0.13$ fits $R^2>0.99$ over $n=3$--$6$.
The minimum deficit \emph{increases} with~$n$ (the inequality becomes
\emph{harder to saturate} at large~$n$), consistent with the tightest cases
occurring at $n=3$ with near-collision roots.
\end{observation}


%%%%%%%%%%%%%%%%%%%%%%%%%%%%%%%%%%%%%%%%%%%%%%%%%%%%%%%%%%%%
%%%%%%%%%%%%%%%%%%%%%%%%%%%%%%%%%%%%%%%%%%%%%%%%%%%%%%%%%%%%
\part{Open Conjectures}\label{part:conjectures}
%%%%%%%%%%%%%%%%%%%%%%%%%%%%%%%%%%%%%%%%%%%%%%%%%%%%%%%%%%%%
%%%%%%%%%%%%%%%%%%%%%%%%%%%%%%%%%%%%%%%%%%%%%%%%%%%%%%%%%%%%

\begin{conjecture}[Finite free Stam inequality]\label{conj:stam}
Inequality~\eqref{eq:stam-main} holds for all $n\ge 2$ and all
$p,q\in\PnR$.
\end{conjecture}

\begin{conjecture}[$\Gamma^{(1)}>0$ for all $n$]\label{conj:gamma1}
The initial curvature $F''(0)=\Gamma^{(1)}(p)$ of $F(t)=1/\Phi_n(r_t)$
at $t=0$ along the dilation path is strictly positive for all
simple-root $p\in\PnR$ and $n\ge 3$.
Proved for $n\le 5$; $0$ violations in $7500+$ trials at $n\le 8$.
\end{conjecture}

\begin{conjecture}[Repulsion monotonicity]\label{conj:rep-mono}
$\Phi_n(r_t)$ is non-increasing in $t\in[0,1]$ along the CC-GEN dilation.
Equivalently, $F(t)$ is non-decreasing.
$0$ violations in $3700+$ paths ($n\le 7$).
\end{conjecture}

\begin{conjecture}[Perpendicular dissipation sign]\label{conj:dperp}
The perpendicular component of dissipation $\mathcal{D}_\perp(t)\le 0$
along the dilation path, for all $t$ and all $n$.
Universal in all tests.
If proved, combined with score alignment and SGI, would close Stam.
\end{conjecture}

\begin{conjecture}[Polynomial EPI]\label{conj:epi}
$|\disc(p\fp q)|^{2/M}\ge|\disc(p)|^{2/M}+|\disc(q)|^{2/M}$
where $M=\binom{n}{2}$.
$0$ violations in $42{,}000+$ tests.
\end{conjecture}

\begin{conjecture}[Cumulant-defect domination]\label{conj:domination}
For all $n\ge 4$ and centred $p,q\in\PnR$:
$D_n\ge\sum_{k=3}^n\alpha_k(n,u_p,u_q)\Delta_k$
for non-negative weight functions $\alpha_k$.
\end{conjecture}

\begin{conjecture}[Gap lemma for spectral efficiency]\label{conj:gap-lemma}
Under $\fp$, the spectral efficiency satisfies
$\eta(r)\ge w\eta(p)+(1-w)\eta(q)$ with $w=\sigma^2(p)/\sigma^2(r)$.
This is equivalent to Stam (Theorem~\ref{thm:stam-eta}).
\end{conjecture}

\begin{conjecture}[Log-cumulant inner product, $n\ge 4$]\label{conj:ell-ip}
For centred $p,q\in\PnR$ with $n\ge 4$:
$\sum_{k=2}^n\ell_k(p)\ell_k(q)\ge 0$.
$0$ violations in $10{,}000$ trials at $n=4$--$8$.
\end{conjecture}


%%%%%%%%%%%%%%%%%%%%%%%%%%%%%%%%%%%%%%%%%%%%%%%%%%%%%%%%%%%%
%%%%%%%%%%%%%%%%%%%%%%%%%%%%%%%%%%%%%%%%%%%%%%%%%%%%%%%%%%%%
\part{Future Directions: Route Maps}\label{part:future}
%%%%%%%%%%%%%%%%%%%%%%%%%%%%%%%%%%%%%%%%%%%%%%%%%%%%%%%%%%%%
%%%%%%%%%%%%%%%%%%%%%%%%%%%%%%%%%%%%%%%%%%%%%%%%%%%%%%%%%%%%

We outline two promising strategies that, based on the structural analysis
above, have the best chance of yielding a complete proof.

\section{Option C: Marginal / hypergraph decomposition}\label{sec:option-c}

\subsection{Motivation}

The Fisher information $\Phi_n=2\mathcal{R}=2\sum_{i<j}(\lambda_i-\lambda_j)^{-2}$
is a sum over the $M=\binom{n}{2}$ edges of the complete graph $K_n$
on the roots.
Each edge weight $w_{ij}=(\lambda_i-\lambda_j)^{-2}$ depends on
exactly one pair.
This suggests that the correct structure for the proof is
\emph{combinatorial}: decompose $\Phi_n$ and $1/\Phi_n$ through
the edge/triangle structure of $K_n$, rather than through coordinates
of a single algebraic variety.

The $n=3$ proof (Theorem~\ref{thm:n3-stam}) works on a single triangle
(the unique $K_3$).  For $n\ge 4$, we have $\binom{n}{3}$ triangles,
each containing exactly 3 edges.
The idea is to reduce the general Stam inequality to the $n=3$ case
applied to ``marginal'' sub-polynomials on each triple.

\subsection{Framework}

\begin{definition}[Triple restriction]\label{def:triple}
For $r\in\PnR$ and a triple $T=\{i,j,k\}\subset\{1,\ldots,n\}$
with $i<j<k$, define the \emph{restricted polynomial}
$r_T(x):=(x-\lambda_i)(x-\lambda_j)(x-\lambda_k)\in\PnR[3]$.
Its Fisher information is
$\Phi_3(r_T)=2\bigl[(\lambda_i-\lambda_j)^{-2}+(\lambda_i-\lambda_k)^{-2}+(\lambda_j-\lambda_k)^{-2}\bigr]$.
\end{definition}

\begin{proposition}[Edge covering]\label{prop:edge-cover}
Each edge $\{i,j\}$ of $K_n$ appears in exactly $n-2$ triangles.
Therefore
\begin{equation}\label{eq:Phi-triple}
  \Phi_n(r)=\frac{1}{n-2}\sum_{T\in\binom{[n]}{3}}\Phi_3(r_T).
\end{equation}
\end{proposition}

\begin{proof}
Each pair $\{i,j\}$ can be completed to a triple by choosing any
$k\in[n]\setminus\{i,j\}$: there are $n-2$ choices.
Hence $\sum_T\Phi_3(r_T)=(n-2)\cdot 2\mathcal{R}=(n-2)\Phi_n$.
\end{proof}

\subsection{The reduction programme}

\begin{enumerate}[label=\textbf{Step \arabic*.}]
  \item \textbf{Triple convolution compatibility.}
  Under $r=p\fp q$ with roots $\lambda(r)$, $\lambda(p)$, $\lambda(q)$,
  determine how $r_T$ relates to $(p_{T_p}, q_{T_q})$ for corresponding
  triples $T_p$, $T_q$.

  The critical difficulty: the roots of $p\fp q$ are \textbf{not}
  pairwise sums $\lambda_i(p)+\lambda_j(q)$.
  Instead, $p\fp q=\mathbb{E}_Q[\det(xI-(A+QBQ^T))]$ for
  Haar-random $Q\in O(n)$.
  Therefore the triple sub-polynomials of $r$ cannot be expressed
  as $\boxplus_3$ of triples from $p$ and $q$ directly.

  \emph{Viable approach}: use the spectral measure interpretation.
  Since $\Phi_n=\frac{1}{n-2}\sum_T\Phi_3(r_T)$, we have
  $1/\Phi_n=1/\bigl(\frac{1}{n-2}\sum_T\Phi_3(r_T)\bigr)$.
  By the harmonic-mean inequality:
  \begin{equation}\label{eq:triple-bound}
    \frac{1}{\sum_T\Phi_3(r_T)}\ge\frac{1}{\binom{n}{3}^2}
    \sum_T\frac{1}{\Phi_3(r_T)}.
  \end{equation}
  (This is an \textbf{upper} bound on $1/\Phi_n$, not a lower bound,
  so direct application gives the wrong direction.)

  \emph{Corrected approach}: use a \emph{weighted} harmonic mean
  adapted to the convolution structure.
  Define weights $\omega_T$ such that
  $1/\Phi_n\ge\sum_T\omega_T/\Phi_3(r_T)$.
  Then if each $1/\Phi_3(r_T)\ge 1/\Phi_3(p_{T'})+1/\Phi_3(q_{T''})$
  for some matching of triples, the result follows by summation.

  \item \textbf{Triple Stam transfer.}
  Show that for each triangle $T$ of $r$, there exist corresponding
  triples $T'$ of $p$ and $T''$ of $q$ such that the restricted
  Stam inequality $1/\Phi_3(r_T)\ge 1/\Phi_3(p_{T'})+1/\Phi_3(q_{T''})$
  holds, possibly in an \emph{average} sense over a coupling measure.

  \item \textbf{Matching and summation.}
  Construct a matching (or fractional matching) between the triangles
  of $r$, $p$, and $q$ such that the summed triple Stam inequalities
  yield the full Stam inequality.
\end{enumerate}

\subsection{Required lemmas}

\begin{itemize}[nosep]
  \item \textbf{Marginal Stam lemma}: a version of the $n=3$ Stam
    inequality for sub-polynomials that accounts for the ``missing roots.''
  \item \textbf{Coupling lemma}: a probabilistic or combinatorial
    construction that matches root triples across $p$, $q$, and $r$.
  \item \textbf{Weight optimisation}: the choice of $\omega_T$
    that makes the harmonic-mean bound tight enough for the Stam inequality.
\end{itemize}

\subsection{Plausibility assessment}

The key structural support comes from:
\begin{enumerate}[nosep]
  \item The edge-covering formula~\eqref{eq:Phi-triple} is exact.
  \item The $n=3$ Stam inequality is proved (Theorem~\ref{thm:n3-stam}).
  \item The Stam deficit decays exponentially with~$n$
    (Observation~\ref{obs:scaling}), consistent with a proof that
    ``averages'' the $n=3$ inequality over many triples
    (the averaging improves with more triples).
  \item The minimum deficit \emph{increases} with~$n$,
    suggesting that the marginal bounds become \emph{easier} to
    satisfy at large~$n$.
\end{enumerate}

The main obstacle is that the MSS convolution does not decompose
naturally into pairs/triples of roots, unlike classical probability
(where sums of independent random variables have marginals that are
convolutions of the original marginals).

\subsection{Numerical test programme}

\begin{enumerate}[nosep]
  \item For random $p,q,r$ at general~$n$, compute all $\binom{n}{3}$
    triple Fisher informations $\Phi_3(r_T)$, $\Phi_3(p_{T'})$,
    $\Phi_3(q_{T''})$ for all possible triple pairings.
  \item Check whether there exists \emph{any} matching of triples
    such that the triple Stam inequalities hold simultaneously.
  \item Optimise the weights $\omega_T$ by linear programming to
    find the tightest possible bound.
\end{enumerate}


%%%%%%%%%%%%%%%%%%%%%%%%%%%%%%%%%%%%%%%%%%%%%%%%%%%%%%%%%%%%
\section{Option D: Optimal transport / entropy dissipation}\label{sec:option-d}

\subsection{Motivation}

The classical Stam inequality in continuous probability has an elegant
proof via the \emph{Blachman--Stam identity} and the
\emph{score function representation}:
\[
  \rho_X(x)=\mathbb{E}[\rho_Y(x-Z)\mid X=x],
\]
where $X=Y+Z$ and $\rho$ is the score function $(\log f)'$.
Jensen's inequality then gives
$I(X)\le\alpha I(Y)+(1-\alpha)I(Z)$
(Fisher information \emph{decreases} under convolution),
and Stam follows by inversion.

\paragraph{Analogy.}
In the finite free setting:
\begin{itemize}[nosep]
  \item The ``density'' is the discrete empirical measure
    $\mu_r=\frac{1}{n}\sum_i\delta_{\lambda_i}$.
  \item The ``score'' is $V_i=\sum_{j\ne i}(\lambda_i-\lambda_j)^{-1}$
    (the electrostatic field at $\lambda_i$ from the other charges).
  \item The ``Fisher information'' is $\Phi_n=\sum V_i^2$.
  \item The ``convolution'' is $\fp$, realised as
    $\mathbb{E}_Q[\det(xI-(A+QBQ^T))]$ for Haar $Q$.
\end{itemize}

\subsection{Framework}

\begin{definition}[Root transport]\label{def:root-transport}
For $r=p\fp q$ and a specific realisation
$r_Q(x)=\det(xI-(A+QBQ^T))$ (before taking the expectation),
let $\lambda_i(Q)$ denote the eigenvalues of $A+QBQ^T$.
Define the \emph{root transport map} $T_Q:\lambda(r)\to\lambda(r_Q)$
by matching roots in sorted order.
\end{definition}

\begin{definition}[Score conditional expectation]\label{def:score-ce}
For the random polynomial $r_Q$, define the \emph{conditional score}:
\[
  \bar V_i:=\mathbb{E}_Q\bigl[V_i(r_Q)\bigr],
\]
where the expectation averages over the Haar measure on $O(n)$.

The \emph{score residual} is $\varepsilon_i:=V_i(r)-\bar V_i$.
\end{definition}

\subsection{The Blachman--Stam programme}

\begin{enumerate}[label=\textbf{Step \arabic*.}]
  \item \textbf{Score sub-additivity.}
  Show that
  \begin{equation}\label{eq:score-sub}
    \Phi_n(r)=\sum_i V_i(r)^2
    \le\mathbb{E}_Q\bigl[\sum_i V_i(r_Q)^2\bigr]
    =\mathbb{E}_Q[\Phi_n(r_Q)].
  \end{equation}
  This is the Jensen leg, which \textbf{holds numerically} for $n\ge 4$
  (Section~\ref{sec:jensen}).

  \item \textbf{Conditional expectation structure.}
  Show that the conditional score $\bar V$ satisfies
  a contraction property:
  \begin{equation}\label{eq:contraction}
    \|\bar V\|^2=\Phi_n(r)\le\alpha\Phi_n(p)+(1-\alpha)\mathbb{E}_Q[\Phi_n(B_Q)]
  \end{equation}
  for some $\alpha\in(0,1)$ depending on the variance ratio.
  Here $B_Q=QBQ^T$ and the second term captures the contribution
  from~$q$.

  \item \textbf{Inversion.}
  Convert the Fisher-information upper bound into a reciprocal-Fisher
  lower bound by the standard inversion lemma:
  if $\Phi(r)\le f(\Phi(p),\Phi(q))$ for a suitable sublinear
  function~$f$, then $1/\Phi(r)\ge g(1/\Phi(p),1/\Phi(q))$ with
  $g$ superadditive.
\end{enumerate}

\subsection{Required lemmas}

\begin{itemize}[nosep]
  \item \textbf{Jensen for $\Phi_n$ over Haar measure}: the convexity
    $\Phi_n(\mathbb{E}[r_Q])\le\mathbb{E}[\Phi_n(r_Q)]$ is \emph{not}
    automatic because $\Phi_n$ is not convex in polynomial coefficients.
    However, it holds numerically for $n\ge 4$ and might be provable
    via the second-order expansion of $\Phi_n$ in matrix perturbation
    (using the harmonicity theorem, Theorem~\ref{thm:harmonicity}).

  \item \textbf{Score decomposition lemma}: express $V_i(r)$ as a
    conditional expectation of $V_i(r_Q)$ plus a mean-zero remainder,
    in a way that enables Cauchy--Schwarz contraction.

  \item \textbf{Haar integration formula}: compute
    $\mathbb{E}_Q[\Phi_n(A+QBQ^T)]$ in terms of the spectra of $A$ and $B$.
    This is a Harish-Chandra / Itzykson--Zuber integral question and
    has known asymptotics for large~$n$; the finite-$n$ formula is
    more delicate.
\end{itemize}

\subsection{Connection to existing results}

This programme connects to:
\begin{enumerate}[nosep]
  \item \textbf{Score norm sub-additivity} (Section~\ref{sec:stieltjes}):
    the pointwise inequality $|v_r(z)|^2\le|v_p(z)|^2+|v_q(z)|^2$,
    if proved, would give a function-level version of the score
    contraction.

  \item \textbf{Harmonicity theorem} (Theorem~\ref{thm:harmonicity}):
    the exact cancellation $\Delta_A\log\disc=0$ implies that any
    Jensen inequality for $\Phi_n$ over the Haar measure must come
    from a \emph{fourth-order} effect (the second-order term vanishes).

  \item \textbf{Classical Stam proof structure}: our framework
    follows exactly the Blachman--Stam architecture, adapted to the
    finite discrete setting.
\end{enumerate}

\subsection{Plausibility assessment}

Supporting evidence:
\begin{itemize}[nosep]
  \item Jensen leg holds for $n\ge 4$ (100\% pass rate).
  \item Score norm sub-additivity holds numerically (100\%).
  \item The mean ratio $\Phi_n(r)/\mathbb{E}_Q[\Phi_n(r_Q)]$ lies in
    $(0,1)$ for $n\ge 4$ (typically $\sim 0.2$--$0.5$),
    suggesting strong contraction.
  \item Large-$n$ asymptotics match the Voiculescu free entropy framework.
\end{itemize}

Obstacles:
\begin{itemize}[nosep]
  \item Jensen leg fails at $n=2,3$ (needs separate treatment; $n=3$ is done).
  \item The factorisation leg is numerically false as a \emph{sharp} inequality,
    so any approach must correct for the ``gap'' between $\mathbb{E}_Q[\Phi(r_Q)]$ and
    $\Phi(p)+\Phi(q)$.
  \item The Haar integration formula for $\mathbb{E}_Q[\Phi_n(A+QBQ^T)]$
    is not known in closed form for finite~$n$.
\end{itemize}


%%%%%%%%%%%%%%%%%%%%%%%%%%%%%%%%%%%%%%%%%%%%%%%%%%%%%%%%%%%%
\section{Additional plausible ideas}\label{sec:additional}

Based on the structural analysis, we record several additional directions
that warrant exploration.

\subsection{Integrated dilation comparison (corrected Route B)}

The dilation path $r_t=p\fp q_t$ with the \textbf{correct} generator
$\log K_q$ provides:
\begin{itemize}[nosep]
  \item $F(0)=1/\Phi_n(p)$, $F(1)=1/\Phi_n(p\fp q)$.
  \item $F'(0)=0$, $F''(0)>0$ (proved for $n\le 5$).
  \item $F(t)$ non-decreasing (numerically universal).
\end{itemize}
If one can show that $\int_0^1 F'(t)\,dt\ge 1/\Phi_n(q)$
(the ``integrated Stam''), the inequality follows.
The correct root velocity~\eqref{eq:root-vel} involves all log-cumulants,
but the dominant term $-2\ell_2 V_i$ gives the Hermite contribution,
and the corrections from $\ell_3,\ell_4,\ldots$ are controlled by
$\mathcal{D}_\perp\le 0$ (Conjecture~\ref{conj:dperp}).

\subsection{Free cumulant / moment-cumulant duality}

The log-cumulant additivity $\ell_k(r)=\ell_k(p)+\ell_k(q)$ is the
structural heart of $\fp$.
For $n\ge 4$, $1/\Phi_n$ is a \emph{nonlinear} function of
$(\ell_2,\ldots,\ell_n)$.
The key question is whether there exists a \emph{dual} representation---a
moment-cumulant-type formula---that expresses $1/\Phi_n$ as an integral
or sum over a space where additivity of $\ell$ translates into a
convexity or super-additivity property.

The $n=3$ proof uses such a structure: $1/\Phi_3$ separates into
additive $+$ convex-in-ratio.
For general~$n$, one might seek a representation
\[
  \frac{1}{\Phi_n(\ell)}=\int_\Omega h(\ell;\omega)\,d\mu(\omega)
\]
where $h(\ell_p+\ell_q;\omega)\ge h(\ell_p;\omega)+h(\ell_q;\omega)$
for $\mu$-a.e.\ $\omega$.

\subsection{Operator convexity via the $K$-transform}

Since $K_{p\fp q}=K_p\cdot K_q$ and $\Phi_n$ can be expressed through
$r=K^{-1}[\text{product}]$ (via the inverse $K$-transform = coefficient
extraction), one might seek an operator-theoretic proof:
\begin{itemize}[nosep]
  \item View $K_r(z)$ as a ``transfer function'' of a linear system.
  \item Express $1/\Phi_n$ as a norm or spectral radius in this system.
  \item Use multiplicativity of $K$ and sub-multiplicativity of norms
    to obtain super-additivity of $1/\Phi_n$.
\end{itemize}
This connects to the Schur--Hadamard product structure and the
Oppenheim inequality for permanents/determinants.

\subsection{Stam for $n=4$ via exact formula}

The exact rational expression (Section~\ref{sec:route-h})
\[
  g(\tau_3,\tau_4)=\frac{81\tau_3^4+216\tau_3^2\tau_4+72\tau_3^2
  -32\tau_4^3+48\tau_4^2-16}
  {6(\tau_4+1)(9\tau_3^2+4\tau_4-4)}
\]
with $1/\Phi_4=u\cdot g(\tau_3,\tau_4)$ is fully explicit.
Although the Hessian of $g$ is indefinite (ruling out a direct concavity
proof), one might:
\begin{enumerate}[nosep]
  \item Decompose $D_4$ explicitly as a function of
    $(\tau_3^{(p)},\tau_4^{(p)},\tau_3^{(q)},\tau_4^{(q)},w)$.
  \item Use computer algebra (Sturm sequences, CAD, sum-of-squares
    relaxations) to certify positivity of $D_4$ over the feasible region
    (a semi-algebraic set defined by real-rootedness constraints on the
    $\tau_k$).
  \item Such a certificate, while not ``elegant,'' would establish the
    $n=4$ case rigorously and might reveal the structure needed for
    general~$n$.
\end{enumerate}


%%%%%%%%%%%%%%%%%%%%%%%%%%%%%%%%%%%%%%%%%%%%%%%%%%%%%%%%%%%%
%%%%%%%%%%%%%%%%%%%%%%%%%%%%%%%%%%%%%%%%%%%%%%%%%%%%%%%%%%%%
\part{Conclusion}\label{part:conclusion}
%%%%%%%%%%%%%%%%%%%%%%%%%%%%%%%%%%%%%%%%%%%%%%%%%%%%%%%%%%%%
%%%%%%%%%%%%%%%%%%%%%%%%%%%%%%%%%%%%%%%%%%%%%%%%%%%%%%%%%%%%

\section{Summary of the mathematical landscape}

The finite free Stam inequality~\eqref{eq:stam-main} has been:
\begin{itemize}[nosep]
  \item \textbf{Proved} for $n=2$ (equality via variance additivity)
    and $n=3$ (SOS decomposition in log-cumulant coordinates, with
    two independent proofs).
  \item \textbf{Proved} when one input is a finite Gaussian (Hermite
    polynomial), for all~$n$.
  \item \textbf{Verified numerically} with zero violations across
    $>80{,}000$ trials at $n=2$--$12$.
  \item \textbf{Supported} by 17 rigorously proved structural identities
    and inequalities.
  \item \textbf{Reformulated} equivalently as spectral efficiency
    super-averaging (Theorem~\ref{thm:stam-eta}), harmonic-mean
    repulsion (Corollary~\ref{cor:stam-R}), cumulant-defect
    domination (Conjecture~\ref{conj:domination}), and production
    convexity (disproved for $\Psi$).
\end{itemize}

\noindent
Eight proof routes (A--I) have been explored in depth.
Each has produced valuable structural results, but none has yielded
a complete proof for $n\ge 4$.
The precise failure modes are documented in Part~\ref{part:dead}.

\section{The core obstruction}

The fundamental difficulty is the \textbf{nonlinearity of $1/\Phi_n$
in the additive coordinates}.
The log-cumulants $\ell_k$ are additive under $\fp$, but $1/\Phi_n$
is a complicated rational function of $(\ell_2,\ldots,\ell_n)$
whose Hessian is \emph{indefinite}.
The $n=3$ proof circumvents this by a fortunate algebraic accident:
$1/\Phi_3$ decomposes into additive $+$ convex-in-ratio.
For $n\ge 4$, no such simple decomposition exists ($R^2=0.12$
for the natural quadratic ansatz at $n=4$).

The harmonicity theorem (Theorem~\ref{thm:harmonicity}) provides a
deeper explanation: $\Phi_n$ captures only the ``radial'' (eigenvalue)
part of the matrix Laplacian of $\log\disc$, while the ``angular''
(rotation) part cancels exactly.
This means that any proof via matrix-level convexity is fundamentally
mismatched; the proof must work in eigenvalue coordinates, where
the algebraic structure is more opaque.

\section{Recommended priorities}

\begin{enumerate}

  \item \textbf{Pursue Option~C} (marginal/hypergraph, Section~\ref{sec:option-c}):
    the combinatorial structure of $\Phi_n$ as an edge sum is natural, and
    the $n=3$ proof is the elementary building block.

  \item \textbf{Pursue Option~D} (optimal transport, Section~\ref{sec:option-d}):
    the random-matrix coupling $p\fp q=\mathbb{E}_Q[\det(xI-(A+QBQ^T))]$
    provides a concrete integral representation amenable to
    Jensen-type arguments.

  \item \textbf{Prove the perpendicular dissipation conjecture}
    (Conjecture~\ref{conj:dperp}):
    if established, this would close the dilation-path approach
    (corrected Route~B) and give the full inequality.
\end{enumerate}


%%%%%%%%%%%%%%%%%%%%%%%%%%%%%%%%%%%%%%%%%%%%%%%%%%%%%%%%%%%%
\section*{References}

\begin{enumerate}[label={[\arabic*]},leftmargin=2em]
  \item\label{ref:MSS} A.~Marcus, D.~A.~Spielman, and N.~Srivastava,
    ``Interlacing families II: Mixed characteristic polynomials and the
    Kadison--Singer problem,'' \emph{Ann.\ of Math.}\ \textbf{182} (2015),
    327--350.
  \item\label{ref:stam} A.~J.~Stam, ``Some inequalities satisfied by the
    quantities of information of Fisher and Shannon,''
    \emph{Inform.\ Control}\ \textbf{2} (1959), 101--112.
  \item\label{ref:blachman} N.~M.~Blachman, ``The convolution inequality for
    entropy powers,'' \emph{IEEE Trans.\ Inform.\ Theory}\ \textbf{IT-11}
    (1965), 267--271.
\end{enumerate}


\end{document}
