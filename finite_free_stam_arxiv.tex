\documentclass[11pt]{amsart}

\usepackage[margin=1in]{geometry}
\usepackage[T1]{fontenc}
\usepackage{lmodern}
\usepackage{microtype}
\usepackage{amsmath,amssymb,amsthm}
\usepackage{mathtools}
\usepackage[colorlinks=true,linkcolor=blue,citecolor=blue,urlcolor=blue]{hyperref}
\usepackage{enumitem}

\allowdisplaybreaks
\setlength{\jot}{8pt}

\newtheorem{theorem}{Theorem}[section]
\newtheorem{lemma}[theorem]{Lemma}
\newtheorem{proposition}[theorem]{Proposition}
\newtheorem{corollary}[theorem]{Corollary}
\theoremstyle{definition}
\newtheorem{definition}[theorem]{Definition}
\newtheorem{conjecture}[theorem]{Conjecture}
\theoremstyle{remark}
\newtheorem{remark}[theorem]{Remark}

\newcommand{\R}{\mathbb{R}}
\newcommand{\C}{\mathbb{C}}
\newcommand{\Pn}{\mathcal{P}_n}
\newcommand{\PnR}{\mathcal{P}_n^{\R}}

\title[Finite free Stam inequality]{Finite free Stam inequality via score-gradient bounds and dilation interpolation:
rigorous partial results, and a counterexample to a convexity heuristic}
\author{}
\date{}

\begin{document}
\begin{abstract}
We study the finite free Stam program for the MSS convolution $\boxplus_n$.
For monic real-rooted degree-$n$ polynomials, the target inequality is
\[
  \frac{1}{\Phi_n(p\boxplus_n q)}\ge \frac{1}{\Phi_n(p)}+\frac{1}{\Phi_n(q)}.
\]
We prove complete low-degree and flow results: the Score-Gradient Inequality,
the Hermite-flow bound, and Stam for $n=2$ (equality) and $n=3$.
For general $n$, we show why a natural dilation-convexity mechanism fails
(explicit counterexample), then develop two alternative frameworks:
Route~B (dilation dissipation decomposition) and Route~C (variational/transport).
In Route~C we prove a finite-free de Bruijn identity, an AM-GM isoperimetric inequality,
and displacement convexity of repulsion, and formulate a discriminant power inequality
(finite-free EPI), proved for $n\le 3$ and strongly supported numerically.

A key new result is the identity $\Gamma^{(1)}=\mathcal{S}/2$
(the weighted score-gap functional equals half the score-gradient energy),
proved via a Hermite-flow chain rule.
This settles the conjecture $\Gamma^{(1)}>0$ for all~$n$ by a three-line
argument, establishes universal positive initial curvature of the dilation excess,
and provides quantitative lower bounds.
We also prove derivative compatibility
$\widetilde{p\boxplus_n q}=\tilde p\boxplus_{n-1}\tilde q$
and systematically catalogue six dead-end strategies.
The paper concludes with a sharpened roadmap whose central open sub-problem
is the sign of the perpendicular dissipation along the dilation path.
\end{abstract}

\maketitle
\tableofcontents

%======================================================================
\section{Setup}
%======================================================================

\subsection{Real-rooted polynomials and convolution}

\begin{definition}[Real-rooted polynomials]
Fix $n\ge 2$. Let $\PnR$ denote the set of monic degree-$n$ polynomials with all roots real.
Every $p\in\PnR$ can be written
\[
  p(x)=\prod_{i=1}^n(x-\lambda_i)=\sum_{k=0}^n a_k\,x^{n-k}
\]
with $\lambda_1\le\cdots\le\lambda_n$ (roots listed with multiplicity).
We say $p$ has \emph{simple roots} if all inequalities are strict.
In the sequel, we shall always state explicitly whenever simple roots are required.
\end{definition}

\begin{definition}[Symmetric additive convolution]
\label{def:conv}
For $p(x)=\sum_{k=0}^n a_k x^{n-k}$ and $q(x)=\sum_{k=0}^n b_k x^{n-k}$ in $\PnR$,
set $r=p\boxplus_n q$ to be the monic degree-$n$ polynomial with coefficients
\[
  r(x)=\sum_{k=0}^n c_k x^{n-k},\qquad
  c_k=\sum_{i+j=k}\frac{(n-i)!\,(n-j)!}{n!\,(n-k)!}\,a_i b_j.
\]
Equivalently (MSS), writing
\[
  T_q:=\sum_{k=0}^n\frac{(n-k)!}{n!}\,b_k\,\partial_x^k,
\]
one has $p\boxplus_n q=T_q p$.
\end{definition}

\begin{theorem}[Marcus--Spielman--Srivastava]
\label{thm:mss}
If $p,q\in\PnR$, then $p\boxplus_n q\in\PnR$. Moreover, $\boxplus_n$ is commutative.
\end{theorem}

\subsection{Scores, Fisher information, and variance}

\begin{definition}[Scores and Fisher information]
\label{def:fisher}
Let $p\in\PnR$ have distinct roots $\lambda_1<\cdots<\lambda_n$.
Define the \emph{score} at $\lambda_i$ and the \emph{finite free Fisher information} by
\[
  V_i:=\sum_{j\ne i}\frac{1}{\lambda_i-\lambda_j},\qquad
  \Phi_n(p):=\sum_{i=1}^n V_i^2.
\]
If $p$ has a repeated root, set $\Phi_n(p):=\infty$ (equivalently $1/\Phi_n(p):=0$).
\end{definition}

\begin{definition}[Score-gradient energy]
\label{def:score-grad}
\[
  \mathcal{S}(p):=\sum_{i<j}\frac{(V_i-V_j)^2}{(\lambda_i-\lambda_j)^2}.
\]
\end{definition}

\begin{definition}[Variance]
\label{def:var}
Let $\bar\lambda:=\frac{1}{n}\sum_{i=1}^n\lambda_i$.
Define
\[
  \sigma^2(p):=\frac{1}{n}\sum_{i=1}^n(\lambda_i-\bar\lambda)^2.
\]
\end{definition}

\begin{remark}[Affine invariances]
$\Phi_n$ and $\mathcal{S}$ are translation-invariant.
Under dilation $p(x)\mapsto p_t(x)=t^{-n}p(tx)$ (i.e. roots scale by $t$), the scores scale as $V_i\mapsto V_i/t$, hence
$\Phi_n\mapsto \Phi_n/t^2$.
\end{remark}

\begin{lemma}[Translation covariance]
\label{lem:translation}
For $c\in\R$ and a monic polynomial $p$, write $(\tau_c p)(x):=p(x-c)$.
Then for all monic degree-$n$ polynomials $p,q$,
\[
  \tau_a p\boxplus_n \tau_b q=\tau_{a+b}(p\boxplus_n q).
\]
In particular, since scores depend only on root differences,
$\Phi_n(\tau_c p)=\Phi_n(p)$ and $\sigma^2(\tau_c p)=\sigma^2(p)$.
\end{lemma}

\begin{proof}
For $p(x)=\sum_{k=0}^n a_k x^{n-k}\in\PnR$, define the \emph{normalized generating function}
\[
  K_p(z):=\sum_{k=0}^n \frac{a_k}{\binom{n}{k}}\,z^k,
\]
so that $K_{p\boxplus_n q}(z)=K_p(z)\,K_q(z)\pmod{z^{n+1}}$
(see~\cite{MSS15}).
Translation by $c$ replaces each root $\lambda_i$ by $\lambda_i+c$.
Since $K_p$ is binomial-normalized, the shift identity
$K_{\tau_c p}(z)=e^{cz}K_p(z)\pmod{z^{n+1}}$
is exactly the standard MSS binomial-convolution relation (see~\cite{MSS15}).
Therefore
$K_{\tau_a p\boxplus_n \tau_b q}=e^{(a+b)z}K_pK_q=K_{\tau_{a+b}(p\boxplus_n q)}\pmod{z^{n+1}}$,
and since $K$ determines the monic polynomial, the identity follows.
The invariance statements follow from the definitions.
\end{proof}

%======================================================================
\section{Preliminary identities}
%======================================================================

\medskip
\noindent\textbf{Standing hypothesis for Section~2.}
Throughout this section, $p\in\PnR$ has simple (i.e.\ distinct) roots $\lambda_1<\cdots<\lambda_n$
and scores $V_i$ as in Definition~\ref{def:fisher}.

\begin{lemma}[Score--derivative relation]
\label{lem:score-deriv}
\[
  V_i=\frac{p''(\lambda_i)}{2\,p'(\lambda_i)}.
\]
\end{lemma}

\begin{proof}
Since $p'(\lambda_i)=\prod_{j\ne i}(\lambda_i-\lambda_j)$, differentiating
$p'(x)=\sum_{i=1}^n\prod_{j\ne i}(x-\lambda_j)$ and evaluating at $x=\lambda_i$ gives
\[
  p''(\lambda_i)=2\,p'(\lambda_i)\sum_{k\ne i}\frac{1}{\lambda_i-\lambda_k}=2\,p'(\lambda_i)\,V_i.\qedhere
\]
\end{proof}

\begin{lemma}[Score identities]
\label{lem:score-id}
\begin{enumerate}[label=\textup{(\roman*)},nosep]
  \item\label{it:score-sum} $\sum_i V_i=0$.
  \item\label{it:score-root} $\sum_i \lambda_i V_i=\binom{n}{2}$.
  \item\label{it:score-centered} $\sum_i (\lambda_i-\bar\lambda)V_i=\binom{n}{2}$.
  \item\label{it:score-gap} $\Phi_n(p)=\sum_{i<j}\frac{V_i-V_j}{\lambda_i-\lambda_j}$.
\end{enumerate}
\end{lemma}

\begin{proof}
(i) is antisymmetry of $(\lambda_i-\lambda_j)^{-1}$ in $(i,j)$.

(ii) Pair $(i,j)$ and $(j,i)$:
$\frac{\lambda_i}{\lambda_i-\lambda_j}+\frac{\lambda_j}{\lambda_j-\lambda_i}=1$.

(iii) follows from (ii) and (i).

(iv) Write $\Phi_n(p)=\sum_i V_i^2=\sum_i V_i\sum_{j\ne i}(\lambda_i-\lambda_j)^{-1}$.
Pairing $(i,j)$ and $(j,i)$ gives $\frac{V_i-V_j}{\lambda_i-\lambda_j}$, and summing over $i<j$ yields the claim.
\end{proof}

\begin{lemma}[Variance via coefficients]
\label{lem:var-coeff}
If $p(x)=\sum_{k=0}^n a_k x^{n-k}$, then
\[
  \sigma^2(p)=\frac{(n-1)a_1^2}{n^2}-\frac{2a_2}{n}.
\]
\end{lemma}

\begin{proof}
By Vieta, $\sum_i\lambda_i=-a_1$ and $\sum_{i<j}\lambda_i\lambda_j=a_2$.
Thus $\sum_i\lambda_i^2=a_1^2-2a_2$, and
$\sigma^2=\frac{1}{n}\sum_i\lambda_i^2-\bar\lambda^2$ with $\bar\lambda=-a_1/n$.
\end{proof}

\begin{lemma}[Variance additivity]
\label{lem:var-add}
$\sigma^2(p\boxplus_n q)=\sigma^2(p)+\sigma^2(q)$.
\end{lemma}

\begin{proof}
From Definition~\ref{def:conv},
$c_1=a_1+b_1$ and $c_2=a_2+\frac{n-1}{n}a_1 b_1+b_2$.
Plugging into Lemma~\ref{lem:var-coeff} and expanding $(a_1+b_1)^2$ shows cross terms cancel.
\end{proof}

%======================================================================
\section{Fisher--variance and the Score-Gradient Inequality}
%======================================================================

\begin{lemma}[Fisher--variance inequality]
\label{lem:FV}
\[
  \Phi_n(p)\,\sigma^2(p)\ge \frac{n(n-1)^2}{4}.
\]
Equality holds iff $V_i=c(\lambda_i-\bar\lambda)$ for some constant $c$.
\end{lemma}

\begin{proof}
By Lemma~\ref{lem:score-id}~\eqref{it:score-centered},
$\sum_i(\lambda_i-\bar\lambda)V_i=\frac{n(n-1)}{2}$.
Apply Cauchy--Schwarz:
\[
  \Bigl(\sum_i(\lambda_i-\bar\lambda)V_i\Bigr)^2
  \le \Bigl(\sum_i(\lambda_i-\bar\lambda)^2\Bigr)\Bigl(\sum_i V_i^2\Bigr)
  = n\,\sigma^2(p)\,\Phi_n(p).\qedhere
\]
\end{proof}

\begin{theorem}[Score-Gradient Inequality]
\label{thm:sgi}
For $p\in\PnR$ with distinct roots,
\begin{equation}\label{eq:sgi}
  \mathcal{S}(p)\,\sigma^2(p)\ge \frac{n-1}{2}\,\Phi_n(p).
\end{equation}
Equality holds iff $V_i=c(\lambda_i-\bar\lambda)$ for some constant $c$.
\end{theorem}

\begin{proof}
Set $T:=n\sigma^2(p)$, $U:=\Phi_n(p)$, $S:=\mathcal{S}(p)$.
We show $ST\ge \frac{n(n-1)}{2}U$.

First, Lemma~\ref{lem:score-id}~\eqref{it:score-centered} and Cauchy--Schwarz give
$\frac{n^2(n-1)^2}{4}\le TU$.
Second, Lemma~\ref{lem:score-id}~\eqref{it:score-gap} writes $U=\sum_{i<j}a_{ij}$
with $a_{ij}:=(V_i-V_j)/(\lambda_i-\lambda_j)$.
Applying Cauchy--Schwarz to the vector $(a_{ij})_{i<j}$ and the all-ones vector gives
$U^2\le \binom{n}{2}\sum_{i<j}a_{ij}^2=\frac{n(n-1)}{2}S$.
Combine:
\[
  ST\ge \frac{2U^2}{n(n-1)}\,T=\frac{2U}{n(n-1)}(TU)
  \ge \frac{2U}{n(n-1)}\,\frac{n^2(n-1)^2}{4}=\frac{n(n-1)}{2}U.\qedhere
\]

\medskip\noindent\emph{Equality characterization.}
Equality in~\eqref{eq:sgi} requires simultaneous equality in both Cauchy--Schwarz applications.
The first (Fisher--variance) gives $V_i=c(\lambda_i-\bar\lambda)$ for a constant $c$.
The second (score-gap) gives $\frac{V_i-V_j}{\lambda_i-\lambda_j}=\mu$ for every $i<j$
and a common value $\mu$.
Substituting $V_i=c(\lambda_i-\bar\lambda)$ into the second condition yields
$c=\mu$, which is consistent, so the equality case is exactly $V_i=c(\lambda_i-\bar\lambda)$.
\end{proof}

%======================================================================
\section{Low-degree Stam: \texorpdfstring{$n=2$ and $n=3$}{n=2 and n=3}}
%======================================================================

\subsection{\texorpdfstring{$n=2$}{n=2}: equality and convexity along the dilation path}

\begin{proposition}[Quadratic case]
\label{prop:n2}
For $n=2$, for all $p,q\in\PnR$ with distinct roots,
\[
  \frac{1}{\Phi_2(p\boxplus_2 q)}=\frac{1}{\Phi_2(p)}+\frac{1}{\Phi_2(q)}.
\]
Moreover, along the dilation path $r_t=p\boxplus_2 q_t$,
$F(t):=1/\Phi_2(r_t)$ is a quadratic polynomial in $t$ with $F''(t)>0$.
\end{proposition}

\begin{proof}
If $p(x)=(x-\lambda_1)(x-\lambda_2)$ with $d=\lambda_2-\lambda_1>0$, then
$V_1=-1/d$, $V_2=1/d$, hence $\Phi_2(p)=2/d^2$ and
$\sigma^2(p)=d^2/4$, so $1/\Phi_2(p)=2\sigma^2(p)$.
Variance additivity gives
$1/\Phi_2(p\boxplus_2 q)=2\sigma^2(p\boxplus_2 q)=2\sigma^2(p)+2\sigma^2(q)$.
Along dilation, $\sigma^2(q_t)=t^2\sigma^2(q)$, hence
$1/\Phi_2(r_t)=2(\sigma^2(p)+t^2\sigma^2(q))$, a quadratic in $t$ with positive constant second derivative.
\end{proof}

\subsection{\texorpdfstring{$n=3$}{n=3}: an explicit computation (centered cubics)}

The following residue formula is valid for all degrees;
we state it here because its only application in this paper is to the cubic case.

\begin{theorem}[Critical-value formula]
\label{thm:critval}
Let $p\in\PnR$ have distinct roots $\lambda_1<\cdots<\lambda_n$, and let
$\zeta_1,\dots,\zeta_{n-1}$ be the simple zeros of $p'$. Then
\begin{equation}\label{eq:critval}
  \Phi_n(p)=-\frac{1}{4}\sum_{j=1}^{n-1}\frac{p''(\zeta_j)}{p(\zeta_j)}.
\end{equation}
\end{theorem}

\begin{proof}
By Lemma~\ref{lem:score-deriv},
$\Phi_n(p)=\frac{1}{4}\sum_i \frac{p''(\lambda_i)^2}{p'(\lambda_i)^2}$.
Consider the meromorphic function on the Riemann sphere:
\[
  F(x):=\frac{p''(x)^2}{p'(x)\,p(x)}.
\]

\emph{Residues at the roots.}
Since $p$ has a simple zero at $\lambda_i$ and $p'(\lambda_i)\ne 0$,
\[
  \operatorname{Res}_{x=\lambda_i}F=\frac{p''(\lambda_i)^2}{p'(\lambda_i)^2}.
\]
Summing over $i$ gives $\sum_i\operatorname{Res}_{\lambda_i}F=4\Phi_n(p)$.

\emph{Residues at the critical points.}
At a simple zero $\zeta_j$ of $p'$, interlacing implies $p(\zeta_j)\ne 0$.
Thus
\[
  \operatorname{Res}_{x=\zeta_j}F=\frac{p''(\zeta_j)^2}{p''(\zeta_j)\,p(\zeta_j)}=\frac{p''(\zeta_j)}{p(\zeta_j)}.
\]

\emph{Residue at infinity.}
We need $\operatorname{Res}_{\infty}F=0$.
Equivalently, the coefficient of $x^{-1}$ in the Laurent expansion of $F$ at $\infty$ must vanish.
Write $p(x)=x^n+a_1x^{n-1}+a_2x^{n-2}+\cdots$, so that
\begin{align*}
  p(x)  &= x^n\bigl(1+a_1x^{-1}+a_2x^{-2}+\cdots\bigr),\\
  p'(x) &= nx^{n-1}\bigl(1+\tfrac{n-1}{n}a_1x^{-1}+\tfrac{n-2}{n}a_2x^{-2}+\cdots\bigr),\\
  p''(x)&= n(n-1)x^{n-2}\bigl(1+\tfrac{n-2}{n-1}a_1x^{-1}+\tfrac{(n-3)(n-2)}{n(n-1)}a_2x^{-2}\cdots\bigr).
\end{align*}
Therefore
\[
  F(x)=\frac{p''(x)^2}{p'(x)\,p(x)}
  =\frac{n(n-1)^2}{x^3}\;\frac{\bigl(1+\tfrac{n-2}{n-1}a_1x^{-1}+\cdots\bigr)^2}
       {\bigl(1+\tfrac{n-1}{n}a_1x^{-1}+\cdots\bigr)\bigl(1+a_1x^{-1}+\cdots\bigr)}.
\]
Expanding numerator and denominator to order $x^{-1}$:
\begin{align*}
  \text{numerator:}\quad &1+\frac{2(n-2)}{n-1}\,a_1x^{-1}+O(x^{-2}),\\
  \text{denominator:}\quad &1+\Bigl(\frac{n-1}{n}+1\Bigr)a_1x^{-1}+O(x^{-2})
  =1+\frac{2n-1}{n}\,a_1x^{-1}+O(x^{-2}).
\end{align*}
Their ratio is
$1+\bigl(\frac{2(n-2)}{n-1}-\frac{2n-1}{n}\bigr)a_1x^{-1}+O(x^{-2})
=1-\frac{n+1}{n(n-1)}a_1x^{-1}+O(x^{-2})$.
Hence
\[
  F(x)=\frac{n(n-1)^2}{x^3}-\frac{(n^2-1)a_1}{x^4}+O(x^{-5}).
\]
Since the expansion starts at $x^{-3}$, no $x^{-1}$ term is present, so $\operatorname{Res}_{\infty}F=0$.

By the global residue theorem, the sum of all residues on the sphere is zero:
$4\Phi_n(p)+\sum_{j=1}^{n-1} \frac{p''(\zeta_j)}{p(\zeta_j)}=0$, proving~\eqref{eq:critval}.
\end{proof}

A centered monic cubic has the form $r(x)=x^3-Sx+T$ with $S\ge 0$.
It has three distinct real roots iff its discriminant $\Delta:=4S^3-27T^2$ is positive.

\begin{proposition}[Closed form for $\Phi_3$]
\label{prop:phi3}
For a centered cubic $r(x)=x^3-Sx+T$ with $\Delta>0$,
\[
  \Phi_3(r)=\frac{18S^2}{\Delta}.
\]
\end{proposition}

\begin{proof}
Apply Theorem~\ref{thm:critval}.
The critical points are $\zeta_\pm=\pm\alpha$ with $\alpha:=\sqrt{S/3}$ and $r''(x)=6x$.
Thus
\[
  4\Phi_3(r)=-\frac{6\alpha}{r(\alpha)}+\frac{6\alpha}{r(-\alpha)}
  =6\alpha\,\frac{r(\alpha)-r(-\alpha)}{r(\alpha)r(-\alpha)}.
\]
Compute $r(\alpha)-r(-\alpha)=-\frac{4S\alpha}{3}$ and
$r(\alpha)r(-\alpha)=T^2-\frac{4S^3}{27}=-\frac{\Delta}{27}$.
Substituting gives $4\Phi_3(r)=\frac{72S^2}{\Delta}$.
\end{proof}

\begin{proposition}[Convolution preserves cubic shape (centered)]
\label{prop:cubic-conv}
If $p(x)=x^3-S_1x+T_1$ and $q(x)=x^3-S_2x+T_2$ are centered, then
$(p\boxplus_3 q)(x)=x^3-(S_1+S_2)x+(T_1+T_2)$.
\end{proposition}

\begin{proof}
For $n=3$ we have $a_0=b_0=1$ (monic), $a_1=b_1=0$ (centered), $a_2=-S_1$, $a_3=T_1$,
and similarly for $q$.
Using Definition~\ref{def:conv} with $n=3$, the convolution coefficients are
\[
  c_k=\sum_{i+j=k}\frac{(3-i)!\,(3-j)!}{3!\,(3-k)!}\,a_ib_j.
\]
A direct substitution gives
$c_0=1$, $c_1=0$, $c_2=a_2+b_2=-(S_1+S_2)$, and $c_3=a_3+b_3=T_1+T_2$.
Therefore $(p\boxplus_3 q)(x)=x^3-(S_1+S_2)x+(T_1+T_2)$.
\end{proof}
\begin{theorem}[Stam for n=3]
\label{thm:stam3}
The finite free Stam inequality holds for $n=3$.
More precisely, after replacing $p,q$ by their centered translates
(which preserves both sides of the inequality by Lemma~\ref{lem:translation}),
equality holds if and only if $T_1=T_2=0$, i.e.\ both polynomials are even.
\end{theorem}

\begin{proof}
By Propositions~\ref{prop:phi3} and~\ref{prop:cubic-conv},
$1/\Phi_3=\Delta/(18S^2)=2S/9-3T^2/(2S^2)$.
Cancelling the linear terms in $S$, the inequality reduces to
\[
  \frac{(T_1+T_2)^2}{(S_1+S_2)^2}\le \frac{T_1^2}{S_1^2}+\frac{T_2^2}{S_2^2},
\]
which follows from convexity of $x\mapsto x^2$.

To see the equality statement, note that three distinct real roots force $\Delta_i=4S_i^3-27T_i^2>0$,
hence $S_i>0$; set $x_i:=T_i/S_i$.
Then
\[
\frac{(T_1+T_2)^2}{(S_1+S_2)^2}=\Bigl(\frac{S_1}{S_1+S_2}x_1+\frac{S_2}{S_1+S_2}x_2\Bigr)^2
\le \frac{S_1}{S_1+S_2}x_1^2+\frac{S_2}{S_1+S_2}x_2^2\le x_1^2+x_2^2.
\]
If equality holds at the endpoints, both inequalities are equalities. The first forces $x_1=x_2$,
and the second then forces $x_1=x_2=0$. Hence $T_1=T_2=0$.
\end{proof}

%======================================================================
\section{Hermite semigroup bound}
%======================================================================

\subsection{Hermite kernel}

\begin{definition}[Hermite kernel]
\label{def:hermite}
For $t\ge 0$, let $G_t\in\PnR$ be the monic degree-$n$ polynomial whose normalized generating function satisfies
\[
  K_{G_t}(z)=\exp\!\Bigl(-\frac{t}{2(n-1)}z^2\Bigr)\pmod{z^{n+1}}.
\]
The \emph{Hermite flow} is $p_t:=p\boxplus_n G_t$.
\end{definition}

\begin{lemma}[Semigroup and variance]
\label{lem:hermite-props}
For $s,t\ge 0$:
\begin{enumerate}[label=\textup{(\roman*)},nosep]
  \item $G_s\boxplus_n G_t=G_{s+t}$.
  \item $\sigma^2(G_t)=t$.
  \item $\sigma^2(p_t)=\sigma^2(p)+t$.
\end{enumerate}
\end{lemma}

\begin{proof}
(i) follows from $K_{G_s}K_{G_t}=K_{G_{s+t}}$ modulo $z^{n+1}$.
(ii) is read from the quadratic term. (iii) is Lemma~\ref{lem:var-add}.
\end{proof}

\subsection{Root ODE and dissipation}

\begin{lemma}[Hermite root ODE]
\label{lem:hermite-ode}
Suppose $p_t$ has simple roots for all $t$ in an open interval $I$.
Then the root trajectories $\lambda_i(t)$ are $C^\infty$ on $I$
(by the implicit function theorem, since $p_t'(\lambda_i(t))\ne 0$),
and
\[
  \dot\lambda_i=\frac{1}{n-1}V_i(t).
\]
\end{lemma}

\begin{proof}
For small $h>0$, expanding the exponential gives
$K_{G_h}(z)=1-\frac{h}{2(n-1)}z^2+O(h^2)$,
so the MSS operator acts as
$T_{G_h}f=f-\frac{h}{2(n-1)}f''+O(h^2)$.
Since $p_{t+h}=p_t\boxplus_n G_h=T_{G_h}\,p_t$, evaluating at $\lambda_i(t+h)$ gives
$0=p_{t+h}(\lambda_i(t+h))=T_{G_h}\,p_t(\lambda_i(t+h))$.
Expand to first order in $h$:
\[
  0 = p_t(\lambda_i+\dot\lambda_i h)
    - \frac{h}{2(n-1)}p_t''(\lambda_i)+O(h^2)
  = p_t'(\lambda_i)\,\dot\lambda_i\,h
    - \frac{h}{2(n-1)}p_t''(\lambda_i)+O(h^2),
\]
where we used $p_t(\lambda_i)=0$.
Dividing by $h\,p_t'(\lambda_i)$ and applying Lemma~\ref{lem:score-deriv}
($V_i=p_t''(\lambda_i)/(2p_t'(\lambda_i))$) yields $\dot\lambda_i=V_i/(n-1)$.
\end{proof}

\begin{lemma}[Hermite dissipation]
\label{lem:hermite-dissip}
Under the same simple-root hypothesis as Lemma~\ref{lem:hermite-ode},
\[
  \frac{d}{dt}\Phi_n(p_t)=-\frac{2}{n-1}\,\mathcal{S}(p_t).
\]
\end{lemma}

\begin{proof}
Differentiate $V_i(t)=\sum_{j\ne i}(\lambda_i-\lambda_j)^{-1}$ using the root ODE
$\dot\lambda_i=V_i/(n-1)$:
\[
  \dot V_i
  =-\sum_{j\ne i}\frac{\dot\lambda_i-\dot\lambda_j}{(\lambda_i-\lambda_j)^2}
  =-\frac{1}{n-1}\sum_{j\ne i}\frac{V_i-V_j}{(\lambda_i-\lambda_j)^2}.
\]
Then
\[
  \dot\Phi_n=2\sum_i V_i\dot V_i
  =-\frac{2}{n-1}\sum_i V_i\sum_{j\ne i}\frac{V_i-V_j}{(\lambda_i-\lambda_j)^2}.
\]
Symmetrizing: for each unordered pair $\{i,j\}$, the contributions from the $i$-sum and the $j$-sum
total $-\frac{2}{n-1}\,\frac{(V_i-V_j)^2}{(\lambda_i-\lambda_j)^2}$.
Summing over pairs gives $\dot\Phi_n=-\frac{2}{n-1}\,\mathcal{S}(p_t)$.
\end{proof}

\begin{lemma}[Non-collision for the Hermite ODE]
\label{lem:noncollision}
Let $p_t=p\boxplus_n G_t$ and assume $p_0$ has simple roots.
Then for all $t\ge 0$ the roots of $p_t$ remain distinct.
\end{lemma}

\begin{proof}[Sketch]
Let $g_i(t):=\lambda_{i+1}(t)-\lambda_i(t)>0$ be an adjacent gap.
From $\dot\lambda_i=V_i/(n-1)$,
\[
  \dot g_i(t)=\frac{V_{i+1}-V_i}{n-1}.
\]
Elementary decomposition gives
\[
  V_{i+1}-V_i
  =\frac{2}{g_i}+\sum_{k\ne i,i+1}\Bigl(\frac{1}{\lambda_{i+1}-\lambda_k}-\frac{1}{\lambda_i-\lambda_k}\Bigr).
\]
The summation term stays bounded while $2/g_i\to+\infty$ as $g_i\downarrow0$.
Hence for sufficiently small $g_i$, one has
$\dot g_i\ge c/g_i$ for some $c>0$, which prevents $g_i$ from hitting zero in finite time.
A continuation argument excludes collisions for all finite $t$.
\end{proof}

\begin{theorem}[Hermite flow bound]
\label{thm:hermite-bound}
Let $p\in\PnR$ have simple roots, $a:=\sigma^2(p)>0$, and $b>0$.
Assume that $p_t:=p\boxplus_n G_t$ has simple roots for all $t\in[0,b]$.
(By Lemma~\ref{lem:noncollision}, this holds globally once $p_0$ is simple.)
Then
\[
  \frac{1}{\Phi_n(p\boxplus_n G_b)}\ge \frac{a+b}{a\,\Phi_n(p)}.
\]
\end{theorem}

\begin{proof}
Apply the Score-Gradient Inequality (Theorem~\ref{thm:sgi}) to $p_t$:
$\mathcal{S}(p_t)\ge \frac{(n-1)\Phi_n(p_t)}{2\sigma^2(p_t)}=\frac{(n-1)\Phi_n(p_t)}{2(a+t)}$.
With Lemma~\ref{lem:hermite-dissip},
$\dot\Phi_n(p_t)\le -\Phi_n(p_t)/(a+t)$.
Since roots remain simple on $[0,b]$, $\Phi_n(p_t)>0$ throughout
(every score is finite), so we may divide by $\Phi_n(p_t)$ to obtain
$(\log\Phi_n)'\le -(a+t)^{-1}$.
Integrating from $0$ to $b$ gives
\[
  \log\Phi_n(p_b)-\log\Phi_n(p_0)
  \le -\int_0^b\frac{dt}{a+t}=\log\frac{a}{a+b},
\]
which rearranges to the displayed inequality.

\medskip\noindent\emph{Iteration remark.}
Even without invoking Lemma~\ref{lem:noncollision}, the same estimate integrates on any
subinterval where roots stay simple; one can then telescope over a finite partition of $[0,b]$.
\end{proof}

%======================================================================
\section{Dilation interpolation and a convexity heuristic}
%======================================================================

\subsection{The dilation path}

\begin{definition}[Dilation family]
\label{def:dilation}
Let $q(x)=\prod_{i=1}^n(x-\mu_i)\in\PnR$. For $t\in[0,1]$, define
\[
  q_t(x):=\prod_{i=1}^n(x-t\mu_i),\qquad r_t:=p\boxplus_n q_t.
\]
\end{definition}

\begin{lemma}[Basic properties]
\label{lem:dilation-props}
Let $a:=\sigma^2(p)$ and $b:=\sigma^2(q)$. Then:
\begin{enumerate}[label=\textup{(\roman*)},nosep]
  \item $r_0=p$ and $r_1=p\boxplus_n q$.
  \item $\sigma^2(q_t)=t^2\sigma^2(q)$ and $\sigma^2(r_t)=a+t^2 b$.
  \item $\Phi_n(q_t)=\Phi_n(q)/t^2$ for $t>0$.
  \item $r_t\in\PnR$ for all $t\in[0,1]$.
\end{enumerate}
\end{lemma}

\begin{proof}
(i) is immediate since $q_0=x^n$ is the identity for $\boxplus_n$.
(ii) follows from scaling of roots and variance additivity.
(iii) is score scaling under dilation.
(iv) is Theorem~\ref{thm:mss}.
\end{proof}

\subsection{The excess functional}

\begin{definition}[Dilation excess]
\label{def:excess}
For the dilation path $r_t$, define
\[
  E(t):=\frac{1}{\Phi_n(r_t)}-\frac{1}{\Phi_n(p)}-\frac{t^2}{\Phi_n(q)}.
\]
\end{definition}

\begin{lemma}[Endpoints]
\label{lem:excess-endpoints}
$E(0)=0$, and $E(1)\ge 0$ is equivalent to the finite free Stam inequality.
\end{lemma}

\begin{proof}
Immediate from the definition and $r_0=p$, $r_1=p\boxplus_n q$.
\end{proof}

\begin{conjecture}[Excess convexity (false in general)]
\label{conj:excess}
Along the dilation path, $E$ is convex on $(0,1)$, i.e. $E''(t)\ge 0$.
Equivalently,
$\frac{d^2}{dt^2}\bigl(1/\Phi_n(r_t)\bigr)\ge 2/\Phi_n(q)$.
\end{conjecture}

\begin{remark}
Conjecture~\ref{conj:excess} is a clean sufficient condition for Stam via
Theorem~\ref{thm:reduction}, but it is \emph{not true} in full generality;
see Appendix~\ref{app:counterexample}.

It is worth stressing a simple diagnostic: if we write $F(t):=1/\Phi_n(r_t)$,
then $E(t)=F(t)-1/\Phi_n(p)-t^2/\Phi_n(q)$ satisfies
\[
  E''(t)=F''(t)-\frac{2}{\Phi_n(q)}.
\]
Thus even if one happens to observe $F''(t)\ge 0$ in a given example,
the subtraction of $t^2/\Phi_n(q)$ shifts the curvature by a negative
constant and can force $E''(t)<0$.
\end{remark}

\begin{lemma}[Vanishing first derivative at t=0]
\label{lem:excess-deriv0}
Assume $q$ is centered (i.e. the sum of its roots is zero, equivalently its $x^{n-1}$ coefficient vanishes).
Then $E'(0)=0$.
\end{lemma}

\begin{proof}
Write $q(x)=\sum_{k=0}^n b_k x^{n-k}$.
Centering means $b_1=0$.

Along the dilation family, $q_t$ has coefficients $b_k(t)=t^k b_k$.
By Definition~\ref{def:conv},
\[
  r_t(x)=\sum_{k=0}^n\frac{(n-k)!}{n!}\,b_k(t)\,p^{(k)}(x)
  =p(x)+\sum_{k=1}^n\frac{(n-k)!}{n!}\,t^k b_k\,p^{(k)}(x).
\]
Since $b_1=0$, the first nonzero perturbation is order $t^2$, so
$\partial_t r_t\big|_{t=0}=0$ as a polynomial.

Now differentiate the identity $r_t(\lambda_i(t))\equiv 0$ in $t$.
Since $r_0=p$ has simple roots, the implicit function theorem
(applied to $r_t(\lambda_i(t))=0$ with $r_0'(\lambda_i)=p'(\lambda_i)\ne 0$)
guarantees that each $\lambda_i(t)$ is $C^\infty$ near $t=0$.
Differentiating gives
\[
  \dot\lambda_i(0)
  =-\frac{\partial_t r_t(\lambda_i)\big|_{t=0}}{r_0'(\lambda_i)}
  =-\frac{0}{p'(\lambda_i)}=0,
\]
so each root trajectory has zero first derivative at $t=0$.
Since $\Phi_n$ is a smooth function of the roots as long as they remain distinct,
this implies
$\bigl.\frac{d}{dt}\frac{1}{\Phi_n(r_t)}\bigr|_{t=0}=0$.
Also $\frac{d}{dt}\bigl(t^2/\Phi_n(q)\bigr)|_{t=0}=0$.
Thus $E'(0)=0$.
\end{proof}

\begin{theorem}[Convexity reduction]
\label{thm:reduction}
If Conjecture~\ref{conj:excess} holds for all $p,q\in\PnR$ with positive variance, then the finite free Stam inequality holds for all such $p,q$.
\end{theorem}

\begin{proof}
By Lemma~\ref{lem:translation}, we may replace $q$ by its centered translate without changing
either side of the Stam inequality; assume henceforth that $q$ is centered.

Assuming Conjecture~\ref{conj:excess}, the convex function $E$ satisfies
$E(t)\ge E(0)+tE'(0)$ for all $t\in[0,1]$.
By Lemma~\ref{lem:excess-endpoints}, $E(0)=0$, and by Lemma~\ref{lem:excess-deriv0}, $E'(0)=0$.
Hence $E(1)\ge 0$, which is exactly the Stam inequality.
\end{proof}

\begin{remark}[What remains for general n]
Appendix~\ref{app:counterexample} rules out global convexity of the dilation excess as a general mechanism.
The remaining task is to find a different one-sided comparison principle along a real-rooted interpolation
(dilation path, constant-variance path, or similar) that still forces $E(1)\ge 0$.
\end{remark}

%======================================================================
\section{Auxiliary toolkit for future attempts}
%======================================================================

The results in this section are \emph{not} used elsewhere in the paper.
They record operator identities and proxy functionals intended as scaffolding
for future general-$n$ attempts at the Stam inequality,
and are included for the convenience of the reader.

\subsection{Operator paths in the normalized derivative basis}

\begin{definition}[Normalized derivative operators]
For $0\le k\le n$, set
\[
  D_k:=\frac{(n-k)!}{n!}\,\partial_x^k.
\]
Then for $q(x)=\sum_{k=0}^n b_k x^{n-k}$,
\[
  T_q=\sum_{k=0}^n b_k D_k,
\]
and $p\boxplus_n q=T_qp$.
\end{definition}

\begin{lemma}[Dilation and Hermite as coefficient rays]
\label{lem:operator-rays}
Let $q_t(x)=\prod_{i=1}^n(x-t\mu_i)=\sum_{k=0}^n t^k b_k x^{n-k}$. Then
\[
  T_{q_t}=\sum_{k=0}^n t^k b_k D_k.
\]
If $G_t$ is the Hermite kernel (Definition~\ref{def:hermite}), then
\[
  T_{G_t}=\sum_{m=0}^{\lfloor n/2\rfloor}\frac{1}{m!}\Bigl(-\frac{t}{2(n-1)}\Bigr)^m\partial_x^{2m}.
\]
In particular, both the dilation family and the Hermite flow are explicit paths in the
commutative algebra generated by $\partial_x$.
\end{lemma}

\begin{proof}
The dilation identity is immediate from coefficient scaling $b_k(t)=t^k b_k$.
For $G_t$, use $K_{G_t}(z)=\exp(-\tfrac{t}{2(n-1)}z^2)\pmod{z^{n+1}}$ and the MSS operator rule
$T_q=\sum_k \frac{(n-k)!}{n!}b_k\partial_x^k$; only even powers occur, with the displayed coefficients.
Commutativity follows because all operators are polynomials in $\partial_x$.
\end{proof}

\begin{remark}[Why this matters]
Lemma~\ref{lem:operator-rays} identifies the operator coefficients as the control variables
for each interpolation, which is the natural format for MSS-style one-sided estimates.
\end{remark}

\subsection{Fisher--repulsion identity and its consequences}

\begin{definition}[Pairwise repulsion energy]
For $p\in\PnR$ with distinct roots $\lambda_1<\cdots<\lambda_n$, define
\[
  \mathcal{R}(p):=\sum_{1\le i<j\le n}\frac{1}{(\lambda_i-\lambda_j)^2}.
\]
\end{definition}

\begin{theorem}[Fisher--repulsion identity]
\label{lem:phi-vs-R}
For $p\in\PnR$ with distinct roots,
\begin{equation}\label{eq:phi-eq-2R}
  \Phi_n(p)=2\,\mathcal{R}(p).
\end{equation}
In particular, $1/\Phi_n(p)=1/(2\mathcal{R}(p))$.
\end{theorem}

\begin{proof}
Expand $\Phi_n=\sum_{i=1}^n V_i^2$ by writing each $V_i$ as a sum:
\[
  \Phi_n
  =\sum_i\sum_{j\ne i}\sum_{k\ne i}
    \frac{1}{(\lambda_i-\lambda_j)(\lambda_i-\lambda_k)}
  =\underbrace{\sum_i\sum_{j\ne i}\frac{1}{(\lambda_i-\lambda_j)^2}}_%
    {\displaystyle=\,2\mathcal{R}}
  +\underbrace{\sum_i\sum_{\substack{j\ne i,\,k\ne i\\j\ne k}}
    \frac{1}{(\lambda_i-\lambda_j)(\lambda_i-\lambda_k)}}_%
    {\displaystyle=:\,C}.
\]
The first sum counts each unordered pair $\{i,j\}$ twice, giving $2\mathcal{R}$.
To show $C=0$, group by unordered triples $\{a,b,c\}$.
Each triple contributes (with two orderings of the non-pivot pair)
\[
  2\Bigl[
    \frac{1}{(\lambda_a-\lambda_b)(\lambda_a-\lambda_c)}
    +\frac{1}{(\lambda_b-\lambda_a)(\lambda_b-\lambda_c)}
    +\frac{1}{(\lambda_c-\lambda_a)(\lambda_c-\lambda_b)}
  \Bigr].
\]
Set $u=\lambda_a-\lambda_b$, $v=\lambda_a-\lambda_c$.
The bracket becomes
$\frac{1}{uv}-\frac{1}{u(v-u)}+\frac{1}{v(v-u)}$.
Putting over the common denominator $uv(v-u)$ gives
$\frac{v-u-v+u}{uv(v-u)}=0$.
Since each bracket vanishes, $C=0$ and $\Phi_n=2\mathcal{R}$.
\end{proof}

\begin{remark}
This identity is purely algebraic: it holds for any list of distinct reals,
with no real-rootedness hypothesis.
It sharpens the Cauchy--Schwarz bound $\Phi_n\le 2(n-1)\mathcal{R}$
by a factor of~$n-1$.
\end{remark}

\begin{corollary}[Stam as a harmonic-mean bound on repulsion]
\label{cor:stam-repulsion}
The Stam inequality
$1/\Phi_n(p\boxplus_n q)\ge 1/\Phi_n(p)+1/\Phi_n(q)$
is equivalent to
\[
  \frac{1}{\mathcal{R}(p\boxplus_n q)}\ge \frac{1}{\mathcal{R}(p)}+\frac{1}{\mathcal{R}(q)},
\]
i.e.~$\mathcal{R}(p\boxplus_n q)$ is bounded above by the harmonic mean of
$\mathcal{R}(p)$ and $\mathcal{R}(q)$.
\end{corollary}

\begin{proof}
Divide~\eqref{eq:phi-eq-2R} through by $2$.
\end{proof}

\begin{corollary}[Improved Fisher--variance via repulsion]
\label{cor:proxy-corridor}
For every $p\in\PnR$ with distinct roots and positive variance,
\[
  \frac{1}{\Phi_n(p)}=\frac{1}{2\mathcal{R}(p)}
  \le \frac{4\,\sigma^2(p)}{n(n-1)^2}.
\]
Equivalently, $\mathcal{R}(p)\,\sigma^2(p)\ge n(n-1)^2/8$.
\end{corollary}

\begin{proof}
Combine Theorem~\ref{lem:phi-vs-R} with Lemma~\ref{lem:FV}.
\end{proof}

\begin{remark}[Reformulation of Stam]
The identity $\Phi_n=2\mathcal{R}$ gives the cleanest formulation of
the Stam inequality: $\mathcal{R}$ depends only on root gaps (no
``scores'' needed), and the target is the classical harmonic-mean
inequality for an energy functional.
\end{remark}

\subsection{Barrier-inspired resolvent quantities}

\begin{definition}[Cauchy transform at height $\eta$]
\label{def:height-eta}
For $\eta>0$ and real $x$, define
\[
  g_{p,\eta}(x):=\frac{1}{n}\,\frac{p'(x+i\eta)}{p(x+i\eta)}
  =\frac{1}{n}\sum_{j=1}^n\frac{1}{x+i\eta-\lambda_j}.
\]
Write
$g_{p,\eta}=u_{p,\eta}+iv_{p,\eta}$ with $u,v\in\R$.
\end{definition}

\begin{lemma}[Uniform bounds away from the real axis]
\label{lem:eta-bounds}
For every $\eta>0$ and every real $x$,
\[
  |u_{p,\eta}(x)|\le \frac{1}{\eta},
  \qquad
  0<v_{p,\eta}(x)\le \frac{1}{\eta}.
\]
\end{lemma}

\begin{proof}
Using Definition~\ref{def:height-eta},
\[
  u_{p,\eta}(x)=\frac{1}{n}\sum_{j=1}^n\frac{x-\lambda_j}{(x-\lambda_j)^2+\eta^2},
  \qquad
  v_{p,\eta}(x)=\frac{1}{n}\sum_{j=1}^n\frac{\eta}{(x-\lambda_j)^2+\eta^2}.
\]
Each summand has absolute value at most $1/\eta$ in the real part and lies in $(0,1/\eta]$
in the imaginary part, and averaging preserves these bounds.
\end{proof}

\begin{remark}[Connection to MSS barriers]
In MSS-style arguments one tracks logarithmic derivatives off the real axis.
Here $g_{p,\eta}(x)$ is precisely that quantity at height $\eta$, and unlike $\Phi_n$
it stays uniformly bounded and avoids collision singularities.
\end{remark}

\subsection{Discussion and next steps}

\begin{enumerate}[label=\textup{(L\arabic*)},leftmargin=2.5em]
  \item\label{log:l1}
  The dilation-path convexity heuristic for $E(t)$ is false globally, so the final proof
  mechanism is likely to be one-sided and robust rather than purely curvature-based.

  \item\label{log:l2}
  The operator identities in Lemma~\ref{lem:operator-rays} provide a clean decomposition of
  both interpolation families into differential-operator coordinates.
  This is the correct format for importing MSS-style barrier ideas.

  \item\label{log:l3}
  The identity $\Phi_n=2\mathcal{R}$ (Theorem~\ref{lem:phi-vs-R}) and
  the resulting harmonic-mean reformulation (Corollary~\ref{cor:stam-repulsion})
  provide the cleanest formulation of the Stam conjecture.

  \item\label{log:l4}
  Numerical experiments (see \texttt{route\_a\_experiments.py} and \texttt{test\_repulsion\_stam.py})
  show that Lorentzian regularisation of the repulsion energy at any fixed $\eta>0$
  \emph{breaks} the super-additivity; thus a direct resolvent-barrier proof at fixed height
  is unlikely to succeed without substantial modification.
  However, the identity $\Phi_n=2\mathcal{R}$ was discovered via this investigation and is
  independently useful.

  \item\label{log:l5}
  The most promising remaining strategies are:
  (a) a dilation-path ODE argument for $1/\mathcal{R}(r_t)$ combined with the SGI and
  the identity $\Phi_n=2\mathcal{R}$ (see Section~\ref{sec:routeB} for Route~B analysis);
  (b) the constant-variance path combined with
  monotonicity of $\Phi_n$ under higher-cumulant perturbation;
  (c) a random-matrix / HCIZ representation of $\mathcal{R}$ as a Haar-averaged trace;
  (d) a variational/transport approach via the discriminant power inequality discovered in
  Section~\ref{sec:routeC}.
  Route~B (Section~\ref{sec:routeB}) provides extensive evidence that $\Phi_n(r_t)$
  is non-increasing along the dilation path and that the ``pointwise dilation Stam''
  (Conjecture~\ref{conj:pointwise-stam}) holds; the key open sub-problem is controlling
  the sign of the perpendicular dissipation (Conjecture~\ref{conj:perp-sign}).

  \item\label{log:l6}
  Route~C (Section~\ref{sec:routeC}) introduces a transport/variational framework
  and discovers the \emph{discriminant power inequality} (Conjecture~\ref{conj:epi}):
  $\mathcal{N}(p\boxplus_n q)\ge\mathcal{N}(p)+\mathcal{N}(q)$,
  the finite-free analogue of Shannon's entropy-power inequality.
  The conjecture survives $29\,000+$ random tests for $n=2$ to~$8$ with zero violations.
  For $n=2$ it reduces to variance additivity (exact equality),
  and for $n=3$ it is proved via the same convexity argument that establishes the Stam inequality.
  Several na\"{\i}ve transport proxies (pairwise gap super-additivity, dilation vs.\ displacement
  comparison, weighted gap coupling) are disproved by numerical counterexample.
\end{enumerate}

%======================================================================
\section{Route C: Variational / transport approach}
\label{sec:routeC}
%======================================================================

We now develop a variational perspective on the Stam inequality based on
optimal-transport ideas and energy functionals of root configurations.
The key outcome is a new conjectured inequality (the \emph{discriminant power inequality},
Conjecture~\ref{conj:epi}) that is the finite-free analogue of Shannon's entropy power inequality.

\subsection{Root measures, log-Vandermonde, and discriminant power}

\begin{definition}[Empirical root measure]
\label{def:root-meas}
For $p\in\PnR$ with roots $\lambda_1\le\cdots\le\lambda_n$, define
$\mu_p:=\frac{1}{n}\sum_{i=1}^n\delta_{\lambda_i}$.
\end{definition}

\begin{definition}[Log-Vandermonde (``entropy'')]
\label{def:log-vand}
For $p\in\PnR$ with distinct roots $\lambda_1<\cdots<\lambda_n$, define
\[
  H(p):=\sum_{1\le i<j\le n}\log(\lambda_j-\lambda_i)
  =\log\prod_{i<j}(\lambda_j-\lambda_i)
  =\tfrac{1}{2}\log\operatorname{disc}(p),
\]
where $\operatorname{disc}(p)=\prod_{i<j}(\lambda_j-\lambda_i)^2$ is the discriminant.
If $p$ has a repeated root, set $H(p):=-\infty$.
\end{definition}

\begin{definition}[Discriminant power (``entropy power'')]
\label{def:disc-power}
With $N:=\binom{n}{2}$, define the \emph{discriminant power}
\[
  \mathcal{N}(p):=\operatorname{disc}(p)^{1/N}=e^{2H(p)/N}
  =\Bigl(\prod_{i<j}(\lambda_j-\lambda_i)^2\Bigr)^{1/N},
\]
i.e.\ the geometric mean of the squared gaps.
Set $\mathcal{N}(p):=0$ when $p$ has a repeated root.
\end{definition}

\begin{remark}
For $n=2$, $N=1$ and $\mathcal{N}(p)=(\lambda_2-\lambda_1)^2=4\sigma^2(p)$.
More generally, $\mathcal{N}(p)$ captures the ``typical squared gap'' of~$p$.
\end{remark}

\subsection{Gradient structure of the log-Vandermonde}

\begin{lemma}[Score as gradient of $H$]
\label{lem:grad-H}
Let $p\in\PnR$ have distinct roots.
Viewing $H$ as a function of the root vector $\boldsymbol{\lambda}=(\lambda_1,\dots,\lambda_n)$
in the open Weyl chamber $W_n:=\{\lambda_1<\cdots<\lambda_n\}\subset\R^n$,
\[
  \frac{\partial H}{\partial\lambda_k}=V_k\qquad(k=1,\dots,n).
\]
In particular $\Phi_n(p)=\|\nabla H(\boldsymbol\lambda)\|^2$.
\end{lemma}

\begin{proof}
Differentiate $H=\sum_{i<j}\log(\lambda_j-\lambda_i)$.
For fixed $k$, every pair $(k,j)$ with $j>k$ contributes
$\partial_{\lambda_k}\log(\lambda_j-\lambda_k)=-1/(\lambda_j-\lambda_k)$,
and every pair $(i,k)$ with $i<k$ contributes
$\partial_{\lambda_k}\log(\lambda_k-\lambda_i)=1/(\lambda_k-\lambda_i)$.
Combining, $\partial_k H=\sum_{j\ne k}1/(\lambda_k-\lambda_j)=V_k$.
\end{proof}

\begin{lemma}[Hessian and trace identity]
\label{lem:hess-H}
For distinct roots,
\[
  \frac{\partial^2 H}{\partial\lambda_i\partial\lambda_j}
  =\begin{cases}
    \displaystyle\frac{1}{(\lambda_i-\lambda_j)^2}&i\ne j,\\[8pt]
    \displaystyle-\sum_{k\ne i}\frac{1}{(\lambda_i-\lambda_k)^2}&i=j.
  \end{cases}
\]
In particular, $\operatorname{tr}\!\bigl(\operatorname{Hess}(H)\bigr)
=-2\mathcal{R}(p)=-\Phi_n(p)$.
Moreover, $\operatorname{Hess}(H)$ is negative semidefinite with kernel
$\operatorname{span}(\mathbf{1})$ (translation direction).
\end{lemma}

\begin{proof}
The formula follows by direct differentiation.
The trace sums the diagonal entries: $\sum_i(-\sum_{k\ne i}(\lambda_i-\lambda_k)^{-2})
=-\sum_i\sum_{k\ne i}(\lambda_i-\lambda_k)^{-2}=-2\mathcal{R}=-\Phi_n$.

For negative semidefiniteness, let $v\in\R^n$ with $\sum_i v_i=0$.
Then
\[
  v^\top\!\operatorname{Hess}(H)\,v
  =\sum_i v_i^2\Bigl(-\sum_{k\ne i}\frac{1}{d_{ik}^2}\Bigr)
   +\sum_{i\ne j}\frac{v_iv_j}{d_{ij}^2}
  =-\sum_{i<j}\frac{(v_i-v_j)^2}{d_{ij}^2}\le 0,
\]
with equality iff all $v_i$ are equal; combined with $\sum v_i=0$, this gives $v=0$.
On the full space, $\mathbf{1}^\top\operatorname{Hess}(H)\mathbf{1}=\operatorname{tr}+2\sum_{i<j}d_{ij}^{-2}=-2\mathcal{R}+2\mathcal{R}=0$,
so $\mathbf{1}$ is in the kernel.
\end{proof}

\subsection{De Bruijn identity and isoperimetric inequality}

\begin{theorem}[Finite free de Bruijn identity]
\label{thm:deBruijn}
Under the same simple-root hypothesis as Lemma~\ref{lem:hermite-ode}, along the Hermite flow $p_t=p\boxplus_n G_t$,
\begin{equation}\label{eq:deBruijn}
  \frac{d}{dt}H(p_t)=\frac{\Phi_n(p_t)}{n-1}.
\end{equation}
\end{theorem}

\begin{proof}
By the chain rule, Lemma~\ref{lem:grad-H}, and the Hermite root ODE $\dot\lambda_i=V_i/(n-1)$:
\[
  \frac{d}{dt}H(p_t)
  =\sum_{k=1}^n \frac{\partial H}{\partial\lambda_k}\,\dot\lambda_k
  =\sum_{k=1}^n V_k\cdot\frac{V_k}{n-1}
  =\frac{1}{n-1}\sum_k V_k^2
  =\frac{\Phi_n(p_t)}{n-1}.\qedhere
\]
\end{proof}

\begin{remark}
Identity~\eqref{eq:deBruijn} is the exact analogue of the classical de Bruijn identity
$\frac{d}{dt}h(X+\sqrt{t}\,Z)=\frac{1}{2}J(X+\sqrt{t}\,Z)$
for the differential entropy $h$ and Fisher information~$J$.
Combined with the dissipation identity $\dot\Phi_n=-\frac{2}{n-1}\mathcal{S}$
(Lemma~\ref{lem:hermite-dissip}), it gives a closed system on $(H,\Phi_n,\mathcal{S})$
along the Hermite flow.
\end{remark}

\begin{lemma}[AM-GM isoperimetric inequality]
\label{lem:amgm-iso}
For $p\in\PnR$ with distinct roots,
\begin{equation}\label{eq:amgm}
  \mathcal{R}(p)\cdot\mathcal{N}(p)\ge N=\binom{n}{2}.
\end{equation}
Equivalently, $\Phi_n(p)\cdot\mathcal{N}(p)\ge 2\binom{n}{2}=n(n-1)$.
Equality holds if and only if all $\binom{n}{2}$ gaps $|\lambda_i-\lambda_j|$ are equal
(i.e.\ the roots are equally spaced).
\end{lemma}

\begin{proof}
By AM-GM applied to the $N$ positive numbers $1/d_{ij}^2$ $(i<j)$:
\[
  \frac{1}{N}\sum_{i<j}\frac{1}{d_{ij}^2}
  \ge \Bigl(\prod_{i<j}\frac{1}{d_{ij}^2}\Bigr)^{1/N}
  =\frac{1}{\mathcal{N}(p)}.
\]
Rearranging, $\mathcal{R}/N\ge 1/\mathcal{N}$.
Equality in AM-GM holds iff all $d_{ij}^{-2}$ coincide.
\end{proof}

\subsection{Displacement convexity of the repulsion energy}

\begin{definition}[Displacement interpolation]
\label{def:displacement}
For two ordered root vectors
$\boldsymbol{\lambda}=(\lambda_1<\cdots<\lambda_n)$ and
$\boldsymbol{\rho}=(\rho_1<\cdots<\rho_n)$ in $W_n$,
define
$\boldsymbol{\gamma}(t):=(1-t)\boldsymbol{\lambda}+t\boldsymbol{\rho}$
for $t\in[0,1]$.
This is the McCann displacement interpolation (monotone-coupling geodesic)
between the empirical measures $\mu_{\boldsymbol\lambda}$ and $\mu_{\boldsymbol\rho}$.
\end{definition}

\begin{lemma}[Displacement convexity of $\mathcal{R}$]
\label{lem:disp-convex}
The repulsion energy $\mathcal{R}$ is convex along displacement interpolations in~$W_n$.
That is, for every $\boldsymbol{\lambda},\boldsymbol{\rho}\in W_n$,
\[
  t\mapsto \mathcal{R}(\boldsymbol{\gamma}(t))
  \quad\text{is convex on }[0,1].
\]
\end{lemma}

\begin{proof}
Each gap along the displacement interpolation is linear:
$\gamma_j(t)-\gamma_i(t)=(1-t)(\lambda_j-\lambda_i)+t(\rho_j-\rho_i)$,
and is positive for all $t\in[0,1]$ because both endpoints are positive
(since $i<j$ and both $\boldsymbol{\lambda},\boldsymbol{\rho}$ are strictly ordered).
Thus
$\mathcal{R}(\boldsymbol{\gamma}(t))=\sum_{i<j}\frac{1}{(\gamma_j(t)-\gamma_i(t))^2}$
is a sum of functions $f\circ\ell$ where $f(d)=1/d^2$ is convex on $(0,\infty)$
(since $f''(d)=6/d^4>0$) and $\ell(t)$ is a positive affine function.
A convex function composed with an affine function is convex, and sums of convex functions are convex.
\end{proof}

\begin{remark}
In optimal-transport terminology, this says the Riesz $(s{=}{-}2)$ interaction energy
is \emph{displacement convex} on $W_n$.
In contrast, the log-energy $H$ is easily verified to be displacement \emph{concave}
(since $\log d$ is concave), and its exponential $\mathcal{N}=e^{2H/N}$
inherits the log-concavity.
\end{remark}

\subsection{The discriminant power inequality}

\begin{conjecture}[Discriminant power inequality (``finite-free EPI'')]
\label{conj:epi}
For all $p,q\in\PnR$,
\begin{equation}\label{eq:epi}
  \mathcal{N}(p\boxplus_n q)\ge \mathcal{N}(p)+\mathcal{N}(q).
\end{equation}
Equivalently,
$\operatorname{disc}(p\boxplus_n q)^{1/N}
\ge \operatorname{disc}(p)^{1/N}+\operatorname{disc}(q)^{1/N}$.
\end{conjecture}

\begin{proposition}[EPI for $n=2$: equality]
\label{prop:epi-n2}
For $n=2$, the discriminant power inequality
is an exact equality and reduces to variance additivity:
$\mathcal{N}(p\boxplus_2 q)=4\sigma^2(p\boxplus_2 q)=4\sigma^2(p)+4\sigma^2(q)=\mathcal{N}(p)+\mathcal{N}(q)$.
\end{proposition}

\begin{proof}
For $n=2$, $N=1$ and $\mathcal{N}(p)=(\lambda_2-\lambda_1)^2=4\sigma^2(p)$.
The result is Lemma~\ref{lem:var-add}.
\end{proof}

\begin{theorem}[EPI for $n=3$]
\label{thm:epi-n3}
The discriminant power inequality~\eqref{eq:epi} holds for $n=3$.
More precisely, after centering (which preserves $\mathcal{N}$ because
all gaps are translation-invariant), write
$p(x)=x^3-S_1x+T_1$ and $q(x)=x^3-S_2x+T_2$ with $S_i>0$ and
$\Delta_i=4S_i^3-27T_i^2>0$.
Then
\[
  \Delta_r^{1/3}\ge \Delta_1^{1/3}+\Delta_2^{1/3},
\]
where $\Delta_r=4(S_1+S_2)^3-27(T_1+T_2)^2$.
\end{theorem}

\begin{proof}
Set $\delta_i:=\Delta_i/(4S_i^3)\in(0,1]$ and $w_i:=S_i/(S_1+S_2)$, so $w_1+w_2=1$.
Then $\Delta_i^{1/3}=4^{1/3}S_i\delta_i^{1/3}$ and the claim is
\begin{equation}\label{eq:epi-n3-reduced}
  \delta_r^{1/3}\ge w_1\delta_1^{1/3}+w_2\delta_2^{1/3}.
\end{equation}
By concavity of $x^{1/3}$, it is enough to prove
\begin{equation}\label{eq:delta-linear}
  \delta_r\ge w_1\delta_1+w_2\delta_2.
\end{equation}
After routine algebra, \eqref{eq:delta-linear} is equivalent to
\[
  \frac{(T_1+T_2)^2}{(S_1+S_2)^2}
  \le \frac{T_1^2}{S_1^2}+\frac{T_2^2}{S_2^2},
\]
which is the weighted Jensen inequality for $x^2$ with $x_i=T_i/S_i$.
\end{proof}

\begin{remark}[Relationship between EPI and Stam for $n=3$]
The intermediate inequality~\eqref{eq:delta-linear} is \emph{equivalent}
to the Stam inequality at $n=3$
(compare the proof of Theorem~\ref{thm:stam3}).
Thus for $n=3$, Stam $\Rightarrow$ EPI (since Stam gives
the stronger linear bound, from which the cube-root bound follows by concavity).
For general~$n$ the two statements appear to be independent --- neither
obviously implies the other --- although both hold in all numerical tests.
\end{remark}

\subsection{Numerical evidence}
\label{sec:routeC-numerics}

We summarize the Route~C experiments (code: \texttt{route\_c\_experiments.py},
\texttt{route\_c\_epi\_deep.py}).
Conjecture~\ref{conj:epi} shows zero violations in the reported sweeps
(including $3000$ centered samples per degree for $n=2,\dots,8$), with
machine-precision equality at $n=2$ and positive margins for $n\ge 3$.
The Stam and EPI excesses are strongly correlated in samples.
Conversely, several natural proxies are decisively false:
pairwise gap super-additivity, dilation-vs-displacement ordering for $\mathcal R$,
raw log-Vandermonde super-additivity, and repulsion-weighted gap coupling.
Finally, midpoint displacement convexity of $\mathcal R$ is numerically stable
in large sweeps, and the normalized transport--repulsion decrement tested in this work
remains positive on all sampled instances.

\subsection{Discussion and next steps for Route~C}

\begin{enumerate}[label=\textup{(C\arabic*)},leftmargin=2.5em]
  \item The discriminant power inequality (Conjecture~\ref{conj:epi}) is numerically robust
  and has a clean algebraic statement:
  $\operatorname{disc}(p\boxplus_n q)^{1/N}\ge\operatorname{disc}(p)^{1/N}+\operatorname{disc}(q)^{1/N}$.
  A proof would constitute a finite-free analogue of Shannon's entropy power inequality.

  \item The de Bruijn identity (Theorem~\ref{thm:deBruijn})
  and the AM-GM isoperimetric (Lemma~\ref{lem:amgm-iso})
  provide the finite-free analogues of the two pillar identities
  in the classical proof of the entropic Stam inequality.
  The question is whether they can be combined to yield either Stam or EPI for general~$n$.

  \item For a proof of the EPI at general~$n$,
  possible approaches include:
  (i) the MSS random-matrix representation
  $p\boxplus_n q(x)=\mathbb{E}_U\det(xI-A-UBU^*)$
  combined with Minkowski-type determinantal inequalities;
  (ii) induction on $n$ using interlacing and root-deletion relations
  for the discriminant;
  (iii) an operator-semigroup argument along the Hermite flow,
  exploiting the de Bruijn identity and the dissipation formula.

  \item Even without a direct EPI$\Rightarrow$Stam implication,
  the EPI serves as an independent ``litmus test'' for proposed approaches:
  any strategy that can prove both Stam \emph{and} EPI likely captures
  the essential structure of root spreading under~$\boxplus_n$.
\end{enumerate}

%======================================================================
\section{Route B: Operator-coefficient monotone path}
\label{sec:routeB}
%======================================================================

We study the dilation path $r_t=p\boxplus_n q_t$ (Definition~\ref{def:dilation})
from the perspective of root-velocity dissipation.
The key outcomes are:
(i)~a general dissipation formula that decomposes $\dot\Phi_n$ into
score-aligned and perpendicular components (Lemma~\ref{lem:gen-dissip}),
(ii)~a proof that the initial root acceleration is proportional to the score
(Proposition~\ref{prop:initial-accel}),
(iii)~a new functional $\Gamma^{(1)}$ whose positivity controls $F''(0)$,
proved positive for $n=3$ (Proposition~\ref{prop:gamma1-n3}),
and (iv)~several conjectures supported by extensive numerical experiments.

\subsection{General dissipation along arbitrary root motions}

\begin{lemma}[General Fisher dissipation]
\label{lem:gen-dissip}
Let $\boldsymbol\lambda(t)=(\lambda_1(t),\dots,\lambda_n(t))$
be a smooth path of distinct ordered configurations in $W_n$,
and set $V_i(t):=\sum_{j\ne i}(\lambda_i(t)-\lambda_j(t))^{-1}$.
Then
\begin{equation}\label{eq:gen-dissip}
  \frac{d}{dt}\Phi_n(\boldsymbol\lambda(t))
  =-2\sum_{i<j}\frac{(V_i-V_j)(\dot\lambda_i-\dot\lambda_j)}{(\lambda_i-\lambda_j)^2}.
\end{equation}
\end{lemma}

\begin{proof}
$\Phi_n=\sum_i V_i^2$.
Differentiating $V_i=\sum_{j\ne i}(\lambda_i-\lambda_j)^{-1}$ gives
$\dot V_i=-\sum_{j\ne i}\frac{\dot\lambda_i-\dot\lambda_j}{(\lambda_i-\lambda_j)^2}$.
Thus $\dot\Phi_n=2\sum_i V_i\dot V_i$.
Substituting $\dot V_i$ and symmetrizing over unordered pairs~$\{i,j\}$
(exactly as in the proof of Lemma~\ref{lem:hermite-dissip}) yields~\eqref{eq:gen-dissip}.
\end{proof}

\begin{definition}[Dissipation pairing]
\label{def:dissip-pairing}
For a root configuration $\boldsymbol\lambda\in W_n$ with scores $V_i$ and
velocity vector $\dot{\boldsymbol\lambda}$,
define
\[
  \mathcal{D}(\boldsymbol\lambda,\dot{\boldsymbol\lambda})
  :=\sum_{i<j}\frac{(V_i-V_j)(\dot\lambda_i-\dot\lambda_j)}{(\lambda_i-\lambda_j)^2}.
\]
Thus $\dot\Phi_n=-2\mathcal{D}$ and
$\dot F:=\frac{d}{dt}\frac{1}{\Phi_n}=\frac{2\mathcal{D}}{\Phi_n^2}$.
\end{definition}

\begin{remark}
Writing $s_{ij}:=(V_i-V_j)/d_{ij}$ and $w_{ij}:=(\dot\lambda_i-\dot\lambda_j)/d_{ij}$,
we have $\mathcal{D}=\sum_{i<j}s_{ij}\,w_{ij}$
and $\mathcal{S}=\sum_{i<j}s_{ij}^2$.
For the Hermite flow ($w_{ij}=s_{ij}/(n-1)$),
$\mathcal{D}=\mathcal{S}/(n-1)>0$ always.
For a general path the sign of $\mathcal{D}$ depends on the alignment between
the score-gradient and velocity-gradient vectors.
\end{remark}

\subsection{Dilation root velocity and initial acceleration}

\begin{lemma}[Dilation root ODE]
\label{lem:dil-root-ode}
Along $r_t=p\boxplus_n q_t$ (with $p,q_t$ having all distinct roots in a neighbourhood),
the root trajectories satisfy
\begin{equation}\label{eq:dil-ode}
  \dot\lambda_i(t)
  =-\frac{(\partial_t r_t)(\lambda_i(t))}{r_t'(\lambda_i(t))},
\end{equation}
where
$\displaystyle
  \partial_t r_t(x)
  =\sum_{k=1}^n k\,t^{k-1}\,b_k\,\frac{(n-k)!}{n!}\,p^{(k)}(x)
$
and $b_k$ are the coefficients of $q(x)=\sum_{k=0}^n b_k x^{n-k}$.
\end{lemma}

\begin{proof}
Direct implicit differentiation of $r_t(\lambda_i(t))\equiv 0$.
The formula for $\partial_t r_t$ follows from the MSS operator representation
$r_t=\sum_k t^k b_k D_k p$ (Lemma~\ref{lem:operator-rays}).
\end{proof}

\begin{proposition}[Initial root acceleration for centered $q$]
\label{prop:initial-accel}
Let $q$ be centered ($b_1=0$).
Then $\dot\lambda_i(0)=0$ and
\begin{equation}\label{eq:initial-accel}
  \ddot\lambda_i(0)=\frac{2\,\sigma^2(q)}{n-1}\,V_i(p),
\end{equation}
where $V_i(p)$ are the scores of $p$.
\end{proposition}

\begin{proof}
\emph{Step 1: $\dot\lambda_i(0)=0$.}
Since $b_1=0$, the term $k=1$ in~\eqref{eq:dil-ode} vanishes at $t=0$,
and all terms $k\ge 2$ carry a factor $t^{k-1}\to 0$, so $(\partial_t r_t)|_{t=0}=0$.
By~\eqref{eq:dil-ode}, $\dot\lambda_i(0)=0$.

\emph{Step 2: second-order expansion.}
Differentiating $r_t(\lambda_i(t))=0$ twice at $t=0$
(using $r_0=p$, $\dot\lambda_i(0)=0$):
\[
  p'(\lambda_i)\ddot\lambda_i(0)+(\partial_t^2 r_t)(\lambda_i)\big|_{t=0}=0.
\]
Now $\partial_t^2 r_t(x)=\sum_{k\ge 2}k(k-1)t^{k-2}b_k D_k p(x)$,
and at $t=0$ only $k=2$ survives:
$(\partial_t^2 r_t)(x)\big|_{t=0}=2b_2\,D_2 p(x)=\frac{2b_2}{n(n-1)}p''(x)$.
Thus:
\[
  \ddot\lambda_i(0)=-\frac{2b_2}{n(n-1)}\,\frac{p''(\lambda_i)}{p'(\lambda_i)}
  =-\frac{4b_2}{n(n-1)}\,V_i(p),
\]
using Lemma~\ref{lem:score-deriv}.
Since $q$ is centered with $b_1=0$, the variance formula (Lemma~\ref{lem:var-coeff}) gives
$\sigma^2(q)=-2b_2/n$, i.e.\ $b_2=-n\sigma^2(q)/2$.
Substituting:
$\ddot\lambda_i(0)=\frac{4\cdot n\sigma^2(q)}{2n(n-1)}\,V_i(p)
=\frac{2\sigma^2(q)}{n-1}\,V_i(p)$.
\end{proof}

\begin{remark}[Score-aligned acceleration]
Equation~\eqref{eq:initial-accel} shows that the initial root acceleration under
dilation is proportional to the \emph{score} $V_i$, with
a positive coefficient $2\sigma^2(q)/(n-1)$.
This is exactly the velocity field of the Hermite flow
(Lemma~\ref{lem:hermite-ode}), scaled by
$2\sigma^2(q)$. In other words, the leading-order dilation perturbation
``looks like'' a Hermite perturbation at rate $2\sigma^2(q)$
per unit~$t^2$.
\end{remark}

\subsection{The weighted score-gap functional $\Gamma^{(1)}$}

\begin{definition}[Weighted score-gap functionals]
\label{def:gamma-s}
For integer $s\ge 0$ and $p\in\PnR$ with distinct roots, define
\[
  \Gamma^{(s)}(p):=\sum_{i<j}\frac{V_j-V_i}{(\lambda_j-\lambda_i)^{1+2s}}.
\]
\end{definition}

\begin{remark}
$\Gamma^{(0)}=\Phi_n$ by Lemma~\ref{lem:score-id}~(iv).
\end{remark}

\begin{proposition}[Initial curvature of $1/\Phi_n$ along dilation]
\label{prop:initial-Fpp}
For centered $q$,
\begin{equation}\label{eq:initial-Fpp}
  F''(0)=\frac{d^2}{dt^2}\Bigl(\frac{1}{\Phi_n(r_t)}\Bigr)\Bigg|_{t=0}
  =\frac{2\,\sigma^2(q)\,\Gamma^{(1)}(p)}{(n-1)\,\mathcal{R}(p)^2}.
\end{equation}
In particular, $F''(0)>0$ if and only if $\Gamma^{(1)}(p)>0$.
\end{proposition}

\begin{proof}
From Proposition~\ref{prop:initial-accel},
$\dot\lambda_i(0)=0$ and $\ddot\lambda_i(0)=\frac{2\sigma^2(q)}{n-1}V_i$.
Using $\Phi_n=2\mathcal{R}$ and $F=1/\Phi_n=1/(2\mathcal{R})$:
\[
  F'(t)=-\frac{\dot{\mathcal{R}}}{2\mathcal{R}^2},\qquad
  \dot{\mathcal{R}}=-2\sum_{i<j}\frac{\dot\lambda_j-\dot\lambda_i}{(\lambda_j-\lambda_i)^3}.
\]
Since all $\dot\lambda_i(0)=0$, we get $\dot{\mathcal{R}}(0)=0$ and $F'(0)=0$.
At second order:
$\ddot{\mathcal{R}}(0)=-2\sum_{i<j}\frac{\ddot\lambda_j(0)-\ddot\lambda_i(0)}{(\lambda_j-\lambda_i)^3}
=-\frac{4\sigma^2(q)}{n-1}\sum_{i<j}\frac{V_j-V_i}{(\lambda_j-\lambda_i)^3}
=-\frac{4\sigma^2(q)}{n-1}\Gamma^{(1)}(p)$.
Since $F'(0)=0$:
$F''(0)=-\frac{\ddot{\mathcal{R}}(0)}{2\mathcal{R}(p)^2}
=\frac{2\sigma^2(q)\,\Gamma^{(1)}(p)}{(n-1)\mathcal{R}(p)^2}$.
\end{proof}

\begin{proposition}[$\Gamma^{(1)}>0$ for $n=3$]
\label{prop:gamma1-n3}
For every $p\in\mathcal{P}_3^{\R}$ with distinct roots,
$\Gamma^{(1)}(p)>0$.
\end{proposition}

\begin{proof}
Let the roots be $\lambda_1<\lambda_2<\lambda_3$ and set
$d:=\lambda_2-\lambda_1>0$, $e:=\lambda_3-\lambda_2>0$.
Computing the scores $V_1=-1/d-1/(d+e)$, $V_2=1/d-1/e$, $V_3=1/(d+e)+1/e$
and substituting into $\Gamma^{(1)}=\sum_{i<j}(V_j-V_i)/(\lambda_j-\lambda_i)^3$,
a mechanical computation (verified with \texttt{sympy}) yields
\begin{equation}\label{eq:gamma1-n3}
  \Gamma^{(1)}
  =\frac{(d^2+de+e^2)\,f(d,e)}{d^4\,e^4\,(d+e)^4},
\end{equation}
where
\[
  f(d,e):=2d^6+6d^5e+3d^4e^2-4d^3e^3+3d^2e^4+6de^5+2e^6.
\]
The denominator and the factor $d^2+de+e^2$ are manifestly positive.
For $f$, note that $f$ is a palindromic polynomial in $t:=e/d$:
$f=d^6(2t^6+6t^5+3t^4-4t^3+3t^2+6t+2)=:d^6\,\tilde f(t)$.
Setting $s:=t+1/t\ge 2$ (by AM-GM), we find
$\tilde f(t)=t^3\,g(s)$ with $g(s)=2s^3+6s^2-3s-16$.
Now $g(2)=18>0$ and $g'(s)=6s^2+12s-3>0$ for $s\ge 2$,
so $g(s)\ge 18$ on $[2,\infty)$, hence $\tilde f(t)>0$ for all $t>0$.
\end{proof}

\begin{remark}[Resolved conjecture]
\label{conj:gamma1}
The former conjecture ``$\Gamma^{(1)}(p)>0$ for all $n$'' is now proved in
Section~\ref{sec:agent-iteration} (Theorem~\ref{thm:gamma1-identity} and
Corollary~\ref{cor:gamma1-pos}).
\end{remark}

\subsection{Velocity decomposition and score alignment}

\begin{definition}[Score alignment and orthogonal decomposition]
\label{def:score-align}
Given root velocities $\dot\lambda_i$ and scores $V_i$,
define the \emph{score-alignment coefficient}
\[
  \alpha:=\frac{(n-1)\,\sum_i V_i\,\dot\lambda_i}{\sum_i V_i^2}
\]
and decompose
$\dot\lambda_i=\frac{\alpha}{n-1}\,V_i+\varepsilon_i$,
where $\sum_i V_i\varepsilon_i=0$
(so $\varepsilon$ is orthogonal to the score vector in $\ell^2$).
The \emph{aligned dissipation} is
$\mathcal{D}_\parallel:=\frac{\alpha}{n-1}\mathcal{S}$,
and the \emph{perpendicular dissipation} is
$\mathcal{D}_\perp:=\mathcal{D}-\mathcal{D}_\parallel$.
\end{definition}

\begin{remark}
In the Hermite case, $\alpha\equiv 1$ and $\varepsilon\equiv 0$,
so $\mathcal{D}=\mathcal{D}_\parallel=\mathcal{S}/(n-1)$.
\end{remark}

The following conjectures are supported by extensive numerical experiments
(code: \texttt{route\_b\_experiments.py} and \texttt{route\_b\_deep.py}).

\begin{conjecture}[Score alignment]
\label{conj:align}
Along the dilation path $r_t=p\boxplus_n q_t$ (with centered $q$, $t\in(0,1]$),
the score-alignment coefficient satisfies $\alpha(t)>0$.
\end{conjecture}

\noindent
\emph{Numerical evidence:}
In the reported sweeps ($n=3,4,5,6$), all sampled trajectories satisfy $\alpha(t)>0$;
this is consistent with Proposition~\ref{prop:initial-accel}, which gives
$\alpha(t)\sim 2\sigma^2(q)t$ as $t\downarrow 0$.

\begin{conjecture}[Perpendicular dissipation sign]
\label{conj:perp-sign}
Along the dilation path,
$\mathcal{D}_\perp(t)\le 0$ for all $t\in(0,1]$.
In other words, the non-Hermite component of the velocity always
\emph{retards} the Fisher dissipation, but does not reverse it.
\end{conjecture}

\noindent
\emph{Numerical evidence:}
Observed negative on all sampled trajectories tested in this work;
moreover $|\mathcal D_\perp|<\mathcal D_\parallel$ in those samples.

\subsection{Repulsion monotonicity and pointwise dilation Stam}

\begin{conjecture}[Repulsion monotonicity]
\label{conj:rep-monotone}
Along the dilation path $r_t=p\boxplus_n q_t$, the repulsion energy is non-increasing:
\[
  \frac{d}{dt}\mathcal{R}(r_t)\le 0\qquad\text{for all }t\in[0,1].
\]
Equivalently, $\Phi_n(r_t)$ is non-increasing, and
$F(t):=1/\Phi_n(r_t)$ is non-decreasing.
\end{conjecture}

\noindent
\emph{Numerical evidence:}
No violations were observed in the reported random-grid sweeps.

\begin{remark}
Conjecture~\ref{conj:rep-monotone} combined with Conjectures~\ref{conj:align}
and~\ref{conj:perp-sign} would follow from
$\mathcal{D}(t)=\mathcal{D}_\parallel(t)+\mathcal{D}_\perp(t)\ge 0$,
which says the (positive) score-aligned dissipation always
dominates the (negative) perpendicular correction.
\end{remark}

\begin{conjecture}[Pointwise dilation Stam]
\label{conj:pointwise-stam}
For all $p,q\in\PnR$ with distinct roots and positive variance,
\begin{equation}\label{eq:pointwise-stam}
  \frac{1}{\Phi_n(r_t)}\ge \frac{1}{\Phi_n(p)}+\frac{t^2}{\Phi_n(q)}
  \qquad\text{for all }t\in[0,1].
\end{equation}
Evaluating at $t=1$ recovers the Stam inequality.
\end{conjecture}

\noindent
\emph{Numerical evidence:}
No violations were observed in the reported random and near-collision sweeps;
the smallest sampled deficits are at machine-precision scale.

\begin{remark}[Hierarchy of conjectures]
The implications are:
\[
  \text{Conj.~\ref{conj:pointwise-stam} (pointwise)}
  \;\Longrightarrow\;
  \text{Stam inequality}
  \;\Longrightarrow\;
  E(1)\ge 0.
\]
Also, Conjecture~\ref{conj:rep-monotone} (monotonicity $F'\ge 0$) alone gives
only $F(1)\ge F(0)=1/\Phi_n(p)$,
which is weaker than Stam.
However, monotonicity \emph{combined with an initial curvature bound}
(Proposition~\ref{prop:initial-Fpp} and $\Gamma^{(1)}>0$)
could potentially yield Stam via an integrated comparison argument.
\end{remark}

\subsection{Route~B checkpoint (condensed)}

Route~B has three durable facts and one bottleneck:
\begin{enumerate}[label=\textup{(B\arabic*)},leftmargin=2.5em]
  \item Initial dilation acceleration is score-aligned
  (Proposition~\ref{prop:initial-accel}), so the path starts in a Hermite-like direction.
  \item Pointwise derivative proxies are false (including $F'(t)\ge 2t/\Phi_n(q)$), so the proof
  must use integrated/ODE control, not a local slope inequality.
  \item Direct Hermite domination fails globally; only a perturbative comparison is realistic.
  \item The decisive open sign is $\mathcal{D}_\perp\le 0$
  (Conjecture~\ref{conj:perp-sign}), now coupled with proved
  $\Gamma^{(1)}=\mathcal{S}/2$ in Section~\ref{sec:agent-iteration}.
\end{enumerate}

%======================================================================
\appendix
\section{A numerical counterexample to dilation convexity}
\label{app:counterexample}

We record an explicit brute-force example showing that neither $t\mapsto 1/\Phi_n(r_t)$ nor
the dilation excess $E(t)$ must be convex.

\medskip
\noindent\textbf{Computational protocol.}
All computations were done in Python~3 with \texttt{numpy} (64-bit IEEE~754).
We approximate $F''(t)$ by the centered stencil
$\bigl(F(t+h)-2F(t)+F(t-h)\bigr)/h^2$ on a uniform $200$-point grid in $[0,1]$,
compute $r_t=p\boxplus_n q_t$ from Definition~\ref{def:conv}, and extract roots with
\texttt{numpy.roots}. Real-rootedness is checked by $|\operatorname{Im}\lambda|<10^{-8}$
(well above observed noise $\sim10^{-15}$). Scripts are available in the cited repository.

\medskip
\noindent\textbf{Example~1 ($n=3$).} $p$ with roots $(-2,-\tfrac32,\tfrac32)$ and $q$ with roots $(-5,2,3)$ (so $q$ is centered).
Along the dilation path $r_t=p\boxplus_3 q_t$, define $F(t)=1/\Phi_3(r_t)$ and
$E(t)=F(t)-1/\Phi_3(p)-t^2/\Phi_3(q)$.

A finite-difference computation (step $h=10^{-5}$, checked on the full $200$-point grid) yields
a negative second derivative:
\[
  F''(t^*)\approx -8.16\quad\text{at }t^*\approx 0.435.
\]
Since $2/\Phi_3(q)\approx 0.965$, this also forces $E''(t^*)\approx -9.12<0$.
Nevertheless $E(1)\approx 2.18>0$, so the Stam inequality holds in this example.

Raw convexity $F''(t)\ge 0$ fails in higher degrees as well.

\medskip
\noindent\textbf{Example~2 ($n=4$).}
Take $p$ with roots
$(-1.10743,-0.81774,-0.36839,0.42118)$ and $q$ with centered roots
$(-1.57864,-1.22305,-0.93765,3.73934)$. A finite-difference computation (step size $h=2\cdot 10^{-4}$)
gives $F''(0.3)\approx -0.14$ (and already $F''(0.2)\approx -0.12$), so $t\mapsto 1/\Phi_4(r_t)$ need not be convex.

This appendix is included to prevent overfitting the analysis to a false convexity narrative.

%======================================================================
\section{Reproducibility note}
\label{app:independent-check}

An independent rerun (2026-02-12) of
\texttt{route\_c\_experiments.py}, \texttt{route\_c\_epi\_deep.py},
\texttt{route\_b\_experiments.py}, \texttt{verify\_gamma1.py}, and
\texttt{test\_repulsion\_stam.py}
reproduced the qualitative numerical claims used in this paper:
the reported false proxies remain false, no sampled violations were found for
the main tested Stam/EPI inequalities, and the Route~B sign patterns were consistent.
These computations are consistency checks only and are not used as proof inputs.

%======================================================================
\section{Agent iteration results: key identity
\texorpdfstring{$\Gamma^{(1)}=\mathcal{S}/2$}{Gamma(1)=S/2}
and consequences}
\label{sec:agent-iteration}
%======================================================================

This section records the results of an automated proof-search iteration
(February~2026).
The principal breakthrough is an identity
$\Gamma^{(1)}=\mathcal{S}/2$ that settles
Conjecture~\ref{conj:gamma1} in full generality;
we also record several corrections and dead ends discovered during the process.

\subsection{The identity $\Gamma^{(1)}=\frac{1}{2}\mathcal{S}$}

\begin{theorem}[$\Gamma^{(1)}$ equals half the score-gradient energy]
\label{thm:gamma1-identity}
For every $n\ge 2$ and every configuration of distinct reals
$\lambda_1<\cdots<\lambda_n$,
\begin{equation}\label{eq:gamma1-half-S}
  \Gamma^{(1)}(p)\;=\;\frac{1}{2}\,\mathcal{S}(p)
  \;=\;\frac{1}{2}\sum_{i<j}\frac{(V_i-V_j)^2}{(\lambda_i-\lambda_j)^2}.
\end{equation}
\end{theorem}

\begin{proof}
Set $W_k:=\sum_{j\ne k}(\lambda_k-\lambda_j)^{-3}$. Pair-symmetrization gives
\begin{equation}\label{eq:gamma1-VW}
  \Gamma^{(1)}=\sum_k V_kW_k.
\end{equation}
Since $\mathcal{R}=\sum_{i<j}(\lambda_j-\lambda_i)^{-2}$,
$\partial_{\lambda_k}\mathcal{R}=-2W_k$, so
\begin{equation}\label{eq:gamma1-inner}
  \Gamma^{(1)}=-\tfrac12\langle\nabla H,\nabla\mathcal{R}\rangle,
\end{equation}
using $V_k=\partial_{\lambda_k}H$ (Lemma~\ref{lem:grad-H}).
Along the Hermite flow $\dot\lambda_i=V_i/(n-1)$,
\[
\frac{d}{dt}\mathcal{R}
=\frac{1}{n-1}\langle\nabla\mathcal{R},\nabla H\rangle
=-\frac{2}{n-1}\Gamma^{(1)}.
\]
But also (Lemma~\ref{lem:hermite-dissip} and $\Phi_n=2\mathcal R$)
\[
\frac{d}{dt}\mathcal{R}=\frac12\frac{d}{dt}\Phi_n=-\frac{1}{n-1}\mathcal{S}.
\]
Hence $\Gamma^{(1)}=\mathcal{S}/2$.
\end{proof}

\begin{remark}[Algebraic status]
The flow argument is only a chain-rule device; the identity is algebraic in the roots.
It is symbolically verified for $n\le 5$ and numerically to machine precision for
$n=3,\ldots,10$.
\end{remark}

\begin{corollary}[$\Gamma^{(1)}>0$ for all $n$]
\label{cor:gamma1-pos}
For every $n\ge 2$ and every $p\in\PnR$ with distinct roots,
$\Gamma^{(1)}(p)>0$.
\end{corollary}

\begin{proof}
$\mathcal{S}(p)=\sum_{i<j}(V_i-V_j)^2/(\lambda_i-\lambda_j)^2>0$
for distinct roots (every summand is a positive ratio of squares;
at least one is nonzero since the scores are not all equal when $n\ge 2$).
\end{proof}

\begin{remark}
This supersedes the coefficient-positivity proofs of $\Gamma^{(1)}>0$
for $n=3,4,5$ (Proposition~\ref{prop:gamma1-n3}
and the agent report's Theorems~2.1--2.3);
those were hard-won case-by-case computations, whereas
Corollary~\ref{cor:gamma1-pos} is a three-line argument valid
for \emph{every}~$n$.
\end{remark}

\subsection{Consequences for the dilation path}

\begin{corollary}[Universal positive initial curvature]
\label{cor:initial-curv}
For all $n\ge 2$, for all $p\in\PnR$ with distinct roots and
all centered $q$ with $\sigma^2(q)>0$,
the dilation excess $F(t):=1/\Phi_n(r_t)$ satisfies
\[
  F''(0)=\frac{2\,\sigma^2(q)\,\Gamma^{(1)}(p)}{(n-1)\,\mathcal{R}(p)^2}
  =\frac{\sigma^2(q)\,\mathcal{S}(p)}{(n-1)\,\mathcal{R}(p)^2}>0.
\]
\end{corollary}

\begin{proof}
Combine Proposition~\ref{prop:initial-Fpp} with Theorem~\ref{thm:gamma1-identity}.
\end{proof}

\begin{corollary}[Quantitative lower bound on $F''(0)$]
\label{cor:Fpp-lower}
Under the same hypotheses,
\[
  F''(0)\ge \frac{2\,\sigma^2(q)}{\sigma^2(p)\,\Phi_n(p)}
  \ge \frac{8\,\sigma^2(q)}{n(n-1)^2}.
\]
\end{corollary}

\begin{proof}
From $\Gamma^{(1)}=\mathcal{S}/2$ and the
Score-Gradient Inequality (Theorem~\ref{thm:sgi}):
$\mathcal{S}\ge(n{-}1)\Phi_n/(2\sigma^2(p))=(n{-}1)\mathcal{R}/\sigma^2(p)$.
Thus
$F''(0)=\sigma^2(q)\mathcal{S}/((n{-}1)\mathcal{R}^2)
\ge\sigma^2(q)/(\sigma^2(p)\mathcal{R})
=2\sigma^2(q)/(\sigma^2(p)\Phi_n)$.
The second bound uses $\Phi_n\sigma^2\ge n(n{-}1)^2/4$.
\end{proof}

\subsection{Gradient-flow interpretation}

\begin{corollary}[Repulsion decreases along the score flow]
\label{cor:score-flow}
Let the ``score flow'' be $\dot\lambda_k=V_k$.
Then
$\frac{d}{dt}\mathcal{R}=-2\,\Gamma^{(1)}=-\mathcal{S}<0$.
In words: the gradient flow of the log-Vandermonde $H$ is a descent flow
for the repulsion energy $\mathcal{R}$.
\end{corollary}

\begin{proof}
$\frac{d}{dt}\mathcal{R}=\langle\nabla\mathcal{R},V\rangle
=\langle\nabla\mathcal{R},\nabla H\rangle=-2\Gamma^{(1)}=-\mathcal{S}<0$.
\end{proof}

\begin{remark}[Geometric meaning]
$\Gamma^{(1)}>0$ is equivalent to
$\langle\nabla H,\nabla\mathcal{R}\rangle<0$: increasing Vandermonde entropy
decreases repulsion, with exact coupling constant $\mathcal{S}/2$.
\end{remark}

\subsection{Derivative compatibility}

\begin{theorem}[Derivative compatibility with $\boxplus_n$]
\label{thm:deriv-compat-full}
For $p,q\in\PnR$, let $\tilde p:=p'(x)/n$ denote the monic derivative
(degree $n{-}1$). Then
\[
  \widetilde{p\boxplus_n q}\;=\;\tilde p\,\boxplus_{n-1}\,\tilde q.
\]
\end{theorem}

\begin{proof}
Let $K_p(z)=\sum_{k=0}^n\frac{a_k}{\binom{n}{k}}z^k$ be the
$K$-transform, so that $K_{p\boxplus_n q}=K_p\cdot K_q\pmod{z^{n+1}}$.
The monic derivative $\tilde p$ has coefficients
$\tilde a_k=\frac{n-k}{n}a_k$ for $0\le k\le n{-}1$.
Then
\[
  K_{\tilde p}(z)
  =\sum_{k=0}^{n-1}\frac{\tilde a_k}{\binom{n{-}1}{k}}z^k
  =\sum_{k=0}^{n-1}\frac{a_k}{\binom{n}{k}}z^k
  =K_p(z)\bmod z^n,
\]
where the second equality uses
$\frac{(n-k)/n}{\binom{n-1}{k}}=\frac{1}{\binom{n}{k}}$.
Therefore
$K_{\widetilde{p\boxplus q}}=K_{p\boxplus q}\bmod z^n
=K_pK_q\bmod z^n
=(K_p\bmod z^n)(K_q\bmod z^n)\bmod z^n
=K_{\tilde p}K_{\tilde q}\bmod z^n
=K_{\tilde p\boxplus_{n-1}\tilde q}$.
\end{proof}

\begin{corollary}[Stam consistency across degrees]
\label{cor:stam-consistency}
If the Stam inequality holds at degree $n{-}1$, then
\[
  \frac{1}{\Phi_{n-1}(\widetilde{p\boxplus_n q})}
  \ge\frac{1}{\Phi_{n-1}(\tilde p)}+\frac{1}{\Phi_{n-1}(\tilde q)}.
\]
\end{corollary}

\begin{remark}[Induction gap]
To close an induction on~$n$, one would need
$\Phi_{n-1}(\tilde p)\ge\Phi_n(p)$ (derivative reduces Fisher),
which is numerically confirmed but would produce the wrong inequality on the LHS:
one needs $1/\Phi_n(p\boxplus q)\ge 1/\Phi_{n-1}(\widetilde{p\boxplus q})$,
i.e.\ $\Phi_{n-1}\ge\Phi_n$ for the convolution --- which goes the right way ---
but the RHS then gives $\ge 1/\Phi_{n-1}(\tilde p)+1/\Phi_{n-1}(\tilde q)$,
which is \emph{larger} than $1/\Phi_n(p)+1/\Phi_n(q)$, not smaller.
Thus the induction does not close without a refined comparison lemma
linking $\Phi_n$ and $\Phi_{n-1}$ on both sides simultaneously.
\end{remark}

%======================================================================
\subsection{Dead ends and corrections}
\label{sec:dead-ends-iter}
%======================================================================

The following approaches were systematically tested and found to
be non-viable:

\begin{enumerate}[label=\textup{(D\arabic*)},leftmargin=2.5em]
  \item \textbf{P\'olya frequency (PF) sequences from $K_p$.}
  The $K$-transform coefficients $w_k=a_k/\binom{n}{k}$ generate a
  Toeplitz matrix that is \emph{not} totally positive:
  for $n=3$, 199/200 random tests have negative $2\!\times\!2$ minors;
  for $n\ge 4$ the failure rate is $100\%$.
  The sequence $w_k$ \emph{is} ultra-log-concave (ULC), i.e.\
  $w_k^2\ge w_{k-1}w_{k+1}$, which is a known result for real-rooted
  polynomials (Newton's inequalities), but ULC alone does not control
  $\Phi_n$ or~$\mathcal{R}$.

  \item \textbf{Zeros of $K_p(z)$ are not all real-negative.}
  For $n\ge 3$, over $90\%$ of random real-rooted polynomials have
  complex $K$-transform zeros.
  In particular, the Aissen--Schoenberg--Whitney total positivity framework
  does not apply.

  \item \textbf{Concavity of $1/\Phi_n$ in $w$-coordinates.}
  In the coordinates where $\boxplus_n$ becomes multiplication of
  $K$-transforms, $1/\Phi_n$ is \emph{not} concave:
  the Hessian has large positive eigenvalues at every tested point
  (100/100 non-concave at $n=3$, 96/100 at $n=4$).
  A direct concavity-implies-Stam argument is therefore impossible
  in these coordinates.

  \item \textbf{Schur-concavity of $1/\mathcal{R}$ in adjacent gaps.}
  $1/\mathcal{R}$ is \emph{not} Schur-concave in the adjacent gap vector
  $(d_1,\ldots,d_{n-1})$: T-transform tests give 29/500 violations at $n=5$.
  The full pairwise-gap vector formulation remains open.

  \item \textbf{Componentwise gap super-additivity.}
  The bound $(\rho_j-\rho_i)^2\ge(\lambda_j-\lambda_i)^2+(\mu_j-\mu_i)^2$
  for individual gap pairs of $p\boxplus_n q$ vs.\ $p$ and $q$ fails
  generically.
  However, the \emph{total} squared-gap sum
  $\sum_{i<j}(\rho_j-\rho_i)^2\ge\sum_{i<j}(\lambda_j-\lambda_i)^2
  +\sum_{i<j}(\mu_j-\mu_i)^2$ holds in all tested cases
  (this follows from the relation
  $\sum_{i<j}d_{ij}^2=n\sum_i\lambda_i^2-(\sum_i\lambda_i)^2$
  and variance additivity).

  \item \textbf{Derivative bound $F'(t)\ge 2t/\Phi_n(q)$ pointwise.}
  Fails in ${\sim}3\%$ of tested cases at $n=3$.
  The integrated comparison $\int_0^1 F'(t)\,dt\ge 1/\Phi_n(q)$ (which is
  just $F(1)-F(0)\ge 1/\Phi_n(q)$, a restatement of Stam) survives,
  but the pointwise derivative bound is too strong.
\end{enumerate}

%======================================================================
\subsection{Numerical landscape summary}
\label{sec:numerical-landscape}
%======================================================================

\begin{center}
\renewcommand{\arraystretch}{1.3}
\begin{tabular}{|l|c|l|}
\hline
\textbf{Claim} & \textbf{Status} & \textbf{Evidence / Method}\\
\hline
$\Gamma^{(1)}=\mathcal{S}/2$ (Thm~\ref{thm:gamma1-identity}) & \textbf{Proved} & Hermite chain rule, symbolic $n\le 5$\\
$\Gamma^{(1)}>0$ for all $n$ (Cor~\ref{cor:gamma1-pos}) & \textbf{Proved} & Immediate from $\mathcal{S}>0$\\
$F''(0)>0$ universally (Cor~\ref{cor:initial-curv}) & \textbf{Proved} & $=\sigma^2(q)\mathcal{S}/((n{-}1)\mathcal{R}^2)$\\
Derivative compatibility (Thm~\ref{thm:deriv-compat-full}) & \textbf{Proved} & $K$-transform truncation\\
Coeff.\ positivity $n=4$ (194~terms) & \textbf{Proved} & Symbolic; superseded by Thm~\ref{thm:gamma1-identity}\\
Coeff.\ positivity $n=5$ (6773~terms) & \textbf{Proved} & Symbolic; superseded by Thm~\ref{thm:gamma1-identity}\\
\hline
Stam for general $n$ & \textbf{Open} & $0$ violations in $29{,}000+$ tests, $n\le 12$\\
EPI (Conj~\ref{conj:epi}) for general $n$ & \textbf{Open} & $0$ violations in $29{,}000+$ tests, $n\le 8$\\
Pointwise dilation Stam ($E(t)/t^2\ge0$) & Plausible & $0$ violations, $n=3,\ldots,6$\\
Score alignment $\alpha(t)>0$ (Conj~\ref{conj:align}) & Plausible & $0$ violations, $n=3,\ldots,6$\\
$\mathcal{D}_\perp\le 0$ (Conj~\ref{conj:perp-sign}) & Plausible & Observed universally\\
Repulsion monotonicity (Conj~\ref{conj:rep-monotone}) & Plausible & $0$ violations in $800+$ tests\\
\hline
PF sequence (D1) & \textbf{False} & 199/200 negative minors at $n=3$\\
$K_p$ zeros real-negative (D2) & \textbf{False} & $>90\%$ complex zeros\\
$1/\Phi_n$ concave in $w$-coords (D3) & \textbf{False} & 100/100 non-concave\\
$1/\mathcal{R}$ Schur-concave in adj.\ gaps (D4) & \textbf{False} & 29/500 violations at $n=5$\\
Componentwise gap super-additivity (D5) & \textbf{False} & Generic failures\\
Pointwise $F'\ge 2t/\Phi_n(q)$ (D6) & \textbf{False} & ${\sim}3\%$ failures\\
\hline
\end{tabular}
\end{center}

%======================================================================
\subsection{Sharpened roadmap for the general proof (compressed handoff)}
\label{sec:sharpened-roadmap}
%======================================================================

\begin{enumerate}[label=\textup{(R\arabic*)},leftmargin=2.5em]

  \item \textbf{[Priority~1] ODE closure on dilation path.}
  With $E(0)=E'(0)=0$ and $E''(0)>0$, set $G(t):=E(t)/t^2$.
  Stam is $G(1)\ge 0$. The remaining obstruction is a sign/control estimate on
  $\mathcal{D}_\perp$ in the ODE for $G$ (via Lemma~\ref{lem:gen-dissip} and
  Definition~\ref{def:score-align}). Any bound of Gr\"onwall type that prevents
  zero-crossing of $G$ closes the proof.

  \item \textbf{[Priority~2] Dilation-aware SGI.}
  Upgrade frozen-time SGI (Theorem~\ref{thm:sgi}) to a pathwise inequality for $r_t$
  strong enough to force a one-sided differential inequality for $G$.

  \item \textbf{[Priority~3] Hermite perturbation route.}
  Treat dilation as Hermite plus non-Hermite correction ($O(t^3)$); combine
  Theorem~\ref{thm:hermite-bound} with explicit control of the correction through
  $\mathcal{D}_\perp$.

  \item \textbf{[Priority~4] ULC-to-Fisher algebraization.}
  Since $w_k=a_k/\binom{n}{k}$ is ULC and multiplicative under $\boxplus_n$,
  seek an explicit bridge from $w$-data to $\mathcal{R}$ (or $1/\Phi_n$)
  via Newton identities/power sums.

  \item \textbf{[Lower priority] EPI cross-validation.}
  A strategy proving both Stam and Conjecture~\ref{conj:epi} is likely structural,
  not accidental.

  \item \textbf{[Lower priority] HCIZ/random-matrix path.}
  The missing analytic link is sub-additivity of
  $\mathbb{E}[1/\Phi_n(A+UBU^*)]$ in $(A,B)$.
\end{enumerate}

\medskip\noindent
\textbf{Key open sub-problem.}
The single most important open question is:
\emph{prove $\mathcal{D}_\perp(t)\le 0$ along the dilation path}
(Conjecture~\ref{conj:perp-sign}).
Combined with the now-proved $\Gamma^{(1)}>0$, score alignment,
and the initial curvature bound, this would close the Stam inequality
for all~$n$.

%======================================================================
\section{Bibliographic notes}
%======================================================================

\begin{thebibliography}{9}

\bibitem{MSS15}
A.~Marcus, D.~A.~Spielman, and N.~Srivastava,
\emph{Interlacing families {II}: Mixed characteristic polynomials and the {K}adison--{S}inger problem},
Ann.\ of\ Math.\ \textbf{182} (2015), 327--350.

\bibitem{Stam59}
A.~J.~Stam,
\emph{Some inequalities satisfied by the quantities of information of {F}isher and {S}hannon},
Inform.\ Control \textbf{2} (1959), 101--112.

\bibitem{Shannon48}
C.~E.~Shannon,
\emph{A mathematical theory of communication},
Bell Syst.\ Tech.\ J.\ \textbf{27} (1948), 379--423, 623--656.

\bibitem{Villani09}
C.~Villani,
\emph{Optimal Transport: Old and New},
Grundlehren der mathematischen Wissenschaften, vol.~338, Springer, 2009.

\end{thebibliography}

\bigskip
\noindent\textbf{Data availability.}
Supplementary Python scripts and further numerical experiments are available at
\url{https://github.com/omegaestable/math-docs}.
Scripts include:
\texttt{route\_a\_core.py} (core library),
\texttt{route\_a\_experiments.py} (Route~A proxies),
\texttt{test\_repulsion\_stam.py} (large-scale Stam tests),
\texttt{route\_b\_experiments.py} (Route~B dilation path experiments),
\texttt{route\_b\_deep.py} (Route~B targeted follow-ups and stress tests),
\texttt{route\_c\_experiments.py} (Route~C transport experiments),
\texttt{route\_c\_epi\_deep.py} (discriminant power inequality stress tests),
\texttt{route\_c\_n3\_epi\_proof.py} ($n{=}3$ EPI proof verification).

\end{document}
