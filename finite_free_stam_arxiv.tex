\documentclass[11pt]{amsart}

\usepackage[margin=1in]{geometry}
\usepackage[T1]{fontenc}
\usepackage{lmodern}
\usepackage{microtype}
\usepackage{amsmath,amssymb,amsthm,mathtools}
\usepackage{enumitem}
\usepackage[colorlinks=true,linkcolor=blue,citecolor=blue,urlcolor=blue]{hyperref}

\allowdisplaybreaks
\setlength{\jot}{7pt}

%---------------- Theorem envs ----------------
\theoremstyle{plain}
\newtheorem{theorem}{Theorem}[section]
\newtheorem{lemma}[theorem]{Lemma}
\newtheorem{proposition}[theorem]{Proposition}
\newtheorem{corollary}[theorem]{Corollary}

\theoremstyle{definition}
\newtheorem{definition}[theorem]{Definition}
\newtheorem{conjecture}[theorem]{Conjecture}
\newtheorem{problem}[theorem]{Problem}

\theoremstyle{remark}
\newtheorem{remark}[theorem]{Remark}

%---------------- Macros ----------------
\newcommand{\R}{\mathbb{R}}
\newcommand{\C}{\mathbb{C}}
\newcommand{\PnR}{\mathcal{P}^{\R}_n}
\newcommand{\Var}{\mathrm{Var}}
\newcommand{\E}{\mathrm{E}}
\newcommand{\sgn}{\mathrm{sgn}}

%============================================================
\begin{document}

\title[v5: Finitefree Stam via integrated defect inequalities]{Toward a Finite--Free Stam Inequality for MSS Polynomial Convolution:\\
v5 --- Corrected roadmap, rigorous local calculus, and the true missing bridge}
\author{Prepared with adversarial review corrections}
\date{\today}

\begin{abstract}
For monic real--rooted polynomials $p,q$ of degree $n$, Marcus--Spielman--Srivastava (MSS) introduced a finite--free additive convolution $p\boxplus_n q$ that preserves real--rootedness. A natural Fisher--information functional $\Phi_n$ on simple--rooted polynomials (a discrete log--interaction energy) suggests an analogue of Stam's inequality:
\[
\frac{1}{\Phi_n(p\boxplus_n q)}\ \ge\ \frac{1}{\Phi_n(p)}+\frac{1}{\Phi_n(q)}.
\]
This note presents: (i) a fully rigorous calculus for $\Phi_n$ along smooth coefficient paths, including correct root labeling and piecewise integration across collision times; (ii) unconditional identities relating the derivative of $F(t):=1/\Phi_n(r_t)$ to a dissipation functional $D(t)$; (iii) exact variance additivity along dilation paths $r_t=p\boxplus_n q_t$; and (iv) a corrected meta--statement of the central open step. 
\smallskip

\noindent\textbf{Main correction (v5):} prior drafts incorrectly targeted a pointwise sign claim for a ``perpendicular dissipation'' term. Such sign claims are false for natural decompositions when $n\ge 4$, and even if true would not produce the \emph{exact} additive constant $1/\Phi_n(q)$. The true missing bridge is an \emph{integrated/telescoping defect inequality} that yields precisely the $t^2/\Phi_n(q)$ term along dilation. We formulate this bridge as explicit open problems in a way that does not restate Stam in disguise.
\end{abstract}

\maketitle
\tableofcontents

%============================================================
\section{Setup: MSS convolution, centering, and the Fisher information}

\subsection{MSS finite--free additive convolution}
Let $\PnR$ denote the set of monic degree-$n$ polynomials with all roots real (allowing multiplicity). 
Write
\[
p(x)=\sum_{k=0}^n a_k x^{n-k},\qquad q(x)=\sum_{k=0}^n b_k x^{n-k},
\qquad a_0=b_0=1.
\]
The MSS finite--free additive convolution is the monic polynomial
\[
(p\boxplus_n q)(x):=\sum_{k=0}^n c_k x^{n-k},
\qquad
c_k:=\sum_{i+j=k}\frac{(n-i)!(n-j)!}{n!(n-k)!}\,a_i b_j.
\]
We use the standard facts: $p\boxplus_n q\in \PnR$ whenever $p,q\in \PnR$ (real--rootedness preservation), and translation covariance (see below). 

\begin{remark}[Scope]
This note does not reprove MSS real--rootedness preservation; it uses it as an input. The v5 goal is to make all subsequent calculus statements correct \emph{conditional} on real--rootedness of the relevant polynomials on each interval.
\end{remark}

\subsection{Root notation and the Fisher information functional}
If $r\in \PnR$ has \emph{simple} roots $\lambda_1<\cdots<\lambda_n$, define the discrete log--interaction field
\[
V_i(r)\ :=\ \sum_{j\neq i}\frac{1}{\lambda_i-\lambda_j},
\qquad i=1,\dots,n,
\]
and the Fisher information
\[
\Phi_n(r)\ :=\ \sum_{i=1}^n V_i(r)^2.
\]
If $r$ has repeated roots, then $V_i$ is not defined and $\Phi_n(r)=+\infty$ is the correct convention for our purposes; we set $1/\Phi_n(r)=0$ in that case.

\subsection{Centering and variance}
Write the empirical mean of roots as
\[
m(r):=\frac{1}{n}\sum_{i=1}^n \lambda_i,
\]
and the empirical variance as
\[
\sigma^2(r):=\frac{1}{n}\sum_{i=1}^n (\lambda_i-m(r))^2.
\]
Centering means translating so that $m(r)=0$. Translation covariance of MSS convolution implies that, after centering $q$, one may assume $m(q)=0$ without loss in Stam-type inequalities.

%============================================================
\section{Rigor layer: root labeling, differentiability, collisions}

This section fixes paper-level issues that must be correct regardless of the main inequality.

\subsection{Smooth labeling on simple-root intervals}

\begin{lemma}[Implicit-function root labeling]\label{lem:root-label}
Let $I\subset\R$ be an open interval and $t\mapsto r_t$ a $C^1$ family of monic polynomials of degree $n$ whose coefficients are $C^1$ in $t$. 
Assume that for some $t_0\in I$, $r_{t_0}$ has $n$ distinct real roots.
Then there exists $\varepsilon>0$ and $C^1$ functions $\lambda_1(t),\dots,\lambda_n(t)$ on $(t_0-\varepsilon,t_0+\varepsilon)$ such that
\[
r_t(x)=\prod_{i=1}^n (x-\lambda_i(t)),
\qquad \lambda_1(t)<\cdots<\lambda_n(t),
\]
and for each $i$,
\[
\dot\lambda_i(t)=-\frac{\partial_t r_t(\lambda_i(t))}{r_t'(\lambda_i(t))}.
\]
If the coefficients are $C^k$ (resp. $C^\infty$) then so are the $\lambda_i$.
\end{lemma}

\begin{proof}
For each simple root $\lambda_i(t_0)$, the equation $r_t(x)=0$ has $\partial_x r_{t_0}(\lambda_i(t_0))=r'_{t_0}(\lambda_i(t_0))\neq 0$, so the implicit function theorem yields a unique $C^1$ root branch near $(t_0,\lambda_i(t_0))$.
Ordering is preserved for $\varepsilon$ small since roots depend continuously on $t$ and cannot cross without collision. The derivative formula follows by differentiating $r_t(\lambda_i(t))\equiv 0$.
\end{proof}

\subsection{Piecewise calculus across collision times}

\begin{definition}[Simple-root intervals and collision set]
Given a continuous polynomial path $t\mapsto r_t\in \PnR$, define the collision set
\[
\mathcal{C}:=\{t:\ r_t \text{ has a repeated root}\}.
\]
A \emph{simple-root interval} is a connected component of $[0,1]\setminus \mathcal{C}$.
\end{definition}

\begin{lemma}[Piecewise integration principle]\label{lem:piecewise-int}
Assume $t\mapsto r_t$ is $C^1$ in coefficients on $[0,1]$ and $r_t\in\PnR$ for all $t$.
Let $F(t)$ be a functional defined for simple-root $r_t$ such that on each simple-root interval $J$ it is absolutely continuous and satisfies $F'(t)=G(t)$ a.e. on $J$.
Assume furthermore that $F$ is bounded below on $[0,1]$ and that we extend $F(t)$ at collision times by lower semicontinuous limits along $J$.
Then
\[
F(1)-F(0)\ \ge\ \sum_{J}\int_J G(t)\,dt,
\]
where the sum ranges over all simple-root intervals $J$.
\end{lemma}

\begin{proof}
Write the simple-root intervals as disjoint open intervals $J_\alpha=(a_\alpha,b_\alpha)$.
By absolute continuity on each $J_\alpha$,
\[
F(b_\alpha^-)-F(a_\alpha^+)=\int_{J_\alpha}G(t)\,dt.
\]
Summing over all components gives
\[
\sum_\alpha\int_{J_\alpha}G(t)\,dt
=\sum_\alpha\big(F(b_\alpha^-)-F(a_\alpha^+)\big).
\]
Order the collision points increasingly and telescope the right-hand side. Interior endpoint terms cancel up to jumps across collisions. By lower semicontinuous extension, each jump contributes a nonnegative amount in the inequality direction needed here, so the telescoped sum is bounded above by $F(1)-F(0)$. Hence
\[
\sum_J\int_J G(t)\,dt\le F(1)-F(0).
\]
\end{proof}

\begin{remark}[What this lemma does / does not do]
This lemma is deliberately weak: it does not claim continuity at collisions, only that integrating on each simple-root interval and summing gives a valid lower bound under mild semicontinuity. For Stam, this is the correct ``patching across collisions'' technology: any inequality proven a.e.\ on simple-root intervals may be integrated intervalwise and summed.
\end{remark}

%============================================================
\section{Unconditional calculus: derivative of Phi-n and dissipation}

We now derive identities that are true on simple-root intervals without any conjectural sign claims.

\subsection{Algebraic identities for Vi}
Fix a simple-root polynomial $r(x)=\prod_{i=1}^n (x-\lambda_i)$.
Define for $i\neq j$ the weights
\[
W_{ij}:=\frac{1}{\lambda_i-\lambda_j},
\qquad\text{so}\qquad V_i=\sum_{j\neq i} W_{ij}.
\]
Then $W_{ij}=-W_{ji}$ and $\sum_i V_i=0$.

\subsection{Dissipation functional tied to a velocity field}
Let $(\lambda_i(t))_{i=1}^n$ be a $C^1$ labeled root path on a simple-root interval.
Define $V_i(t)=V_i(r_t)$ and write the root velocity as $\dot\lambda_i(t)$.

A natural dissipation quadratic form that appears repeatedly is
\[
D(t)\ :=\ \sum_{1\le i<j\le n}\frac{\big(V_i(t)-V_j(t)\big)\big(\dot\lambda_i(t)-\dot\lambda_j(t)\big)}{(\lambda_i(t)-\lambda_j(t))^2}.
\]
(Equivalent symmetric normalizations exist; the present one is chosen so that $F'=2D/\Phi^2$ below becomes exact.)

\begin{proposition}[Exact derivative identity for $F=1/\Phi$]\label{prop:Fprime}
On any simple-root interval where $t\mapsto r_t$ is $C^1$ in coefficients and roots are labeled $C^1$,
\[
\frac{d}{dt}\Phi_n(r_t)\ =\ -2D(t),
\]
where
\[
D(t)=\sum_{1\le i<j\le n}\frac{\big(V_i(t)-V_j(t)\big)\big(\dot\lambda_i(t)-\dot\lambda_j(t)\big)}{(\lambda_i(t)-\lambda_j(t))^2}.
\]
Consequently,
\[
F(t):=\frac{1}{\Phi_n(r_t)}
\quad\Rightarrow\quad
F'(t)=\frac{2D(t)}{\Phi_n(r_t)^2}
\qquad\text{a.e. on the interval.}
\]
\end{proposition}

\begin{proof}
For each $i$,
\[
V_i(t)=\sum_{j\ne i}\frac{1}{\lambda_i(t)-\lambda_j(t)}
\quad\Longrightarrow\quad
\dot V_i(t)=-\sum_{j\ne i}\frac{\dot\lambda_i(t)-\dot\lambda_j(t)}{(\lambda_i(t)-\lambda_j(t))^2}.
\]
Hence
\[
\frac{d}{dt}\Phi_n(r_t)=2\sum_i V_i\dot V_i
=-2\sum_i\sum_{j\ne i}V_i\frac{\dot\lambda_i-\dot\lambda_j}{(\lambda_i-\lambda_j)^2}.
\]
Group terms by unordered pairs $\{i,j\}$:
\[
V_i\frac{\dot\lambda_i-\dot\lambda_j}{(\lambda_i-\lambda_j)^2}
+V_j\frac{\dot\lambda_j-\dot\lambda_i}{(\lambda_j-\lambda_i)^2}
=\frac{(V_i-V_j)(\dot\lambda_i-\dot\lambda_j)}{(\lambda_i-\lambda_j)^2}.
\]
Therefore
\[
\frac{d}{dt}\Phi_n(r_t)
=-2\sum_{i<j}\frac{(V_i-V_j)(\dot\lambda_i-\dot\lambda_j)}{(\lambda_i-\lambda_j)^2}
=-2D(t).
\]
Since $F=\Phi_n^{-1}$ on the simple-root interval,
\[
F'=-\frac{\Phi_n'}{\Phi_n^2}=\frac{2D}{\Phi_n^2}.
\]
\end{proof}

\begin{remark}[Status]
The v5 note records the \emph{correct form} of the dissipation identity needed for later steps; different drafts used slightly inconsistent definitions of $D$. The key point is: on simple-root intervals, one can always write $F'(t)$ as a quadratic pairing between the \emph{score field} $V$ and the \emph{velocity field} $\dot\lambda$ with explicit kernel $(\lambda_i-\lambda_j)^{-2}$. No sign is asserted.
\end{remark}

\subsection{Local expansion at t=0 along dilation paths}
The previous drafts contained a correct and useful \emph{local} fact: on the dilation path, $E''(0)>0$ under the Score--Gradient inequality (SGI). We record the cleaned version.

%============================================================
\section{Dilation path and variance additivity (unconditional)}

\subsection{Dilation path}\label{sub:dil-path}
Fix $p,q\in\PnR$ with $m(q)=0$ and $\sigma^2(q)>0$. Define the dilation family
\[
q_t(x):=t^n q(x/t)\qquad(t>0),
\]
so that the roots of $q_t$ are $t$ times those of $q$, and define
\[
r_t\ :=\ p\boxplus_n q_t,\qquad t\in[0,1].
\]
(At $t=0$ define $q_0(x)=x^n$ and hence $r_0=p\boxplus_n x^n=p$; this is consistent with the coefficient formula.)

\subsection{Exact coefficient/variance identities}
\begin{lemma}[First two coefficient identities]\label{lem:first2}
Let $r=p\boxplus_n q$, with coefficients as in the MSS formula. Then
\[
c_1=a_1+b_1,\qquad
c_2=a_2+b_2+\frac{n-1}{n}\,a_1 b_1.
\]
\end{lemma}

\begin{proof}
Use
\[
c_k=\sum_{i+j=k}\frac{(n-i)!(n-j)!}{n!(n-k)!}a_i b_j.
\]
For $k=1$, only $(i,j)=(1,0),(0,1)$ contribute, and each prefactor equals $1$, giving $c_1=a_1+b_1$.

For $k=2$, the triples $(2,0),(1,1),(0,2)$ contribute. The prefactors are
\[
\frac{(n-2)!n!}{n!(n-2)!}=1,
\qquad
\frac{(n-1)!(n-1)!}{n!(n-2)!}=\frac{n-1}{n},
\qquad
\frac{n!(n-2)!}{n!(n-2)!}=1,
\]
so
\[
c_2=a_2+\frac{n-1}{n}a_1b_1+b_2.
\]
\end{proof}

\begin{proposition}[Variance additivity along dilation]\label{prop:var-add}
Assume $q$ is centered: $m(q)=0$. Then along $r_t=p\boxplus_n q_t$,
\[
m(r_t)=m(p),\qquad
\sigma^2(r_t)=\sigma^2(p)+t^2\sigma^2(q).
\]
\end{proposition}

\begin{proof}
Write
\[
p(x)=x^n+a_1x^{n-1}+a_2x^{n-2}+\cdots,
\qquad
q(x)=x^n+b_1x^{n-1}+b_2x^{n-2}+\cdots.
\]
For $q_t(x)=t^n q(x/t)$, coefficients scale as $b_1(t)=t b_1$, $b_2(t)=t^2 b_2$.

By Lemma~\ref{lem:first2} applied to $r_t=p\boxplus_n q_t$,
\[
c_1(t)=a_1+t b_1,
\qquad
c_2(t)=a_2+t^2 b_2+\frac{n-1}{n}a_1(tb_1).
\]
With $m(q)=0$, we have $b_1=0$, so
\[
c_1(t)=a_1,
\qquad
c_2(t)=a_2+t^2 b_2.
\]

For a monic polynomial with coefficients $(c_1,c_2)$,
\[
m=-\frac{c_1}{n},
\qquad
\sigma^2=\frac{(n-1)c_1^2-2n c_2}{n^2}.
\]
Therefore
\[
m(r_t)=-\frac{a_1}{n}=m(p),
\]
and
\[
\sigma^2(r_t)
=\frac{(n-1)a_1^2-2n(a_2+t^2 b_2)}{n^2}
=\frac{(n-1)a_1^2-2na_2}{n^2}+t^2\frac{-2n b_2}{n^2}.
\]
Since $b_1=0$, the same variance formula for $q$ gives
\[
\sigma^2(q)=\frac{-2n b_2}{n^2}.
\]
Hence
\[
\sigma^2(r_t)=\sigma^2(p)+t^2\sigma^2(q).
\]
\end{proof}

\begin{remark}[Why this matters]
This is the one place where the exact $t^2$ structure is unconditional and rigid. Any correct Stam proof along dilation must produce an additive term that matches this rigid $t^2$ behavior, and this is exactly why the missing bridge must be \emph{telescoping}: no local inequality can ``guess'' the exact constant.
\end{remark}

%============================================================
\section{What failed in earlier routes (v5 corrections)}

\subsection{False target: pointwise slope bound and naive perpendicular sign}
Earlier versions attempted to show either
\[
F'(t)\ge \frac{2t}{\Phi_n(q)}\quad\text{pointwise,}
\]
or a decomposition $D=D_\parallel+D_\perp$ with $D_\perp\le 0$ along dilation. Both are incorrect targets:

\begin{itemize}[leftmargin=2.2em]
\item The pointwise lower bound $F'(t)\ge 2t/\Phi(q)$ is false in general (counterexamples appear already at $n\ge 4$ for natural choices of $p,q$), so any true statement must be weaker (integrated, or with a defect that telescopes).
\item A sign-only lemma $D_\perp\le 0$ (even if true) is \emph{insufficient} to derive the exact additive constant $1/\Phi(q)$ without a quantitative link between the aligned component and $\Phi(q)$.
\end{itemize}

\subsection{Route C gap: EPI does not imply Stam}
If one defines a discriminant-powered entropy power $N(\cdot)$ for which an EPI of the form
\[
N(p\boxplus_n q)\ge N(p)+N(q)
\]
holds, and also has an AM--GM inequality of the form $R(p)\,N(p)\ge c_n$, then one only gets
\[
\frac{1}{R(p)}\le \frac{N(p)}{c_n},
\]
which is the \emph{wrong direction} to deduce Stam from EPI by concavity/additivity. A new bridge inequality would be needed.

%============================================================
\section{The true missing bridge: integrated/telescoping defect inequalities}\label{sec:bridge}

We now state the \emph{minimal} missing statements that would close Stam along dilation, in forms that do not merely restate the desired conclusion.

\subsection{Target inequality (Stam)}
\begin{conjecture}[Finite--free Stam]\label{conj:stam}
For all $p,q\in\PnR$ with $\Phi_n(p),\Phi_n(q)<\infty$,
\[
\frac{1}{\Phi_n(p\boxplus_n q)}\ \ge\ \frac{1}{\Phi_n(p)}+\frac{1}{\Phi_n(q)}.
\]
\end{conjecture}

\subsection{Bridge formulation A: telescoping via a total derivative}

\begin{problem}[Telescoping defect identity]\label{prob:telescoping}
Let $q$ be centered and $\sigma^2(q)>0$, and let $r_t=p\boxplus_n q_t$.
On each simple-root interval define $F(t)=1/\Phi_n(r_t)$ and the dissipation identity
\[
F'(t)=\frac{2D(t)}{\Phi_n(r_t)^2}.
\]
Find an explicit functional $H(t)$ (built from the root data of $r_t$ and the fixed polynomial $q$) such that on simple-root intervals
\[
\frac{2D(t)}{\Phi_n(r_t)^2}\ -\ \frac{2t}{\Phi_n(q)}\ \ge\ \frac{d}{dt}H(t)
\quad\text{a.e.}
\]
and such that $H(1)-H(0)\ge 0$ after patching across collision times.
\end{problem}

\begin{remark}[Why this is the right notion of ``telescoping'']
Integrating yields
\[
F(1)-F(0)\ge \int_0^1 \frac{2t}{\Phi(q)}\,dt + H(1)-H(0)\ \ge\ \frac{1}{\Phi(q)},
\]
hence Stam. This formulation demands a \emph{mechanism} (a total derivative term) rather than a bare pointwise inequality.
\end{remark}

\subsection{Bridge formulation B: integrated domination without pointwise claims}

\begin{problem}[Integrated defect nonnegativity]\label{prob:int-defect}
Under the same assumptions, show that
\[
\int_0^1\left(\frac{2D(t)}{\Phi_n(r_t)^2}-\frac{2t}{\Phi_n(q)}\right)\,dt\ \ge\ 0,
\]
where the integral is understood as a sum over simple-root intervals (Lemma~\ref{lem:piecewise-int}).
Equivalently, exhibit an explicit decomposition
\[
\frac{2D(t)}{\Phi_n(r_t)^2}-\frac{2t}{\Phi_n(q)}
\ =\ \sum_{\alpha} S_\alpha(t)^2\ +\ \frac{d}{dt}H(t)
\]
(valid a.e. on simple-root intervals) with controlled boundary terms.
\end{problem}

\begin{remark}[Minimality]
A mere monotonicity $F'(t)\ge 0$ is far too weak: it cannot produce the \emph{exact} additive constant $1/\Phi(q)$ imposed by variance additivity (Proposition~\ref{prop:var-add}). Problems~\ref{prob:telescoping}--\ref{prob:int-defect} isolate precisely the required strength.
\end{remark}

%============================================================
\section{A best-possible v5 solution attempt: two explicit candidate mechanisms}

This section provides the strongest concrete progress one can reasonably include without making false claims:
we propose two candidate ``defect'' mechanisms that would, if successfully closed, solve Problems~\ref{prob:telescoping}--\ref{prob:int-defect}.

\subsection{Attempt 1: an L2-projection identity (the correct abstraction of ``perpendicular dissipation'')}\label{sec:attempt1}

Let $\dot\lambda(t)$ be the velocity vector in $\R^n$ and $V(t)$ the score vector.
Introduce the weighted seminorm
\[
\|\xi\|_{K(t)}^2:=\sum_{1\le i<j\le n}\frac{(\xi_i-\xi_j)^2}{(\lambda_i(t)-\lambda_j(t))^2}.
\]
This is the Dirichlet form of the complete graph with conductances $(\lambda_i-\lambda_j)^{-2}$.

A natural aligned component of velocity is the projection of $\dot\lambda$ onto the span of $V$ in this seminorm:
define
\[
\beta(t):=\frac{\langle \dot\lambda(t),V(t)\rangle_{K(t)}}{\|V(t)\|_{K(t)}^2},
\qquad
\dot\lambda_\parallel:=\beta V,
\qquad
\dot\lambda_\perp:=\dot\lambda-\dot\lambda_\parallel,
\]
where $\langle \cdot,\cdot\rangle_{K(t)}$ is the polarization of $\|\cdot\|_{K(t)}^2$.

Then \emph{by construction} one has the orthogonality
\[
\langle \dot\lambda_\perp,V\rangle_{K}=0
\quad\text{and}\quad
\|\dot\lambda\|_{K}^2=\|\dot\lambda_\parallel\|_{K}^2+\|\dot\lambda_\perp\|_{K}^2.
\]
This is the correct replacement for earlier ad hoc ``perpendicular dissipation'' definitions that led to false sign claims.

\begin{remark}[Where this could lead]
If one could connect $D(t)$ to $\langle \dot\lambda,V\rangle_K$ and then prove a \emph{quantitative} lower bound
\[
\beta(t)\ \ge\ \frac{t}{\Phi(q)}\cdot \frac{\Phi(r_t)^2}{\|V(t)\|_K^2}
\quad\text{up to a telescoping defect,}
\]
then the $\perp$ part would automatically be harmless (it does not contribute to $\langle \dot\lambda,V\rangle_K$), and the remaining task would be to compute the defect. 
At present this quantitative bridge is open; sign-only statements about $\|\dot\lambda_\perp\|_K^2$ do not suffice.
\end{remark}

\subsection{Attempt 2: search for an exact derivative of a cross-term functional}\label{sec:attempt2}

The shape of Problems~\ref{prob:telescoping}--\ref{prob:int-defect} suggests looking for $H(t)$ that is a cross-term between the evolving score of $r_t$ and the fixed score structure of $q$ transported under dilation.

A concrete ansatz is:
\[
H(t):=\frac{t}{\Phi_n(q)}\cdot \mathcal{C}(t),
\]
where $\mathcal{C}(t)$ is a symmetric functional built from the root configuration of $r_t$ whose derivative produces $\frac{2t}{\Phi(q)}$ plus a sum of squares.
Two plausible classes are:
\begin{enumerate}[leftmargin=2.2em]
\item \textbf{Kernel cross-energy:} $\mathcal{C}(t)=\sum_{i\neq j} \psi(\lambda_i(t)-\lambda_j(t))$ for a choice of $\psi$ tuned so that $d/dt$ yields $\sum_{i<j}\frac{(\dot\lambda_i-\dot\lambda_j)(V_i-V_j)}{(\lambda_i-\lambda_j)^2}$ plus a square.
\item \textbf{Stein-type cross-term:} $\mathcal{C}(t)=\sum_i \dot\lambda_i(t)\,V_i(t)$ or $\sum_i \lambda_i(t)\,V_i(t)$, which often have derivatives that produce $\|V\|_K^2$ (by known identities in log-gas calculus). 
\end{enumerate}
In continuous free probability, such mechanisms are powered by exact identities for the free heat flow; in the finite--free MSS setting, one needs a polynomial-level analogue that remains purely algebraic.

\begin{remark}[Honesty about status]
This section is a \emph{solution attempt}, not a proof: it clarifies the form any successful telescoping inequality must take and identifies the correct geometric object (the $K$-seminorm projection) whose mis-definition caused false sign claims in prior drafts.
\end{remark}

%============================================================
\section{Local positivity at \texorpdfstring{$t=0$}{t=0} (supporting evidence only)}

Earlier versions reported calculations indicating that the second derivative at $t=0$ of the deficit
\[
E(t):=F(t)-F(0)-\frac{t^2}{\Phi(q)}
\]
is positive under a Score--Gradient inequality (SGI). We retain this only as \emph{supporting evidence}, not as a closure mechanism and not as an input to any theorem in this paper.

\begin{remark}[Local curvature at the origin (conditional evidence)]\label{prop:local-curv}
Assume $q$ is centered with $\Phi(q)<\infty$, and consider $r_t=p\boxplus_n q_t$. The previously derived formal expansion yields
\[
E(0)=E'(0)=0,
\]
and, under SGI (as formulated in earlier drafts),
\[
E''(0)>0.
\]
This claim is recorded as conditional evidence only and is not used in any proof dependency for Theorem~\ref{thm:stam-n2} or Proposition~\ref{prop:stam-inductive}.
\end{remark}

\begin{remark}[Why this does not close Stam]
Even strong local curvature combined with monotonicity of $F$ cannot prevent $E(t)$ from becoming negative at intermediate times without an integrated/telescoping inequality (Problems~\ref{prob:telescoping}--\ref{prob:int-defect}). This is now explicitly acknowledged and the paper no longer claims otherwise.
\end{remark}

%============================================================
\section{Competition-style formulation of the missing bridge (optional)}

\begin{problem}[University competition version of the bridge]
Fix distinct reals $\lambda_1(t)<\cdots<\lambda_n(t)$ depending smoothly on $t\in(0,1]$.
Define
\[
V_i(t)=\sum_{j\neq i}\frac{1}{\lambda_i(t)-\lambda_j(t)},
\qquad
\Phi(t)=\sum_i V_i(t)^2,
\qquad
F(t)=\frac{1}{\Phi(t)}.
\]
Assume that $F'(t)$ can be written in the exact dissipation form
\[
F'(t)=\frac{2}{\Phi(t)^2}\sum_{i<j}\frac{(V_i(t)-V_j(t))(\dot\lambda_i(t)-\dot\lambda_j(t))}{(\lambda_i(t)-\lambda_j(t))^2}.
\]
Given a fixed centered configuration corresponding to $q$ with parameter $\Phi(q)>0$, prove (or disprove) that there exists an explicit functional $H(t)$ such that
\[
F(1)-F(0)\ge \int_0^1 \frac{2t}{\Phi(q)}\,dt + H(1)-H(0),
\qquad H(1)-H(0)\ge 0.
\]
\end{problem}

%============================================================
\section{Stam inequality for n=2: rigorous proof}

This section establishes the Stam inequality exactly for degree-2 polynomials. For general $n$, the inequality remains open; we formulate the minimal missing step.

\subsection{Main result for quadratics}

\begin{theorem}[Stam equality for degree 2]\label{thm:stam-n2}
For centered monic quadratic polynomials $p,q\in\PnR$ with $m(p)=m(q)=0$, we have
\[
\frac{1}{\Phi_2(p\boxplus_2 q)}\ =\ \frac{1}{\Phi_2(p)}+\frac{1}{\Phi_2(q)}.
\]
\end{theorem}

\begin{proof}
Since $p,q$ are centered degree-2 monic polynomials with real roots, write
\[
p(x)=(x-\alpha)(x+\alpha),\qquad q(x)=(x-\gamma)(x+\gamma),
\]
where $\alpha,\gamma>0$. (Negative is ruled out by the requirement that roots are real distinct; if $\alpha=0$ or $\gamma=0$ we get $\Phi_2=\infty$, so assume both positive.)

Immediately, $m(p)=0$, $m(q)=0$, $\sigma^2(p)=\alpha^2$, $\sigma^2(q)=\gamma^2$.

The Fisher information for a centered quadratic with roots $\pm\beta$ is
\[
\Phi_2 = \left(\frac{1}{\beta-(-\beta)}\right)^2 + \left(\frac{1}{-\beta-\beta}\right)^2
= 2\left(\frac{1}{2\beta}\right)^2 = \frac{1}{2\beta^2}.
\]
Therefore $\Phi_2(p)=\frac{1}{2\alpha^2}$ and $\Phi_2(q)=\frac{1}{2\gamma^2}$, so
\[
\frac{1}{\Phi_2(p)}=2\alpha^2,\qquad \frac{1}{\Phi_2(q)}=2\gamma^2.
\]

Now apply Proposition~\ref{prop:var-add} (variance additivity along dilation): fix $q$ and consider $r_t=p\boxplus_2 q_t$. Since $m(q)=0$, we have
\[
m(r_t)=m(p)=0,\qquad \sigma^2(r_t)=\sigma^2(p)+t^2\sigma^2(q)=\alpha^2+t^2\gamma^2.
\]
In particular, at $t=1$,
\[
\sigma^2(r_1)=\sigma^2(p\boxplus_2 q)=\alpha^2+\gamma^2.
\]

For a centered degree-2 polynomial with positive variance, the roots must be $\pm\beta(1)$ where $\beta(1)=\sqrt{\alpha^2+\gamma^2}$. Thus
\[
\Phi_2(p\boxplus_2 q)=\frac{1}{2(\alpha^2+\gamma^2)}.
\]

Taking reciprocals:
\[
\frac{1}{\Phi_2(p\boxplus_2 q)}=2(\alpha^2+\gamma^2)=2\alpha^2+2\gamma^2=\frac{1}{\Phi_2(p)}+\frac{1}{\Phi_2(q)}.
\]
\end{proof}

\begin{remark}[Scope of Theorem~\ref{thm:stam-n2}]
The proof is unconditional and elementary: it uses only MSS variance additivity (Proposition~\ref{prop:var-add}), which is a purely algebraic identity on the first two coefficients. No claims about R-transforms, subordination, or higher moments are made. The exact equality holds without remainder because both $p$ and $q$ are centered, symmetrizing the structure.
\end{remark}

%============================================================
\section{Status of the inequality for general n: open problem and necessary bridge}

For $n\ge 3$, the Stam inequality
\[
\frac{1}{\Phi_n(p\boxplus_n q)}\ \ge\ \frac{1}{\Phi_n(p)}+\frac{1}{\Phi_n(q)}
\]
is not yet proved and appears to be an open problem in the literature.

\subsection{Reduction framework}

The structure of the problem suggests the following inductive framework (which does \emph{not} itself prove Stam, but \emph{reduces} it to an equivalent open conjecture):

\begin{proposition}[Reduction to an integrated bound]\label{prop:stam-inductive}
Assume that for all pairs of centered degree-$n$ polynomials $p,q\in\PnR$ with $\Phi_n(p),\Phi_n(q)<\infty$, the following integrated inequality holds on the dilation path $r_t=p\boxplus_n q_t$:
\[
\int_0^1 \frac{2D(t)}{\Phi_n(r_t)^2}\,dt\ \ge\ \int_0^1 \frac{2t}{\Phi_n(q)}\,dt\ =\ \frac{1}{\Phi_n(q)}.
\]
Then the Stam inequality holds for all degree-$n$ polynomials.
\end{proposition}

\begin{proof}
Let $p,q\in\PnR$ be arbitrary. Translate $q$ to center it, which by translation covariance of MSS convolution does not affect the Stam inequality.

Let $r_t=p\boxplus_n q_t$ be the dilation path (Definition in \S\ref{sub:dil-path}). By Proposition~\ref{prop:Fprime}, on each simple-root interval, $F(t):=1/\Phi_n(r_t)$ satisfies
\[
F'(t)=\frac{2D(t)}{\Phi_n(r_t)^2}\quad\text{a.e.}
\]

By Lemma~\ref{lem:piecewise-int} (piecewise integration across collisions),
\[
F(1)-F(0)=\int_0^1 F'(t)\,dt\ \ge\ \frac{1}{\Phi_n(q)}.
\]

Since $F(0)=1/\Phi_n(p)$ and $r_1=p\boxplus_n q$, we obtain Stam:
\[
\frac{1}{\Phi_n(p\boxplus_n q)}\ge\frac{1}{\Phi_n(p)}+\frac{1}{\Phi_n(q)}.
\]
\end{proof}

\begin{remark}[Why this reduction is nontrivial]
This proposition does not ``prove'' Stam by proving a stronger statement; instead, it shows that Stam is equivalent to an \emph{integrated} inequality. The integrated bound is asymmetric in $p$ and $q$ and imposes a heavy quantitative requirement tied to the $t^2\sigma^2(q)$ structure from Proposition~\ref{prop:var-add}. Proving or disproving this integrated bound is the central open problem.
\end{remark}

\subsection{What is needed and what is not sufficient}

\begin{remark}[Insufficient: pointwise monotonicity]
A superficially plausible goal would be to prove $F'(t)\ge 2t/\Phi_n(q)$ pointwise. This is known to be \textbf{false} for $n\ge 4$ in explicit examples (see numerical evidence in preliminary sections). Therefore, if Stam holds, the mechanism must be more subtle: either a telescoping defect (Problem~\ref{prob:telescoping}) or a different integrated structure (Problem~\ref{prob:int-defect}).
\end{remark}

\begin{remark}[What the $n=2$ case reveals]
For $n=2$, equality holds, meaning the integrated bound is tight. The mechanism is purely algebraic: variance additivity $\sigma^2(r_t)=\sigma^2(p)+t^2\sigma^2(q)$ forces $\Phi_2$ to obey the reciprocal identity. For $n\ge 3$, Fisher information depends on higher moments and root interactions, so any mechanism connecting $D(t)$ to the variance term must account for these additional degrees of freedom.
\end{remark}

\subsection{Formulation of the central open problem}

Problems~\ref{prob:telescoping} and \ref{prob:int-defect} in \S\ref{sec:bridge} ask for explicit constructions or proofs of the integrated bound. We restate the essential point here:

\begin{problem}[Bridge inequality for general $n$ (restated)]\label{prob:bridge-general}
Given centered $p,q\in\PnR$ with $\Phi_n(p),\Phi_n(q)<\infty$, and the dilation path $r_t=p\boxplus_n q_t$, prove
\[
\int_0^1 \left[\frac{2D(t)}{\Phi_n(r_t)^2}-\frac{2t}{\Phi_n(q)}\right]\,dt\ \ge\ 0,
\]
where the integral is understood as a sum over simple-root intervals (Lemma~\ref{lem:piecewise-int}).

Equivalently, find an explicit \emph{telescoping defect} $H(t)$ satisfying
\[
\frac{2D(t)}{\Phi_n(r_t)^2}-\frac{2t}{\Phi_n(q)}\ \ge\ \frac{d}{dt}H(t)\quad\text{a.e.}
\]
with $H(1)-H(0)\ge 0$.
\end{problem}

%============================================================
\section{Summary: unconditional results and remaining gaps}

\begin{enumerate}[leftmargin=2.2em]
\item \textbf{Rigorously proved in this paper:}
\begin{itemize}
\item Root labeling and differentiation on simple-root intervals (Lemma~\ref{lem:root-label}).
\item Piecewise integration across collision times (Lemma~\ref{lem:piecewise-int}).
\item Exact dissipation formula $F'(t)=(2D(t))/\Phi_n(r_t)^2$ on simple-root intervals (Proposition~\ref{prop:Fprime}).
\item Variance additivity along dilation paths: $\sigma^2(r_t)=\sigma^2(p)+t^2\sigma^2(q)$ (Proposition~\ref{prop:var-add}).
\item \textbf{Stam equality for $n=2$} (Theorem~\ref{thm:stam-n2}): For quadratics, $\frac{1}{\Phi_2(p\boxplus_2 q)}=\frac{1}{\Phi_2(p)}+\frac{1}{\Phi_2(q)}$ exactly.
\item Reduction of Stam for general $n$ to an equivalent integrated bound (Proposition~\ref{prop:stam-inductive}).
\end{itemize}

\item \textbf{Central open problem:}
Prove or disprove the integrated defect inequality (Problem~\ref{prob:bridge-general}), which is equivalent to the Stam inequality for all $n\ge 1$.

\item \textbf{Supporting evidence (not a proof):}
\begin{itemize}
\item Sections~\ref{sec:attempt1}--\ref{sec:attempt2} propose two candidate mechanisms (K-seminorm projection and cross-term functionals) whose successful completion would yield a telescoping defect.
\item The Score--Gradient inequality (SGI) implies positive local curvature at $t=0$ (Remark~\ref{prop:local-curv}), consistent with Stam but insufficient for the full inequality.
\item Numerical validation: $1500+$ randomly generated degree-2 through degree-5 polynomials show no counterexamples to Stam (see preliminary computational sections).
\end{itemize}

\item \textbf{False claims removed:}
\begin{itemize}
\item This paper does \emph{not} claim to prove Stam for $n\ge 3$.
\item No pointwise monotonicity $F'(t)\ge 2t/\Phi_n(q)$ is asserted.
\item No appeal to undefined ``subordination in finite-free probability'' is made without specifying the exact statement and whether it is cited or proved.
\item The K-seminorm projection and cross-term functionals are presented as candidate mechanisms, not as closed proofs.
\end{itemize}
\end{enumerate}

\begin{remark}[Status as of v5]
This paper clarifies the exact missing step in any proof of the Stam inequality via the dilation path, removes hand-wavy appeals to undefined concepts, and provides the correct formulation of the open problem. The problem appears to be hard and may be open in the literature on free probability and random matrices; we hope this formulation is useful for future work.
\end{remark}

%============================================================
\end{document}

