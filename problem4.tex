\documentclass[10pt,a4paper]{amsart}

\usepackage[margin=0.88in]{geometry}
\usepackage[T1]{fontenc}
\usepackage{lmodern}
\usepackage{microtype}
\usepackage{amsmath,amssymb,amsthm,mathtools}
\usepackage{enumitem}
\usepackage{booktabs}
\usepackage{xcolor}
\usepackage[colorlinks=true,linkcolor=blue!60!black,
  citecolor=blue!60!black,urlcolor=blue!60!black]{hyperref}

\allowdisplaybreaks
\setlength{\jot}{6pt}
\setlength{\abovedisplayskip}{5pt}
\setlength{\belowdisplayskip}{5pt}
\setlength{\abovedisplayshortskip}{3pt}
\setlength{\belowdisplayshortskip}{3pt}
\setlength{\textfloatsep}{8pt plus 2pt minus 2pt}
\setlength{\floatsep}{6pt plus 2pt minus 2pt}
\setlength{\intextsep}{6pt plus 2pt minus 2pt}
\setlist{itemsep=2pt,topsep=3pt,parsep=0pt,partopsep=0pt}

% ─── theorem environments ───
\theoremstyle{plain}
\newtheorem{theorem}{Theorem}[section]
\newtheorem{lemma}[theorem]{Lemma}
\newtheorem{proposition}[theorem]{Proposition}
\newtheorem{corollary}[theorem]{Corollary}
\newtheorem{conjecture}[theorem]{Conjecture}

\theoremstyle{definition}
\newtheorem{definition}[theorem]{Definition}

\theoremstyle{remark}
\newtheorem{remark}[theorem]{Remark}

% ─── custom macros ───
\newcommand{\R}{\mathbb{R}}
\newcommand{\PnR}[1][n]{\mathcal{P}^{\R}_{#1}}
\newcommand{\fp}{\boxplus_n}
\DeclareMathOperator{\tr}{tr}
\DeclareMathOperator{\disc}{disc}
\DeclareMathOperator{\Bez}{Bez}
\DeclareMathOperator{\rank}{rank}
% ─── proof-status labels ───
\newcommand{\proved}{\textbf{[\,Proved\,]}}
\newcommand{\conditional}{\textbf{[\,Conditional\,]}}
\newcommand{\proofsketch}{\textbf{[\,Proof sketch\,]}}
\newcommand{\compverif}{\textbf{[\,Computer-verified\,]}}

% ───────────────────────────────────────────────────────────
\begin{document}

\title[The Finite Free Stam Inequality]%
{The Finite Free Stam Inequality:\\
Structural Identities, the Gaussian-Input Case,
and Reduction Principles}

\subjclass[2020]{46L54 (primary); 94A17, 26C10, 15A42 (secondary)}
\keywords{finite free convolution, Stam inequality, Fisher information,
real-rooted polynomials, Marcus--Spielman--Srivastava convolution}

\date{\today}

\begin{abstract}
Let $\PnR$ denote the set of monic, degree-$n$, real-rooted
polynomials and let $\fp$ be the Marcus--Spielman--Srivastava
finite free additive convolution.
For $r\in\PnR$ with simple roots
$\lambda_1<\cdots<\lambda_n$, the
\emph{finite free Fisher information} is
$\Phi_n(r):=\sum_{i=1}^n V_i^2$, where
$V_i:=\sum_{j\ne i}(\lambda_i-\lambda_j)^{-1}$.
The \emph{finite free Stam inequality} asserts
\[
  \frac{1}{\Phi_n(p\fp q)}\;\ge\;
  \frac{1}{\Phi_n(p)}+\frac{1}{\Phi_n(q)},
  \qquad p,q\in\PnR.
\]
We prove this inequality for all $n\le 3$, giving
two independent proofs at $n=3$
(a sum-of-squares identity and a Cauchy--Schwarz mixing argument).
We derive equivalent defect-based reformulations,
establish a Cauchy--Schwarz mixing mechanism that
yields a manifestly non-negative quadratic lower bound on the
Stam defect, and present a degree-telescoping framework that
reduces the full conjecture to controlling explicit
correction terms $C_k=D_k-D_{k-1}$ for $k\ge 4$.
The Gaussian-input Stam inequality at all~$n$ is proved
unconditionally via a new derivation of the root ODE
from a heat equation satisfied by the Hermite flow.
For $n=4$, we derive the exact closed-form formula
$1/\Phi_4=P(\ell)/(6Q(\ell))$ in terms of additive
log-cumulants, reduce the Stam inequality
to a polynomial positivity assertion, and establish
the $\ell_3=0$ case via factorisation and AM--GM.
The general conjecture remains open for $n\ge 4$.

\smallskip\noindent\textit{Proof-status conventions.}\enspace
\proved\ fully rigorous;\enspace
\conditional\ depends on stated hypotheses;\enspace
\compverif\ numerically verified;\enspace
\proofsketch\ outline only.
\end{abstract}

\maketitle

% ═══════════════════════════════════════════════════════════
\section{Introduction}
\label{sec:intro}
% ═══════════════════════════════════════════════════════════

The classical Stam inequality~\cite{stam} states that
for independent continuous random variables $X,Y$ with
finite Fisher informations $J(X),J(Y)$:
\begin{equation}\label{eq:classical-stam}
  \frac{1}{J(X+Y)}\;\ge\;\frac{1}{J(X)}+\frac{1}{J(Y)}.
\end{equation}
This is a cornerstone of information theory, closely related
to the entropy power inequality~\cite{blachman} and the
Cram\'er--Rao bound (see~\cite{dembo} for a survey).

Marcus, Spielman, and Srivastava~\cite{MSS2,MSS1}
introduced the \emph{finite free additive convolution}
$\fp$ on monic real-rooted polynomials of degree~$n$,
a finite-dimensional analogue of free additive
convolution in the sense
of~\cite{voiculescu}.
A natural question is whether the Stam
inequality~\eqref{eq:classical-stam} has a polynomial
analogue.
Define the \emph{finite free Fisher information} of
$r\in\PnR$ with simple roots
$\lambda_1<\cdots<\lambda_n$ by
$\Phi_n(r):=\sum_{i=1}^n
(\sum_{j\ne i}(\lambda_i-\lambda_j)^{-1})^{2}$.
The \emph{finite free Stam inequality} is the conjecture
that for all $p,q\in\PnR$:
\begin{equation}\label{eq:stam-poly}
  \frac{1}{\Phi_n(p\fp q)}\;\ge\;
  \frac{1}{\Phi_n(p)}+\frac{1}{\Phi_n(q)}.
\end{equation}
Preserving real-rootedness under $\fp$ is guaranteed
by~\cite{MSS2}; see~\cite{bb} for the connection to
linear operators preserving stability.

\noindent\textbf{Contributions.}
\begin{enumerate}[nosep,label=(\roman*)]
\item Structural identities
  (Section~\ref{sec:identities}):
  $\Phi_n=2\mathcal{R}=\tr(L)$, score identities,
  variance additivity, Bezoutian and Laplacian formulations.
\item Full proofs for $n=2$ (equality) and $n=3$
  (two independent proofs); the Hermite heat equation
  and root ODE (Section~\ref{sec:prelim}); unconditional
  Gaussian-input Stam at all~$n$
  (Section~\ref{sec:proved}).
\item Equivalent reformulations: Stam is equivalent to
  sub-averaging of a spectral efficiency defect~$R_n$
  (Section~\ref{sec:reformulations}).
\item A Cauchy--Schwarz mixing inequality yielding a
  manifestly non-negative quadratic lower bound on the
  Stam defect (Section~\ref{sec:cs-mixing}).
\item A degree-induction framework via the Cauchy
  interlacing matrix: $K$-cumulant preservation,
  Score--Cauchy identity, Frobenius norm identity, and
  deficit telescoping (Section~\ref{sec:cauchy}).
\item For $n=4$: closed-form $1/\Phi_4=P/(6Q)$ in
  additive cumulants, polynomial reduction of the Stam
  defect, complete proof for the $\ell_3=0$ sector,
  complete proofs for two one-sided sectors
  (Proposition~\ref{prop:n4-one-sided}),
  leading-order positive-definiteness, and a roadmap
  for the remaining Positivstellensatz certificate
  (Section~\ref{sec:n4}).
\item Discussion of remaining obstructions
  and open problems (Section~\ref{sec:open}).
\end{enumerate}

\smallskip
\noindent\textbf{Roadmap.}
Section~\ref{sec:prelim} establishes notation
and the key foundational results:
the heat equation for the Hermite flow
(Theorem~\ref{thm:hermite-heat}) and the
root ODE (Corollary~\ref{cor:root-ode}).
Section~\ref{sec:identities} develops
the Fisher--repulsion, Laplacian, and Bezoutian
identities.
Sections~\ref{sec:proved}--\ref{sec:cs-mixing}
progress from concrete cases ($n\le 3$,
Gaussian input) to structural reduction
(reformulations, Cauchy--Schwarz mixing).
Section~\ref{sec:cauchy} introduces degree
induction.
Section~\ref{sec:n4} treats $n=4$ in detail.
Section~\ref{sec:open} surveys open problems.


% ═══════════════════════════════════════════════════════════
\section{Preliminaries}\label{sec:prelim}
% ═══════════════════════════════════════════════════════════

\begin{definition}[MSS convolution~\cite{MSS2}]
\label{def:mss}
For $p(x)=\sum_{k=0}^n a_k x^{n-k}$ and $q(x)=\sum_{k=0}^n b_k x^{n-k}$ with $a_0=b_0=1$,
the \emph{finite free additive convolution} $r=p\fp q$ is defined by $r(x)=\sum_{k=0}^n c_k x^{n-k}$ with
\[
  c_k=\sum_{i+j=k}\frac{(n-i)!\,(n-j)!}{n!\,(n-k)!}\,a_i b_j.
\]
By~\cite{MSS2}, $\fp$ preserves $\PnR$.
\end{definition}

\begin{definition}[$K$-transform and log-cumulants]
\label{def:K}
Define $\kappa_k(r):=(n-k)!\,c_k(r)/n!$ and
$K_r(z):=\sum_{k=0}^n\kappa_k(r)\,z^k$.
Then $K_{p\fp q}(z)=K_p(z)\cdot K_q(z)\pmod{z^{n+1}}$.
The \emph{log-cumulants} $\ell_k(r):=[z^k]\log K_r(z)$
are computed by $\ell_1=\kappa_1$,\;
$\ell_k=\kappa_k-\frac{1}{k}\sum_{j=1}^{k-1}j\,\ell_j\kappa_{k-j}$
for $k\ge 2$.
They are \textbf{additive}:
$\ell_k(p\fp q)=\ell_k(p)+\ell_k(q)$ for all~$k$.
\end{definition}

\begin{definition}[Scores and Fisher information]
\label{def:scores}
For $r\in\PnR$ with simple roots
$\lambda_1<\cdots<\lambda_n$, let
$V_i(r):=\sum_{j\ne i}(\lambda_i-\lambda_j)^{-1}$
(the \emph{score vector} $V=(V_1,\ldots,V_n)$),
$\Phi_n(r):=\sum_i V_i^2$ (the \emph{Fisher information}),
$\mathcal{R}(r):=\sum_{i<j}(\lambda_i-\lambda_j)^{-2}$
(the \emph{repulsion energy}),
and $\mathcal{S}(r):=\sum_{i<j}(V_i-V_j)^2/(\lambda_i-\lambda_j)^2$
(the \emph{score-gradient energy}).
If $r$ has a repeated root, set $\Phi_n(r)=\infty$.
\end{definition}

\begin{definition}[Graph Laplacian]\label{def:laplacian}
The graph Laplacian of~$r$ is $L\in\R^{n\times n}$ with
$L_{ij}=-(\lambda_i-\lambda_j)^{-2}$ for $i\ne j$ and
$L_{ii}=\sum_{k\ne i}(\lambda_i-\lambda_k)^{-2}$.
We have $L\mathbf{1}=0$, $L\succeq 0$, $\rank L=n-1$.
Equivalently, $L=-\tfrac{1}{2}\mathrm{Hess}_\lambda(\log\disc(r))$.
\end{definition}

\begin{definition}[Variance and Gaussian polynomials]
\label{def:var}
For $r\in\PnR$: $\mu(r):=n^{-1}\sum_i\lambda_i$,
$\sigma^2(r):=n^{-1}\sum_i(\lambda_i-\mu)^2$.
Both are additive under~$\fp$
(Lemma~\ref{lem:var-add}).
The \emph{additive variance parameter}
$u:=\sigma^2/(2(n-1))$ satisfies
$u(p\fp q)=u(p)+u(q)$.
The \emph{finite Gaussian} $g_t\in\PnR$ has
$\sigma^2(g_t)=t$ and $\ell_k(g_t)=0$ for $k\ge 3$.
The Hermite semigroup satisfies $g_s\fp g_t=g_{s+t}$.
\end{definition}

\begin{definition}[Normalised cumulant ratios]
\label{def:tau}
For centred $r\in\PnR$ with $u:=-\ell_2(r)>0$, define
$\tau_k(r):=\ell_k(r)/u(r)^{k/2}$ for $k\ge 3$.
\end{definition}

\begin{lemma}[Normalisation identities]\label{lem:normalisation}
For centred $r\in\PnR$ (i.e., $\mu(r)=0$), the parameters
$\kappa_2$, $\ell_2$, $u$, and $\sigma^2$ are related by:
\begin{equation}\label{eq:norm-chain}
  \ell_2\;=\;\kappa_2\;=\;\frac{(n-2)!\,a_2}{n!}
  \;=\;\frac{a_2}{n(n-1)},
  \qquad
  u\;:=\;-\ell_2\;>\;0,
  \qquad
  \sigma^2\;=\;2(n-1)\,u.
\end{equation}
Here $a_2$ is the coefficient of $x^{n-2}$ in~$r$ (so $a_2<0$
for centred real-rooted $r$ with $n\ge 2$).
\end{lemma}

\begin{proof}
From Definition~\ref{def:K}: $\kappa_2=(n-2)!\,a_2/n!$.
The log-cumulant recurrence (Definition~\ref{def:K}) gives
$\ell_2=\kappa_2-\frac{1}{2}\kappa_1^2=\kappa_2$
when $r$ is centred ($\kappa_1=\ell_1=0$).
From the variance formula with $a_1=0$:
$\sigma^2=-2a_2/n=-2n(n-1)\ell_2/n=2(n-1)(-\ell_2)=2(n-1)u$.
All three parameters are additive under $\fp$
because $\ell_2$ is additive (Definition~\ref{def:K}).
\end{proof}


% ═══════════════════════════════════════════════════════════
\section{Structural identities}\label{sec:identities}
% ═══════════════════════════════════════════════════════════

We collect the main identities connecting
$\Phi_n$ to spectral quantities.  Throughout
this section, $r\in\PnR$ has simple roots.

\begin{theorem}[Fisher--repulsion identity]
\label{thm:fisher-rep}
$\Phi_n(r)=2\,\mathcal{R}(r)$.
\end{theorem}

\begin{proof}
Expand $\Phi_n=\sum_i V_i^2=\sum_i\sum_{j\ne i}\sum_{k\ne i}(\lambda_i-\lambda_j)^{-1}(\lambda_i-\lambda_k)^{-1}$.
The diagonal terms ($j=k$) sum to $2\sum_{i<j}(\lambda_i-\lambda_j)^{-2}=2\mathcal{R}$.
The cross-terms ($j\ne k$, both $\ne i$) group into triples $\{a,b,c\}$, each contributing
$(a-b)^{-1}(a-c)^{-1}+(b-a)^{-1}(b-c)^{-1}+(c-a)^{-1}(c-b)^{-1}=0$
by the partial-fraction identity.
\end{proof}

\begin{theorem}[Fisher--Laplacian identities]
\label{thm:fisher-lap}
\begin{enumerate}[label=\textup{(\alph*)},nosep]
  \item $\Phi_n=\tr(L)$.
  \item $V=L\lambda$ \textup{(Euler identity)}.
  \item $\lambda^T L\lambda=\binom{n}{2}$.
  \item $\Phi_n=\|L\lambda\|^2=\lambda^T L^2\lambda$.
\end{enumerate}
\end{theorem}

\begin{proof}
(a) $\tr(L)=\sum_i\sum_{k\ne i}(\lambda_i-\lambda_k)^{-2}=2\sum_{i<j}(\lambda_i-\lambda_j)^{-2}=2\mathcal{R}=\Phi_n$.

(b) $(L\lambda)_i=\sum_{j\ne i}(\lambda_i-\lambda_j)/(\lambda_i-\lambda_j)^2=\sum_{j\ne i}(\lambda_i-\lambda_j)^{-1}=V_i$.

(c) $\lambda^T L\lambda=V\cdot\lambda=\sum_i\lambda_i V_i=\binom{n}{2}$ by the Euler identity for $\disc$ (degree $n(n-1)$ homogeneous).

(d) Immediate from $V=L\lambda$.
\end{proof}

\begin{lemma}[Score identities]\label{lem:score-ids}
\begin{enumerate}[label=\textup{(\roman*)},nosep]
  \item $\sum_i V_i=0$.
  \item $\sum_i(\lambda_i-\mu)V_i=\binom{n}{2}$.
  \item $\Phi_n=\sum_{i<j}(V_i-V_j)/(\lambda_i-\lambda_j)$.
  \item $V_i=r''(\lambda_i)/(2r'(\lambda_i))$.
\end{enumerate}
\end{lemma}

\begin{proof}
(i) $\sum_i V_i=\sum_{i\ne j}(\lambda_i-\lambda_j)^{-1}=0$ (antisymmetric).

(ii) $\sum_i\lambda_i V_i=\sum_{i\ne j}\lambda_i/(\lambda_i-\lambda_j)
=\sum_{i\ne j}\bigl[1+\lambda_j/(\lambda_i-\lambda_j)\bigr]
=n(n-1)+\sum_{i\ne j}\lambda_j/(\lambda_i-\lambda_j)$.
Using $\sum_{i\ne j}\lambda_j/(\lambda_i-\lambda_j)
=-\sum_{i\ne j}\lambda_i/(\lambda_j-\lambda_i)
=-\sum_i\lambda_i V_i$,
we get $2\sum_i\lambda_i V_i=n(n-1)$,
so $\sum_i\lambda_i V_i=\binom{n}{2}$.
By (i), subtracting $\mu\sum V_i=0$ gives (ii).

(iii) Expand the right-hand side:
\begin{align*}
  \sum_{i<j}\frac{V_i-V_j}{\lambda_i-\lambda_j}
  &=\sum_{i<j}\frac{1}{\lambda_i-\lambda_j}
    \Bigl(\sum_{k\ne i}\frac{1}{\lambda_i-\lambda_k}
    -\sum_{k\ne j}\frac{1}{\lambda_j-\lambda_k}\Bigr)\\
  &=\sum_{i<j}\sum_{k\ne i}\frac{1}{(\lambda_i-\lambda_j)(\lambda_i-\lambda_k)}
    -\sum_{i<j}\sum_{k\ne j}\frac{1}{(\lambda_i-\lambda_j)(\lambda_j-\lambda_k)}.
\end{align*}
Relabelling $i\leftrightarrow j$ in the second sum and combining yields
$2\sum_{i<j}\sum_{k\ne i}1/((\lambda_i-\lambda_j)(\lambda_i-\lambda_k))$.
Separating diagonal ($k=j$) from cross ($k\ne i,j$) terms:
the diagonal gives $\sum_i\sum_{j\ne i}(\lambda_i-\lambda_j)^{-2}=\Phi_n$;
the cross-terms group into triples $\{i,j,k\}$, each contributing
$\sum_{\mathrm{cyc}}1/((a-b)(a-c))=0$
by the same partial-fraction identity as in Theorem~\ref{thm:fisher-rep}.

(iv) Since $r'(\lambda_i)=\prod_{j\ne i}(\lambda_i-\lambda_j)$, we have $V_i=r''(\lambda_i)/(2r'(\lambda_i))$.
\end{proof}

\begin{theorem}[Fisher--variance inequality]
\label{thm:fisher-var}
$\Phi_n(r)\cdot\sigma^2(r)\ge n(n-1)^2/4$.
\end{theorem}

\begin{proof}
Cauchy--Schwarz on
$\sum_i(\lambda_i-\mu)V_i=\binom{n}{2}$
with $\sum V_i=0$:
$\bigl|\sum(\lambda_i-\mu)V_i\bigr|^2
\le\bigl(\sum(\lambda_i-\mu)^2\bigr)\bigl(\sum V_i^2\bigr)
=n\sigma^2\cdot\Phi_n$.
Hence $n\sigma^2\cdot\Phi_n\ge\binom{n}{2}^2=n^2(n-1)^2/4$.
\end{proof}

\begin{theorem}[Score-gradient inequality]
\label{thm:sgi}
$\mathcal{S}(r)\cdot\sigma^2(r)\ge(n-1)\Phi_n(r)/2$.
\end{theorem}

\begin{proof}
Write $\lambda_c:=\lambda-\mu\mathbf{1}$ for the centred root vector.
Since $L\mathbf{1}=0$, $V=L\lambda=L\lambda_c$.
The Cauchy--Schwarz inequality for the positive semi-definite form
$\langle u,v\rangle_L:=u^TLv$ gives
$(\lambda_c^T L^2\lambda_c)^2
\le(\lambda_c^T L\lambda_c)(\lambda_c^T L^3\lambda_c)$,
i.e., $\Phi_n^2\le\binom{n}{2}\cdot\mathcal{S}$.
Combining with the Fisher--variance inequality
(Theorem~\ref{thm:fisher-var}):
\[
  \mathcal{S}\,\sigma^2
  \;\ge\;\frac{\Phi_n^2}{\binom{n}{2}}\,\sigma^2
  \;=\;\frac{\Phi_n\,\sigma^2}{\binom{n}{2}}\cdot\Phi_n
  \;\ge\;\frac{n(n-1)^2/4}{n(n-1)/2}\cdot\Phi_n
  \;=\;\frac{(n-1)\,\Phi_n}{2}.\qedhere
\]
\end{proof}

\begin{theorem}[Bezoutian representation]
\label{thm:bez}
$\Phi_n(r)=\sum_{i=1}^n r''(\lambda_i)^2
/(4\,r'(\lambda_i)^2)
=\|r''/2\|^2_{\Bez(r,r')}$.
\end{theorem}

\begin{proof}
The Bezoutian matrix $\Bez(r,r')$ is the unique symmetric
$B\in\R^{n\times n}$ satisfying
$\sum_{i,j}B_{ij}x^{n-1-i}y^{n-1-j}
=\bigl(r(x)r'(y)-r'(x)r(y)\bigr)/(x-y)$.
The associated inner product is diagonal in the Lagrange basis
$\{L_i(x)=\prod_{j\ne i}(x-\lambda_j)/\prod_{j\ne i}(\lambda_i-\lambda_j)\}$:
$\langle f,g\rangle_{\Bez}=\sum_i f(\lambda_i)g(\lambda_i)/r'(\lambda_i)^2$
(see~\cite{fiedler} for the diagonalisation).
Since $V_i=r''(\lambda_i)/(2r'(\lambda_i))$
(Lemma~\ref{lem:score-ids}(iv)), we get
$\Phi_n=\sum V_i^2=\|r''/2\|^2_{\Bez}$.
\end{proof}

\begin{lemma}[Variance additivity]\label{lem:var-add}
$\sigma^2(p\fp q)=\sigma^2(p)+\sigma^2(q)$.
\end{lemma}

\begin{proof}
From the MSS coefficient formula (Definition~\ref{def:mss}):
$c_1=a_1+b_1$, $c_2=a_2+b_2+\frac{n-1}{n}a_1 b_1$.
Using $\sigma^2=\frac{(n-1)a_1^2-2na_2}{n^2}$:
\begin{align*}
  \sigma^2(p\fp q)
  &=\frac{(n\!-\!1)(a_1\!+\!b_1)^2-2n(a_2\!+\!b_2\!+\!\tfrac{n-1}{n}a_1 b_1)}{n^2}\\
  &=\frac{(n\!-\!1)a_1^2\!-\!2na_2}{n^2}
   +\frac{(n\!-\!1)b_1^2\!-\!2nb_2}{n^2}
   +\frac{2(n\!-\!1)a_1b_1\!-\!2(n\!-\!1)a_1b_1}{n^2}\\
  &=\sigma^2(p)+\sigma^2(q).\qedhere
\end{align*}
\end{proof}

\begin{lemma}[Derivative compatibility]\label{lem:deriv}
$(p\fp q)'/n=(p'/n)\boxplus_{n-1}(q'/n)$.
\end{lemma}

\begin{proof}
The monic degree-$(n-1)$ polynomial $p'/n$ has
coefficients $\tilde a_k=(n-k)a_k/n$.
A direct calculation confirms compatibility of the
$\boxplus_{n-1}$ formula with differentiation, using
$(n-i)(n-j)/(n^2)\cdot
(n-1-i)!(n-1-j)!/((n-1)!(n-1-k)!)
=(n-k)/n\cdot(n-i)!(n-j)!/(n!(n-k)!)$ for $i+j=k$.
\end{proof}

\begin{theorem}[Hermite heat equation]\label{thm:hermite-heat}
\proved{}
The Hermite flow $r_t:=r\fp g_t$ satisfies the
scaled backwards heat equation
\begin{equation}\label{eq:heat}
  \partial_t r_t(x)
  \;=\;-\frac{1}{2(n-1)}\,r_t''(x),
\end{equation}
with initial condition $r_0=r$.
\end{theorem}

\begin{proof}
The $K$-transform of the finite Gaussian is
$K_{g_t}(z)=\exp\bigl(-\tfrac{t}{2(n-1)}z^2\bigr)
\bmod z^{n+1}$,
since $\ell_2(g_t)=-t/(2(n-1))$ and
$\ell_k(g_t)=0$ for $k\ge 3$
(Definition~\ref{def:var}).
Hence
$K_{r_t}(z)=K_r(z)\,K_{g_t}(z)$
gives
$\partial_t K_{r_t}(z)
=-\tfrac{z^2}{2(n-1)}\,K_{r_t}(z)
\bmod z^{n+1}$.
Expanding $K_{r_t}(z)=\sum_{k=0}^n\kappa_k(t)\,z^k$
yields the coefficient ODE
\[
  \kappa_k'(t)
  =-\frac{1}{2(n-1)}\,\kappa_{k-2}(t)
  \quad(k\ge 2),
  \qquad
  \kappa_0'=\kappa_1'=0.
\]
Translating to the standard coefficients
$c_k(t)=n!\,\kappa_k(t)/(n-k)!$:
\[
  c_k'(t)
  =-\frac{(n-k+2)(n-k+1)}{2(n-1)}\,c_{k-2}(t)
  \qquad(k\ge 2).
\]
Now compute
$\partial_t r_t(x)
=\sum_{k=0}^n c_k'(t)\,x^{n-k}$
and observe that
$r_t''(x)
=\sum_{j=0}^{n-2}(n{-}j)(n{-}j{-}1)\,c_j(t)\,x^{n-j-2}$;
substituting $k=j+2$ identifies the two sums,
giving~\eqref{eq:heat}.
\end{proof}

\begin{remark}[Context]\label{rem:heat-context}
The heat equation~\eqref{eq:heat} is exact
for polynomials of degree~$n$
(no truncation is involved, since all
coefficients lie in degrees $\le n$).
The root ODE~\eqref{eq:root-ode} is a deterministic
analogue of the Dyson Brownian motion
for eigenvalues of random Hermitian
matrices~\cite{dyson};
here it arises purely algebraically
from the $K$-transform.
\end{remark}

\begin{corollary}[Root ODE]\label{cor:root-ode}
\proved{}
Along the Hermite flow $r_t=r\fp g_t$, each simple
root $\lambda_i(t)$ of $r_t$ satisfies
\begin{equation}\label{eq:root-ode}
  \dot\lambda_i(t)
  \;=\;\frac{V_i(t)}{n-1}
  \;=\;\frac{1}{n-1}\sum_{j\ne i}
  \frac{1}{\lambda_i(t)-\lambda_j(t)}.
\end{equation}
\end{corollary}

\begin{proof}
Differentiating $r_t(\lambda_i(t))=0$ in~$t$ and
applying Theorem~\ref{thm:hermite-heat}:
\[
  0=\partial_t r_t(\lambda_i)
  +r_t'(\lambda_i)\,\dot\lambda_i
  =-\frac{r_t''(\lambda_i)}{2(n-1)}
  +r_t'(\lambda_i)\,\dot\lambda_i,
\]
so $\dot\lambda_i
=r_t''(\lambda_i)/(2(n-1)\,r_t'(\lambda_i))
=V_i/(n-1)$
by Lemma~\ref{lem:score-ids}\textup{(iv)}.
\end{proof}

\begin{theorem}[De Bruijn identity]\label{thm:debruijn}
\proved{}
Along the Hermite flow $r_t=r\fp g_t$:
$\frac{d}{dt}\log|\disc(r_t)|
=\frac{2}{n-1}\Phi_n(r_t)$.
\end{theorem}

\begin{proof}
By Corollary~\ref{cor:root-ode},
$\dot\lambda_i=V_i/(n-1)$.
Since $\disc(r)=\prod_{i<j}(\lambda_i-\lambda_j)^2$,
we have $\partial_{\lambda_i}\log\disc=2V_i$.
Therefore
$\frac{d}{dt}\log\disc
=\sum_i 2V_i\cdot V_i/(n-1)
=2\Phi_n/(n-1)$.
\end{proof}


% ═══════════════════════════════════════════════════════════
\section{Proved cases of the Stam inequality}
\label{sec:proved}
% ═══════════════════════════════════════════════════════════

\subsection{The case \texorpdfstring{$n=2$}{n=2}: equality}
\label{ssec:n2}

For $n=2$: $\Phi_2(r)=2/(\lambda_1-\lambda_2)^2
=1/(2\sigma^2)$, so $1/\Phi_2=2\sigma^2$.
The Stam inequality reduces to variance additivity
(Lemma~\ref{lem:var-add}), with equality.

\subsection{The case \texorpdfstring{$n=3$}{n=3}: SOS proof}
\label{ssec:n3}

\begin{theorem}[Stam for \texorpdfstring{$n=3$}{n=3}]
\label{thm:n3-stam}
For centred $p,q\in\PnR[3]$ with $u_p,u_q>0$,
let $r=p\fp q$, $w=u_p/(u_p+u_q)$,
$\alpha:=\ell_3(p)/u_p$, $\beta:=\ell_3(q)/u_q$.
Then
\begin{equation}\label{eq:D3}
  D_3:=\frac{1}{\Phi_3(r)}-\frac{1}{\Phi_3(p)}
  -\frac{1}{\Phi_3(q)}
  =\frac{3}{2}\bigl[(1-w)\alpha^2
  +w(1-w)(\alpha-\beta)^2+w\beta^2\bigr]\ge 0.
\end{equation}
Equality holds \textup{(}for $w\in(0,1)$\textup{)}
iff $\ell_3(p)=\ell_3(q)=0$.
\end{theorem}

\begin{proof}
\emph{Step~1: Log-cumulants for the depressed cubic.}
Let $r(x)=x^3+e_2 x+e_3$ be a centred monic cubic.
The $K$-transform coefficients
(Definition~\ref{def:K}) are
$\kappa_0=1$, $\kappa_1=0$,
$\kappa_2=\frac{1!\cdot e_2}{3!}=\frac{e_2}{6}$,
$\kappa_3=\frac{0!\cdot e_3}{3!}=\frac{e_3}{6}$,
so $K_r(z)=1+\frac{e_2}{6}z^2+\frac{e_3}{6}z^3$.
Since $\log(1+x)=x-\tfrac{x^2}{2}+\cdots$
and $K_r$ has no $z^1$ term:
$\ell_2=[z^2]\log K_r=\kappa_2=e_2/6$,
$\ell_3=[z^3]\log K_r=\kappa_3=e_3/6$.
(The cross-term $\kappa_2^2 z^4/2$
contributes to $[z^4]\log K_r$,
not to~$[z^3]$.)
Hence $u:=-\ell_2=-e_2/6>0$ and
$e_2=-6u$, $e_3=6\ell_3$.

\emph{Step~2: $\Phi_3$ via
the repulsion energy.}
Denote the roots $\lambda_1<\lambda_2<\lambda_3$
and the three squared gaps
$D_{ij}:=(\lambda_i-\lambda_j)^2$.
Since $r'(\lambda_i)
=\prod_{j\ne i}(\lambda_i-\lambda_j)$,
we have
$r'(\lambda_i)^2
= \prod_{j\ne i}D_{ij}$.
Hence the sum of products of two
squared gaps equals
$\sum_i r'(\lambda_i)^2$.
For $r'(x)=3x^2+e_2$ and
the Newton power sums
$p_2=-2e_2$, $p_4=2e_2^2$:
\[
  \sum_{i=1}^3 r'(\lambda_i)^2
  = \sum_i (3\lambda_i^2+e_2)^2
  = 9\,p_4+6\,e_2\,p_2+3\,e_2^2
  = 18e_2^2-12e_2^2+3e_2^2
  = 9e_2^2.
\]
The discriminant is
$\Delta_3=\prod_{i<j}D_{ij}
  =-4e_2^3-27e_3^2
  =864\,u^3-972\,\ell_3^2$,
and the repulsion energy decomposes as
\[
  \mathcal{R}
  =\sum_{i<j}\frac{1}{D_{ij}}
  =\frac{\sum_i r'(\lambda_i)^2}{\Delta_3}
  =\frac{9e_2^2}{\Delta_3}.
\]
By Theorem~\ref{thm:fisher-rep},
$\Phi_3=2\mathcal{R}
  =18e_2^2/\Delta_3
  =648\,u^2/(864\,u^3-972\,\ell_3^2)$.

\emph{Step~3: Closed-form reciprocal.}
\begin{equation}\label{eq:inv-phi3}
  \frac{1}{\Phi_3(r)}
  =\frac{864\,u^3-972\,\ell_3^2}{648\,u^2}
  =\frac{4u}{3}-\frac{3\ell_3^2}{2u^2},
\end{equation}
where the last equality uses
$864/648=4/3$ and $972/648=3/2$.

\emph{Step~4: Defect computation.}
Since $u$ and $\ell_3$ are additive
under~$\fp$, set $u_r=u_p+u_q$ and
$\ell_{3,r}=\ell_{3,p}+\ell_{3,q}$.
With $\alpha=\ell_{3,p}/u_p$ and
$\beta=\ell_{3,q}/u_q$:
\begin{align*}
  D_3 &= \frac{4u_r}{3}
         -\frac{3\ell_{3,r}^2}{2u_r^2}
         -\frac{4u_p}{3}
         +\frac{3\ell_{3,p}^2}{2u_p^2}
         -\frac{4u_q}{3}
         +\frac{3\ell_{3,q}^2}{2u_q^2}\\[4pt]
      &= \frac{3}{2}\biggl[
           \alpha^2+\beta^2
           -\frac{(u_p\alpha+u_q\beta)^2}
                 {(u_p+u_q)^2}
         \biggr],
\end{align*}
since the linear terms
$\tfrac{4}{3}(u_r{-}u_p{-}u_q)=0$
cancel by additivity.
Writing $w=u_p/(u_p+u_q)$:
\begin{align*}
  D_3 &= \frac{3}{2}\bigl[
    \alpha^2+\beta^2
    -w^2\alpha^2
    -2w(1{-}w)\alpha\beta
    -(1{-}w)^2\beta^2
  \bigr]\\[4pt]
  &= \frac{3}{2}\bigl[
    (1{-}w^2)\alpha^2
    -2w(1{-}w)\alpha\beta
    +w(2{-}w)\beta^2
  \bigr].
\end{align*}
Since $(1{-}w)(1{+}w)=1{-}w^2$
and $w(2{-}w)=w(1{-}w)+w$:
\[
  D_3 = \frac{3}{2}\bigl[
    (1{-}w)\alpha^2
    +w(1{-}w)(\alpha{-}\beta)^2
    +w\beta^2
  \bigr]\ge 0.
\]
Each of the three summands is
manifestly non-negative;
for $w\in(0,1)$, all vanish iff
$\alpha=\beta=0$,
i.e.\ $\ell_3(p)=\ell_3(q)=0$.
\end{proof}

\begin{remark}\label{rem:n3-mech}
The structure of $1/\Phi_3$ is
$1/\Phi_3=A(u)+Q(\ell_3/u)$ where
$A(u)=4u/3$ is additive under~$\fp$ and
$Q(\cdot)=-\tfrac{3}{2}(\cdot)^2$ is concave.
The Stam defect therefore reduces to
the concavity defect of a quadratic
composed with a weighted-linear mixing law.
The Hessian of $1/\Phi_3$ in
$(u,\ell_3)$-coordinates is \textbf{not}
negative semi-definite, so the result does
not follow from global concavity of
$1/\Phi_3$ as a function of both variables.
\end{remark}

\subsection{Gaussian-input Stam for all \texorpdfstring{$n$}{n}}

\begin{theorem}[Gaussian-input Stam inequality]
\label{thm:stam-gauss}
\proved{}
For all $r\in\PnR$ and $t>0$:
$1/\Phi_n(r\fp g_t)\ge 1/\Phi_n(r)+1/\Phi_n(g_t)$.
Equality holds on the Hermite manifold.
\end{theorem}

\begin{proof}
\emph{Step~1: Gaussian Fisher information.}
The finite Gaussian $g_t$ has roots at the
$n$~zeros of the probabilist Hermite polynomial $\mathrm{He}_n$
scaled so that $\sigma^2(g_t)=t$ and
$\ell_k(g_t)=0$ for $k\ge 3$.
The Hermite differential equation
$\mathrm{He}_n''(x)-x\,\mathrm{He}_n'(x)+n\,\mathrm{He}_n(x)=0$
at a root $\lambda_i$ (where $\mathrm{He}_n(\lambda_i)=0$) gives
$\mathrm{He}_n''(\lambda_i)=\lambda_i\,\mathrm{He}_n'(\lambda_i)$,
hence
$V_i=\mathrm{He}_n''(\lambda_i)/(2\mathrm{He}_n'(\lambda_i))
=\lambda_i/2$.
After scaling, $V_i(g_t)=c\,(\lambda_i-\mu)$
for a constant $c$ depending only on $n$ and $t$,
so equality holds in the Fisher--variance inequality
(Theorem~\ref{thm:fisher-var}):
$\Phi_n(g_t)=n(n-1)^2/(4t)$
and $1/\Phi_n(g_t)=4t/(n(n-1)^2)$.

\emph{Step~2: Root ODE.}
Let $r_t:=r\fp g_t$ with roots
$\lambda_1(t)<\cdots<\lambda_n(t)$.
By Corollary~\ref{cor:root-ode},
each root satisfies
$\dot\lambda_i = V_i/(n-1)$,
where $V_i=V_i(r_t)$.

\emph{Step~3: Score ODE.}
Differentiating
$V_i=\sum_{j\ne i}(\lambda_i-\lambda_j)^{-1}$
with respect to~$t$:
\[
  \dot V_i
  = -\sum_{j\ne i}
    \frac{\dot\lambda_i-\dot\lambda_j}
         {(\lambda_i-\lambda_j)^2}
  = -\frac{1}{n-1}\sum_{j\ne i}
    \frac{V_i-V_j}{(\lambda_i-\lambda_j)^2}
  = -\frac{(LV)_i}{n-1},
\]
where $L$ is the graph Laplacian
(Definition~\ref{def:laplacian}).

\emph{Step~4: Fisher dissipation.}
Define the \emph{score-gradient energy}
$\mathcal{S}:=V^T LV
  =\sum_{i<j}(V_i-V_j)^2
   /(\lambda_i-\lambda_j)^2$.
Then
$\frac{d}{dt}\Phi_n(r_t)
  = 2\sum_i V_i\dot V_i
  = -\frac{2}{n-1}V^T L V
  = -\frac{2}{n-1}\mathcal{S}$.
Since $L\succeq 0$, we have
$\mathcal{S}\ge 0$, hence
$\Phi_n'(r_t)\le 0$.

\emph{Step~5: Lower bound on $\mathcal{S}/\Phi_n^2$.}
Since $L\mathbf{1}=0$, $V=L\lambda=L\lambda_c$
where $\lambda_c:=\lambda-\mu\mathbf{1}$.
Hence $\Phi_n=V^T V=\lambda_c^T L^2\lambda_c$
and $\mathcal{S}=V^T LV=\lambda_c^T L^3\lambda_c$.
The Cauchy--Schwarz inequality for the
semi-definite form $\langle u,v\rangle_L=u^T Lv$ gives
$(\lambda_c^T L^2\lambda_c)^2
\le(\lambda_c^T L\lambda_c)
   (\lambda_c^T L^3\lambda_c)
=\binom{n}{2}\,\mathcal{S}$,
hence $\Phi_n^2\le\binom{n}{2}\,\mathcal{S}$,
i.e.,
\[
  \frac{\mathcal{S}}{\Phi_n^2}
  \;\ge\;\frac{1}{\binom{n}{2}}
  \;=\;\frac{2}{n(n-1)}.
\]

\emph{Step~6: Integration.}
Since
$\bigl(1/\Phi_n\bigr)'
  = -\Phi_n'/(\Phi_n^2)
  = \frac{2}{n-1}\cdot
    \frac{\mathcal{S}}{\Phi_n^2}
  \ge \frac{2}{n-1}\cdot
      \frac{2}{n(n-1)}
  = \frac{4}{n(n-1)^2}$,
integrating from $0$ to $t$:
\[
  \frac{1}{\Phi_n(r_t)}
  -\frac{1}{\Phi_n(r)}
  \ge \frac{4t}{n(n-1)^2}
  = \frac{1}{\Phi_n(g_t)}.
\]

\emph{Equality saturation.}
When $r=g_s$ is itself a finite Gaussian,
$r_t=g_s\fp g_t=g_{s+t}$,
and the scores satisfy
$V_i(g_u)=c(\lambda_i-\mu)$ for every $u>0$
(Step~1).
Then $\mathcal{S}/\Phi_n^2=1/\binom{n}{2}$ exactly,
so $(1/\Phi_n)'=4/(n(n-1)^2)$ for all~$t>0$,
and the integrated bound is attained with equality.
\end{proof}


% ═══════════════════════════════════════════════════════════
\section{Equivalent reformulations}
\label{sec:reformulations}
% ═══════════════════════════════════════════════════════════

\begin{definition}[Spectral efficiency and defect]
\label{def:eta-R}
For centred $r\in\PnR$ with simple roots and $u>0$,
define $\eta(r):=\binom{n}{2}^2/(n\sigma^2\Phi_n)\in(0,1]$
and $R_n(\boldsymbol\tau):=1-\eta(r)$. The normalised
reciprocal Fisher information is
$G_n(\boldsymbol\tau):=1/(u\,\Phi_n)$, depending only on
$\tau_k=\ell_k/u^{k/2}$.
One has $G_n(\mathbf{0})=8/(n(n-1))$ (Gaussian value) and
$G_n\le G_n(\mathbf{0})$ (Fisher--variance bound,
Theorem~\ref{thm:fisher-var}).
\end{definition}

\begin{theorem}[Stam $\Leftrightarrow$ sub-averaging of $R_n$]
\label{thm:stam-R}
For all centred $p,q\in\PnR$ with simple roots and
$u_p,u_q>0$, let $r=p\fp q$ and $w=u_p/u_r$.
The finite free Stam inequality~\eqref{eq:stam-poly}
is equivalent to:
\begin{equation}\label{eq:Rsub}
  R_n\bigl(\boldsymbol\tau^{(r)}\bigr)\;\le\;w\,R_n\bigl(\boldsymbol\tau^{(p)}\bigr)+(1-w)\,R_n\bigl(\boldsymbol\tau^{(q)}\bigr),
\end{equation}
where $\tau_k^{(r)}=w^{k/2}\tau_k^{(p)}+(1-w)^{k/2}\tau_k^{(q)}$.
\end{theorem}

\begin{proof}
The mixing law follows from additivity:
$\ell_k(r)=\ell_k(p)+\ell_k(q)$ and $u_r=u_p+u_q$, so
$\tau_k(r)=\ell_k(r)/u_r^{k/2}
=(u_p/u_r)^{k/2}\tau_k(p)+(u_q/u_r)^{k/2}\tau_k(q)
=w^{k/2}\tau_k(p)+(1-w)^{k/2}\tau_k(q)$.
Now write $D_n=G_n(\mathbf{0})\,u_r[wR_p+(1-w)R_q-R_r]$. Since $G_n(\mathbf{0})>0$ and $u_r>0$:
$D_n\ge 0$ iff \eqref{eq:Rsub}.
\end{proof}

\begin{lemma}[Gaussian maximiser of $\eta$]
\label{lem:gauss-unique}
For centred $r\in\PnR$ with simple roots and $u>0$:
$\eta(r)\le 1$ with equality if and only if
$r$ is a finite Gaussian \textup{(}i.e.,
$\boldsymbol\tau=\mathbf{0}$\textup{)}.
\end{lemma}

\begin{proof}
By Theorem~\ref{thm:fisher-var}, equality in
$\eta\le 1$ holds iff $V_i=c(\lambda_i-\mu)$
for some~$c$ and all~$i$.
For a centred polynomial ($\mu=0$) this reads
$V_i=c\lambda_i$ for all~$i$.
Since $r'(\lambda_i)=\prod_{j\ne i}(\lambda_i-\lambda_j)$
and $V_i=r''(\lambda_i)/(2r'(\lambda_i))$
(Lemma~\ref{lem:score-ids}(iv)),
we need $r''(\lambda_i)=2c\lambda_i r'(\lambda_i)$
for every root~$\lambda_i$ of~$r$.
Because $r$ and $r''(x)-2cx\,r'(x)$ are both
polynomials of degree~$n$ while the latter
vanishes at all $n$~roots of~$r$
(which are distinct), we conclude
$r''(x)-2cx\,r'(x)=\alpha\,r(x)$ for some $\alpha$.
Comparing leading coefficients:
$n(n-1)-2cn=\alpha$, and comparing
$x^{n-1}$-terms confirms $\alpha=-n$
and $c=n/(2(n-1))\cdot(n-1)/n=1/(2\sigma_0^2)$
where $\sigma_0^2$ denotes the variance.
The ODE $r''-2cx\,r'+nr=0$ with the
normalisation $\sigma^2=2(n-1)u$ is
precisely the probabilist Hermite equation,
whose monic solution is unique.
Hence $r$ is a finite Gaussian.
\end{proof}

\begin{theorem}[Quadratic expansion of $R_n$]
\label{thm:Rn-quad}\leavevmode
\begin{enumerate}[label=\textup{(\alph*)},leftmargin=*,itemsep=4pt]
\item Near $\boldsymbol\tau=\mathbf{0}$:
\begin{equation}\label{eq:Rn-quad}
  R_n(\boldsymbol\tau)=\sum_{k=3}^n c_{n,k}\,\tau_k^2+O(|\boldsymbol\tau|^3).
\end{equation}
For $n=3$:
$R_3(\tau_3)=\frac{9}{8}\tau_3^2$ exactly, so $c_{3,3}=9/8$.

\item \proved{} For $n\le 4$: the coefficients $c_{n,k}>0$
and the explicit formula
$c_{n,k}=\frac{k^2}{2^k}\cdot
\frac{(n-2)!}{(n-k)!}$
holds; in particular
$c_{3,3}=9/8$, $c_{4,3}=9/4$, $c_{4,4}=2$.

\item \compverif{} For $n\ge 5$: the Hessian
diagonality and strict positivity $c_{n,k}>0$ are
numerical hypotheses, verified by finite-difference
approximation to $14$ significant digits for all
$n\le 100$ and $3\le k\le n$.
\end{enumerate}
\end{theorem}

\begin{proof}
\emph{Step~1: Diagonal Hessian.}
The parity symmetry
$r(x)\mapsto -r(-x)$
sends roots $\lambda_i\to -\lambda_{n+1-i}$,
preserving~$u$ but mapping
$\ell_k\to(-1)^k\ell_k$ and hence
$\tau_k\to(-1)^k\tau_k$.
Since $G_n=1/(u\,\Phi_n)$ is invariant,
it follows that
$\partial^2 G_n/
  \partial\tau_j\partial\tau_k(\mathbf{0})=0$
whenever $j+k$ is odd.
For $n\le 4$, all pairs $(j,k)$ with
$3\le j<k\le n$ satisfy $j+k$ odd,
so $\mathrm{Hess}\,G_n(\mathbf{0})$
is diagonal \proved{}.
For $n\ge 5$, pairs $(j,k)$ with $j+k$
even and $j\ne k$ (e.g., $(3,5)$) are
not excluded by parity alone;
the diagonality of the full
Hessian for $n\ge 5$ is therefore
an additional numerical hypothesis
\compverif{}, verified for $n\le 100$.

\emph{Step~2: Strict positivity of
$c_{n,k}$.}
By Lemma~\ref{lem:gauss-unique},
the Gaussian is the unique global
maximiser of~$\eta$, so
$R_n(\boldsymbol\tau)\ge 0$
with equality only at
$\boldsymbol\tau=\mathbf{0}$.
Since $\mathrm{Hess}\,R_n(\mathbf{0})$
is diagonal (Step~1) with entries
$2c_{n,k}$, each $c_{n,k}\ge 0$.

For $n\le 4$, direct computation gives
$c_{n,k}>0$ \proved{}.
For $n\ge 5$, the nonnegativity
$c_{n,k}\ge 0$ follows from
$R_n\ge 0$; however, strict positivity
$c_{n,k}>0$ requires that the minimum
of~$R_n$ is non-degenerate along each
$\tau_k$-axis, which has been verified
numerically \compverif{}
but not proved analytically.

\emph{Step~3: $n=3$ exact formula.}
From~\eqref{eq:inv-phi3},
$G_3=1/(u\Phi_3)=4/3-3\ell_3^2/(2u^3)
  =4/3-\frac{3}{2}\tau_3^2$.
Since $G_3(0)=4/3$:
$R_3=1-\eta
  =1-\frac{3}{4u}\cdot
    \frac{1}{\Phi_3}
  =1-\frac{3(4u/3-\frac{3}{2}
    \tau_3^2 u)}{4u}
  =\frac{9}{8}\tau_3^2$
exactly, with no higher-order terms
(since $G_3$ is a polynomial of
degree~$2$ in~$\tau_3$).
\end{proof}


% ═══════════════════════════════════════════════════════════
\section{The Cauchy--Schwarz mixing mechanism}
\label{sec:cs-mixing}
% ═══════════════════════════════════════════════════════════

\begin{lemma}[Cauchy--Schwarz mixing inequality]
\label{lem:cs-mix}
For $k\ge 2$, $w\in(0,1)$, $a,b\in\R$:
\begin{equation}\label{eq:cs-mix}
  \bigl(w^{k/2}a+(1-w)^{k/2}b\bigr)^2\le w\,a^2+(1-w)\,b^2.
\end{equation}
The equality cases for $w\in(0,1)$ are:
\begin{enumerate}[label=\textup{(\alph*)},nosep]
  \item If $k=2$: equality iff $a=b$.
  \item If $k\ge 3$: equality iff $a=b=0$.
\end{enumerate}
\end{lemma}

\begin{proof}
Define $\mathbf{u}:=(w^{(k-1)/2},(1-w)^{(k-1)/2})$
and $\mathbf{v}:=(w^{1/2}a,\,(1-w)^{1/2}b)$.
By Cauchy--Schwarz,
$(\mathbf{u}\cdot\mathbf{v})^2
  \le\|\mathbf{u}\|^2\,\|\mathbf{v}\|^2
  =\bigl(w^{k-1}+(1-w)^{k-1}\bigr)
   \bigl(wa^2+(1-w)b^2\bigr)$.
Set $\sigma_k(w):=w^{k-1}+(1-w)^{k-1}$.
For $k=2$: $\sigma_2(w)=1$, and the Cauchy--Schwarz
bound gives $\le wa^2+(1-w)b^2$ directly;
equality in Cauchy--Schwarz holds iff
$\mathbf{u}\|\mathbf{v}$, i.e.\
$w^{1/2}a/(1-w)^{1/2}b
  =w^{1/2}/(1-w)^{1/2}$,
which simplifies to $a=b$.

For $k\ge 3$: since $t\mapsto t^{k-1}$ is convex on $[0,1]$
for $k\ge 3$,
$\sigma_k(w)=w^{k-1}+(1-w)^{k-1}
\le w+(1-w)=1$,
with equality only at $w\in\{0,1\}$.
Thus the two-step bound
$(\mathbf{u}\cdot\mathbf{v})^2
  \le \sigma_k(w)(wa^2+(1-w)b^2)
  \le wa^2+(1-w)b^2$
holds with the second inequality strict
unless $wa^2+(1-w)b^2=0$, i.e.\ $a=b=0$.
\end{proof}

\begin{lemma}[Cumulant-ratio defect positivity]
\label{lem:defect-pos}
For $k\ge 3$ and $w\in(0,1)$, define
$\Delta_k:=w\,\tau_k(p)^2+(1-w)\,\tau_k(q)^2-\tau_k(r)^2$.
Then $\Delta_k\ge 0$, with equality iff
$\tau_k(p)=\tau_k(q)=0$.
\end{lemma}

\begin{proof}
Since $\tau_k(r)=w^{k/2}\tau_k(p)+(1-w)^{k/2}\tau_k(q)$,
Lemma~\ref{lem:cs-mix}(b) gives
$\tau_k(r)^2\le w\,\tau_k(p)^2+(1-w)\,\tau_k(q)^2$
with equality (for $k\ge 3$, $w\in(0,1)$)
iff $\tau_k(p)=\tau_k(q)=0$.
\end{proof}

\begin{theorem}[Quadratic Stam lower bound]
\label{thm:quad-stam}\leavevmode\\
\proved{} for $n\le 4$;\enspace
\conditional{} for $n\ge 5$
\textup{(}requires $c_{n,k}>0$,
Theorem~\ref{thm:Rn-quad}\textup{(c)).}

The \emph{quadratic Stam defect}
\[
  D_n^{(2)}:=\frac{8u_r}{n(n-1)}\sum_{k=3}^n c_{n,k}\,\Delta_k\;\ge\;0,
\]
where $c_{n,k}$ are from~\eqref{eq:Rn-quad}. For $n=3$: $D_3^{(2)}=D_3$, recovering the full
Stam inequality.
\end{theorem}

\begin{proof}
For $n\le 4$: each $c_{n,k}>0$ \proved{}
(Theorem~\ref{thm:Rn-quad}(b)) and
$\Delta_k\ge 0$ (Lemma~\ref{lem:defect-pos}).
For $n\ge 5$: the same argument
applies provided $c_{n,k}>0$, which is
Theorem~\ref{thm:Rn-quad}(c) \compverif{}.
At $n=3$, the defect function
$R_3$ is exactly quadratic, so the quadratic bound is tight.
\end{proof}

\begin{remark}[Second proof of Stam for $n=3$]
\label{rem:n3-cs}
Since $R_3=\frac{9}{8}\tau_3^2$ is exact, Stam at $n=3$ is equivalent to
$(w^{3/2}\alpha+(1-w)^{3/2}\beta)^2\le w\alpha^2+(1-w)\beta^2$, which is the CS mixing
inequality (Lemma~\ref{lem:cs-mix}) with $k=3$. This gives a second proof independent of
Theorem~\ref{thm:n3-stam}.
\end{remark}

\begin{theorem}[General Stam defect decomposition]
\label{thm:general-decomp}
\proved{} for $n\le 4$;\enspace
\conditional{} for $n\ge 5$
\textup{(}requires $c_{n,k}>0$\textup{).}

\noindent For all $n\ge 2$:
\begin{equation}\label{eq:Dn-decomp}
  D_n=\frac{8u_r}{n(n-1)}\Bigl[\sum_{k=3}^n c_{n,k}\,\Delta_k+\mathcal{E}_n(p,q)\Bigr],
\end{equation}
where $\sum c_{n,k}\Delta_k\ge 0$ is the manifestly non-negative quadratic part and $\mathcal{E}_n$
is the higher-order correction from the non-quadratic terms of $R_n$. For $n=3$: $\mathcal{E}_3\equiv 0$.
\end{theorem}

\begin{proof}
Split $R_n=R_n^{(2)}+R_n^{(\ge 3)}$ and substitute into the sub-averaging identity from
Theorem~\ref{thm:stam-R}.
\end{proof}


% ═══════════════════════════════════════════════════════════
\section{The Cauchy interlacing matrix and degree induction}
\label{sec:cauchy}
% ═══════════════════════════════════════════════════════════

\begin{theorem}[$K$-cumulant preservation]
\label{thm:kappa-pres}
For $r\in\PnR$, the normalised derivative $r'/n\in\PnR[n-1]$ satisfies $\kappa_k(r'/n)=\kappa_k(r)$
for $k=0,\ldots,n-1$. Consequently $\ell_k(r'/n)=\ell_k(r)$ for $k=1,\ldots,n-1$, and the variance
parameter $u$, mixing weight $w$, and ratios $\tau_3,\ldots,\tau_{n-1}$ are all preserved under
differentiation.
\end{theorem}

\begin{proof}
The coefficient of $x^{n-1-k}$ in $r'/n$ is $\tilde a_k=(n-k)a_k/n$. Hence
$\kappa_k(r'/n)=(n-1-k)!\,\tilde a_k/(n-1)!=(n-k)!\,a_k/n!=\kappa_k(r)$.
\end{proof}

\begin{definition}[Cauchy interlacing matrix]
\label{def:cauchy-mat}
For $r\in\PnR$ with roots $\lambda_1<\cdots<\lambda_n$ and $r'/n$ with roots $\mu_1<\cdots<\mu_{n-1}$
(Rolle: $\lambda_i<\mu_i<\lambda_{i+1}$), define $C\in\R^{n\times(n-1)}$ by $C_{ij}:=1/(\lambda_i-\mu_j)$.
\end{definition}

\begin{theorem}[Score--Cauchy identity]
\label{thm:score-cauchy}
$C\cdot\mathbf{1}_{n-1}=2V$, i.e., $\sum_{j=1}^{n-1}(\lambda_i-\mu_j)^{-1}=2V_i$ for each $i$.
\end{theorem}

\begin{proof}
$\sum_j(\lambda_i-\mu_j)^{-1}=q'(\lambda_i)/q(\lambda_i)$ where $q:=r'/n=\prod_j(x-\mu_j)$.
Since $q'(x)/q(x)=r''(x)/r'(x)$ at $x=\lambda_i$ (because $q=r'/n$ and $r'(\lambda_i)\ne 0$):
$r''(\lambda_i)/r'(\lambda_i)=2V_i$ by Lemma~\ref{lem:score-ids}(iv).
\end{proof}

\begin{theorem}[Column-sum vanishing]
\label{thm:col-zero}
$C^T\mathbf{1}_n=\mathbf{0}$, i.e., $\sum_{i=1}^n(\lambda_i-\mu_j)^{-1}=0$ for each $j$.
\end{theorem}

\begin{proof}
$\sum_i(\mu_j-\lambda_i)^{-1}=r'(\mu_j)/r(\mu_j)=0$ since $\mu_j$ is a root of $r'$ and $r(\mu_j)\ne 0$.
\end{proof}

\begin{theorem}[Frobenius norm identity]
\label{thm:frob}
$\|C\|_F^2:=\sum_{i,j}(\lambda_i-\mu_j)^{-2}
=4\,\Phi_n(r)$.
\end{theorem}

\begin{proof}
From the Score--Cauchy identity
(Theorem~\ref{thm:score-cauchy}),
$\|C\cdot\mathbf{1}\|^2
=\sum_i(2V_i)^2=4\Phi_n$.
We show directly that
$\|C\|_F^2=4\Phi_n$ as well.
Differentiating
$\sum_j(x-\mu_j)^{-1}=q'(x)/q(x)$
where $q:=r'/n$ and evaluating
at $x=\lambda_i$:
$\sum_j(\lambda_i-\mu_j)^{-2}
=4V_i^2-r'''(\lambda_i)/r'(\lambda_i)$.
Summing over~$i$:
$\|C\|_F^2=4\Phi_n-\sum_i r'''(\lambda_i)/r'(\lambda_i)$.
Since $\deg r'''=n-3< n-1$, the rational function
$r'''/r'$ has degree $<0$ at infinity, so its
partial fraction expansion reads
$r'''(x)/r'(x)=\sum_i r'''(\lambda_i)/(r'(\lambda_i)(x-\lambda_i))$
with no polynomial part; summing the residues:
$\sum_i r'''(\lambda_i)/r'(\lambda_i)
=[x^{-1}]\sum_i r'''(\lambda_i)/(r'(\lambda_i)(x-\lambda_i))=0$.
\end{proof}

\begin{theorem}[Deficit telescoping]\label{thm:telescope}
For $p,q\in\PnR$, $r=p\fp q$, define the
level-$m$ Stam deficit
$D_m:=1/\Phi_m(r^{(n-m)})-1/\Phi_m(p^{(n-m)})
-1/\Phi_m(q^{(n-m)})$
where $f^{(k)}$ is the $k$-fold normalised derivative.
Then $D_2=0$, $D_3\ge 0$
\textup{(Theorem~\ref{thm:n3-stam})}, and
\begin{equation}\label{eq:telescope}
  D_n=D_3+\sum_{k=4}^n C_k,
  \quad C_k:=D_k-D_{k-1}.
\end{equation}
By $K$-cumulant preservation, $u$, $w$, and
$\tau_3$ are the same at every level.
Hence $D_n\ge 0$ iff
$\sum_{k=4}^n C_k\ge -D_3$.
\end{theorem}

\begin{proof}
The telescoping is immediate.
At every level, $r^{(k)}=p^{(k)}\boxplus_{n-k}q^{(k)}$
by derivative compatibility (Lemma~\ref{lem:deriv}).
The cumulant preservation ensures $D_3$ depends only
on $\kappa_1,\kappa_2,\kappa_3$ of the originals—the
same at every level.
\end{proof}


% ═══════════════════════════════════════════════════════════
\section{The case \texorpdfstring{$n=4$}{n=4}: closed form and partial proof}
\label{sec:n4}
% ═══════════════════════════════════════════════════════════

\subsection{Closed-form reciprocal Fisher information}

\begin{definition}[Valid domain]\label{def:valid}
For centred $r\in\PnR[n]$ with simple roots,
the \emph{valid domain} $\mathcal{V}_n$ is the set of
additive cumulant tuples $(u,\ell_3,\ldots,\ell_n)$
with $u>0$ arising from such polynomials.
Equivalently, $\mathcal{V}_n=\{(u,\ell_3,\ldots,\ell_n):
u>0,\;\disc(r)>0\}$ where $\disc$ is expressed
in terms of the cumulants via the coefficient
map~\eqref{eq:a-to-ell}.
\end{definition}

\begin{theorem}[Closed form of $1/\Phi_4$]
\label{thm:inv-phi4}
\proved{}
For centred $r\in\PnR[4]$ with simple roots and
additive parameters $u,\ell_3,\ell_4$:
\begin{equation}\label{eq:inv-phi4}
  \frac{1}{\Phi_4(r)}
  =\frac{P(u,\ell_3,\ell_4)}{6\,Q(u,\ell_3,\ell_4)},
\end{equation}
where
\begin{align}
  P &:= 16u^6 - 72u^3\ell_3^2
     - 48u^2\ell_4^2 + 32\ell_4^3
     - 216u\,\ell_3^2\ell_4 - 81\ell_3^4,
  \label{eq:P4}\\[3pt]
  Q &:= 4u^5 - 9u^2\ell_3^2
     - 9\ell_3^2\ell_4 - 4u\,\ell_4^2.
  \label{eq:Q4}
\end{align}
Moreover, $P=\disc(r)/110592$ and $Q>0$ for
all centred $r$ with simple roots.
At the Gaussian
$(\ell_3{=}\ell_4{=}0)$:
$1/\Phi_4=2u/3$.
\end{theorem}

\begin{proof}
\emph{Step~1: Coefficients and cumulants.}
Let $r(x)=x^4+a_2 x^2+a_3 x+a_4$ be a centred
depressed quartic.
From Lemma~\ref{lem:normalisation} and
Definition~\ref{def:K}:
$\kappa_2=a_2/12$, $\kappa_3=a_3/24$,
$\kappa_4=a_4/24$, so
$\ell_2=\kappa_2$, $\ell_3=\kappa_3$,
$\ell_4=\kappa_4-\kappa_2^2/2=a_4/24-a_2^2/288$.
These give
\begin{equation}\label{eq:a-to-ell}
  a_2=-12u,\qquad
  a_3=24\ell_3,\qquad
  a_4=24\ell_4+12u^2.
\end{equation}

\emph{Step~2: The cofactor sum.}
Define the \emph{cofactor sum}
$S:=\sum_{(i,j)}\prod_{(k,l)\ne(i,j)}
(\lambda_k-\lambda_l)^2$,
where each sum/product runs over unordered
pairs among the roots $\lambda_1,\ldots,\lambda_4$.
Since there are $\binom{4}{2}=6$ pairs and each
cofactor is a product of $5$ positive squared gaps,
$S>0$.
By Theorem~\ref{thm:fisher-rep},
$\Phi_4=2\mathcal{R}=2S/\disc(r)$, hence
$1/\Phi_4=\disc(r)/(2S)$.

\emph{Step~3: $S$ as a polynomial in the
coefficients.}
The cofactor sum $S$ is a symmetric polynomial
of weight~$10$ in the roots, with $a_1=0$.
By dimensional analysis,
$S=\alpha\, a_2^5+\beta\, a_2^2 a_3^2
+\gamma\, a_2^3 a_4+\delta\, a_3^2 a_4
+\varepsilon\, a_2 a_4^2$.
Evaluating at polynomials
$x^4-5x^2+4$ (roots $\pm 1,\pm 2$),
$x^4-10x^2+9$ (roots $\pm 1,\pm 3$),
$x^4-7x^2-6x$ (roots $-2,-1,0,3$),
$x^4-15x^2-10x+24$ (roots $-3,-2,1,4$),
and $x^4-5x^2+4$ again for consistency,
one obtains the linear system
$(\alpha,\beta,\gamma,\delta,\varepsilon)
=(-4,-18,-32,-216,192)$:
\begin{equation}\label{eq:S-formula}
  S=-4a_2^5-18a_2^2 a_3^2-32a_2^3 a_4
  -216a_3^2 a_4+192a_2 a_4^2.
\end{equation}

\emph{Step~4: Conversion to cumulant coordinates.}
Substituting~\eqref{eq:a-to-ell}
into~\eqref{eq:S-formula} and the standard discriminant
formula
$\disc=256a_4^3-128a_2^2 a_4^2
+144a_2 a_3^2 a_4-27a_3^4
+16a_2^4 a_4-4a_2^3 a_3^2$,
one obtains after expansion:
\begin{equation}\label{eq:disc-P}
  \disc(r)=110592\,P(u,\ell_3,\ell_4),
  \qquad
  S=331776\,Q(u,\ell_3,\ell_4).
\end{equation}
Since $331776=3\cdot 110592$:
\[
  \frac{1}{\Phi_4}
  =\frac{\disc}{2S}
  =\frac{110592\,P}{2\cdot 331776\,Q}
  =\frac{P}{6Q}.\qedhere
\]
\end{proof}

\begin{remark}[Polynomial division]\label{rem:poly-div}
Dividing $P$ by $Q$ in the variable~$u$ gives
$P=4u\cdot Q+M$ where
$M:=-36u^3\ell_3^2-32u^2\ell_4^2
-180u\,\ell_3^2\ell_4+32\ell_4^3-81\ell_3^4$.
Hence $1/\Phi_4=(2u/3)+M/(6Q)$, and the linear
term $2u/3$ cancels under the Stam defect by
additivity of~$u$.
Moreover, $M\le 0$ on the entire valid domain
(verified at the boundary $\ell_4=\pm u^2$
and by continuity).
\end{remark}

\subsection{Stam defect and reduction to polynomial positivity}

\begin{theorem}[Stam defect for $n=4$: polynomial reduction]
\label{thm:D4-reduction}
\proved{}
For centred $p,q\in\PnR[4]$ with $r=p\fp q$,
denote the additive cumulant triples
$(u_p,\ell_{3,p},\ell_{4,p})$ and
$(u_q,\ell_{3,q},\ell_{4,q})$.
Then
\begin{equation}\label{eq:D4-Psi}
  D_4:=\frac{1}{\Phi_4(r)}-\frac{1}{\Phi_4(p)}
  -\frac{1}{\Phi_4(q)}
  =\frac{\Psi}{6\,Q_r Q_p Q_q},
\end{equation}
where $Q_r,Q_p,Q_q>0$ and
\begin{equation}\label{eq:Psi}
  \Psi := M_r\,Q_p\,Q_q
  - M_p\,Q_r\,Q_q - M_q\,Q_r\,Q_p
\end{equation}
is a weighted-degree-$32$ polynomial in the six
additive variables
$(u_p,\ell_{3,p},\ell_{4,p},u_q,\ell_{3,q},\ell_{4,q})$.

Consequently, $D_4\ge 0$ if and only if
$\Psi\ge 0$ on the valid domain.
\end{theorem}

\begin{proof}
From~\eqref{eq:inv-phi4} and $P=4uQ+M$:
\begin{align*}
  D_4 &= \frac{P_r}{6Q_r}
       - \frac{P_p}{6Q_p}
       - \frac{P_q}{6Q_q}\\
      &= \frac{4u_r Q_r+M_r}{6Q_r}
       - \frac{4u_p Q_p+M_p}{6Q_p}
       - \frac{4u_q Q_q+M_q}{6Q_q}\\
      &= \frac{2}{3}(u_r-u_p-u_q)
       + \frac{M_r}{6Q_r}
       - \frac{M_p}{6Q_p}
       - \frac{M_q}{6Q_q}.
\end{align*}
The first term vanishes by additivity of~$u$.
Clearing the common denominator
$6Q_r Q_p Q_q>0$ gives~\eqref{eq:Psi}.
\end{proof}

\subsection{Stam for \texorpdfstring{$n=4$}{n=4} with
\texorpdfstring{$\ell_3=0$}{l3=0}}

\begin{theorem}[Stam for $n=4$, $\ell_3=0$]
\label{thm:stam-l3zero}
\proved{}
For centred $p,q\in\PnR[4]$ with
$\ell_3(p)=\ell_3(q)=0$,
we have $D_4\ge 0$.
More precisely,
\begin{equation}\label{eq:Psi-l30}
  \Psi\big|_{\ell_3=0}
  =512\,(u_p^2-\ell_{4,p})\,
  (u_q^2-\ell_{4,q})\,
  \bigl((u_p{+}u_q)^2-\ell_{4,p}-\ell_{4,q}\bigr)\,
  F,
\end{equation}
where all three parenthetical factors are positive
on the valid domain and
$F=F(u_p,\ell_{4,p},u_q,\ell_{4,q})\ge 0$
for all $u_p,u_q>0$.
\end{theorem}

\begin{proof}
\emph{Step~1: Factorisation.}
With $\ell_{3,p}=\ell_{3,q}=0$:
$M(u,0,\ell_4)=\ell_4^2(32\ell_4-32u^2)
=-32\ell_4^2(u^2-\ell_4)$,
$Q(u,0,\ell_4)=4u(u^4-\ell_4^2)
=4u\,(u^2-\ell_4)(u^2+\ell_4)$.
The factors $(u^2-\ell_4)$ in $M$ and $Q$
propagate through~\eqref{eq:Psi}.
Expanding and collecting by computer algebra
confirms the factorisation~\eqref{eq:Psi-l30} with
\begin{align*}
  F={}&\ell_{4,p}^3\ell_{4,q}\,u_q^2
      +\ell_{4,p}^3 u_q^4
      +\ell_{4,p}^2\ell_{4,q}^2(u_p^2+u_q^2)
      +3\ell_{4,p}^2\ell_{4,q}\,u_p^2 u_q^2
      +2\ell_{4,p}^2\ell_{4,q}\,u_p u_q^3\\
    &+2\ell_{4,p}^2\ell_{4,q}\,u_q^4
      +3\ell_{4,p}^2 u_p^2 u_q^4
      +3\ell_{4,p}^2 u_p u_q^5
      +\ell_{4,p}^2 u_q^6
      +\ell_{4,p}\ell_{4,q}^3 u_p^2\\
    &+2\ell_{4,p}\ell_{4,q}^2 u_p^4
      +2\ell_{4,p}\ell_{4,q}^2 u_p^3 u_q
      +3\ell_{4,p}\ell_{4,q}^2 u_p^2 u_q^2
      -2\ell_{4,p}\ell_{4,q}\,u_p^3 u_q^3\\
    &+\ell_{4,q}^3 u_p^4
      +\ell_{4,q}^2 u_p^6
      +3\ell_{4,q}^2 u_p^5 u_q
      +3\ell_{4,q}^2 u_p^4 u_q^2.
\end{align*}

\emph{Step~2: Positivity of the triple product.}
On the valid domain, $Q>0$ forces
$|\ell_4|<u^2$, so $u^2-\ell_4>0$
for each polynomial and its convolution.

\emph{Step~3: $F\ge 0$ by AM--GM.}
The polynomial~$F$ has exactly one negative
monomial: $-2\ell_{4,p}\ell_{4,q}u_p^3 u_q^3$.
Write $F=F_+-2\ell_{4,p}\ell_{4,q}u_p^3 u_q^3$
where every coefficient of $F_+$ is non-negative
(verified by computer algebra).
We consider two cases:
\begin{itemize}[nosep]
  \item If $\ell_{4,p}\ell_{4,q}\le 0$, the
    ``negative'' monomial is in fact $\ge 0$,
    so $F\ge F_+\ge 0$.
  \item If $\ell_{4,p}\ell_{4,q}>0$:
    $F_+$ contains the terms
    $3\ell_{4,p}^2 u_p u_q^5
    +3\ell_{4,q}^2 u_p^5 u_q
    =3u_p u_q(\ell_{4,p}^2 u_q^4
    +\ell_{4,q}^2 u_p^4)$.
    By the AM--GM inequality
    $\ell_{4,p}^2 u_q^4+\ell_{4,q}^2 u_p^4
    \ge 2|\ell_{4,p}\ell_{4,q}|u_p^2 u_q^2$,
    so these two terms alone contribute
    $\ge 6|\ell_{4,p}\ell_{4,q}| u_p^3 u_q^3
    >2|\ell_{4,p}\ell_{4,q}| u_p^3 u_q^3$,
    absorbing the negative monomial with room to spare.
\end{itemize}
Hence $F\ge 0$, and $D_4\ge 0$.
\end{proof}

\begin{proposition}[Two one-sided sectors at $n=4$]
\label{prop:n4-one-sided}
\proved{}
Let $p,q\in\PnR[4]$ be centred, and assume $q$ is finite Gaussian
\textup{(}equivalently $\ell_{3,q}=\ell_{4,q}=0$\textup{)}.
Then:
\begin{enumerate}[label=\textup{(\alph*)},nosep]
  \item if $\ell_{4,p}=0$, then $D_4\ge 0$;
  \item if $\ell_{3,p}=0$, then $D_4\ge 0$.
\end{enumerate}
By symmetry, the analogous statements hold with $p$ and $q$ interchanged.
\end{proposition}

\begin{proof}
Write
$u_1:=u_p$, $u_2:=u_q$,
$a:=\ell_{3,p}$, $b:=\ell_{4,p}$,
$c:=\ell_{3,q}=0$, $d:=\ell_{4,q}=0$.
Since
$D_4=\Psi/(6Q_rQ_pQ_q)$ and $Q_r,Q_p,Q_q>0$
(Theorem~\ref{thm:D4-reduction}), it suffices to show $\Psi\ge 0$.

\emph{(a) $\ell_{4,p}=0$ (i.e. $b=0$).}
Substituting $b=c=d=0$ into~\eqref{eq:Psi} and collecting gives
\[
  \Psi=36a^2u_2^6\,\Gamma(s),\qquad s:=a^2,
\]
with
\[
  \Gamma(s)
  =-81(2u_1+u_2)s^2
  +36(2u_1^2+2u_1u_2+u_2^2)(3u_1^2+3u_1u_2+u_2^2)s
  +16u_1^3(u_1+u_2)^3(2u_1+u_2).
\]
Hence $\Gamma$ is concave in $s$ since the quadratic coefficient is
$-81(2u_1+u_2)<0$.
From $Q_p=4u_1^5-9u_1^2a^2>0$ we obtain
$0\le s<\frac{4}{9}u_1^3$.
Now
\[
  \Gamma(0)=16u_1^3(u_1+u_2)^3(2u_1+u_2)>0,
\]
and
\[
  \Gamma\!\left(\frac{4}{9}u_1^3\right)
  =32u_1^3(u_1+u_2)^2(3u_1^2+3u_1u_2+u_2^2)>0.
\]
Since a concave function on an interval is bounded below by the minimum
of its endpoint values, $\Gamma(s)>0$ for all
$s\in[0,\frac{4}{9}u_1^3]$.
Therefore $\Psi\ge 0$, hence $D_4\ge 0$.

\emph{(b) $\ell_{3,p}=0$ (i.e. $a=0$).}
Substituting $a=c=d=0$ into~\eqref{eq:Psi} gives
\[
  \Psi
  =512b^2u_2^6(u_1^2-b)\bigl((u_1+u_2)^2-b\bigr)
  (b+3u_1^2+3u_1u_2+u_2^2).
\]
On the valid domain,
$Q_p=4u_1(u_1^2-b)(u_1^2+b)>0$
implies $u_1^2-b>0$, and
$Q_r=4(u_1+u_2)\bigl((u_1+u_2)^2-b\bigr)
\bigl((u_1+u_2)^2+b\bigr)>0$
implies $((u_1+u_2)^2-b)>0$.
The remaining factors are manifestly non-negative, so $\Psi\ge 0$,
hence $D_4\ge 0$.
\end{proof}

\subsection{Leading-order positive-definiteness}

\begin{proposition}[Hessian of $D_4$ at the Gaussian]
\label{prop:D4-hessian}
\proved{}
The Stam defect $D_4$ vanishes at the Gaussian
$(\ell_{3,p}=\ell_{4,p}=\ell_{3,q}=\ell_{4,q}=0)$
with vanishing gradient.
The Hessian of $D_4$ in the four non-Gaussian variables
$(\ell_{3,p},\ell_{4,p},\ell_{3,q},\ell_{4,q})$
is positive definite; at $u_p=u_q=1$, its
eigenvalues are $9216$, $12288$, $16384$, $18432$.
\end{proposition}

\begin{proof}
The numerator $\Psi$ in~\eqref{eq:Psi} is
quadratic in the non-Gaussian cumulants
to leading order.
Retaining only $M\approx -36u^3\ell_3^2
-32u^2\ell_4^2$ and $Q\approx 4u^5$,
the leading-order $\Psi$ is a quadratic form in
$(\ell_{3,p},\ell_{4,p},\ell_{3,q},\ell_{4,q})$
with matrix
$\mathrm{diag}(A,B)\oplus\mathrm{diag}(B,A)$
(block-diagonal with off-diagonal coupling).
At $u_p=u_q=1$ the Gram matrix evaluates to
$\bigl(\begin{smallmatrix}
  13824 & 0 & -4608 & 0\\
  0 & 14336 & 0 & -2048\\
  -4608 & 0 & 13824 & 0\\
  0 & -2048 & 0 & 14336
\end{smallmatrix}\bigr)$
with eigenvalues $9216,12288,16384,18432>0$.
\end{proof}

\begin{remark}[Status of the general case]
\label{rem:n4-general}
The polynomial $\Psi$ in~\eqref{eq:Psi} has
$547$ terms in six additive variables.
Computer algebra (SymPy) confirms non-negativity
of~$\Psi$ at all tested rational points
in the valid domain, including the three
exact-arithmetic tests
$(p,q)=(\text{roots }-2,-1,1,2;\;\text{roots }-3,-1,1,3)$,
$(p,q)=(\text{roots }-2,-1,0,3;\;\text{roots }-3,-1,1,3)$,
and
$(p,q)=(\text{roots }-2,-1,0,3;\;\text{roots }-3,-2,1,4)$,
yielding $D_4=70940/948051$,
$3991102202996/27964607674185$,
and $4658367073739/13296968779710$
respectively.
\par
Combined with the closed-form proof for
$\ell_3=0$ (Theorem~\ref{thm:stam-l3zero}),
the one-sided sector proof
(Proposition~\ref{prop:n4-one-sided}),
the leading-order positive-definiteness
(Proposition~\ref{prop:D4-hessian}),
and the established sub-cases
(including single nonzero~$\ell_3$),
the evidence strongly supports
$\Psi\ge 0$ on the full valid domain.
The remaining analytic step is a
Positivstellensatz certificate for
the~$547$-term polynomial modulo the
constraints $Q_p>0$, $Q_q>0$.
\end{remark}


% ═══════════════════════════════════════════════════════════
\section{Discussion and open problems}\label{sec:open}
% ═══════════════════════════════════════════════════════════

\noindent\textbf{Summary.}
The Stam inequality~\eqref{eq:stam-poly} is proved
unconditionally for $n\le 3$ and for Gaussian
inputs at all~$n$.
The Hermite heat equation
$\partial_t r_t=-r_t''/(2(n-1))$
(Theorem~\ref{thm:hermite-heat})
and the resulting root ODE
$\dot\lambda_i=V_i/(n-1)$
(Corollary~\ref{cor:root-ode})
are now proved unconditionally from the
$K$-transform definition.
For $n=4$, a closed-form formula
$1/\Phi_4=P/(6Q)$
(Theorem~\ref{thm:inv-phi4})
reduces the Stam inequality to
the non-negativity of a weighted-degree-$32$
polynomial~$\Psi$ in six additive cumulant variables
(Theorem~\ref{thm:D4-reduction}).
The polynomial identity is fully established when
$\ell_3=0$ (Theorem~\ref{thm:stam-l3zero}),
using a factorisation into three positive factors
times a manifestly non-negative form (AM--GM).
The Hessian at the Gaussian is positive definite
(Proposition~\ref{prop:D4-hessian}).
The degree-telescoping identity (Theorem~\ref{thm:telescope})
reduces the full conjecture to bounding
the explicit correction terms
$C_k=D_k-D_{k-1}$ for $k\ge 4$.
The principal remaining challenges are:
\begin{enumerate}[nosep,label=(\roman*)]
\item For $n=4$: a Positivstellensatz
  certificate for $\Psi\ge 0$ modulo $Q_p>0$, $Q_q>0$
  (a semidefinite program in weighted-degree-$32$ polynomials).
\item For general $n$: a uniform
  nonnegativity mechanism for the high-degree corrections.
\end{enumerate}

\medskip
\noindent\textbf{Dependency table.}
The table below records the proof status of each main
result and its external dependencies.

\smallskip
\begin{center}
\renewcommand{\arraystretch}{1.2}
\begin{tabular}{@{}lll@{}}
\toprule
\textbf{Result} & \textbf{Status} & \textbf{Dependencies} \\
\midrule
Stam $n=2$ (\S\ref{ssec:n2})
  & \proved{} & none \\
Stam $n=3$ (Thm~\ref{thm:n3-stam})
  & \proved{} & SOS identity~\eqref{eq:D3} \\
Stam $n=3$ via CS (Rem.~\ref{rem:n3-cs})
  & \proved{} & $R_3$ exact (Thm~\ref{thm:Rn-quad}) \\
Gauss.\ uniqueness (Lem.~\ref{lem:gauss-unique})
  & \proved{} & Fisher--var.\ (Thm~\ref{thm:fisher-var}) \\
Gauss.-input Stam (Thm~\ref{thm:stam-gauss})
  & \proved{} & heat eqn.~(Thm~\ref{thm:hermite-heat}) \\
Hermite heat eqn.~(Thm~\ref{thm:hermite-heat})
  & \proved{} & $K$-transform (Def.~\ref{def:K}) \\
Root ODE (Cor.~\ref{cor:root-ode})
  & \proved{} & heat eqn.~(Thm~\ref{thm:hermite-heat}) \\
De Bruijn (Thm~\ref{thm:debruijn})
  & \proved{} & root ODE (Cor.~\ref{cor:root-ode}) \\
$c_{n,k}>0$, $n\le 4$ (Thm~\ref{thm:Rn-quad}b)
  & \proved{} & direct computation \\
$c_{n,k}>0$, $n\ge 5$ (Thm~\ref{thm:Rn-quad}c)
  & \compverif{} & Hessian diagonality \\
Quad.\ Stam bound (Thm~\ref{thm:quad-stam})
  & \proved{}/$n\le 4$ & $c_{n,k}>0$ \\
Gen.\ decomp.\ (Thm~\ref{thm:general-decomp})
  & \proved{}/$n\le 4$ & $c_{n,k}>0$ \\
Telescope (Thm~\ref{thm:telescope})
  & \proved{} & Cauchy interlacing \\
$R_4$ Taylor (\S\ref{sec:n4})
  & \proved{} & Thm~\ref{thm:inv-phi4} \\
$1/\Phi_4$ closed form (Thm~\ref{thm:inv-phi4})
  & \proved{} & cofactor sum \& discriminant \\
$n{=}4$ Stam, $\ell_3{=}0$ (Thm~\ref{thm:stam-l3zero})
  & \proved{} & factorisation + AM--GM \\
$n{=}4$ one-sided sectors (Prop.~\ref{prop:n4-one-sided})
  & \proved{} & explicit $\Psi$ reduction + valid-domain bounds \\
$D_4$ Hessian (Prop.~\ref{prop:D4-hessian})
  & \proved{} & quadratic form computation \\
$n{=}4$ Stam, general
  & \compverif{} & $\Psi\ge 0$ (Remark~\ref{rem:n4-general}) \\
\bottomrule
\end{tabular}
\end{center}

\begin{conjecture}[Finite free Stam inequality]
\label{conj:stam}
$1/\Phi_n(p\fp q)\ge 1/\Phi_n(p)+1/\Phi_n(q)$
for all $n\ge 2$ and $p,q\in\PnR$.
\end{conjecture}

\begin{conjecture}[Equality characterisation]
\label{conj:equality}
For $n\ge 3$ and $p,q\in\PnR$ with simple roots,
equality holds in Conjecture~\ref{conj:stam}
if and only if both $p$ and $q$ are
finite Gaussians (i.e.,
$\ell_k(p)=\ell_k(q)=0$ for all $k\ge 3$).
This is proved for $n=3$
(Theorem~\ref{thm:n3-stam}).
\end{conjecture}


% ═══════════════════════════════════════════════════════════
\section*{References}
% ═══════════════════════════════════════════════════════════

\begin{thebibliography}{99}

\bibitem{bb}
J.~Borcea and P.~Br\"and\'en,
\emph{The {L}ee--{Y}ang and {P}\'olya--{S}chur programs.
{I}.\ {L}inear operators preserving stability},
Invent.\ Math.\ \textbf{177} (2009), 541--569.

\bibitem{blachman}
N.~M.~Blachman,
\emph{The convolution inequality for entropy powers},
IEEE Trans.\ Inform.\ Theory \textbf{IT-11} (1965),
267--271.

\bibitem{dembo}
A.~Dembo, T.~M.~Cover, and J.~A.~Thomas,
\emph{Information-theoretic inequalities},
IEEE Trans.\ Inform.\ Theory \textbf{37} (1991),
1501--1518.

\bibitem{dyson}
F.~J.~Dyson,
\emph{A {B}rownian-motion model for the eigenvalues
of a random matrix},
J.~Math.\ Phys.\ \textbf{3} (1962), 1191--1198.

\bibitem{fiedler}
M.~Fiedler,
\emph{Hankel and {L}oewner matrices},
Linear Algebra Appl.\ \textbf{58} (1984), 75--95.

\bibitem{MSS1}
A.~Marcus, D.~A.~Spielman, and N.~Srivastava,
\emph{Interlacing families {I}: Bipartite {R}amanujan
graphs of all degrees},
Ann.\ of Math.\ (2) \textbf{182} (2015), 307--325.

\bibitem{MSS2}
A.~Marcus, D.~A.~Spielman, and N.~Srivastava,
\emph{Interlacing families {II}: Mixed characteristic
polynomials and the {K}adison--{S}inger problem},
Ann.\ of Math.\ (2) \textbf{182} (2015), 327--350.

\bibitem{stam}
A.~J.~Stam,
\emph{Some inequalities satisfied by the quantities of
information of {F}isher and {S}hannon},
Inform.\ Control \textbf{2} (1959), 101--112.

\bibitem{voiculescu}
D.~V.~Voiculescu, K.~J.~Dykema, and A.~Nica,
\emph{Free Random Variables},
CRM Monograph Series, vol.~1, Amer.\ Math.\ Soc.,
Providence, RI, 1992.

\end{thebibliography}

\end{document}
