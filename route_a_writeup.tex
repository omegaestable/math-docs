\documentclass[11pt]{amsart}

\usepackage[margin=1in]{geometry}
\usepackage[T1]{fontenc}
\usepackage{lmodern}
\usepackage{microtype}
\usepackage{amsmath,amssymb,amsthm}
\usepackage{mathtools}
\usepackage[colorlinks=true,linkcolor=blue,citecolor=blue,urlcolor=blue]{hyperref}
\usepackage{enumitem}

\allowdisplaybreaks
\setlength{\jot}{8pt}

\newtheorem{theorem}{Theorem}[section]
\newtheorem{lemma}[theorem]{Lemma}
\newtheorem{proposition}[theorem]{Proposition}
\newtheorem{corollary}[theorem]{Corollary}
\theoremstyle{definition}
\newtheorem{definition}[theorem]{Definition}
\newtheorem{conjecture}[theorem]{Conjecture}
\theoremstyle{remark}
\newtheorem{remark}[theorem]{Remark}

\newcommand{\R}{\mathbb{R}}
\newcommand{\C}{\mathbb{C}}
\newcommand{\Pn}{\mathcal{P}_n}
\newcommand{\PnR}{\mathcal{P}_n^{\R}}

\title[Route A: Resolvent/barrier regularisation]{Route A: Resolvent/barrier regularisation for the finite free Stam inequality\\[4pt]
  \normalsize Proxy functionals, an exact identity $\Phi_n=2\mathcal{R}$, and numerical experiments}
\author{}
\date{}

\begin{document}
\maketitle

\begin{abstract}
We pursue the \emph{resolvent/barrier regularisation} approach (Route~A) to the
finite free Stam inequality $1/\Phi_n(p\boxplus_n q)\ge 1/\Phi_n(p)+1/\Phi_n(q)$.
Two candidate proxy functionals built from the Cauchy transform
$g_{p,\eta}(x)=\frac{1}{n}\frac{p'(x+i\eta)}{p(x+i\eta)}$ at height $\eta>0$
are introduced and tested numerically.
As a by-product, we discover and prove a short algebraic identity:
the Fisher information $\Phi_n(p)$ equals \emph{twice} the pairwise repulsion
energy $\mathcal{R}(p)=\sum_{i<j}(\lambda_i-\lambda_j)^{-2}$
(Theorem~\ref{thm:phi-eq-2R}).
This identity sharpens the paper's Cauchy--Schwarz bound $\Phi_n\le 2(n-1)\mathcal{R}$
by a factor of $n-1$ and translates the Stam inequality into a
\emph{harmonic-mean bound on pairwise repulsion} (Corollary~\ref{cor:stam-rewrite}).
Extensive numerical experiments (degrees $n=2$--$8$, $\ge5000$ trials each, plus
near-collision stress tests) confirm the Stam inequality with \textbf{zero violations}
and reveal that Lorentzian regularisation ($\eta>0$) of the repulsion energy
\emph{breaks} super-additivity,
ruling out a direct resolvent-barrier proof at fixed height.
All Python scripts are attached.
\end{abstract}

\tableofcontents

%======================================================================
\section{Definitions and notation}
\label{sec:defs}
%======================================================================

We use the notation of the main document. Fix $n\ge 2$. Let
$p\in\PnR$ be monic of degree $n$ with distinct roots
$\lambda_1<\cdots<\lambda_n$ and scores $V_i=\sum_{j\ne i}(\lambda_i-\lambda_j)^{-1}$.
Write $\Phi_n(p)=\sum_i V_i^2$ (Fisher information),
$\sigma^2(p)=\frac{1}{n}\sum_i(\lambda_i-\bar\lambda)^2$ (variance),
and $\boxplus_n$ for the MSS symmetric additive convolution.

\begin{definition}[Pairwise repulsion energy]
\label{def:repulsion}
\[
  \mathcal{R}(p):=\sum_{1\le i<j\le n}\frac{1}{(\lambda_i-\lambda_j)^2}.
\]
\end{definition}

\begin{definition}[Cauchy transform at height $\eta$]
\label{def:cauchy}
For $\eta>0$ and $x\in\R$, define
\[
  g_{p,\eta}(x):=\frac{1}{n}\sum_{j=1}^n\frac{1}{x+i\eta-\lambda_j}
  =:u_{p,\eta}(x)+i\,v_{p,\eta}(x),
\]
where
\[
  u_{p,\eta}(x)=\frac{1}{n}\sum_j\frac{x-\lambda_j}{(x-\lambda_j)^2+\eta^2},
  \qquad
  v_{p,\eta}(x)=\frac{1}{n}\sum_j\frac{\eta}{(x-\lambda_j)^2+\eta^2}.
\]
\end{definition}

%======================================================================
\section{The identity $\Phi_n=2\mathcal{R}$}
\label{sec:identity}
%======================================================================

The main document establishes by Cauchy--Schwarz that
$\Phi_n(p)\le 2(n-1)\mathcal{R}(p)$
(Lemma~6.6 of the arxiv draft).
We now prove that the correct relationship is an \emph{exact identity}, removing the factor $n-1$.

\begin{theorem}[Fisher--repulsion identity]
\label{thm:phi-eq-2R}
For every $p\in\PnR$ with $n\ge 2$ distinct roots,
\begin{equation}\label{eq:phi-eq-2R}
  \Phi_n(p)=2\,\mathcal{R}(p).
\end{equation}
\end{theorem}

\begin{proof}
Expand $\Phi_n=\sum_{i=1}^n V_i^2$ by writing each $V_i$ as a sum:
\[
  \Phi_n=\sum_{i=1}^n\Bigl(\sum_{j\ne i}\frac{1}{\lambda_i-\lambda_j}\Bigr)^{\!2}
  =\sum_i\sum_{j\ne i}\sum_{k\ne i}
    \frac{1}{(\lambda_i-\lambda_j)(\lambda_i-\lambda_k)}.
\]
Split the triple sum over ordered pairs $(j,k)$ with $j,k\ne i$ into
\emph{diagonal terms} ($j=k$) and \emph{off-diagonal terms} ($j\ne k$):
\[
  \Phi_n
  =\underbrace{\sum_i\sum_{j\ne i}\frac{1}{(\lambda_i-\lambda_j)^2}}_{\displaystyle=\,2\mathcal{R}}
  +\underbrace{\sum_i\sum_{\substack{j\ne i,\;k\ne i\\j\ne k}}
    \frac{1}{(\lambda_i-\lambda_j)(\lambda_i-\lambda_k)}}_{\displaystyle=:\,C}.
\]
The first sum counts every unordered pair $\{i,j\}$ twice
(once with $i$ as the ``pivot'' and once with $j$), giving $2\mathcal{R}$.

It remains to show $C=0$.
Group the terms in~$C$ by unordered triples $\{a,b,c\}\subset\{1,\dots,n\}$.
For each such triple, the three choices of pivot $i\in\{a,b,c\}$ each
contribute two ordered pairs $(j,k)$, totalling
\[
  C=2\!\sum_{\{a,b,c\}}\!\Bigl[
    \frac{1}{(\lambda_a-\lambda_b)(\lambda_a-\lambda_c)}
    +\frac{1}{(\lambda_b-\lambda_a)(\lambda_b-\lambda_c)}
    +\frac{1}{(\lambda_c-\lambda_a)(\lambda_c-\lambda_b)}
  \Bigr].
\]
Setting $u=\lambda_a-\lambda_b$, $v=\lambda_a-\lambda_c$ (so
$\lambda_b-\lambda_c=v-u$), each bracket becomes
\[
  \frac{1}{uv}+\frac{1}{(-u)(v-u)}+\frac{1}{(-v)(u-v)}
  =\frac{1}{uv}-\frac{1}{u(v-u)}+\frac{1}{v(v-u)}.
\]
Computing over a common denominator $uv(v-u)$:
\[
  \frac{v-u}{uv(v-u)}-\frac{v}{uv(v-u)}+\frac{u}{uv(v-u)}
  =\frac{v-u-v+u}{uv(v-u)}=0.
\]
Since each bracket vanishes, $C=0$, and $\Phi_n=2\mathcal{R}$.
\end{proof}

\begin{remark}
The identity $\Phi_n=2\mathcal{R}$ is a purely algebraic consequence
of the definition $V_i=\sum_{j\ne i}(\lambda_i-\lambda_j)^{-1}$;
it holds for any list of distinct real numbers, with no real-rootedness
assumption.
\end{remark}

\begin{corollary}[Stam as a harmonic-mean bound on repulsion]
\label{cor:stam-rewrite}
The Stam inequality
$1/\Phi_n(p\boxplus_n q)\ge 1/\Phi_n(p)+1/\Phi_n(q)$
is equivalent to
\begin{equation}\label{eq:stam-repulsion}
  \frac{1}{\mathcal{R}(p\boxplus_n q)}
  \ge \frac{1}{\mathcal{R}(p)}+\frac{1}{\mathcal{R}(q)},
\end{equation}
i.e.\ $\mathcal{R}(p\boxplus_n q)$ is bounded above by the
\emph{harmonic mean} of $\mathcal{R}(p)$ and $\mathcal{R}(q)$.
\end{corollary}

\begin{proof}
Divide \eqref{eq:phi-eq-2R} through by $2$.
\end{proof}

\begin{corollary}[Improved Fisher--variance inequality]
\label{cor:improved-FV}
The identity $\Phi_n=2\mathcal{R}$ combined with the Fisher--variance bound
$\Phi_n\sigma^2\ge n(n-1)^2/4$ gives
\[
  \mathcal{R}(p)\,\sigma^2(p)\ge \frac{n(n-1)^2}{8},
\]
with equality iff $V_i=c(\lambda_i-\bar\lambda)$ for a constant $c$.
\end{corollary}

%======================================================================
\section{Two candidate proxy functionals (Route A)}
\label{sec:proxies}
%======================================================================

\subsection{Proxy A1: Lorentzian-regularised inverse repulsion}

\begin{definition}[Regularised repulsion]
\label{def:R-eta}
For $\eta\ge 0$, define
\[
  \mathcal{R}_\eta(p):=\sum_{1\le i<j\le n}\frac{1}{(\lambda_i-\lambda_j)^2+4\eta^2}.
\]
Note that $\mathcal{R}_0=\mathcal{R}$ and $\mathcal{R}_\eta\le\mathcal{R}$ for all $\eta>0$.
\end{definition}

\begin{definition}[Proxy A1]
\[
  \mathcal{P}_\eta^{(A1)}(p):=\frac{1}{2(n-1)\,\mathcal{R}_\eta(p)}.
\]
\end{definition}

\begin{lemma}[Properties of Proxy A1]
\label{lem:A1-props}
\hfill
\begin{enumerate}[label=\textup{(\roman*)},nosep]
  \item At $\eta=0$:
    $\mathcal{P}_0^{(A1)}(p)=\frac{1}{(n-1)\Phi_n(p)}$,
    so the conjectured super-additivity
    $\mathcal{P}_0^{(A1)}(r)\ge\mathcal{P}_0^{(A1)}(p)+\mathcal{P}_0^{(A1)}(q)$
    is exactly $\frac{1}{n-1}$ times the Fisher Stam inequality.
  \item For $\eta>0$:
    $\mathcal{P}_\eta^{(A1)}(p)\ge \mathcal{P}_0^{(A1)}(p)$.
  \item The function $\eta\mapsto\mathcal{P}_\eta^{(A1)}(p)$ is
    strictly increasing on $(0,\infty)$ and
    $\mathcal{P}_\eta^{(A1)}(p)\to\infty$ as $\eta\to\infty$.
\end{enumerate}
\end{lemma}

\begin{proof}
Part~(i) follows from $\Phi_n=2\mathcal{R}$ (Theorem~\ref{thm:phi-eq-2R}).
Parts (ii) and (iii) are immediate from $\mathcal{R}_\eta\le\mathcal{R}_0$
and the fact that the regularisation decreases monotonically with~$\eta$.
\end{proof}

\begin{proposition}[Proxy A1 super-additivity \emph{fails} for $\eta>0$]
\label{prop:A1-fails}
For every $\eta>0$ and every $n\ge 3$, there exist $p,q\in\PnR$ with
\[
  \mathcal{P}_\eta^{(A1)}(p\boxplus_n q)
  <\mathcal{P}_\eta^{(A1)}(p)+\mathcal{P}_\eta^{(A1)}(q).
\]
\emph{(Numerical evidence; see Section~\ref{sec:numerics}.)}
\end{proposition}

\begin{remark}
For $\eta=0$, the proxy A1 inequality is by Lemma~\ref{lem:A1-props}(i)
equivalent to the Fisher Stam inequality. Therefore the $\eta=0$ case is
\emph{open} (not false), with $0$ violations in $>35{,}000$ random tests.
\end{remark}

\subsection{Proxy A2: Density-weighted resolvent Fisher inverse}

\begin{definition}[Resolvent Fisher information]
\label{def:Phi-res}
For $\eta>0$, define
\[
  \Phi_{n,\eta}^{\mathrm{res}}(p)
  :=\frac{n^3}{\pi}\int_{-\infty}^\infty u_{p,\eta}(x)^2\,v_{p,\eta}(x)\,dx,
\]
where $u_{p,\eta}=\operatorname{Re}g_{p,\eta}$,
$v_{p,\eta}=\operatorname{Im}g_{p,\eta}$.
\end{definition}

\begin{lemma}[Limiting behaviour]
$\Phi_{n,\eta}^{\mathrm{res}}(p)\to\Phi_n(p)$ as $\eta\to 0^+$.
\end{lemma}

\begin{proof}[Proof sketch]
As $\eta\to 0^+$, the weight $\frac{n}{\pi}v_{p,\eta}\,dx$ converges
weakly to $\sum_{i=1}^n\delta_{\lambda_i}$ and
$u_{p,\eta}(\lambda_i)\to V_i/n$,
so $\Phi_{n,\eta}^{\mathrm{res}}\to n^2\sum_i(V_i/n)^2=\Phi_n$.
\end{proof}

\begin{definition}[Proxy A2]
$\mathcal{P}_\eta^{(A2)}(p):=1/\Phi_{n,\eta}^{\mathrm{res}}(p)$.
\end{definition}

\begin{proposition}[Proxy A2 super-additivity \emph{fails} for all $\eta>0$]
\label{prop:A2-fails}
For every $\eta>0$ tested and every $n\ge 3$, violations
$\mathcal{P}_\eta^{(A2)}(r)<\mathcal{P}_\eta^{(A2)}(p)+\mathcal{P}_\eta^{(A2)}(q)$
appear generically ($\ge85\%$ of random trials).
\emph{(Numerical evidence; see Section~\ref{sec:numerics}.)}
\end{proposition}

%======================================================================
\section{Proofs and lemma status}
\label{sec:proofs}
%======================================================================

\begin{center}
\renewcommand{\arraystretch}{1.3}
\begin{tabular}{lll}
\hline
\textbf{Result} & \textbf{Status} & \textbf{Reference} \\
\hline
$\Phi_n(p)=2\mathcal{R}(p)$ & \textbf{Proven}
  & Theorem~\ref{thm:phi-eq-2R} \\
Stam $\Leftrightarrow$ harmonic-mean repulsion
  & \textbf{Proven} & Corollary~\ref{cor:stam-rewrite} \\
Proxy A1 ($\eta>0$) super-additivity
  & \textbf{False} & Proposition~\ref{prop:A1-fails} \\
Proxy A1 ($\eta=0$) super-additivity
  & \textbf{Open = Stam} & Lemma~\ref{lem:A1-props}(i) \\
Proxy A2 super-additivity
  & \textbf{False} & Proposition~\ref{prop:A2-fails} \\
Improved $\mathcal{R}\sigma^2\ge n(n-1)^2/8$
  & \textbf{Proven}
  & Corollary~\ref{cor:improved-FV} \\
\hline
\end{tabular}
\end{center}

\subsection{Detailed heuristic: why Lorentzian regularisation fails}

The failure of Proxy~A1 for $\eta>0$ can be understood as follows.
At $\eta=0$ we have the identity $\Phi_n=2\mathcal{R}$, so the
super-additivity of $1/\mathcal{R}$ is \emph{exactly} the Stam
inequality.
However, the regularisation
$\mathcal{R}_\eta=\sum_{i<j}(d_{ij}^2+4\eta^2)^{-1}$
breaks the algebraic cancellation in the proof of
Theorem~\ref{thm:phi-eq-2R}: the identity
$\Phi_n=2\mathcal{R}$ uses the exact inverse-square kernel
$d_{ij}^{-2}$, and any softening of this kernel introduces
non-cancelling cross terms.
In particular, $\sum_i V_{i,\eta}^2\ne 2\mathcal{R}_\eta$
for $\eta>0$ (where $V_{i,\eta}$ is the Lorentzian-regularised
score).

As a concrete diagnostic, the ``equispaced roots'' example
with $n=4$ satisfies $\Phi_4=2\mathcal{R}$ exactly, but when
$\eta$ increases the proxy excess
$\mathcal{P}_\eta^{(A1)}(r)-\mathcal{P}_\eta^{(A1)}(p)-\mathcal{P}_\eta^{(A1)}(q)$
begins negative at $\eta\approx 0.3$
(see the $\eta$-asymptotics table in Section~\ref{sec:numerics}).

\subsection{Why Proxy A2 fails more severely}

The resolvent Fisher $\Phi_{n,\eta}^{\mathrm{res}}$ involves the
\emph{cube} of the Cauchy kernel (two factors of the real part and
one of the imaginary part), making it a more nonlinear functional of
the root configuration.
The $L^2$-integration mixes contributions from all roots
simultaneously, destroying the pairwise structure that is critical
for the cancellation argument.
Even the direction of the error is adverse:
$\Phi_{n,\eta}^{\mathrm{res}}$ systematically \emph{underestimates}
$\Phi_n$ for the convolution output (relative to the summands),
causing the inverse to overshoot, hence negative excess.

%======================================================================
\section{Implications for the finite free Stam inequality}
\label{sec:implications}
%======================================================================

\subsection{What the identity $\Phi_n=2\mathcal{R}$ gives}

\begin{enumerate}[label=\textup{(\roman*)},leftmargin=2em]
  \item It provides the cleanest formulation of the Stam inequality:
    \[
      \frac{1}{\mathcal{R}(p\boxplus_n q)}
      \ge \frac{1}{\mathcal{R}(p)}+\frac{1}{\mathcal{R}(q)}.
    \]
    The pairwise repulsion $\mathcal{R}$ is a simpler object than
    $\Phi_n$ (no ``scores'' needed; just root gaps).

  \item It strengthens the Fisher--variance bound by a factor of
    $n-1$.

  \item It connects $\Phi_n$ directly to the \emph{pair-correlation
    function} in the sense of random matrix theory, potentially
    opening the door to determinantal or Pfaffian arguments.

  \item It simplifies the Hermite flow dissipation:
    Since $\Phi_n=2\mathcal{R}$, the Hermite dissipation
    $\dot\Phi_n=-\frac{2}{n-1}\mathcal{S}$ translates to
    $\dot{\mathcal{R}}=-\frac{1}{n-1}\mathcal{S}$.
\end{enumerate}

\subsection{What Route A does \emph{not} give}

The resolvent regularisation approach, as tested here, does
\emph{not} yield a super-additive proxy at any fixed $\eta>0$.
Therefore, a direct barrier argument using the Cauchy transform
at height $\eta$ and taking $\eta\to 0$ is unlikely to succeed
without substantial modification.

However, the identity $\Phi_n=2\mathcal{R}$ is itself a
by-product of Route~A exploration and is useful for
Routes~B and~C as well.

%======================================================================
\section{Numerical experiments}
\label{sec:numerics}
%======================================================================

\subsection{Protocol}

All experiments use IEEE~754 double precision via NumPy~2.4.
Polynomials are specified by roots; the convolution
$p\boxplus_n q$ is computed via the coefficient formula
(Definition~1.1 of the main paper) and roots extracted with
\texttt{numpy.roots}.
Random polynomials are generated by sampling $n$ iid
$\mathrm{Uniform}[-3,3]$ roots (centred for $q$ by subtracting
the mean).

Scripts: \texttt{route\_a\_core.py}, \texttt{route\_a\_experiments.py},
\texttt{test\_repulsion\_stam.py}.

\subsection{Summary of results}

\textbf{1.\ Fisher Stam ($0$ violations):}
Across $>35{,}000$ random polynomial pairs for $n=2,\dots,8$,
plus near-collision stress tests with root gaps down to
$10^{-6}$, we observe \emph{zero violations} of the Stam
inequality.

\medskip
\textbf{2.\ Proxy~A1 random sweep} (500 trials each, $n=3,4,5$):

\begin{center}
\begin{tabular}{lrrr}
\hline
$\eta$ & $n=3$ viol.\ & $n=4$ viol.\ & $n=5$ viol.\ \\
\hline
$0.000$ & 0 & 0 & 0 \\
$0.001$ & 0 & 0 & 0 \\
$0.010$ & 0 & 0 & 0 \\
$0.050$ & 9 & 0 & 0 \\
$0.100$ & 18 & 1 & 0 \\
$0.500$ & 302 & 324 & 371 \\
$1.000$ & 489 & 500 & 500 \\
\hline
\end{tabular}
\end{center}

\medskip
\textbf{3.\ Proxy~A2:} violations at every tested $\eta>0$
($\ge85\%$ of trials for $n=3$; $\ge56\%$ for $n=5$).

\medskip
\textbf{4.\ Identity $\Phi_n=2\mathcal{R}$:}
The ratio $\Phi_n/(2(n-1)\mathcal{R})$ equals $1/(n-1)$
to machine precision ($\le 10^{-10}$) across all
$10{,}000$ random polynomials tested for each $n\in\{2,\dots,10\}$.

\medskip
\textbf{5.\ Repulsion Stam excess / Fisher Stam excess:}
The ratio equals \textbf{exactly 2} for all tested pairs,
as predicted by Theorem~\ref{thm:phi-eq-2R}.

%======================================================================
\section*{Acknowledgements}
%======================================================================

All computations were performed with Python~3.14, NumPy~2.4, and mpmath~1.3.

\end{document}
