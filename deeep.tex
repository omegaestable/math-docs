\documentclass{article}
\usepackage{amsmath, amssymb, amsthm}
\usepackage{enumitem}
\usepackage[colorlinks]{hyperref}

\newtheorem{theorem}{Theorem}
\newtheorem{lemma}[theorem]{Lemma}
\newtheorem{corollary}[theorem]{Corollary}
\newtheorem{proposition}[theorem]{Proposition}
\theoremstyle{definition}
\newtheorem{definition}[theorem]{Definition}
\theoremstyle{remark}
\newtheorem{remark}[theorem]{Remark}
\newtheorem{conjecture}[theorem]{Conjecture}

\newcommand{\Pn}{\mathcal{P}_n}
\newcommand{\PnR}{\mathcal{P}_n^{\mathbb{R}}}

\begin{document}

\title{On a Finite Free Stam Inequality: Progress Report}
\author{}
\date{}
\maketitle

\begin{abstract}
We investigate the analog of Stam's inequality from information theory in the setting of finite free probability. Given degree-$n$ real-rooted monic polynomials $p$ and $q$ with positive variance, denote by $\Phi_n$ the finite free Fisher information and by $\boxplus_n$ the symmetric additive convolution of Marcus, Spielman, and Srivastava. We analyze the following inequality:
\[
\frac{1}{\Phi_n(p \boxplus_n q)} \;\ge\; \frac{1}{\Phi_n(p)} + \frac{1}{\Phi_n(q)}.
\]
Using a Score--Gradient Inequality and a genuine second-order generator, we prove a Hermite semigroup flow bound. We also point out the gap in the proof via fractional flows for general $q$, and outline an alternative route based on a critical point identity.
\end{abstract}

\tableofcontents

%======================================================================
\section{Introduction}\label{sec:intro}
%======================================================================

\subsection{Background and Motivation}

Stam's inequality~\cite{Stam59} in information theory asserts: if $X$ and $Y$ are independent random variables with finite Fisher information $I(X)$ and $I(Y)$, then
\[
\frac{1}{I(X+Y)} \;\ge\; \frac{1}{I(X)} + \frac{1}{I(Y)}.
\]
This fundamental inequality---equivalent to the Shannon--Stam entropy power inequality---captures the essential nature that disorder strictly increases after convolution of independent sources.

Finite free probability, introduced by Marcus, Spielman, and Srivastava~\cite{MSS15}, is a polynomial analog of free probability: random variables are replaced by real-rooted polynomials, and addition is replaced by a deterministic convolution operation $\boxplus_n$.
Within this framework, a natural question arises:

\begin{quote}
\emph{Does the finite free additive convolution also satisfy a Stam inequality?}
\end{quote}

\noindent
This article aims to develop partial results and outline several feasible routes for a proof in the general case.

\subsection{Statement of the Main Result}

Let $\Pn$ denote the space of degree-$n$ monic polynomials with real coefficients, and $\PnR \subset \Pn$ the subset of all real-rooted polynomials.
For $p \in \PnR$, if its roots are distinct $\lambda_1 < \cdots < \lambda_n$, define the \emph{scores}
\[
V_i = \sum_{j \neq i} \frac{1}{\lambda_i - \lambda_j}
\]
and the \emph{finite free Fisher information}
\[
\Phi_n(p) = \sum_{i=1}^n V_i^2.
\]
The \emph{symmetric additive convolution} $p \boxplus_n q$ will be reviewed in Section~\ref{sec:prelim}.

\begin{conjecture}[Finite Free Stam Inequality]\label{conj:stam}
For $p,q \in \PnR$ with positive variance,
\begin{equation}\label{eq:stam}
\frac{1}{\Phi_n(p \boxplus_n q)} \;\ge\; \frac{1}{\Phi_n(p)} + \frac{1}{\Phi_n(q)}.
\end{equation}
\end{conjecture}

Using a Score--Gradient Inequality (Theorem~\ref{thm:sgi}) and a dissipative identity for the Hermite semigroup (Lemma~\ref{lem:dissip}), we prove an exact Hermite flow bound.
The general case is reduced to an explicit comparison problem outlined in Section~\ref{sec:alt-route}.
% In this process, we obtain a critical value formula for $\Phi_n$ via residue calculus (Theorem~\ref{thm:critval}) and explicitly verify the case $n=3$ using this formula (Theorem~\ref{thm:stam3}).

\medskip
\noindent\textbf{Convention.}
When $p$ has repeated roots, we set $\Phi_n(p) := \infty$ (equivalently $1/\Phi_n(p) := 0$).
The results below establish the Hermite flow bound under the assumption that $p$ has distinct roots; Remark~\ref{rem:boundary} discusses how a proof of Conjecture~\ref{conj:stam} for distinct roots can be extended to the boundary of $\PnR$.

%======================================================================
\section{Preliminaries}\label{sec:prelim}
%======================================================================

\subsection{Root Statistics}

Let
\[
p(x) = \prod_{i=1}^n (x - \lambda_i) = \sum_{k=0}^n a_k\,x^{n-k}
\]
with $a_0=1$. Then the mean and variance of the root distribution are
\[
\bar\lambda = \frac{1}{n}\sum_{i=1}^n \lambda_i, \qquad \sigma^2(p) = \frac{1}{n}\sum_{i=1}^n (\lambda_i - \bar\lambda)^2.
\]
\begin{lemma}\label{lem:var-coeff}
\[
\sigma^2(p) = \frac{(n-1)\,a_1^2}{n^2} - \frac{2\,a_2}{n}.
\]
\end{lemma}

\begin{proof}
By Vieta's formulas, $\sum_i \lambda_i = -a_1$, $\sum_{i<j}\lambda_i\lambda_j = a_2$, hence
\[
\sum_{i=1}^n \lambda_i^2 = a_1^2 - 2a_2.
\]
Combining with $\sigma^2 = \frac{1}{n}\sum_i\lambda_i^2 - \bar\lambda^2$ yields the result.
\end{proof}

\subsection{Symmetric Additive Convolution}

Let $A$ and $B$ be real symmetric matrices with characteristic polynomials $p$ and $q$ respectively. Finite free additive convolution is defined via orthogonal group averaging:
\[
(p \boxplus_n q)(x) = \int_{O(n)} \det\!\bigl(xI - (A + QBQ^T)\bigr)\,d\mu_{\mathrm{Haar}}(Q).
\]
According to the MSS theorem~\cite{MSS15}, this operation can be expressed in differential operator form: if $q(x)=\sum_{k=0}^n b_k x^{n-k}$, then
\begin{equation}\label{eq:mss}
(p\boxplus_n q)(x) = T_q p(x), \qquad
T_q = \sum_{k=0}^n \frac{(n-k)!}{n!} b_k \partial_x^k.
\end{equation}
Let $r = p\boxplus_n q$, $r(x)=\sum_k c_k x^{n-k}$; its coefficients satisfy
\begin{equation}\label{eq:conv-coeff}
c_k = \sum_{i+j=k} \frac{(n-i)!\,(n-j)!}{n!\,(n-k)!} a_i b_j.
\end{equation}

We will repeatedly use the following two basic properties:

\begin{theorem}[{\cite{MSS15}}]\label{thm:preserve}
If $p,q \in \PnR$, then $p\boxplus_n q \in \PnR$.
\end{theorem}

\begin{lemma}[Variance Additivity]\label{lem:var-add}
$\sigma^2(p\boxplus_n q) = \sigma^2(p) + \sigma^2(q)$.
\end{lemma}

\begin{proof}
From~\eqref{eq:conv-coeff} we obtain
\[
c_1 = a_1 + b_1 \quad\text{and}\quad c_2 = a_2 + \frac{n-1}{n}\,a_1 b_1 + b_2.
\]
Substituting into Lemma~\ref{lem:var-coeff} and expanding $(a_1+b_1)^2$, the cross term $\frac{2(n-1)a_1 b_1}{n^2}$ cancels with $-\frac{2(n-1)a_1 b_1}{n^2}$, yielding $\sigma^2(p\boxplus_n q) = \sigma^2(p)+\sigma^2(q)$.
\end{proof}

\subsection{Scores and Fisher Information}

\begin{definition}\label{def:score-fisher}
For $p \in \PnR$ with distinct roots $\lambda_1<\cdots<\lambda_n$, define the \emph{score} at root $\lambda_i$ and the \emph{finite free Fisher information} as
\[
V_i = \sum_{j \neq i} \frac{1}{\lambda_i - \lambda_j}, \qquad \Phi_n(p) = \sum_{i=1}^n V_i^2.
\]
The \emph{score--gradient energy} is defined by
\[
\mathcal{S}(p) = \sum_{i<j} \frac{(V_i - V_j)^2}{(\lambda_i - \lambda_j)^2}.
\]
\end{definition}

\begin{lemma}\label{lem:score-deriv}
$V_i = \dfrac{p''(\lambda_i)}{2\,p'(\lambda_i)}$.
\end{lemma}

\begin{proof}
Since $p'(\lambda_i) = \prod_{j \neq i}(\lambda_i-\lambda_j)$, differentiating once more gives
\[
p''(\lambda_i) = 2\sum_{k \neq i} \prod_{\substack{j \neq i \\ j \neq k}} (\lambda_i - \lambda_j) = 2\,p'(\lambda_i)\,V_i.
\]
\end{proof}

\begin{lemma}[Score Identities]\label{lem:score-id}
\begin{enumerate}[label=\textup{(\roman*)},nosep]
\item $\displaystyle\sum_{i=1}^n V_i = 0$. \label{it:score-sum}
\item $\displaystyle\sum_{i=1}^n \lambda_i V_i = \binom{n}{2}$. \label{it:score-root}
\item $\displaystyle\sum_{i=1}^n (\lambda_i-\bar\lambda) V_i = \binom{n}{2}$. \label{it:score-centered}
\item $\displaystyle\Phi_n(p) = \sum_{i<j}\frac{V_i-V_j}{\lambda_i-\lambda_j}$. \label{it:score-gap}
\end{enumerate}
\end{lemma}

\begin{proof}
\ref{it:score-sum}:
$\sum_i V_i = \sum_{i \neq j}(\lambda_i-\lambda_j)^{-1} = 0$ (by antisymmetry).

\ref{it:score-root}:
$\sum_i \lambda_i V_i = \sum_{i \neq j}\frac{\lambda_i}{\lambda_i-\lambda_j}
= \sum_{i<j}\bigl(\frac{\lambda_i}{\lambda_i-\lambda_j} + \frac{\lambda_j}{\lambda_j-\lambda_i}\bigr)
= \sum_{i<j}1 = \binom{n}{2}$.

\ref{it:score-centered}: Follows immediately from \ref{it:score-root} and \ref{it:score-sum}.

\ref{it:score-gap}:
$\sum_i V_i^2 = \sum_{i \neq j}\frac{V_i}{\lambda_i-\lambda_j}
= \sum_{i<j}\frac{V_i-V_j}{\lambda_i-\lambda_j}$.
\end{proof}

\begin{lemma}[Fisher--Variance Inequality]\label{lem:FV}
$\Phi_n(p)\,\sigma^2(p) \ge \dfrac{n(n-1)^2}{4}$.
\end{lemma}

\begin{proof}
Apply the Cauchy--Schwarz inequality to Lemma~\ref{lem:score-id}\ref{it:score-centered}:

\[
\frac{n^2(n-1)^2}{4} \;\le\; \Bigl(\sum_i(\lambda_i-\bar\lambda)^2\Bigr) \Bigl(\sum_i V_i^2\Bigr) = n\,\sigma^2(p)\,\Phi_n(p). \qedhere
\]
\end{proof}

% [Comment: Critical value formula and low-degree cases are omitted here as they are not essential for the general proof and are kept for reference only.]
%
%%======================================================================
%\section{Critical Value Formula for $\Phi_n$}\label{sec:critval}
%%======================================================================
%...
%
%%======================================================================
%\section{Low-degree Cases}\label{sec:small}
%%======================================================================
%...

%======================================================================
\section{Score--Gradient Inequality}\label{sec:sgi}
%======================================================================

The following algebraic inequality is a key input for the general proof.

\begin{theorem}[Score--Gradient Inequality]\label{thm:sgi}
For $n \ge 2$ and $p \in \PnR$ with distinct roots,
\begin{equation}\label{eq:sgi}
\mathcal{S}(p)\,\sigma^2(p) \;\ge\; \frac{n-1}{2}\,\Phi_n(p),
\end{equation}
with equality iff there exists a constant $c$ such that $V_i = c(\lambda_i-\bar\lambda)$.
\end{theorem}

\begin{proof}
Let $T = n\,\sigma^2(p)$, $U = \Phi_n(p)$, $S = \mathcal{S}(p)$.
We need to prove $S\,T \ge \frac{n(n-1)}{2}\,U$.

\medskip\noindent\textbf{Step 1.}
By Lemma~\ref{lem:score-id}\ref{it:score-centered} and the Cauchy--Schwarz inequality,
\begin{equation}\label{eq:cs1}
\frac{n^2(n-1)^2}{4} \;\le\; T\,U.
\end{equation}

\noindent\textbf{Step 2.}
By Lemma~\ref{lem:score-id}\ref{it:score-gap} and the Cauchy--Schwarz inequality,
\begin{equation}\label{eq:cs2}
U^2 \;\le\; S\cdot\binom{n}{2}.
\end{equation}

\noindent\textbf{Step 3.}
Combine:
\[
S\,T \ge \frac{2U^2}{n(n-1)} \cdot T = \frac{2U}{n(n-1)} \cdot T\,U \ge \frac{2U}{n(n-1)} \cdot \frac{n^2(n-1)^2}{4} = \frac{n(n-1)}{2}\,U.
\]

\medskip\noindent\textbf{Equality condition.}
Equality $S\,T = \frac{n(n-1)}{2}\,U$ requires equality in both Step 2 and Step 1.

\smallskip
\noindent\emph{Equality in Step 1:}
Cauchy--Schwarz in~\eqref{eq:cs1} is
$\bigl(\sum_i(\lambda_i{-}\bar\lambda)V_i\bigr)^2 \le \bigl(\sum_i(\lambda_i{-}\bar\lambda)^2\bigr)\bigl(\sum_i V_i^2\bigr)$;
equality holds iff $V_i = c(\lambda_i-\bar\lambda)$ for some constant $c$.

\smallskip
\noindent\emph{Equality in Step 2:}
Cauchy--Schwarz in~\eqref{eq:cs2} is
$\bigl(\sum_{i<j}\frac{V_i{-}V_j}{\lambda_i{-}\lambda_j}\cdot 1\bigr)^2 \le \bigl(\sum_{i<j}\frac{(V_i{-}V_j)^2}{(\lambda_i{-}\lambda_j)^2}\bigr)\bigl(\sum_{i<j}1\bigr)$;
equality holds iff $\frac{V_i-V_j}{\lambda_i-\lambda_j}$ is constant $k$ for all $i<j$.

\smallskip
\noindent\emph{Consistency:}
If $V_i = c(\lambda_i-\bar\lambda)$, then for all $i<j$, $\frac{V_i-V_j}{\lambda_i-\lambda_j}=c$, so equality in Step 1 automatically implies equality in Step 2.
Conversely, if $\frac{V_i-V_j}{\lambda_i-\lambda_j}=k$ for all $i<j$, then $V_i - k\lambda_i$ is independent of $i$; using $\sum_i V_i=0$ (Lemma~\ref{lem:score-id}\ref{it:score-sum}), this constant equals $-k\bar\lambda$, hence $V_i = k(\lambda_i-\bar\lambda)$.
Thus the two equality conditions are equivalent.
\end{proof}

\begin{remark}
The equality condition $V_i = c(\lambda_i-\bar\lambda)$ (up to affine transformation) characterizes the zeros of the Hermite polynomial $H_n$:
evaluating the ODE $H_n''-2xH_n'+2nH_n=0$ at its zeros $x_k$ gives $V_k = x_k$.
For $n=2$, this condition holds for any distinct root configuration.
% For $n=3$ this reduces to $T=0$, consistent with Theorem~\ref{thm:stam3}.
\end{remark}

%======================================================================
\section{Hermite Semigroup Flow}\label{sec:flow}
%======================================================================

\subsection{Hermite Kernel and Flow}

Fix $p \in \PnR$, let $a = \sigma^2(p) > 0$. We define a semigroup $(G_t)_{t\ge 0}$ via its generating polynomial.

\begin{definition}[Hermite Kernel]\label{def:hermite-kernel}
Let
\begin{equation}\label{eq:Kgt}
K_{G_t}(z) = \exp\!\Bigl(-\frac{t}{2(n-1)}z^2\Bigr) \pmod{z^{n+1}}.
\end{equation}
Define $G_t\in\Pn$ as the degree-$n$ polynomial whose normalized coefficients equal those of $K_{G_t}(z)$.
The \emph{Hermite flow} is
\[
p_t = p \boxplus_n G_t.
\]
\end{definition}

\begin{lemma}[Semigroup Property]\label{lem:semigroup}
For any $s,t\ge 0$,
\[
G_s \boxplus_n G_t = G_{s+t}.
\]
\end{lemma}

\begin{proof}
By multiplicativity of $K$ under $\boxplus_n$,
\[
K_{G_s \boxplus_n G_t}(z) = K_{G_s}(z)\,K_{G_t}(z) = \exp\!\Bigl(-\frac{s+t}{2(n-1)}z^2\Bigr) \pmod{z^{n+1}},
\]
which by definition is $K_{G_{s+t}}(z)$.
\end{proof}

\begin{lemma}[Real-rootedness and Variance]\label{lem:gt-real}
For each $t\ge 0$, the polynomial $G_t$ has $n$ distinct real roots, and
\[
\sigma^2(G_t) = t.
\]
\end{lemma}

\begin{proof}
$G_1$ is, up to affine scaling, the Hermite polynomial $H_n$ in probability theory; its roots are all simple and real.
Via scaling, for $t>0$, $G_t(x)=t^{n/2}G_1(x/\sqrt{t})$ still has simple real roots, and $G_0=x^n$.
Since $K_{G_t}(z)=\exp(-\tfrac{t}{2(n-1)}z^2)$, we have $\kappa_1(G_t)=0$, $\kappa_2(G_t)=-\tfrac{t}{2(n-1)}$; substituting into Lemma~\ref{lem:var-coeff} yields $\sigma^2(G_t)=t$.
\end{proof}

\begin{lemma}[Variance of the Flow]\label{lem:var-flow}
For any $t\ge 0$, $\sigma^2(p_t) = a + t$.
\end{lemma}

\begin{proof}
By variance additivity (Lemma~\ref{lem:var-add}) and Lemma~\ref{lem:gt-real}, $\sigma^2(p_t)=\sigma^2(p)+\sigma^2(G_t)=a+t$.
\end{proof}

\subsection{Perturbation Analysis}

\begin{lemma}\label{lem:root-shift}
Let $\lambda_i(t)$ be the roots of $p_t$. Then
\[
\lambda_i(t+h) = \lambda_i(t) + \frac{h}{n-1}\,V_i(t) + O(h^2).
\]
\end{lemma}

\begin{proof}
By the semigroup property, $p_{t+h} = p_t \boxplus_n G_h$ and $\sigma^2(G_h)=h$.
Since $K_{G_h}(z)=\exp(-\tfrac{h}{2(n-1)}z^2)$, the operator $T_{G_h}$ contains no first-order term and only the second-order term at order $h$, thus
\[
T_{G_h} r(x) = r(x) - \frac{h}{2(n-1)}\,r''(x) + O(h^2).
\]
Set $\lambda_i(t+h)=\lambda_i(t)+\delta_i$ and substitute into $T_{G_h} p_t(\lambda_i(t+h))=0$; solving to first order gives
\begin{align*}
\delta_i &= \frac{h}{2(n-1)}\cdot\frac{p_t''(\lambda_i)}{p_t'(\lambda_i)} + O(h^2) \\
&= \frac{h}{n-1}\,V_i(t) + O(h^2)
\end{align*}
(using Lemma~\ref{lem:score-deriv} in the last step).
\end{proof}

\begin{lemma}\label{lem:phi-change}
\[
\Phi_n(p_{t+h}) = \Phi_n(p_t) - \frac{2h}{n-1}\,\mathcal{S}(p_t) + O(h^2).
\]
\end{lemma}

\begin{proof}
Let $\epsilon = h/(n-1)$, temporarily suppress the $t$ dependence.
By Lemma~\ref{lem:root-shift}, the perturbed score is
\[
V_i(h) = \sum_{j \neq i} \frac{1}{(\lambda_i - \lambda_j) + \epsilon(V_i - V_j) + O(h^2)} = V_i - \epsilon \sum_{j \neq i} \frac{V_i - V_j}{(\lambda_i - \lambda_j)^2} + O(h^2).
\]
Squaring and summing:
\[
\Phi_n(p_{t+h}) = \sum_i V_i^2 - 2\epsilon \sum_{i \neq j} \frac{V_i(V_i - V_j)}{(\lambda_i - \lambda_j)^2} + O(h^2).
\]
Pairing $(i,j)$ and $(j,i)$:
\[
\sum_{i \neq j} \frac{V_i(V_i - V_j)}{(\lambda_i - \lambda_j)^2} = \sum_{i<j} \frac{(V_i - V_j)^2}{(\lambda_i - \lambda_j)^2} = \mathcal{S}(p_t).
\]
\end{proof}

\begin{remark}[Consistency of Error Terms]\label{rem:error}
The $O(h^2)$ remainders in Lemma~\ref{lem:root-shift} and Lemma~\ref{lem:phi-change} depend on the minimal root spacing $\delta_{\min}(t)=\min_i(\lambda_{i+1}(t)-\lambda_i(t))$:
the implicit constant grows like $\delta_{\min}(t)^{-m}$ for some $m$ depending on $n$.
Lemma~\ref{lem:flow-real} below shows that for a fixed $b>0$, there exists a uniform lower bound $\delta_{\min}(t)\ge\delta_*>0$ on $t\in[0,b]$, so the error terms are uniformly bounded on $[0,b]$, allowing passage to the derivatives
$\dot\lambda_i = \frac{1}{n-1}V_i$,
$\dot\Phi_n = -\frac{2}{n-1}\mathcal{S}$.
\end{remark}

\begin{lemma}\label{lem:flow-real}
For any $b>0$, the polynomials $p_t$ have $n$ simple real roots for $t\in[0,b]$.
\end{lemma}

\begin{proof}
The coefficients of $p_t$ are smooth in $t$ (Definition~\ref{def:hermite-kernel}), so the roots $\lambda_i(t)$ vary continuously.
Since $p_0=p$ has simple real roots, there exists a maximal interval $[0,T)$ where $p_t$ remains simple real-rooted; continuity ensures $T>0$.

\medskip\noindent\textbf{Lyapunov function.}
On $[0,T)$, define the logarithmic Vandermonde function
\[
W(t) = \sum_{i<j} \log\bigl(\lambda_j(t) - \lambda_i(t)\bigr).
\]
By Lemma~\ref{lem:root-shift}, on $[0,T)$ the roots satisfy $\dot\lambda_i = \frac{1}{n-1}V_i$, hence
\begin{align*}
\dot W(t) &= \sum_{i<j} \frac{\dot\lambda_j-\dot\lambda_i}{\lambda_j-\lambda_i}
= \frac{1}{n-1}\sum_{i<j} \frac{V_j-V_i}{\lambda_j-\lambda_i} \\
&= \frac{1}{n-1}\,\Phi_n(p_t) \;\ge\; 0,
\end{align*}
where the penultimate equality follows from Lemma~\ref{lem:score-id}\ref{it:score-gap}.
Therefore for any $t \in [0,T)$, $W(t) \ge W(0)$, i.e.,
\begin{equation}\label{eq:vand-lower}
\prod_{i<j}\bigl(\lambda_j(t)-\lambda_i(t)\bigr) \;\ge\;
\prod_{i<j}\bigl(\lambda_j(0)-\lambda_i(0)\bigr) =: D_0 > 0.
\end{equation}

\medskip\noindent\textbf{Uniform spacing lower bound.}
Since $\sigma^2(p_t)=a+t\le a+b$ for $t\in[0,b]$, the roots satisfy $\sum_i(\lambda_i-\bar\lambda)^2 = n(a+t) \le n(a+b)$,
hence all roots lie within a distance $R=\sqrt{n(a+b)}$ from $\bar\lambda(t)$.
Any two roots differ by at most $2R$.
For a given adjacent spacing $\delta_k(t) = \lambda_{k+1}(t)-\lambda_k(t)$, we can bound all other factors in~\eqref{eq:vand-lower} above by $2R$:
\[
D_0 \;\le\; \prod_{i<j}\bigl(\lambda_j(t) - \lambda_i(t)\bigr) \;\le\; (2R)^{\binom{n}{2}-1}\,\delta_k(t),
\]
yielding the uniform lower bound
\[
\delta_k(t) \;\ge\; \delta_* := D_0\,(2R)^{1-\binom{n}{2}} > 0
\]
for each $k$ and each $t\in[0,T)$.

\medskip\noindent\textbf{Extendability.}
Since all spacings are uniformly greater than $\delta_*>0$ on $[0,T)$, continuity implies that the roots of $p_T$ remain distinct.
This contradicts the maximality of $T$, unless $T\ge b$.
\end{proof}

\subsection{Dissipation and Integral Identity}

\begin{lemma}[Dissipation]\label{lem:dissip}
\[
\frac{d}{dt}\Phi_n(p_t) = -\frac{2}{n-1}\,\mathcal{S}(p_t).
\]
\end{lemma}

\begin{proof}
From Lemma~\ref{lem:phi-change},
$\frac{\Phi_n(p_{t+h})-\Phi_n(p_t)}{h} = -\frac{2}{n-1}\,\mathcal{S}(p_t) + O(h)$.
By Lemma~\ref{lem:flow-real}, $p_t$ has distinct roots for $t\in[0,b]$, hence the scores $V_i(t)$ and $\mathcal{S}(p_t)$ are continuous in $t$.
Letting $h\to 0$ gives the result.
\end{proof}

\begin{corollary}[Integral Identity]\label{cor:integral}
\begin{equation}\label{eq:integral}
\frac{1}{\Phi_n(p\boxplus_n G_b)} - \frac{1}{\Phi_n(p)}
= \frac{2}{n-1}\int_0^b \frac{\mathcal{S}(p_t)}{\Phi_n(p_t)^2}\,dt.
\end{equation}
\end{corollary}

\begin{proof}
Let $f(t)=1/\Phi_n(p_t)$. By the chain rule and Lemma~\ref{lem:dissip},
\[
f'(t) = \frac{2}{n-1} \cdot \frac{\mathcal{S}(p_t)}{\Phi_n(p_t)^2} \ge 0.
\]
By the fundamental theorem of calculus, $f(b)-f(0) = \int_0^b f'(t)\,dt$.
Substituting $f(0) = 1/\Phi_n(p_0) = 1/\Phi_n(p)$, $f(b) = 1/\Phi_n(p_b) = 1/\Phi_n(p\boxplus_n G_b)$ yields~\eqref{eq:integral}.
\end{proof}

%======================================================================
\section{Obstacles in General Fractional Flows}\label{sec:gap}
%======================================================================

The fractional family based on $K_q(z)^t$ fails to generate an effective second-order generator for general $q$. When $\log K_q$ contains non-zero coefficients of degree $\ge 3$, the first-order expansion of $T_{q_h}$ involves higher-order derivatives, and the root ODE and dissipative identity used in the Hermite flow proof cease to hold. Moreover, for non-integer $t$, $q_t$ and $p\boxplus_n q_t$ may lose real-rootedness, making the Lyapunov argument concerning root simplicity inapplicable.

\begin{remark}[Identified Obstacles]\label{rem:learned}
The following points summarize the main issues encountered in existing flow-based proofs and natural comparison attempts:
\begin{enumerate}[label=\textup{(\roman*)},nosep]
\item The generator of a fractional flow is second-order iff $\log K_q$ is purely quadratic; this uniquely singles out the Hermite kernel.
\item For non-integer $t$, fractional families do not preserve real-rootedness; both $q_t$ and $p\boxplus_n q_t$ may develop non-real roots, rendering the Lyapunov argument invalid.
\item Asymmetric forward/reverse bounds derived from fractional flows do not constitute a universal inequality---even under the anticipated case distinction. This conclusion is drawn from experimental observation and has not been used in any formal argument.
\item Naive monotonicity comparison $\Phi_n(p\boxplus_n q)\le \Phi_n(p\boxplus_n G_b)$ fails numerically; therefore the Hermite bound cannot be directly extended via substitution.
\end{enumerate}
\end{remark}

%======================================================================
\section{Another Route: Critical Point Identities}\label{sec:alt-route}
%======================================================================

We outline an alternative approach that does not rely on flows.
The critical point identity
\[
\Phi_n(p) = -\frac{1}{4}\sum_{p'(\zeta)=0} \frac{p''(\zeta)}{p(\zeta)}
\]
suggests that controlling $\Phi_n(T_q p)$ can be achieved by comparing $(T_q p)''/(T_q p)$ with $p''/p$ at the critical points of $T_q p$.

\subsection{Critical Point Identity}

\begin{lemma}\label{lem:critval}
Let $p \in \PnR$ have distinct roots, and let $\zeta_1,\ldots,\zeta_{n-1}$ be the simple zeros of $p'$. Then
\begin{equation}\label{eq:critval}
\Phi_n(p) = -\frac14\sum_{j=1}^{n-1}\frac{p''(\zeta_j)}{p(\zeta_j)}.
\end{equation}
\end{lemma}

\begin{proof}
By Lemma~\ref{lem:score-deriv}, $\Phi_n = \frac14\sum_{i=1}^n \frac{p''(\lambda_i)^2}{p'(\lambda_i)^2}$.
Consider the meromorphic function
\[
F(x) = \frac{p''(x)^2}{p'(x)\,p(x)}.
\]
\noindent\emph{Poles at roots.}
Since $p$ has a simple zero at $\lambda_i$ and $p'(\lambda_i)\ne 0$,
$\operatorname{Res}{x=\lambda_i}F = p''(\lambda_i)^2/p'(\lambda_i)^2$.
Summing gives $\sum_i\operatorname{Res}{\lambda_i}F = 4\Phi_n$.

\noindent\emph{Poles at critical points.}
At a simple zero $\zeta_j$ of $p'$, by the interlacing property $p(\zeta_j)\ne 0$,
hence $\operatorname{Res}_{x=\zeta_j}F = p''(\zeta_j)/p(\zeta_j)$.

\noindent\emph{Pole at infinity.}
As $x\to\infty$, $F(x)=n(n-1)^2/x^3+O(x^{-4})$, therefore $\operatorname{Res}_\infty F=0$.

\noindent
The sum of all residues on $\mathbb{P}^1$ is zero, i.e., $4\Phi_n + \sum_j p''(\zeta_j)/p(\zeta_j)=0$.
\end{proof}

\subsection{Comparison Scheme}

Let $r = T_q p$. By Lemma~\ref{lem:critval},
\[
\Phi_n(r) = -\frac{1}{4}\sum_{r'(\xi)=0} \frac{r''(\xi)}{r(\xi)}.
\]
Thus, if we can compare $r''/r$ pointwise with $p''/p$ at the critical points of $r$, this translates into a global bound on $\Phi_n(r)$.

\begin{lemma}[Comparison Target]\label{lem:comparison-target}
There exist constants $\alpha(q)$ and $\beta(q)$, depending only on the variance and low-order normalized coefficients of $q$, such that for any real-rooted polynomial $p$ with distinct roots and any critical point $\xi$ of $T_q p$,

\[
\frac{(T_q p)''(\xi)}{(T_q p)(\xi)} \;\le\; \alpha(q) + \beta(q)\,\frac{p''(\xi)}{p(\xi)}.
\]
\end{lemma}

If Lemma~\ref{lem:comparison-target} can be established with $\alpha(q)=0$, $\beta(q) \le 1/\sigma^2(q)$, then summing over critical points yields a bound on $\Phi_n(p\boxplus_n q)$ that implies Conjecture~\ref{conj:stam}. Proving this comparison is the central unsolved problem of this route.

\begin{remark}[Simplification Steps]
The comparison lemma can be approached progressively through the following steps:
\begin{enumerate}[label=\textup{(\roman*)},nosep]
\item Express $r''/r$ as a rational function of the derivatives of $p$:
\[
\frac{r''}{r} = \frac{\sum_k c_k\,p^{(k+2)}}{\sum_k c_k\,p^{(k)}}
\]
where the coefficients $c_k$ are determined by $q$.
\item Using interlacing and sign regularity of $p^{(k)}$ evaluated between roots, obtain inequalities for the ratios $p^{(k+2)}/p^{(k)}$ at the critical points of $r$.
\item Show that the right-hand side can be controlled by an affine function of $p''/p$ depending only on low-order normalized coefficients of $q$.
\end{enumerate}
This route would not rely on any flow and could directly prove Conjecture~\ref{conj:stam}.
\end{remark}

%======================================================================
\section{Hermite Flow Bound}\label{sec:general}
%======================================================================

\begin{theorem}[Hermite Flow Bound]\label{thm:general}
Let $p\in\PnR$, $a=\sigma^2(p)>0$, $b>0$. Then
\begin{equation}\label{eq:hermite-forward}
\frac{1}{\Phi_n(p\boxplus_n G_b)} \;\ge\; \frac{a+b}{a\,\Phi_n(p)}.
\end{equation}
\end{theorem}

\begin{proof}
Let $a=\sigma^2(p)$.

\medskip
\noindent\textbf{Step 1 (Differential inequality).}
Applying the Score--Gradient Inequality (Theorem~\ref{thm:sgi}) to $p_t$ yields
\[
\mathcal{S}(p_t) \ge \frac{(n-1)\,\Phi_n(p_t)}{2\,\sigma^2(p_t)}.
\]
Substituting into Lemma~\ref{lem:dissip}:
\[
\frac{d}{dt}\Phi_n(p_t) \;\le\; -\frac{1}{a+t}\,\Phi_n(p_t).
\]
Integrating $(\log\Phi_n(p_t))' \le -1/(a+t)$ from $0$ to $t$ gives
\begin{equation}\label{eq:ode-bound}
\frac{1}{\Phi_n(p_t)} \;\ge\; \frac{a+t}{a\,\Phi_n(p)}.
\end{equation}

\medskip
\noindent\textbf{Step 2 (Forward bound).}
By Corollary~\ref{cor:integral} and the Score--Gradient Inequality:
\[
\frac{1}{\Phi_n(p \boxplus_n G_b)} - \frac{1}{\Phi_n(p)} \;\ge\; \int_0^b \frac{dt}{(a+t)\,\Phi_n(p_t)}.
\]
Substituting~\eqref{eq:ode-bound}, the factors $(a+t)$ cancel:
\begin{equation}\label{eq:fwd}
\frac{1}{\Phi_n(p\boxplus_n G_b)} \;\ge\; \frac{a+b}{a\,\Phi_n(p)}.
\end{equation}
\qedhere
\end{proof}

\begin{remark}
The Hermite bound~\eqref{eq:hermite-forward} is sharp for Hermite inputs, but does not itself imply Conjecture~\ref{conj:stam}.
The asymmetric reverse bound used in fractional flow proofs has not been successfully justified for general inputs.
\end{remark}

\begin{remark}
Strict inequality typically holds.
Equality in the Hermite bound~\eqref{eq:hermite-forward} occurs iff the roots of $p$ are an affine scaling of the zeros of the Hermite polynomial $H_n$.
For $n=2$, this condition is satisfied for any distinct root configuration.
% For $n=3$ this reduces to $T_1=T_2=0$ (Theorem~\ref{thm:stam3}).
\end{remark}

\begin{remark}[Boundary Behavior]\label{rem:boundary}
Under the convention $1/\Phi_n:=0$ (for repeated roots), if Conjecture~\ref{conj:stam} is proved for the distinct-root case, it can be extended to the boundary of $\PnR$ as follows.
When both $p$ and $q$ have repeated roots, both sides of the inequality are zero and it holds trivially.
When exactly one factor (say $p$) has repeated roots, the inequality reduces to $\Phi_n(q)\ge\Phi_n(p\boxplus_n q)$, a monotonicity statement.
To verify this, approximate $p$ by polynomials $p_\varepsilon\to p$ with distinct roots; the proven inequality gives
$1/\Phi_n(p_\varepsilon\boxplus_n q) \ge 1/\Phi_n(p_\varepsilon)+1/\Phi_n(q) \ge 1/\Phi_n(q)$.
Since convolution is continuous in coefficients, $p_\varepsilon\boxplus_n q\to p\boxplus_n q$.
If $p\boxplus_n q$ has distinct roots, then $\Phi_n(p_\varepsilon\boxplus_n q)\to\Phi_n(p\boxplus_n q)$ and the bound carries over to the limit.
The remaining case---$p\boxplus_n q$ itself has repeated roots while $q$ has distinct roots---cannot occur: in the matrix model
$(p\boxplus_n q)(x)=\int_{O(n)}\det(xI-(A+QBQ^T))\,d\mu(Q)$,
if $B$ has at least two distinct eigenvalues, then for $\mu$-almost every $Q$, the polynomial in the integrand has $n$ distinct real roots;
the averaged polynomial (which is real-rooted by~\cite{MSS15}) can acquire new repeated roots only at the boundaries of its interlacing families, and assuming $\sigma^2(q)>0$ (positive variance) excludes such boundary cases.
\end{remark}

%======================================================================
\begin{thebibliography}{9}

\bibitem{MSS15}
A.~Marcus, D.~A.~Spielman, and N.~Srivastava,
\emph{Interlacing families {II}: Mixed characteristic polynomials
and the {K}adison--{S}inger problem},
Ann.\ of Math.\ \textbf{182} (2015), 327--350.

\bibitem{Stam59}
A.~J.~Stam,
\emph{Some inequalities satisfied by the quantities of information
of {F}isher and {S}hannon},
Inform.\ Control \textbf{2} (1959), 101--112.

\end{thebibliography}

\end{document}
