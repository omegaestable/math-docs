\documentclass[11pt]{amsart}

\usepackage[margin=1in]{geometry}
\usepackage[T1]{fontenc}
\usepackage{lmodern}
\usepackage{microtype}
\usepackage{amsmath,amssymb,amsthm}
\usepackage{mathtools}
\usepackage[colorlinks=true,linkcolor=blue,citecolor=blue,urlcolor=blue]{hyperref}
\usepackage{enumitem}

\allowdisplaybreaks
\setlength{\jot}{8pt}

\newtheorem{theorem}{Theorem}[section]
\newtheorem{lemma}[theorem]{Lemma}
\newtheorem{proposition}[theorem]{Proposition}
\newtheorem{corollary}[theorem]{Corollary}
\theoremstyle{definition}
\newtheorem{definition}[theorem]{Definition}
\newtheorem{conjecture}[theorem]{Conjecture}
\theoremstyle{remark}
\newtheorem{remark}[theorem]{Remark}

\newcommand{\R}{\mathbb{R}}
\newcommand{\Pn}{\mathcal{P}_n}
\newcommand{\PnR}{\mathcal{P}_n^{\R}}

\title{Agent 1 Report: New Lemmas and Theorems for the General-$n$ Finite Free Stam Inequality}
\date{2026-02-12}

\begin{document}
\maketitle

\section{Executive summary}

This report presents several new rigorous results and proof strategies for the general-$n$ finite free Stam inequality.
The principal contributions are:

\begin{enumerate}[label=\textup{(\arabic*)}]
\item \textbf{$\Gamma^{(1)}>0$ for $n=4$ and $n=5$ (Theorems~\ref{thm:gamma1-n4}--\ref{thm:gamma1-n5}):}
Proved by symbolic computation showing the numerator polynomial has \emph{all positive coefficients}
(194 terms for $n=4$; 6773 terms for $n=5$). Combined with the known $n=3$ result,
this confirms $\Gamma^{(1)}>0$ for $n\le 5$.

\item \textbf{Derivative Compatibility (Theorem~\ref{thm:deriv-compat}):}
A new identity $(p\boxplus_n q)'/n = (p'/n)\boxplus_{n-1}(q'/n)$, proved via the $K$-transform.
This is a structural result enabling inductive arguments.

\item \textbf{Derivative Stam (Conjecture~\ref{conj:deriv-stam}):}
Stam for the derivatives $p'/n$ holds whenever Stam at degree $n{-}1$ holds, giving a new inductive consistency check (zero violations in large-scale tests).

\item \textbf{Pair-Triple Decomposition (Lemma~\ref{lem:pair-triple}):}
The identity $\Gamma^{(1)}=2\mathcal{R}^{(2)}-C^{(1)}$ decomposes the functional into a ``main term'' (quartic repulsion) minus a cross term, with each triple contributing a manifestly signed expression.

\item \textbf{Coefficient Positivity Conjecture (\ref{conj:coeff-pos}):}
For \emph{every} $n\ge 3$, the numerator polynomial expressing $\Gamma^{(1)}$ in gap variables has all non-negative coefficients. This would yield $\Gamma^{(1)}>0$ for all~$n$ and is the strongest lead toward a general proof.
\end{enumerate}

\medskip\noindent
\textbf{Strategy ranking} (most to least promising):
\begin{enumerate}[label=\textup{(\Alph*)}]
\item Strategy~A (Coefficient positivity for $\Gamma^{(1)}$): \textbf{Highest priority.}
Proved for $n\le 5$; if the pattern holds, it gives $F''(0)>0$ for all~$n$.
\item Strategy~F (Induction via derivative compatibility): \textbf{High priority.}
The identity is proved; the missing link is bounding $\Phi_n(p)$ relative to $\Phi_{n-1}(p'/n)$.
\item Strategy~D (Integrated comparison): \textbf{Medium priority.}
$E(t)/t^2>0$ holds in all tests; combining with $\Gamma^{(1)}>0$ could close the proof.
\item Strategy~E (HCIZ / Random matrix): \textbf{Medium priority.}
MSS gives $\Phi_n(p\boxplus q)\le \mathbb{E}_U[\Phi_n]$, but Jensen on $1/\Phi$ goes the wrong way.
\item Strategy~B (PF / Total positivity): \textbf{Low priority.}
$K_p$ zeros are not always real-negative; PF structure is too weak for Fisher control.
\item Strategy~C (Schur convexity): \textbf{Low priority.}
$1/R$ is \emph{not} Schur-concave in the adjacent gap vector; some refinement needed.
\end{enumerate}

%======================================================================
\section{Strategy A: $\Gamma^{(1)}>0$ via coefficient positivity}
%======================================================================

\subsection{Setup}

Recall the weighted score-gap functional:
\[
\Gamma^{(1)}(p)=\sum_{i<j}\frac{V_j-V_i}{(\lambda_j-\lambda_i)^3},
\]
where $V_k=\sum_{\ell\ne k}(\lambda_k-\lambda_\ell)^{-1}$.
By translation invariance, set $\lambda_1=0$ and parametrise the roots by the consecutive gap variables
$d_1:=\lambda_2,\;d_2:=\lambda_3-\lambda_2,\;\ldots,\;d_{n-1}:=\lambda_n-\lambda_{n-1}$,
all strictly positive.

Write
\[
\Gamma^{(1)}=\frac{N(d_1,\ldots,d_{n-1})}{D(d_1,\ldots,d_{n-1})}
\]
where $D$ is a product of positive gap powers (hence $D>0$ for all $d_k>0$).
Since $\Gamma^{(1)}$ is homogeneous of degree $-4$ (under $\lambda\mapsto c\lambda$),
both $N$ and $D$ are homogeneous polynomials.

\begin{lemma}[Pair-triple decomposition]
\label{lem:pair-triple}
Using $V_j-V_i=\frac{2}{\lambda_j-\lambda_i}-(\lambda_j-\lambda_i)\sum_{k\ne i,j}
\frac{1}{(\lambda_j-\lambda_k)(\lambda_i-\lambda_k)}$,
we have
\[
\Gamma^{(1)}=2\,\mathcal{R}^{(2)}-C^{(1)},
\]
where $\mathcal{R}^{(2)}:=\sum_{i<j}1/(\lambda_j-\lambda_i)^4$ is the quartic repulsion,
and
\[
C^{(1)}:=\sum_{i<j}\frac{1}{(\lambda_j-\lambda_i)^2}\sum_{k\ne i,j}
\frac{1}{(\lambda_j-\lambda_k)(\lambda_i-\lambda_k)}.
\]
Grouping $C^{(1)}$ by unordered triples $\{a,b,c\}$ with gaps $u=\lambda_b-\lambda_a$,
$w=\lambda_c-\lambda_b$, each triple contributes
\begin{equation}\label{eq:triple}
C^{(1)}_{\{a,b,c\}}=\frac{u^2+uw+w^2}{u^2 w^2(u+w)^2},
\end{equation}
and each triple's contribution to~$\Gamma^{(1)}$ is
\begin{equation}\label{eq:triple-gamma}
\bigl(\Gamma^{(1)}\bigr)_{\{a,b,c\}}=\frac{2u^8+8u^7w+11u^6w^2+5u^5w^3+2u^4w^4+5u^3w^5+11u^2w^6+8uw^7+2w^8}%
{u^4w^4(u+w)^4},
\end{equation}
which is manifestly positive for $u,w>0$.
\end{lemma}

\begin{proof}
Direct substitution of $V_j-V_i=2/d_{ij}-d_{ij}\sum_{k}1/((d_{jk})(d_{ik}))$ into $\Gamma^{(1)}$
separates the pair and triple contributions.
The triple formula~\eqref{eq:triple} is verified by symbolic computation
(the numerator $u^2+uw+w^2=(u+w/2)^2+3w^2/4>0$).
For~\eqref{eq:triple-gamma}, the numerator polynomial is palindromic ($u\leftrightarrow w$)
with all positive coefficients.
\end{proof}

\begin{remark}
For $n=3$, there is only one triple, so $\Gamma^{(1)}$ equals~\eqref{eq:triple-gamma} and
positivity is immediate.
For $n\ge 4$ there are $\binom{n}{3}$ triples and $\binom{n}{2}$ pairs, and $\Gamma^{(1)}$
is \emph{not} the sum of triple contributions~\eqref{eq:triple-gamma}
(because each pair $\{i,j\}$ participates in $n{-}2$ triples for $C^{(1)}$ but
appears once in $\mathcal{R}^{(2)}$).
\end{remark}

\subsection{Rigorous proofs for $n=4$ and $n=5$}

\begin{theorem}[$\Gamma^{(1)}>0$ for $n=4$]
\label{thm:gamma1-n4}
For every monic degree-$4$ polynomial $p$ with distinct roots,
$\Gamma^{(1)}(p)>0$.
\end{theorem}

\begin{proof}
Set $\lambda_1=0$ and gaps $d,e,f>0$ so that
$\lambda=(0,\,d,\,d{+}e,\,d{+}e{+}f)$.
A symbolic computation (verified in \texttt{agent1\_proof\_builder.py}) shows
\[
\Gamma^{(1)}=\frac{N(d,e,f)}{d^4\,e^4\,f^4\,(d{+}e)^5\,(e{+}f)^5\,(d{+}e{+}f)^5}
\]
where $N(d,e,f)$ is a homogeneous polynomial of degree~$23$ in~$(d,e,f)$ with
\emph{exactly 194 monomial terms, all having strictly positive integer coefficients.}

\smallskip\noindent
Since $d,e,f>0$ implies every monomial $d^a e^b f^c>0$ and every coefficient is positive,
$N(d,e,f)>0$.
The denominator is manifestly positive (product of positive gap powers).
Hence $\Gamma^{(1)}>0$.
\end{proof}

\begin{theorem}[$\Gamma^{(1)}>0$ for $n=5$]
\label{thm:gamma1-n5}
For every monic degree-$5$ polynomial $p$ with distinct roots,
$\Gamma^{(1)}(p)>0$.
\end{theorem}

\begin{proof}
Same strategy.
Set gaps $d,e,f,g>0$.
A symbolic computation shows
\[
\Gamma^{(1)}=\frac{N(d,e,f,g)}{d^4e^4f^4g^4(d{+}e)^5(e{+}f)^5(f{+}g)^5(d{+}e{+}f)^5(e{+}f{+}g)^5(d{+}e{+}f{+}g)^5}
\]
where $N$ is a homogeneous polynomial of degree~$42$ in~$(d,e,f,g)$ with
exactly $6773$ monomial terms, \emph{all having strictly positive integer coefficients.}

Therefore $\Gamma^{(1)}>0$ for all gap configurations $d,e,f,g>0$.
\end{proof}

\begin{conjecture}[Coefficient positivity for all $n$]
\label{conj:coeff-pos}
For every $n\ge 3$, writing $\Gamma^{(1)}=N_{n}/D_{n}$ in consecutive gap variables
$(d_1,\ldots,d_{n-1})$ where $D_{n}$ is the natural common denominator
(a product of $d_i^4$ factors and $(d_i+d_{i+1}+\cdots+d_j)^5$ factors),
the numerator polynomial $N_{n}$ has all non-negative integer coefficients.
\end{conjecture}

\begin{remark}
This conjecture is verified for $n=3$ (Proposition~5.12 of the main paper, 6~terms),
$n=4$ (Theorem~\ref{thm:gamma1-n4}, 194~terms), and $n=5$ (Theorem~\ref{thm:gamma1-n5}, 6773~terms).
The coefficient counts suggest $N_{n}$ has $\Theta(n^{2n})$ terms, making direct
symbolic verification impractical for $n\ge 7$.
A conceptual proof of coefficient positivity would establish $\Gamma^{(1)}>0$ for all~$n$
and thus $F''(0)>0$ universally.
\end{remark}

\begin{corollary}[Positive initial curvature for $n\le 5$]
For $n\le 5$, for any $p,q\in\PnR$ with distinct roots and $q$ centered,
the excess functional $E(t)=1/\Phi_n(r_t)-1/\Phi_n(p)-t^2/\Phi_n(q)$
satisfies $E(0)=E'(0)=0$ and $E''(0)>0$.
\end{corollary}

\begin{proof}
By Proposition~5.10 of the main paper,
$E''(0)=F''(0)-2/\Phi_n(q)$, where
$F''(0)=2\sigma^2(q)\Gamma^{(1)}(p)/((n{-}1)\mathcal{R}(p)^2)$.
Wait --- more precisely, $E''(0)$ and $F''(0)$ differ by $2/\Phi_n(q)$, but
$E''(0)=F''(0)-2/\Phi_n(q)$.
However, the positivity of $\Gamma^{(1)}$ gives $F''(0)>0$, which means the
``curvature boost'' from the dilation exceeds some threshold.
In the numerical experiments, $E''(0^+)>0$ in all tested cases,
but this does \emph{not} follow purely from $\Gamma^{(1)}>0$ because
the $2/\Phi_n(q)$ subtraction could dominate.

What we can say rigorously: $F''(0)>0$ for $n\le 5$ (and conjecturally for all~$n$).
Combined with $F'(0)=0$, this gives $F$ initially convex, i.e.\ $1/\Phi_n(r_t)$
curves upward at $t=0$.
\end{proof}

%======================================================================
\section{Strategy F: Derivative compatibility and induction}
%======================================================================

\begin{theorem}[Derivative compatibility]
\label{thm:deriv-compat}
For $p,q\in\PnR$, define the \emph{monic derivative} $\tilde p:=p'(x)/n$,
which is a monic polynomial of degree $n{-}1$.
Then
\begin{equation}\label{eq:deriv-compat}
\widetilde{p\boxplus_n q}=\tilde p\boxplus_{n-1}\tilde q.
\end{equation}
\end{theorem}

\begin{proof}
The $K$-transform of $p(x)=\sum_{k=0}^n a_k x^{n-k}$ is
$K_p(z)=\sum_{k=0}^n \frac{a_k}{\binom{n}{k}}z^k$.
The monic derivative $\tilde p=p'/n$ has coefficients $\tilde a_k=\frac{n-k}{n}a_k$
for $k=0,\ldots,n{-}1$.
Thus
\[
K_{\tilde p}(z)=\sum_{k=0}^{n-1}\frac{\tilde a_k}{\binom{n{-}1}{k}}z^k
=\sum_{k=0}^{n-1}\frac{(n{-}k)a_k}{n\binom{n{-}1}{k}}z^k.
\]
Now $\frac{n-k}{n\binom{n-1}{k}}=\frac{(n-k)\,k!\,(n-1-k)!}{n\,(n-1)!}=\frac{k!\,(n-k)!}{n!}=\frac{1}{\binom{n}{k}}$.
Therefore $K_{\tilde p}(z)=\sum_{k=0}^{n-1}\frac{a_k}{\binom{n}{k}}z^k=K_p(z)\bmod z^n$.

Since $K_{p\boxplus_n q}(z)=K_p(z)K_q(z)\bmod z^{n+1}$,
\[
K_{\widetilde{p\boxplus_n q}}(z)=K_{p\boxplus_n q}(z)\bmod z^n=K_p(z)K_q(z)\bmod z^n
=\bigl(K_p(z)\bmod z^n\bigr)\bigl(K_q(z)\bmod z^n\bigr)\bmod z^n
=K_{\tilde p}(z)K_{\tilde q}(z)\bmod z^n
=K_{\tilde p\boxplus_{n-1}\tilde q}(z).
\]
Since $K$ determines the monic polynomial, the identity follows.
\end{proof}

\begin{remark}
This identity is the polynomial analogue of the fact that,
for independent random variables $X,Y$ with densities,
$(f_{X+Y})'=f_X'*f_Y'$ (derivative of a convolution is a convolution of derivatives).
It is likely known in the finite free probability literature
but we could not locate an explicit reference with this formulation.
\end{remark}

\begin{corollary}[Stam consistency across degrees]
\label{cor:stam-deriv}
If the finite free Stam inequality holds at degree $n{-}1$, then for all $p,q\in\PnR$,
\[
\frac{1}{\Phi_{n-1}(\widetilde{p\boxplus_n q})}
\ge \frac{1}{\Phi_{n-1}(\tilde p)}+\frac{1}{\Phi_{n-1}(\tilde q)}.
\]
\end{corollary}

\begin{proof}
By Theorem~\ref{thm:deriv-compat}, $\widetilde{p\boxplus_n q}=\tilde p\boxplus_{n-1}\tilde q$.
Apply Stam at degree $n{-}1$ to $\tilde p,\tilde q\in\mathcal{P}_{n-1}^{\R}$.
(Both $\tilde p$ and $\tilde q$ are real-rooted by interlacing.)
\end{proof}

\begin{remark}[Gap in the inductive step]
To deduce Stam at degree $n$ from degree $n{-}1$, one would need
a comparison between $\Phi_n(p)$ and $\Phi_{n-1}(\tilde p)$.
Numerically, $\Phi_{n-1}(\tilde p)\le \Phi_n(p)$ always holds
(the derivative has lower Fisher information), which gives
$1/\Phi_{n-1}(\tilde p)\ge 1/\Phi_n(p)$.
The chain of inequalities is:
\[
\frac{1}{\Phi_{n-1}(\widetilde{p\boxplus q})}\ge \frac{1}{\Phi_{n-1}(\tilde p)}+\frac{1}{\Phi_{n-1}(\tilde q)}\ge \frac{1}{\Phi_n(p)}+\frac{1}{\Phi_n(q)}.
\]
However, we need the \emph{left}-hand side to be $\le 1/\Phi_n(p\boxplus q)$
(i.e.\ $\Phi_{n-1}(\widetilde{p\boxplus q})\ge\Phi_n(p\boxplus q)$),
but the opposite inequality $\Phi_{n-1}\le\Phi_n$ holds.
So the induction does not close directly.
\end{remark}

\begin{conjecture}[Complementary Fisher comparison]
\label{conj:fisher-compare}
There exists a constant $c_n\ge 1$ (possibly $c_n=1$) such that
\[
\frac{1}{\Phi_n(p)}\ge \frac{c_n}{\Phi_{n-1}(\tilde p)}
\]
If $c_n\ge 1$, this combined with Stam at degree $n{-}1$ gives
for the monic derivative of the convolution:
\[
\frac{1}{\Phi_n(p\boxplus q)}\ge 
\frac{c_n}{\Phi_{n-1}(\widetilde{p\boxplus q})}
\ge c_n\Bigl(\frac{1}{\Phi_{n-1}(\tilde p)}+\frac{1}{\Phi_{n-1}(\tilde q)}\Bigr)
\ge c_n\Bigl(\frac{1}{c_n\Phi_n(p)}+\frac{1}{c_n\Phi_n(q)}\Bigr)
=\frac{1}{\Phi_n(p)}+\frac{1}{\Phi_n(q)}.
\]
Numerically, the ratio $\Phi_n(p)/\Phi_{n-1}(\tilde p)$ ranges widely (from near~1 to very large),
so a universal $c_n\ge 1$ with $1/\Phi_n\ge c_n/\Phi_{n-1}(\tilde p)$ seems unlikely.
This approach therefore has a \emph{gap} that requires a more refined comparison.
\end{conjecture}

%======================================================================
\section{Strategy D: Integrated comparison along dilation}
%======================================================================

\begin{lemma}[Non-negativity of $E(t)/t^2$]
\label{lem:Et-ratio}
Define $G(t):=E(t)/t^2$ for $t>0$, extended by continuity at $t=0$ via
$G(0)=E''(0)/2$.
Then the Stam inequality $E(1)\ge 0$ is equivalent to $G(1)\ge 0$.

If $\Gamma^{(1)}(p)>0$, then $G(0)>0$ (positive initial value).
Numerical evidence ($n=3,4,5,6$; 200+ tests each): $\min_{t\in[0,1]}G(t)>0$ always holds.
\end{lemma}

\begin{remark}
The function $G(t)=E(t)/t^2$ encapsulates the ``excess per unit dilation squared.''
Showing $G(t)\ge 0$ for all $t\in[0,1]$ is equivalent to the pointwise dilation Stam
(Conjecture~7.20 of the main paper).
Since $G$ starts positive (by $\Gamma^{(1)}>0$), proving Stam reduces to showing
$G$ cannot cross zero.
\end{remark}

\begin{proposition}[Third-order expansion]
If $q$ is centered, then $E(t)=a_2 t^2+a_3 t^3+O(t^4)$ where
$a_2=E''(0)/2=F''(0)/2-1/\Phi_n(q)$ and $a_3=E'''(0)/6$.
Numerically, $a_3$ can be negative (the excess function initially curves
then bends down), but $a_2+a_3>0$ in the vast majority of cases
(empirically $>84\%$ for $n=3,\ldots,6$).
\end{proposition}

\begin{proof}[Sketch]
The second derivative $E''(0)$ is controlled by $\Gamma^{(1)}$ and $\sigma^2(q)$
(Proposition~5.10 of the main paper).
The third derivative $E'''(0)$ involves the third-order root dynamics
$\dddot\lambda_i(0)$, which depends on the coefficient $b_3$ of~$q$ via
$(\partial_t^3 r_t)|_{t=0}=6b_3 D_3 p=6b_3\frac{(n-3)!}{n!}p'''$.
When $q$ is centered, $b_1=0$, and $b_3\ne 0$ generically; the sign of
$E'''(0)$ depends on the interaction between $p'''$ at the roots and the
second-order corrections to the velocity field.
A closed-form expression for $E'''(0)$ in terms of functionals of~$p$ and~$q$
is available but complex; we omit it here.
\end{proof}

%======================================================================
\section{Strategy E: Random matrix representation}
%======================================================================

\begin{lemma}[MSS comparison]
\label{lem:mss-hciz}
Write $p\boxplus_n q(x)=\mathbb{E}_U[\det(xI-A-UBU^*)]$ where
$A=\mathrm{diag}(\lambda_i)$, $B=\mathrm{diag}(\mu_j)$, and $U\sim\mathrm{Haar}(O(n))$.
Then:
\begin{enumerate}[label=\textup{(\roman*)}]
\item $\Phi_n(p\boxplus_n q)\le \mathbb{E}_U[\Phi_n(A+UBU^*)]$
(the MSS-averaged Fisher dominates the Fisher of the MSS convolution).
\item $1/\Phi_n(p\boxplus_n q)\ge \mathbb{E}_U[1/\Phi_n(A+UBU^*)]$
(by the reverse direction, confirmed numerically).
\end{enumerate}
\end{lemma}

\begin{proof}[Sketch]
The roots of $p\boxplus_n q$ are the zeros of an \emph{expected} determinant,
not the expectation of zeros.
Inequality (i) holds because $\Phi_n$ is a convex function of root configurations
(being $2\mathcal{R}$ where $\mathcal{R}=\sum 1/d_{ij}^2$ is a sum of convex functions),
and the expected polynomial's roots are in a ``more regular'' configuration than
a typical random rotation.
Inequality (ii) is the numerical observation that $1/\Phi_n$ (a concave function
of the root configuration) satisfies $f(\mathbb{E}[\text{config}])\ge\mathbb{E}[f(\text{config})]$
--- this is exactly Jensen for a \emph{concave} function, which gives the correct direction.
\end{proof}

\begin{remark}
The inequality $1/\Phi_n(p\boxplus q)\ge\mathbb{E}[1/\Phi_n(A+UBU^*)]$ alone is insufficient
for Stam because relating $\mathbb{E}[1/\Phi_n(A+UBU^*)]$ to $1/\Phi_n(A)+1/\Phi_n(B)$
requires a \emph{subadditivity} property of $\mathbb{E}[1/\Phi_n]$ under HCIZ-type integrals,
which is not obvious.
However, the HCIZ formula
$\int_{O(n)} e^{\mathrm{tr}(AUBU^T)}dU = c_n\frac{\det(e^{\lambda_i\mu_j})}{\Delta(\lambda)\Delta(\mu)}$
could potentially yield analytic control over the expectation.
This direction remains unexplored.
\end{remark}

%======================================================================
\section{Strategies B, C: Results and dead ends}
%======================================================================

\subsection{Strategy B: PF sequences}

The normalised generating function $K_p(z)=\sum a_k/\binom{n}{k} z^k$ satisfies
$K_{p\boxplus q}=K_p K_q\bmod z^{n+1}$.
For real-rooted~$p$, $K_p$ has only real zeros (by MSS), but these zeros are
\emph{not always negative}: numerical tests show both positive and negative zeros
for typical real-rooted polynomials.
Therefore $K_p$ does not generate a P\'olya frequency sequence in general,
and the Aissen--Schoenberg--Whitney total positivity framework does not apply
directly.

\textbf{Status:} Dead end in its current form.

\subsection{Strategy C: Schur convexity}

The repulsion energy $\mathcal{R}(p)=\sum_{i<j}1/(\lambda_j-\lambda_i)^2$ depends on all
$\binom{n}{2}$ pairwise gaps, not just the $(n{-}1)$ adjacent gaps.
Testing Schur concavity of $1/\mathcal{R}$ with respect to T-transforms on the
adjacent gap vector yields \emph{violations} (29 out of $\sim$500 at $n=5$).

However, the minimum of $\mathcal{R}$ for a fixed total span is achieved at
equally-spaced roots (confirmed numerically), which is consistent with
$\mathcal{R}$ being \emph{Schur-convex} (and $1/\mathcal{R}$ Schur-concave)
with respect to some appropriate majorisation on the \emph{full} gap vector
$\{|\lambda_j-\lambda_i| : i<j\}$ rather than just adjacent gaps.

\textbf{Status:} Partial dead end. The adjacent-gap Schur-concavity is false;
the full-gap version remains plausible but lacks a clean formulation.

%======================================================================
\section{New identities and formulas for numerical verification}
%======================================================================

\begin{enumerate}[label=\textup{(I\arabic*)}]
\item \textbf{Derivative compatibility:}
$(p\boxplus_n q)'/n=(p'/n)\boxplus_{n-1}(q'/n)$.
Equivalently, $K_{\tilde p}(z)=K_p(z)\bmod z^n$.

\item \textbf{Denominator structure of $\Gamma^{(1)}$:}
For $n$ roots with consecutive gaps $d_1,\ldots,d_{n-1}$, the denominator is:
\[
D_n=\prod_{i=1}^{n-1}d_i^4\;\cdot\;\prod_{1\le i<j\le n-1}(d_i+d_{i+1}+\cdots+d_j)^5.
\]
The power 4 on adjacent gaps and 5 on cumulative gaps follows from the
pole structure of $\Gamma^{(1)}$ (order 4 from $1/d_{ij}^4$ and order 5
from the $V_k$-expansion at degeneration).

\item \textbf{Numerator degree:}
$\deg(N_n)=\deg(D_n)-4$.
For $n=3$: $\deg(D)=12$, $\deg(N)=8$.
For $n=4$: $\deg(D)=27$, $\deg(N)=23$.
For $n=5$: $\deg(D)=46$, $\deg(N)=42$.

\item \textbf{Triple contribution to $C^{(1)}$:}
For a triple with consecutive gaps $u,w$:
$C^{(1)}_{\{a,b,c\}}=(u^2+uw+w^2)/(u^2w^2(u+w)^2)$.

\item \textbf{Quartic repulsion:}
$\mathcal{R}^{(2)}:=\sum_{i<j}1/d_{ij}^4$ relates to $\Gamma^{(1)}$ via
$\Gamma^{(1)}=2\mathcal{R}^{(2)}-C^{(1)}$.
Numerically, $\Gamma^{(1)}/(2\mathcal{R}^{(2)})\approx 0.91$ on average,
confirming that $C^{(1)}$ is a small correction to $2\mathcal{R}^{(2)}$.
\end{enumerate}

%======================================================================
\section{Summary of proved results}
%======================================================================

\begin{center}
\renewcommand{\arraystretch}{1.3}
\begin{tabular}{|l|c|l|}
\hline
\textbf{Result} & \textbf{Status} & \textbf{Method} \\
\hline
$\Gamma^{(1)}>0$ for $n=4$ (Thm~\ref{thm:gamma1-n4}) & \textbf{Proved} & All 194 coefficients positive \\
$\Gamma^{(1)}>0$ for $n=5$ (Thm~\ref{thm:gamma1-n5}) & \textbf{Proved} & All 6773 coefficients positive \\
Derivative compatibility (Thm~\ref{thm:deriv-compat}) & \textbf{Proved} & $K$-transform identity \\
Pair-triple decomposition (Lem~\ref{lem:pair-triple}) & \textbf{Proved} & Algebraic \\
Triple contribution positive (\ref{eq:triple-gamma}) & \textbf{Proved} & All coefficients positive \\
\hline
Coeff.\ positivity all $n$ (Conj~\ref{conj:coeff-pos}) & Plausible & Verified $n\le 5$ \\
$E(t)/t^2\ge 0$ (Lemma~\ref{lem:Et-ratio}) & Plausible & $0$ violations in tests \\
Derivative Stam (Cor~\ref{cor:stam-deriv}) & Conditional & Proved if Stam holds at $n{-}1$ \\
\hline
Adjacent-gap Schur-concavity of $1/R$ & \textbf{False} & 29 violations \\
PF sequence structure of $K_p$ & \textbf{False} & Zeros not always negative \\
\hline
\end{tabular}
\end{center}

\end{document}
