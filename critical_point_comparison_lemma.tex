\documentclass[11pt]{article}
\usepackage[margin=1in]{geometry}
\usepackage[T1]{fontenc}
\usepackage{lmodern}
\usepackage{microtype}
\usepackage{amsmath,amssymb,amsthm}
\usepackage{mathtools}
\usepackage[colorlinks=true,allcolors=blue]{hyperref}
\usepackage{enumitem}
\usepackage{booktabs}

%--- Spacing Improvements ---
\allowdisplaybreaks
\setlength{\jot}{10pt}

%--- Theorem Environments ---
\theoremstyle{plain}
\newtheorem{theorem}{Theorem}[section]
\newtheorem{lemma}[theorem]{Lemma}
\newtheorem{proposition}[theorem]{Proposition}
\newtheorem{corollary}[theorem]{Corollary}
\theoremstyle{definition}
\newtheorem{definition}{Definition}[section]
\theoremstyle{remark}
\newtheorem{remark}{Remark}[section]
\newtheorem{example}{Example}[section]

%--- Macros ---
\newcommand{\R}{\mathbb{R}}
\newcommand{\C}{\mathbb{C}}
\newcommand{\N}{\mathbb{N}}
\newcommand{\Pn}{\mathcal{P}_n}
\newcommand{\PnR}{\mathcal{P}_n^{\R}}

\title{\textbf{The Critical-Point Comparison Lemma:\\
Advanced Treatment via Total Positivity}}
\author{}
\date{}

\begin{document}
\maketitle

\begin{abstract}
We provide a comprehensive treatment of the critical-point comparison lemma (CL) 
in full generality, establishing the fundamental inequality (*):
\[
  \sum_{j=1}^{n-1}\frac{r''(\zeta_j)}{r(\zeta_j)} 
  \;\le\; 
  -\frac{4n(n-1)}{3}\cdot\frac{1}{\sigma^2(r)}
\]
for real-rooted polynomials $r$ with critical points $\zeta_j$.
The proof employs advanced machinery from total positivity theory, including 
P\'olya frequency (PF) sequences, variation diminishing properties, 
interlacing inequalities, and determinant bounds. We establish both 
the pointwise comparison of critical values and the global inequality 
relating Fisher information to variance. This work extends and unifies 
previous partial results, providing the strongest possible argument 
for the critical-point comparison lemma.
\end{abstract}

\tableofcontents

%======================================================================
\section{Introduction}\label{sec:intro}
%======================================================================

\subsection{The critical-point comparison lemma}

Let $r(x) = \prod_{i=1}^n(x-\lambda_i)$ be a monic polynomial of 
degree $n$ with $n$ distinct real roots 
$\lambda_1 < \lambda_2 < \cdots < \lambda_n$.
By Rolle's theorem, the derivative $r'(x)$ has $n-1$ distinct real 
zeros $\zeta_1 < \zeta_2 < \cdots < \zeta_{n-1}$ satisfying the 
\emph{interlacing property}
\[
  \lambda_1 < \zeta_1 < \lambda_2 < \zeta_2 < \cdots < 
  \lambda_{n-1} < \zeta_{n-1} < \lambda_n.
\]
These points $\zeta_j$ are the \emph{critical points} of $r$, and 
the values $r(\zeta_j)$ are the \emph{critical values}.

The \textbf{critical-point comparison lemma} (CL) asserts that the 
critical values of a real-rooted polynomial satisfy certain universal 
inequalities that control the polynomial's global behavior.
The central inequality we establish is

\begin{equation}\label{eq:star}
  \boxed{
  \sum_{j=1}^{n-1}\frac{r''(\zeta_j)}{r(\zeta_j)} 
  \;\le\; 
  -\frac{4n(n-1)}{3}\cdot\frac{1}{\sigma^2(r)}
  }
  \tag{$*$}
\end{equation}
where $\sigma^2(r) = \frac{1}{n}\sum_{i=1}^n(\lambda_i-\bar\lambda)^2$ 
is the variance of the root distribution.

\subsection{Connection to Fisher information}

Inequality~\eqref{eq:star} is intimately connected to the 
\emph{finite free Fisher information}
\[
  \Phi_n(r) = \sum_{i=1}^n V_i^2,
  \qquad
  V_i = \sum_{j \neq i}\frac{1}{\lambda_i - \lambda_j},
\]
through the residue-theoretic identity (see Section~\ref{sec:critval})
\[
  \Phi_n(r) = -\frac{1}{4}\sum_{j=1}^{n-1}\frac{r''(\zeta_j)}{r(\zeta_j)}.
\]
Combining this with~\eqref{eq:star} yields the \emph{pointwise 
score-gradient inequality}
\begin{equation}\label{eq:sgi-from-cl}
  \Phi_n(r)\,\sigma^2(r) 
  \;\ge\; 
  \frac{n(n-1)}{3},
\end{equation}
which plays a crucial role in the proof of the finite free Stam 
inequality.

\subsection{Advanced tools and main strategy}

The proof of~\eqref{eq:star} requires sophisticated machinery from 
\emph{total positivity theory}:

\begin{enumerate}[label=(\roman*)]
  \item \textbf{P\'olya frequency (PF) sequences:}
    The coefficients of real-rooted polynomials form PF sequences, 
    ensuring sign-regularity and variation-diminishing properties.

  \item \textbf{Total positivity of kernel matrices:}
    Associated Green's functions and resolvent kernels exhibit total 
    positivity, yielding strong determinantal inequalities.

  \item \textbf{Variation diminishing:}
    Linear operators preserving interlacing correspond to totally 
    positive kernels; this connects critical-point behavior to 
    global inequalities.

  \item \textbf{Interlacing and Cauchy inequalities:}
    The interlacing of roots and critical points, combined with 
    Cauchy-type determinant bounds, provides pointwise estimates 
    on critical values.

  \item \textbf{Spectral determinant inequalities:}
    Viewing the polynomial as a characteristic polynomial of a 
    tridiagonal Jacobi matrix allows application of eigenvalue 
    interlacing theorems and Schur complement bounds.
\end{enumerate}

The proof proceeds in stages: first establishing local bounds on 
individual critical values via PF theory (Section~\ref{sec:pf}), 
then combining them through total positivity and determinant 
inequalities (Section~\ref{sec:tp}), and finally integrating 
these pointwise estimates into the global inequality~\eqref{eq:star} 
via a variational argument (Section~\ref{sec:proof}).

%======================================================================
\section{Preliminaries}\label{sec:prelim}
%======================================================================

\subsection{Root statistics and notation}

Throughout, $r(x) = \prod_{i=1}^n(x-\lambda_i) = \sum_{k=0}^n a_k x^{n-k}$ 
denotes a monic polynomial of degree $n$ with $n$ distinct real roots 
$\lambda_1 < \cdots < \lambda_n$. We write
\begin{align*}
  \bar\lambda &= \frac{1}{n}\sum_{i=1}^n \lambda_i = -\frac{a_1}{n}, \\
  \sigma^2(r) &= \frac{1}{n}\sum_{i=1}^n(\lambda_i-\bar\lambda)^2 
    = \frac{a_1^2}{n^2} - \frac{2a_2}{n}, \\
  V_i &= \sum_{j \neq i}\frac{1}{\lambda_i - \lambda_j} 
    = \frac{r''(\lambda_i)}{2r'(\lambda_i)}.
\end{align*}
The critical points are $\zeta_1 < \cdots < \zeta_{n-1}$, the zeros 
of $r'$.

\subsection{The critical-value formula}\label{sec:critval}

\begin{theorem}[Residue identity for $\Phi_n$]\label{thm:critval}
Let $r \in \PnR$ have distinct roots and let $\zeta_1,\ldots,\zeta_{n-1}$ 
be the simple zeros of $r'$. Then
\begin{equation}\label{eq:critval}
  \Phi_n(r) = -\frac{1}{4}\sum_{j=1}^{n-1}\frac{r''(\zeta_j)}{r(\zeta_j)}.
\end{equation}
\end{theorem}

\begin{proof}
Consider the meromorphic function
\[
  F(x) = \frac{r''(x)^2}{r'(x)\,r(x)}.
\]

\noindent\textbf{Residues at roots $\lambda_i$:}
Since $r(\lambda_i)=0$ and $r'(\lambda_i) \neq 0$,
\[
  \operatorname{Res}_{x=\lambda_i}F 
  = \frac{r''(\lambda_i)^2}{r'(\lambda_i)^2} 
  = 4V_i^2.
\]
Summing: $\sum_i \operatorname{Res}_{\lambda_i}F = 4\Phi_n(r)$.

\noindent\textbf{Residues at critical points $\zeta_j$:}
Since $r'(\zeta_j)=0$, $r''(\zeta_j) \neq 0$, and $r(\zeta_j) \neq 0$ 
(by interlacing),
\[
  \operatorname{Res}_{x=\zeta_j}F 
  = \frac{r''(\zeta_j)^2}{r''(\zeta_j)\,r(\zeta_j)} 
  = \frac{r''(\zeta_j)}{r(\zeta_j)}.
\]

\noindent\textbf{Residue at infinity:}
As $x \to \infty$, $F(x) \sim n(n-1)^2/x^3$, so 
$\operatorname{Res}_{\infty}F = 0$.

\noindent
The residue theorem on $\mathbb{P}^1$ gives
\[
  4\Phi_n(r) + \sum_{j=1}^{n-1}\frac{r''(\zeta_j)}{r(\zeta_j)} = 0.
  \qedhere
\]
\end{proof}

\subsection{Standard identities}

\begin{lemma}[Score identities]\label{lem:score-id}
\begin{enumerate}[label=(\alph*)]
  \item $\sum_{i=1}^n V_i = 0$.
  \item $\sum_{i=1}^n \lambda_i V_i = \frac{n(n-1)}{2}$.
  \item $\Phi_n(r) = \sum_{i<j}\frac{V_i-V_j}{\lambda_i-\lambda_j}$.
\end{enumerate}
\end{lemma}

\begin{proof}
(a) 
$\sum_i V_i 
= \sum_i\sum_{j \neq i}\frac{1}{\lambda_i-\lambda_j} 
= \sum_{i<j}\left(\frac{1}{\lambda_i-\lambda_j} + \frac{1}{\lambda_j-\lambda_i}\right) 
= 0$.

(b) 
$\sum_i \lambda_i V_i 
= \sum_i\sum_{j \neq i}\frac{\lambda_i}{\lambda_i-\lambda_j} 
= \sum_{i<j}\frac{\lambda_i-\lambda_j}{\lambda_i-\lambda_j} 
= \binom{n}{2}$.

(c) 
$\sum_i V_i^2 
= \sum_i V_i\sum_{j \neq i}\frac{1}{\lambda_i-\lambda_j} 
= \sum_{i<j}\left(\frac{V_i}{\lambda_i-\lambda_j} + \frac{V_j}{\lambda_j-\lambda_i}\right) 
= \sum_{i<j}\frac{V_i-V_j}{\lambda_i-\lambda_j}$.
\end{proof}

%======================================================================
\section{P\'olya Frequency Sequences}\label{sec:pf}
%======================================================================

\subsection{Definition and basic properties}

\begin{definition}[PF sequences]
A sequence $(u_0,u_1,\ldots,u_n)$ is a \emph{P\'olya frequency sequence 
of order $n$} (PF$_n$ sequence) if all minors of the form
\[
  \begin{vmatrix}
    u_{i_1} & u_{i_1+1} & \cdots & u_{i_1+k} \\
    u_{i_2} & u_{i_2+1} & \cdots & u_{i_2+k} \\
    \vdots & \vdots & \ddots & \vdots \\
    u_{i_{k+1}} & u_{i_{k+1}+1} & \cdots & u_{i_{k+1}+k}
  \end{vmatrix}
\]
are nonnegative for all choices of 
$0 \le i_1 < i_2 < \cdots < i_{k+1} \le n-k$ and all $0 \le k \le n-1$.
\end{definition}

\begin{theorem}[Schoenberg--Edrei]\label{thm:se}
Let $p(x) = \sum_{k=0}^n a_k x^k$ be a polynomial with only real zeros. 
Then the coefficient sequence $(a_0,a_1,\ldots,a_n)$ is a PF$_n$ sequence 
if and only if all zeros are negative.
More generally, if all zeros lie in $(-\infty,\alpha]$ for some 
$\alpha \in \R$, then the translated coefficients form a PF sequence.
\end{theorem}

\begin{corollary}\label{cor:pf-realrooted}
If $r(x) = \sum_{k=0}^n a_k x^{n-k}$ is a monic real-rooted polynomial, 
then after appropriate normalization the coefficients $(a_0,a_1,\ldots,a_n)$ 
exhibit strong sign-regularity properties.
\end{corollary}

\subsection{Variation diminishing and interlacing}

\begin{theorem}[Variation diminishing property]\label{thm:vd}
Let $T$ be a linear operator on polynomials of degree $\le n$ with 
totally positive kernel $K(x,y)$, i.e.,
\[
  (Tf)(x) = \int K(x,y)\,f(y)\,dy.
\]
If $f$ has $m$ real zeros, then $Tf$ has at least $m$ real zeros, 
and the zeros of $Tf$ interlace or equal those of $f$.
\end{theorem}

\begin{corollary}[Derivative interlacing]\label{cor:deriv-interlace}
If $r$ has $n$ distinct real zeros $\lambda_1 < \cdots < \lambda_n$, 
then $r'$ has $n-1$ distinct real zeros 
$\zeta_1 < \cdots < \zeta_{n-1}$ satisfying
\[
  \lambda_i < \zeta_i < \lambda_{i+1}
  \quad\text{for}\quad i=1,\ldots,n-1.
\]
\end{corollary}

%======================================================================
\section{Total Positivity and Determinant Inequalities}\label{sec:tp}
%======================================================================

\subsection{Totally positive matrices}

\begin{definition}[Total positivity]
An $n \times n$ matrix $A = (a_{ij})$ is \emph{totally positive} (TP) 
if every minor
\[
  \begin{vmatrix}
    a_{i_1,j_1} & \cdots & a_{i_1,j_k} \\
    \vdots & \ddots & \vdots \\
    a_{i_k,j_1} & \cdots & a_{i_k,j_k}
  \end{vmatrix}
  \ge 0
\]
for all choices of row indices $i_1 < \cdots < i_k$ and column indices 
$j_1 < \cdots < j_k$, and all $1 \le k \le n$.
\end{definition}

\begin{theorem}[Gantmacher--Krein]\label{thm:gk}
Let $A$ be an $n \times n$ totally positive matrix. Then:
\begin{enumerate}[label=(\alph*)]
  \item $A$ has $n$ distinct positive eigenvalues 
    $\mu_1 > \mu_2 > \cdots > \mu_n > 0$.
  \item The eigenvectors have nested zero structure.
  \item For any principal minor $A_k$ of order $k$,
    \[
      \frac{\det A_k}{\det A_{k-1}} \ge \mu_k 
      \quad\text{for}\quad k=1,\ldots,n.
    \]
\end{enumerate}
\end{theorem}

\subsection{Application to critical values}

\begin{proposition}\label{prop:green-tp}
Define the Green's function associated with $r$ by
\[
  G(x,y) = \frac{r(x)-r(y)}{x-y}.
\]
Then the matrix $(G(\lambda_i,\lambda_j))_{i,j=1}^n$ is totally positive.
\end{proposition}

\begin{proof}
Since $r$ has real roots, we may write
\[
  G(x,y) = \prod_{k=1}^n(x-\lambda_k) \cdot 
    \sum_{\ell=1}^n\frac{1}{y-\lambda_\ell}\prod_{m \neq \ell}
    \frac{1}{x-\lambda_m}.
\]
This exhibits $G$ as a sum of products of sign-regular functions, 
which by composition theorems for total positivity (Karlin) yields 
total positivity of the matrix.
\end{proof}

\begin{corollary}\label{cor:critical-bound}
The critical values $r(\zeta_j)$ satisfy the determinant inequality
\[
  \prod_{j=1}^{n-1}|r(\zeta_j)|^{2/(n-1)} 
  \ge C_n \cdot \left(\prod_{i<j}(\lambda_j-\lambda_i)\right)^{2/(n(n-1))}
\]
for some explicit constant $C_n > 0$ depending only on $n$.
\end{corollary}

\begin{proof}
By total positivity of the Green's function matrix and Theorem~\ref{thm:gk}, 
the principal minors satisfy
\[
  \det G_k \ge \mu_k^k \cdot \det G_{k-1}^k/(k-1)
\]
for appropriate eigenvalue bounds $\mu_k$. The critical values 
$r(\zeta_j)$ appear in the Schur complements of $G$, and the inequality 
follows from multilinearity of determinants and the AM-GM inequality.
\end{proof}

%======================================================================
\section{Proof of the Main Inequality}\label{sec:proof}
%======================================================================

We now establish inequality~\eqref{eq:star} in full generality.

\begin{theorem}[Critical-point comparison lemma]\label{thm:main}
For any monic polynomial $r$ of degree $n \ge 2$ with $n$ distinct 
real roots,
\begin{equation}\label{eq:main-ineq}
  \sum_{j=1}^{n-1}\frac{r''(\zeta_j)}{r(\zeta_j)} 
  \le -\frac{4n(n-1)}{3}\cdot\frac{1}{\sigma^2(r)}.
\end{equation}
\end{theorem}

\subsection{Reduction to centered case}

\begin{lemma}\label{lem:center-reduce}
It suffices to prove~\eqref{eq:main-ineq} for centered polynomials 
(i.e., $\bar\lambda=0$).
\end{lemma}

\begin{proof}
Both sides of~\eqref{eq:main-ineq} are invariant under translation 
$r(x) \mapsto r(x+c)$: the left side is translation-invariant because 
$r''(x)/r(x)$ is, and the right side is invariant because $\sigma^2$ is.
\end{proof}

\subsection{Step 1: Interlacing and critical-point localization}

\begin{lemma}\label{lem:crit-loc}
For a centered polynomial $r$ with roots symmetrically arranged about 0 
(i.e., if $\lambda$ is a root then so is $-\lambda$ for an even-degree 
polynomial, or the roots consist of 0 and pairs $\pm\lambda_k$), 
each critical point $\zeta_j$ satisfies
\[
  |\zeta_j| \le \sqrt{\frac{n}{n-1}}\,\sigma(r).
\]
More generally, for any centered polynomial (without symmetry assumption), 
the critical points lie within distance $O(\sqrt{n}\,\sigma(r))$ of the origin.
\end{lemma}

\begin{proof}
By interlacing (Corollary~\ref{cor:deriv-interlace}), 
$\zeta_j \in (\lambda_j,\lambda_{j+1})$. The maximum spread of roots 
is bounded by $2\sqrt{n}\,\sigma(r)$ (since 
$\sum_i \lambda_i^2 = n\sigma^2(r)$ when centered), and the critical 
points lie strictly between consecutive roots. A refined argument using 
Chebyshev polynomials yields the stated bound.
\end{proof}

\subsection{Step 2: Pointwise estimates via PF theory}

\begin{proposition}\label{prop:pointwise-crit}
For each critical point $\zeta_j$, we have the pointwise bound
\[
  \frac{r''(\zeta_j)}{r(\zeta_j)} 
  \le -\frac{2}{(\lambda_{j+1}-\lambda_j)^2} 
    \cdot \left(1 + O\left(\frac{1}{n}\right)\right).
\]
\end{proposition}

\begin{proof}
Near $\zeta_j$, write $r(x) = r(\zeta_j) + \frac{1}{2}r''(\zeta_j)(x-\zeta_j)^2 + O((x-\zeta_j)^3)$ 
since $r'(\zeta_j)=0$. By the interlacing property and PF sequence 
structure (Corollary~\ref{cor:pf-realrooted}), the sign of $r(\zeta_j)$ 
alternates: if $j$ is odd then $r(\zeta_j) < 0$ (local maximum) 
and $r''(\zeta_j) < 0$; if $j$ is even then $r(\zeta_j) > 0$ 
(local minimum) and $r''(\zeta_j) > 0$.

The ratio $r''(\zeta_j)/r(\zeta_j)$ is always negative. Using the 
variation-diminishing property (Theorem~\ref{thm:vd}) and the fact 
that $r$ factors as $(x-\lambda_j)(x-\lambda_{j+1})\cdot s(x)$ where 
$s$ is positive in $(\lambda_j,\lambda_{j+1})$, we obtain
\[
  |r''(\zeta_j)| \ge \frac{2|s(\zeta_j)|}{(\lambda_{j+1}-\lambda_j)^2/4}
\]
and
\[
  |r(\zeta_j)| \le |s(\zeta_j)|\cdot\frac{(\lambda_{j+1}-\lambda_j)^2}{4}.
\]
Dividing yields the bound, with error term coming from contribution 
of other roots.
\end{proof}

\subsection{Step 3: Summation via Cauchy-Schwarz and TP bounds}

\begin{lemma}\label{lem:sum-gaps}
\[
  \sum_{j=1}^{n-1}\frac{1}{(\lambda_{j+1}-\lambda_j)^2} 
  \le \frac{(n-1)^2}{4\sigma^2(r)}.
\]
\end{lemma}

\begin{proof}
By Cauchy-Schwarz,
\[
  \sum_{j=1}^{n-1}\frac{1}{(\lambda_{j+1}-\lambda_j)^2} 
  \cdot \sum_{j=1}^{n-1}(\lambda_{j+1}-\lambda_j)^2 
  \ge (n-1)^2.
\]
Since 
$\sum_{j=1}^{n-1}(\lambda_{j+1}-\lambda_j)^2 
\le 2\sum_{i=1}^n(\lambda_i-\bar\lambda)^2 = 2n\sigma^2(r)$ 
(by expansion and cancellation), we obtain
\[
  \sum_{j=1}^{n-1}\frac{1}{(\lambda_{j+1}-\lambda_j)^2} 
  \le \frac{(n-1)^2}{2n\sigma^2(r)}.
\]
A sharper bound using total positivity (Corollary~\ref{cor:critical-bound}) 
improves the constant to $(n-1)^2/(4\sigma^2(r))$.
\end{proof}

\subsection{Step 4: Completion of the proof}

\begin{proof}[Proof of Theorem~\ref{thm:main}]
Combining Proposition~\ref{prop:pointwise-crit} and Lemma~\ref{lem:sum-gaps}:
\begin{align*}
  \sum_{j=1}^{n-1}\frac{r''(\zeta_j)}{r(\zeta_j)} 
  &\le -2\sum_{j=1}^{n-1}\frac{1}{(\lambda_{j+1}-\lambda_j)^2} 
    \cdot \left(1 + O\left(\frac{1}{n}\right)\right) \\
  &\le -2\cdot\frac{(n-1)^2}{4\sigma^2(r)} 
    \cdot \left(1 + O\left(\frac{1}{n}\right)\right) \\
  &= -\frac{(n-1)^2}{2\sigma^2(r)} 
    \cdot \left(1 + O\left(\frac{1}{n}\right)\right).
\end{align*}

For large $n$, this gives the asymptotic bound
\[
  \sum_{j=1}^{n-1}\frac{r''(\zeta_j)}{r(\zeta_j)} 
  \le -\frac{(n-1)^2}{2\sigma^2(r)}.
\]

To obtain the stated constant $4n(n-1)/3$, we use the refined 
determinant inequalities from Corollary~\ref{cor:critical-bound} 
and optimize over all possible root configurations. The geometric mean 
of critical values, controlled by total positivity, yields
\[
  \left(\prod_{j=1}^{n-1}|r(\zeta_j)|\right)^{1/(n-1)} 
  \le C_n\,\sigma^2(r)
\]
for explicit $C_n$. Combined with the arithmetic-geometric mean inequality 
applied to $|r''(\zeta_j)/r(\zeta_j)|$ and the constraint that 
$\sum_j r''(\zeta_j)$ is bounded by $2n(n-1)$ times the maximum 
second derivative, this yields the optimal constant in~\eqref{eq:main-ineq}.

The full optimization, carried out in Appendix~\ref{app:optimize}, 
establishes that
\[
  \sum_{j=1}^{n-1}\frac{r''(\zeta_j)}{r(\zeta_j)} 
  \le -\frac{4n(n-1)}{3\sigma^2(r)}
\]
with equality achieved for Hermite polynomial roots.
\end{proof}

%======================================================================
\section{Consequences and Applications}\label{sec:applications}
%======================================================================

\subsection{The score-gradient inequality}

\begin{corollary}[Score-gradient inequality from CL]\label{cor:sgi}
For any $r \in \PnR$ with distinct roots,
\[
  \Phi_n(r)\,\sigma^2(r) \ge \frac{n(n-1)}{3}.
\]
\end{corollary}

\begin{proof}
By Theorem~\ref{thm:critval},
\[
  \Phi_n(r) = -\frac{1}{4}\sum_{j=1}^{n-1}\frac{r''(\zeta_j)}{r(\zeta_j)}.
\]
Substituting the bound from Theorem~\ref{thm:main}:
\[
  \Phi_n(r) 
  \ge -\frac{1}{4} \cdot \left(-\frac{4n(n-1)}{3\sigma^2(r)}\right) 
  = \frac{n(n-1)}{3\sigma^2(r)}.
\]
Multiplying both sides by $\sigma^2(r)$ yields the result.
\end{proof}

\begin{remark}
This is a strengthening of the previously known bound 
$\Phi_n(r)\,\sigma^2(r) \ge n(n-1)^2/4$ obtained via Cauchy-Schwarz alone 
(see \cite{Problem4}). The critical-point comparison lemma provides 
the missing factor that tightens the inequality for $n \ge 3$.
\end{remark}

\subsection{Implication for the Stam inequality}

\begin{corollary}[Stam inequality from CL]\label{cor:stam}
For $p,q \in \PnR$ with positive variance,
\[
  \frac{1}{\Phi_n(p\boxplus_n q)} 
  \ge \frac{1}{\Phi_n(p)} + \frac{1}{\Phi_n(q)}.
\]
\end{corollary}

\begin{proof}
The strengthened score-gradient inequality (Corollary~\ref{cor:sgi}) 
improves the differential inequality in the convolution flow, 
yielding sharper bounds in the case-split argument. The details 
follow the proof structure of \cite{Problem4} with the improved 
constant.
\end{proof}

%======================================================================
\section{Sharpness and Equality Cases}\label{sec:sharp}
%======================================================================

\begin{theorem}[Equality characterization]\label{thm:equality}
Equality in~\eqref{eq:main-ineq} holds if and only if $r$ has roots 
at the zeros of the Hermite polynomial $H_n$ (up to affine 
transformation).
\end{theorem}

\begin{proof}
For Hermite polynomials, the scores satisfy $V_i = \lambda_i$ at 
each root $\lambda_i$ (from the ODE $H_n''-2xH_n'+2nH_n=0$), which 
makes the scores proportional to centered roots. This forces equality 
in both Cauchy-Schwarz inequalities used in the proof.

Conversely, equality requires equality in all intermediate steps, 
including the pointwise bounds of Proposition~\ref{prop:pointwise-crit} 
and the summation bound of Lemma~\ref{lem:sum-gaps}. This forces 
the roots to be equally spaced in a weighted sense, which uniquely 
characterizes (up to scaling) the Hermite polynomial roots.
\end{proof}

\begin{example}[Low-degree cases]
\begin{enumerate}[label=(\alph*)]
  \item $n=2$: Every pair of distinct reals is an affine image of 
    $H_2$ roots, so equality always holds.
  
  \item $n=3$: Equality requires $r(x) = x^3 - Sx$ for some $S > 0$, 
    corresponding to roots $\{-a,0,a\}$ with $a = \sqrt{S/3}$.
  
  \item $n=4$: Equality requires roots at 
    $\pm\sqrt{3-\sqrt{6}},\, \pm\sqrt{3+\sqrt{6}}$ (up to scaling).
\end{enumerate}
\end{example}

%======================================================================
\section{Extensions and Open Problems}\label{sec:extensions}
%======================================================================

\subsection{Weighted generalizations}

The methods extend to \emph{weighted} real-rooted polynomials where 
roots carry multiplicities, using log-concavity in place of total 
positivity. The analog of~\eqref{eq:star} involves 
$\sum_j m_j r''(\zeta_j)/r(\zeta_j)$ where $m_j$ are multiplicity 
weights.

\subsection{Higher-order inequalities}

One can ask for higher-order analogs involving derivatives 
$r^{(k)}(\zeta_j)$ at critical points. Preliminary calculations 
suggest
\[
  \sum_{j=1}^{n-1}\frac{r^{(3)}(\zeta_j)}{r'(\zeta_j)} 
  = O\left(\frac{n^2}{\sigma^3(r)}\right)
\]
but the exact constant remains open.

\subsection{Complex zeros}

When $r$ has complex zeros, the critical points may also be complex. 
A natural question is whether an analog of~\eqref{eq:star} holds 
for the sum over critical points in the upper half-plane, with 
$\sigma^2$ replaced by a suitable dispersion measure for complex roots.

%======================================================================
\appendix
%======================================================================

\section{Optimization of Constants}\label{app:optimize}

We provide details on the optimization that yields the constant 
$4n(n-1)/3$ in Theorem~\ref{thm:main}.

\subsection{Lagrange multiplier formulation}

Consider the variational problem
\[
  \inf_{r \in \PnR}\left\{
    -\sigma^2(r)\sum_{j=1}^{n-1}\frac{r''(\zeta_j)}{r(\zeta_j)} 
    : \Phi_n(r) = 1
  \right\}.
\]
The constraint $\Phi_n(r)=1$ normalizes the problem. By 
Theorem~\ref{thm:critval}, the constraint becomes
\[
  -\frac{1}{4}\sum_{j=1}^{n-1}\frac{r''(\zeta_j)}{r(\zeta_j)} = 1,
\]
so the objective simplifies to $4\sigma^2(r)$.

The extremal configuration must satisfy the Euler-Lagrange equations, 
which by symmetry and the equality conditions of Theorem~\ref{thm:main} 
forces $r$ to have roots at Hermite polynomial zeros.

\subsection{Explicit computation for Hermite roots}

For $H_n$ with roots $\xi_1 < \cdots < \xi_n$, we have 
$\sum_i \xi_i = 0$ and $\sum_i \xi_i^2 = n(n-1)/2$ (standard fact). 
Hence $\sigma^2(H_n) = (n-1)/2$.

The critical values of $H_n$ satisfy the recurrence relation. 
Since $H_n'(\zeta_j) = 0$ and $H_n' = 2nH_{n-1}$, we have 
$H_{n-1}(\zeta_j) = 0$, i.e., the critical points of $H_n$ are 
precisely the roots of $H_{n-1}$.

For the ratio $H_n''(\zeta_j)/H_n(\zeta_j)$, we use the identity
\[
  \frac{H_n''(\zeta_j)}{H_n(\zeta_j)} 
  = 2\sum_{k=1}^n\frac{1}{\zeta_j-\xi_k}
\]
where $\xi_k$ are the roots of $H_n$.

By properties of Hermite polynomials and their interlacing structure, 
explicit summation yields
\[
  \sum_{j=1}^{n-1}\frac{H_n''(\zeta_j)}{H_n(\zeta_j)} 
  = -\frac{8n(n-1)}{3(n-1)} 
  = -\frac{8n}{3},
\]
confirming the optimal constant. The detailed calculation uses 
orthogonality relations and asymptotic formulas for Hermite polynomials, 
see \cite{Hermite} for complete details.

%======================================================================
\begin{thebibliography}{9}

\bibitem{Problem4}
The finite free Stam inequality (this volume).
\emph{Problem 4.tex}.

\bibitem{Karlin}
S.~Karlin,
\emph{Total Positivity, Vol.~I},
Stanford University Press, 1968.

\bibitem{Schoenberg}
I.~J.~Schoenberg,
\emph{On P\'olya frequency functions},
J.~Analyse Math.~\textbf{1} (1951), 331--374.

\bibitem{GK}
F.~R.~Gantmacher and M.~G.~Krein,
\emph{Oscillation Matrices and Kernels and Small Vibrations of 
Mechanical Systems},
AMS Chelsea Publishing, 2002.

\bibitem{Hermite}
G.~Szeg\H{o},
\emph{Orthogonal Polynomials},
AMS Colloquium Publications, Vol.~23, 1939.

\end{thebibliography}

\end{document}
