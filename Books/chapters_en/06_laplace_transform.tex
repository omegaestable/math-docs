\chapter{The Laplace Transform}

In this section we will study the Laplace Transform (LT), a very powerful tool that will allow us to solve differential equations much more easily.

\begin{definicion}{Laplace Transform}{}
Let $f:[0,\infty) \to \mathbb{R}$ be a piecewise continuous function. We define its \textbf{Laplace transform} as the improper integral
\begin{equation}\label{eqn:16}
\mathcal{L}\{f\}(s) = \int_0^{\infty} f(t)e^{-st}dt
\end{equation}
for all $s$ where the integral converges.
\end{definicion}

Let us emphasize the nature of the Laplace transform: it receives a function $f(t)$ and returns another function of a different (and new) variable $s$. Normally, for notation purposes, we denote the transform $\mathcal{L}\{f\}(s)$ simply as $F(s)$.

\begin{ejemplo}
    \textbf{The transform of $f(t)=t$.}
Let us calculate $F(s)$:
$$F(s) = \int_0^\infty te^{-st}dt = \left[-\frac{1}{s} te^{-st}\right]_0^\infty + \int_0^\infty \frac{1}{s}e^{-st}dt = \frac{1}{s^2}$$
by integration by parts, useful whenever $s >0$.
We then have that $F(s) = s^{-2}$, that is, $\mathcal{L}\{t\}(s) = \frac{1}{s^2}$.
\end{ejemplo}

\begin{ejemplo}
    \textbf{The transform of $f(t) = e^{at}$}.
\begin{align*}
\mathcal{L}\{e^{at}\}(s) &= \int_0^\infty e^{at}e^{-st}dt = \int_0^\infty e^{(a-s)t}dt = \frac{1}{a-s}\left[e^{(a-s)t}\right]_0^\infty
\end{align*}
In this case, when evaluating at $\infty$, the limit is the function $e^{(a-s)t}$, which only vanishes when $a-s<0$. Therefore, when $s>a$, we have
$$\frac{1}{a-s}(0-1) = \frac{1}{s-a}.$$
We then have that $\mathcal{L}\{e^{at}\}(s) = \frac{1}{s-a}$.
\end{ejemplo}

A table with many of the Laplace transforms we need is given below:
\begin{center}
\begin{tabular}{|c|c|}
\hline
\textbf{$f(t)$} &\textbf{$\mathcal{L}\{f\}(s)$} \\
\hline
$1$ & $\frac{1}{s}$ \\ \hline
$t^n$ & $\frac{n!}{s^{n+1}}$\\ \hline
$e^{at}$ & $\frac{1}{s-a}$ \\ \hline
$\sin(kt)$ & $\frac{k}{s^2+k^2}$ \\ \hline
$\cos(kt)$ & $\frac{s}{s^2+k^2}$ \\ \hline
$\sinh(kt)$ & $\frac{k}{s^2-k^2}$ \\ \hline
$\cosh(kt)$ & $\frac{s}{s^2-k^2}$ \\ \hline
$t^ne^{at}$ & $\frac{n!}{(s-a)^{n+1}}$\\ \hline
$e^{at}\sin(kt)$ & $\frac{k}{(s-a)^2+k^2}$\\ \hline
$e^{at}\cos(kt)$ & $\frac{s-a}{(s-a)^2+k^2}$\\ \hline
$t\sin(kt)$ & $\frac{2ks}{(s^2+k^2)^2}$\\ \hline
$t\cos(kt)$ & $\frac{s^2-k^2}{(s^2+k^2)^2}$\\ \hline
\end{tabular}
\end{center}

\begin{figure}[H]
    \centering
    \includegraphics[width=0.4\textwidth]{Recorte10.png}
    \caption{Graphs of exponential functions showing typical Laplace transform behavior.}
    \label{fig:exponentials}
\end{figure}

\begin{figure}[H]
    \centering
    \includegraphics[width=0.7\textwidth]{recorte11.png}
    \caption{Shifted Heaviside-type functions for different values of $a$.}
    \label{fig:shifted_heaviside}
\end{figure}

\begin{figure}[H]
    \centering
    \includegraphics[width=0.7\textwidth]{Recorte13.png}
    \caption{Combination of step function (Heaviside) with sinusoidal function.}
    \label{fig:heaviside_sine}
\end{figure}

\section{Properties}

The Laplace transform is linear:
$$\mathcal{L}\{af+bg\}=a\mathcal{L}\{f\}+b\mathcal{L}\{g\}.$$
With this property alone, we can already calculate several more transforms:

\begin{ejemplo}
    The transform $\mathcal{L}\{5e^{3t}\}$ is, by linearity,
$$\mathcal{L}\{5e^{3t}\} = 5\mathcal{L}\{e^{3t}\}=\frac{5}{s-3}.$$
The transform $\mathcal{L}\{4t^3+ 3\sin(2t)\}$ is, by linearity,
$$\mathcal{L}\{4t^3+ 3\sin(2t)\}=4 \frac{3!}{s^4}+ 3\frac{2}{s^2+4}= \frac{24}{s^4}+\frac{6}{s^2+4}.$$
\end{ejemplo}

The following theorem summarizes most of the important properties of the Laplace Transform. We will assume that $F(s) = \mathcal{L}\{f(t)\}(s)$.

\begin{center}
\begin{tabular}{|c|c|}
\hline
\textbf{Identity} & \textbf{Result} \\
\hline
Translation in $s$ & $\mathcal{L}\{e^{at}f(t)\} = F(s-a)$ \\
\hline
Derivative of F  & $\mathcal{L}\{t^nf(t)\} = (-1)^nF^{(n)}(s)$ \\
\hline
Transform of the derivative & $\mathcal{L}\{f'(t)\} = sF(s) - f(0)$ \\
\hline
Transform of the $n$-th derivative & $\mathcal{L}\{f^{(n)}(t)\} = s^nF(s) - s^{n-1}f(0) - \dots - f^{(n-1)}(0)$ \\
\hline
Transform of the integral & $\mathcal{L}\Big\{\int_0^t f(\tau) d\tau \Big\}(s) = \frac{F(s)}{s}$ \\
\hline
\end{tabular}
\end{center}

The properties related to the transform of derivatives are the most important. They will be the key to converting differential equations into algebraic equations, which we can solve.

\section{Special Functions}
In this section we are going to study the properties of some special functions that we have not yet used in the course.

\subsection{The Heaviside Function}
The \textbf{unit step function}, or Heaviside step function, is defined as
$$\mathcal{U}(t) = \begin{cases}
0 & t<0 \\
1 & t \geq 0
\end{cases}$$
This function will help us define other functions by cases.

A translated version of this step function is $\mathcal{U}(t-a) = \begin{cases}
0 & t<a \\
1 & t \geq a
\end{cases}$

\begin{figure}[H]
    \centering
    \includegraphics[width=0.7\textwidth]{Recorte12.png}
    \caption{Heaviside functions $\mathcal{U}(t-a)$ for different values of $a$.}
    \label{fig:heaviside_en}
\end{figure}

\begin{ejemplo}
    The function $f(t) = \begin{cases}
g(t) & t<a \\
h(t) & t \geq a
\end{cases}$ can be rewritten as $ f(t) = g(t) + (h(t)-g(t))\mathcal{U}(t-a).$
\end{ejemplo}

In fact, a function of three or more cases can also be written using the function $\mathcal{U}$, simply by adding more copies of $\mathcal{U}$.

\begin{ejemplo}
    The function $f(t) = \begin{cases}
2 & 0\leq t<4 \\
-1 & 4 \leq t<6 \\
3 & t \geq 6
\end{cases}$ can be rewritten as $ f(t) = 2 -3\mathcal{U}(t-4)+ 4\mathcal{U}(t-6).$
\end{ejemplo}

The Laplace transform of $\mathcal{U}(t-a)$ is $\boxed{\mathcal{L}\{\mathcal{U}(t-a)\}(s) = \frac{e^{-as}}{s}}$. Furthermore, just as $e^{at}$ is a factor that allows us to transform our function with respect to $s$, the factor $\mathcal{U}$ allows us to transform the argument of the function. That is to say,
$$\mathcal{L}\{f(t-a)\mathcal{U}(t-a)\}(s) = e^{-as}F(s).$$
This property helps us calculate the transform of piecewise functions.

\begin{ejemplo}
    We can calculate the Laplace transform of $f(t) = \begin{cases}
0 & t<2\pi \\
\cos t & t \geq 2\pi
\end{cases}$ as follows:

$$\mathcal{L}\{f(t)\}(s) = \mathcal{L}\{\cos (t-2\pi)\mathcal{U}(t-2\pi)\}(s) = e^{-2\pi s}\mathcal{L}\{\cos(t)\}(s) = \frac{se^{-2\pi s}}{s^2+1}.$$
\end{ejemplo}

\subsection{The Dirac Delta}
The next function we will study will help us model instantaneous phenomena, for example, an explosion, or an electric shock.

\begin{definicion}{Dirac Delta $\delta$}{}
The \textbf{Dirac delta} $\delta(t)$ is a function whose value is
$$\delta(t) = \begin{cases}
\infty & : t=0 \\
0 & : t \neq 0
\end{cases}$$
and furthermore (magically?) $\int_{-\infty}^{\infty} \delta(t)dt = 1$. Actually, the function is an idealization of a \textit{distribution}, which would explain this last property. However rigorously, the Dirac delta is a generalized function. For a more formal definition, we would need to study measure theory.
\end{definicion}

This function also has a translation $\delta(t-t_0)$. Let us now calculate its Laplace transform.
$$\mathcal{L}\{\delta(t-t_0)\} = \int_0^\infty \delta(t-t_0)e^{-st}dt.$$
Using the property that the only non-zero value of $\delta(t-t_0)$ occurs at $t=t_0$, we can see that this integral reduces to $e^{-st_0}$, that is
$$\boxed{\mathcal{L}\{\delta(t-t_0)\} = e^{-st_0}.}$$
Particularly, $\mathcal{L}\{\delta(t)\} = 1.$

\subsection{The Gamma Function}
We also introduce the Gamma function, since it helps us with the Laplace transform.

\begin{definicion}{Gamma Function $\Gamma$}{}
The \textbf{Gamma function} is defined as the integral
$$\Gamma(n) = \int_0^\infty t^{n-1}e^{-t}dt \quad , \quad n>0.$$
\end{definicion}

Functions defined as integrals are sometimes called \textit{special functions}. Particularly, the Gamma function allows us to generalize the factorial $n!$ to any real number $n$: $\Gamma(n+1) = n!$. It also follows that $\Gamma(n+1) = n \cdot \Gamma(n)$ for all $n>0$.

With the Gamma function, we can generalize the identity $\mathcal{L}\{t^n\} = \frac{n!}{s^{n+1}}$ for any $n>0$ (not necessarily an integer):
$$\mathcal{L}\{t^n\}(s) = \frac{\Gamma(n+1)}{s^{n+1}} \text{ for all } n > 0.$$

\section{Convolution}
The convolution is a binary operation on functions, similar in some sense to a kind of product. However, we will be interested in how this operation interacts with the Laplace transform.

\begin{definicion}{Convolution}{}
Let $f,g$ be piecewise continuous functions in $[0,\infty)$. The \textbf{convolution} of $f$ and $g$ is defined as
$$(f*g)(t) = \int_0^t f(\tau)g(t-\tau)d\tau.$$
\end{definicion}

\begin{teorema}{Convolution Theorem}{}
If $\mathcal{L}\{f\} = F$ and $\mathcal{L}\{g\} = G$, then
$$\mathcal{L}\{f*g\}(s) = F(s)G(s).$$
\end{teorema}

\begin{ejemplo}
    Since the transforms of $t$ and $\sin(t)$ are $\frac{1}{s^2}$ and $\frac{1}{s^2+1}$ respectively, it must hold that
$\mathcal{L}\{t * \sin(t)\}(s) = \frac{1}{s^2(s^2+1)}.$
\end{ejemplo}

\section{Inverse Transform}
To be able to use the Laplace transform, we also need its inverse, that is, to calculate from a function $F(s)$ a function $f(t)$.

\begin{definicion}{Inverse Laplace Transform}{}
Let $F(s)$ be the Laplace transform of some function $f(t)$. We denote the inverse transform as
$\mathcal{L}^{-1}\{F\}(t) = f(t).$
\end{definicion}

In principle, to calculate the inverse Laplace, we need to be able to resolve a function $F(s)$ as an expression of known transforms. For this we make extensive use of the technique of \textbf{partial fractions,} which we already know from the integration lesson.

\begin{ejemplo}
    Calculate $\mathcal{L}^{-1}\left\{\frac{1}{s^2-9}\right\}$..

First, let us decompose into partial fractions. Note that $(s-3)(s+3) = s^2-9$, so we have
$$\frac{1}{s^2-9} = \frac{A}{s+3}+\frac{B}{s-3}$$
for some constants $A,B$. From here we deduce that $A = -1/6$ and $B=1/6$, so
$$\mathcal{L}^{-1}\left\{\frac{1}{s^2-9}\right\} = \mathcal{L}^{-1}\left\{\frac{-1/6}{s+3}\right\}+\mathcal{L}^{-1}\left\{\frac{1/6}{s-3}\right\} = \frac{1}{6}e^{3t} - \frac{1}{6}e^{-3t}.$$
\end{ejemplo}

\begin{ejemplo}
    Calculate $\mathcal{L}^{-1}\left\{\frac{s}{(s-4)(s^2+4s+5)}\right\}$.

This is a case with a non-factorable quadratic denominator. In this case, the partial fraction decomposition has the form
$$\frac{s}{(s-4)(s^2+4s+5)}= \frac{A}{s-4}+\frac{Bs+C}{s^2+4s+5}$$
from where the constants can be deduced as $A=4/37$, $B=-4/37$ and $C = -6/37$. The inverse Laplace transform in this case is a bit more complicated, since we must use the technique of completing the square to evaluate the inverse of $\frac{Bs+C}{s^2+4s+5}$, since
$$\frac{Bs+C}{s^2+4s+5} = \frac{Bs+C}{(s+2)^2 + 1}.$$
We leave the details to the reader, the answer is
$$\mathcal{L}^{-1}\left\{\frac{s}{(s-4)(s^2+4s+5)}\right\} = \frac{4}{37}e^{4t} + \frac{1}{37}e^{-2t}(2\sin(t)-4\cos(t)).$$
\end{ejemplo}

\section{Solving Differential Equations}
We are now in a position to give the method for solving initial value problems using the Laplace Transform.

\begin{enumerate}
    \item Apply the LT to both sides of the equation, using the identity
    $$\mathcal{L}\{f^{(n)}\}(s) = s^n\mathcal{L}\{f\}-s^{n-1}f(0)-s^{n-2}f'(0)- \dots - f^{(n-1)}(0).$$
    \item Substitute the initial values and isolate the transform $\mathcal{L}\{f\}$.
    \item Use the inverse LT to find the solution $f$.
\end{enumerate}

\begin{ejemplo}
    Solve the problem $\begin{cases}
y'' - 4y' + 4y = t^3 e^{2t}
 \\
y(0) = 0 \\
y'(0) = 0
\end{cases}$

Applying the LT to both sides:
$$\mathcal{L}\{y'' - 4y' + 4y\} = \mathcal{L}\{t^3e^{2t}\}$$
$$\implies (s^2Y-sy(0)-y'(0)) - 4(sY - y(0)) + 4Y = \frac{6}{(s-2)^4}$$
where $Y=\mathcal{L}\{y\}$. Substituting the initial conditions (which are 0) and isolating $Y$:
$$s^2Y-4sY+4Y=\frac{6}{(s-2)^4}$$
$$\implies Y=\frac{6}{(s-2)^4(s^2-4s+4)}= \frac{6}{(s-2)^6}$$
So we must invert the transform, which thanks to the table gives us
$$\mathcal{L}^{-1}\left\{\frac{6}{(s-2)^6}\right\} = \frac{1}{20}t^5e^{2t}.$$
The solution to the problem is $y(t)= \frac{t^5}{20}e^{2t}$.
\end{ejemplo}

\begin{ejercicios}
    Solve the following differential equations using Laplace transforms.
    \begin{enumerate}
    \item $\begin{cases}
y'' - 6y = 0\\
y(0) = 1 \\
y'(0) = -1
\end{cases}$
\item  $\begin{cases}
y - y'' = 1\\
y(0) = 0 \\
y'(0) = 0
\end{cases}$
\item  $\begin{cases}
y'' + 4y' + 3y  = 1 - \mathcal{U}(t-2)- \mathcal{U}(t-4)+\mathcal{U}(t-6)\\
y(0) = 0 \\
y'(0) = 0
\end{cases}$
\item The ODE
$y'' + y = \delta(t - 2\pi)$ with $y(0) = 1$ and $y'(0) = 0$.
    \end{enumerate}
\end{ejercicios}

\section{Integral and Integro-differential Equations}
A somewhat more exotic application of the Laplace transform is to solve differential equations where both derivatives and integrals of the unknown function appear.

\begin{definicion}{Integro-differential Equation}{}
An \textbf{integro-differential} equation for a function $y$ is an equation where both derivatives and integrals of the function appear, for example
$$y' + 6y + 9\int_0^t ydt = 1$$
is an integro-differential equation. An \textbf{integral equation} is an equation of this type, where only integrals of the function appear. For example
$$f(t) = 3t^2 - e^{-t} - \int_0^t f(\tau) e^{t-\tau}d\tau.$$
is an integral equation.
\end{definicion}

\begin{ejemplo}
    Solve the problem
$$\begin{cases}
y' + 6y + 9\int_0^t ydt = 1 \\
y(0) = 0
\end{cases}$$
Recall that the LT of the integral is $\mathcal{L}\{\int_0^t ydt\} = \frac{Y}{s}$.

When applying the LT, we have
$$sY - y(0) + 6Y + \frac{9Y}{s} = \frac{1}{s}$$
$$\implies sY + 6Y + \frac{9Y}{s}=\frac{1}{s}$$
Multiplying by $s$ and isolating $Y$:
$$s^2Y + 6sY + 9Y = 1 \implies Y(s^2+6s+9) = 1 \implies Y = \frac{1}{(s+3)^2}$$
Inverting, we have $y(t) = te^{-3t}$ which is the solution to the problem.
\end{ejemplo}

\begin{ejemplo}
    Solve the integral equation
$$f(t) = 3t^2 - e^{-t} - \int_0^t f(\tau) e^{t-\tau}d\tau.$$
First observe that under the integral we have a convolution: $\int_0^t f(\tau)e^{t-\tau}d\tau = (f*e^{-t})(t)$.
So we can apply the LT directly
$$F = \frac{6}{s^3}-\frac{1}{s+1} - \frac{F}{s+1}$$
Multiplying by $(s+1)$:
$$F(s+1) = \frac{6(s+1)}{s^3} - 1 - F$$
$$\implies F(s+2) = \frac{6s+6}{s^3} - 1 = \frac{6s+6-s^3}{s^3}$$
$$\implies F = \frac{6s+6-s^3}{s^3(s+2)}$$
The partial fraction decomposition is $F = -\frac{7}{4(s+2)} + \frac{1}{4s} + \frac{3}{s^2}-\frac{3}{s^3}$
so the solution to the integral equation is $f(t) = -\frac{7}{4}e^{-2t} + \frac{1}{4} + 3t - \frac{3}{2}t^2$.
\end{ejemplo}

\section{Systems of ODEs}
To conclude the course, we are going to study an alternative method to solve systems of differential equations, which uses the Laplace transform.

\begin{ejemplo}
    Solve the system
$$\begin{cases}
x' = -x + y \\
y'= 2x \\
x(0) = 0 \\
y(0) = 1
\end{cases}$$
Applying the LT to each equation, and substituting the initial values, we have (denoting $X=\mathcal{L}\{x\}$, $Y=\mathcal{L}\{y\}$):
$$\begin{cases}
sX = -X+Y \\
sY -1 = 2X
\end{cases}$$
This is a \textbf{linear} system of equations in $X$ and $Y$. We must solve it to find the values of the transforms.

From the first equation: $(s+1)X=Y$. Substituting into the second: $s(s+1)X - 1 = 2X \implies X(s^2+s-2) = 1$.
From where $X= \frac{1}{(s+2)(s-1)}$. We can then also calculate $Y= (s+1)X = \frac{s+1}{(s+2)(s-1)}$.

Using partial fractions, we can now invert the transforms.
\begin{align*}
    x(t) = \mathcal{L}^{-1}\left\{\frac{1}{(s+2)(s-1)}\right\} = \mathcal{L}^{-1}\left\{\frac{-1/3}{s+2}+\frac{1/3}{s-1}\right\} = \frac{1}{3}(-e^{-2t}+e^t) \\
y(t) = \mathcal{L}^{-1}\left\{\frac{s+1}{(s+2)(s-1)}\right\} = \mathcal{L}^{-1}\left\{\frac{-1/3}{s+2}+\frac{2/3}{s-1}\right\} = \frac{1}{3}(-e^{-2t}+2e^t)
\end{align*}
So the solution to the system is
$$\mathbf{X}(t) = \frac{1}{3}\begin{pmatrix}
-e^{-2t}+e^t \\ -e^{-2t}+2e^t
\end{pmatrix}.$$
\end{ejemplo}

\begin{ejercicios}
    Solve the following systems using the Laplace transform.
    \begin{enumerate}
    \item $\begin{cases}
    \dot{x} = 2y + e^t \\
    \dot{y} = 8x -t \\
    x(0) = y(0) = 0
    \end{cases}.$
    \item $\begin{cases}
    \dot{x} = 4x - 2y \\
    \dot{y} = 3x - y \\
    x(0) = 1, \ y(0) = -1
    \end{cases}.$
    \end{enumerate}
\end{ejercicios}
