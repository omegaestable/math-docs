\documentclass[11pt]{article}
\usepackage{amsmath,amssymb,amsthm}

\newtheorem{theorem}{Theorem}
\newtheorem{lemma}{Lemma}
\newtheorem{proposition}{Proposition}
\newtheorem{remark}{Remark}

\begin{document}

\section*{Problem}
Let $p(x)$ and $q(x)$ be monic real-rooted polynomials of degree $n$. Define
\[
(p \boxplus_n q)(x) = \sum_{k=0}^n c_k x^{n-k},\qquad
c_k = \sum_{i+j=k} \frac{(n-i)!(n-j)!}{n!(n-k)!} a_i b_j.
\]
For $p(x)=\prod_{i=1}^n (x-\lambda_i)$, define
\[
\Phi_n(p) := \sum_{i=1}^n \Big(\sum_{j\ne i} \frac{1}{\lambda_i-\lambda_j}\Big)^2,
\]
with $\Phi_n(p)=\infty$ if $p$ has a multiple root. Is it true that
\[
\frac{1}{\Phi_n(p\boxplus_n q)} \ge \frac{1}{\Phi_n(p)} + \frac{1}{\Phi_n(q)}?
\]

\section*{Answer (Status)}
\begin{remark}
The full inequality for all $n\ge 3$ is currently open in this setting. The best known results include the ``half-Stam'' inequality
\[\frac{2}{\Phi_n(p\boxplus_n q)} \ge \frac{1}{\Phi_n(p)} + \frac{1}{\Phi_n(q)}\]
and a weaker logarithmic bound. See the notes in \texttt{borrador.tex} for the analytic flow proof sketch.
\end{remark}

\section*{Proof for $n=2$}
\begin{proposition}
For $n=2$, the desired inequality holds with equality.
\end{proposition}
\begin{proof}
Let $p(x)=(x-\lambda_1)(x-\lambda_2)$. Then
\[
V_1=\frac{1}{\lambda_1-\lambda_2},\qquad V_2=\frac{1}{\lambda_2-\lambda_1},
\]
so
\[
\Phi_2(p)=V_1^2+V_2^2=\frac{2}{(\lambda_1-\lambda_2)^2}.
\]
Let $m=(\lambda_1+\lambda_2)/2$ and $\sigma^2(p)=\frac{1}{2}\sum_{i=1}^2 (\lambda_i-m)^2$.
Then $\sigma^2(p)=(\lambda_1-\lambda_2)^2/4$, hence
\[
\frac{1}{\Phi_2(p)}=2\sigma^2(p).
\]
The symmetric additive convolution satisfies variance additivity,
$\sigma^2(p\boxplus_2 q)=\sigma^2(p)+\sigma^2(q)$. Therefore
\[
\frac{1}{\Phi_2(p\boxplus_2 q)}=2\sigma^2(p\boxplus_2 q)
=2\sigma^2(p)+2\sigma^2(q)
=\frac{1}{\Phi_2(p)}+\frac{1}{\Phi_2(q)}.
\]
\end{proof}

\section*{Attempt for general $n$ (detailed outline and gap)}
We give a detailed convolution-flow attempt. The argument is complete up
to a missing functional inequality on the root statistics. This does
\emph{not} resolve the full Stam inequality for $n\ge3$.

\subsection*{Setup}
For $p(x)=\prod_{i=1}^n (x-\lambda_i)$ with distinct roots define
\[
V_i:=\sum_{j\ne i}\frac{1}{\lambda_i-\lambda_j},\qquad
\Phi_n(p)=\sum_{i=1}^n V_i^2,
\]
and the quadratic form
\[
\mathcal{S}(p):=\sum_{i<j}\frac{(V_i-V_j)^2}{(\lambda_i-\lambda_j)^2}.
\]
Assume $q$ is centered with variance $\sigma^2(q)$.

\subsection*{Fractional convolution semigroup}
Write $q(x)=\sum_{k=0}^n b_k x^{n-k}$ with $b_0=1$ and set
\[\kappa_k:=\frac{(n-k)!}{n!}\,b_k\qquad (k=0,\ldots,n).
\]
Define
\begin{equation}
q_t(x)=\sum_{k=0}^n b_k(t) x^{n-k},
\qquad b_k(t)=\frac{n!}{(n-k)!}\,\kappa_k^t,
\label{eq:semigroup-def}
\end{equation}
so that $q_0(x)=x^n$ and $q_1=q$. The coefficients depend real-analytically
on $t$ and satisfy
\begin{equation}
q_s\boxplus_n q_t=q_{s+t}
\qquad (s,t\ge0,\ s+t\le1).
\label{eq:semigroup}
\end{equation}
When $q$ is real-rooted, $q_t$ remains real-rooted for $t\in[0,1]$, and
$\sigma^2(q_t)=t\sigma^2(q)$. Define
\[p_t:=p\boxplus_n q_t.
\]

\begin{lemma}[Root-derivative formula]
Let $r_t(x)$ be a monic polynomial with simple roots
$\lambda_1(t),\ldots,\lambda_n(t)$ that are differentiable in $t$. Then
\begin{equation}
\dot{\lambda}_i(t)=-\frac{\partial_t r_t(\lambda_i(t))}{r_t'(\lambda_i(t))}.
\label{eq:root-derivative}
\end{equation}
\end{lemma}
\begin{proof}
Differentiate $r_t(\lambda_i(t))=0$ in $t$ and solve for $\dot{\lambda}_i$.
\end{proof}

\subsection*{Perturbative root shift}
Let $q$ be centered with small variance $\sigma^2(q)=\epsilon^2$.

\begin{lemma}[Second-order shift]
Let $p$ be real-rooted with simple roots $\lambda_i$ and set
$p_t=p\boxplus_n q_t$. Then
\begin{equation}
\lambda_i(t)=\lambda_i(0)+\frac{t\epsilon^2}{n-1}V_i+O(t^2\epsilon^4).
\label{eq:root-shift}
\end{equation}
In particular, the roots $\mu_i$ of $p\boxplus_n q$ satisfy
$\mu_i=\lambda_i+\frac{\epsilon^2}{n-1}V_i+O(\epsilon^4)$.
\end{lemma}
\begin{proof}
By \eqref{eq:root-derivative},
$\dot{\lambda}_i(0)=-\partial_t p_t(\lambda_i)/p'(\lambda_i)$.
The coefficient formula for $p\boxplus_n q_t$ shows
$\partial_t p_t|_{t=0}$ corresponds to adding variance
$\epsilon^2$ in the linearized convolution, yielding
$\dot{\lambda}_i(0)=\frac{\epsilon^2}{n-1}V_i$. The second derivative is
uniformly bounded in terms of $p$, giving \eqref{eq:root-shift}.
\end{proof}

\begin{lemma}[Infinitesimal drop of $\Phi_n$]
For centered $q$ with variance $\epsilon^2$,
\begin{equation}
\Phi_n(p\boxplus_n q)
=\Phi_n(p)-\frac{2\epsilon^2}{n-1}\sum_{i<j}
\frac{(V_i-V_j)^2}{(\lambda_i-\lambda_j)^2}+O(\epsilon^4).
\label{eq:phi-drop}
\end{equation}
\end{lemma}
\begin{proof}
Insert \eqref{eq:root-shift} into the definition of $\Phi_n$. Linear terms
cancel by $\sum_i V_i=0$, and the quadratic term yields the stated sum.
\end{proof}

\subsection*{Energy dissipation along the flow}
\begin{lemma}[Dissipation identity]
For $p_t=p\boxplus_n q_t$,
\begin{equation}
\frac{d}{dt}\Phi_n(p_t)
=-\frac{2\sigma^2(q)}{n-1}\,\mathcal{S}(p_t).
\label{eq:dissipation}
\end{equation}
\end{lemma}
\begin{proof}
By the semigroup property \eqref{eq:semigroup},
$p_{t+h}=p_t\boxplus_n q_h$ and $\sigma^2(q_h)=h\sigma^2(q)$.
Apply \eqref{eq:phi-drop} to $p_t$ with variance $h\sigma^2(q)$, divide by
$h$, and let $h\downarrow0$.
\end{proof}

Consequently,
\begin{equation}
\frac{d}{dt}\Big(\frac{1}{\Phi_n(p_t)}\Big)
=\frac{2\sigma^2(q)}{n-1}\,\frac{\mathcal{S}(p_t)}{\Phi_n(p_t)^2}.
\label{eq:inv-dissipation}
\end{equation}
Integrating from $0$ to $1$ yields
\begin{equation}
\frac{1}{\Phi_n(p\boxplus_n q)}-\frac{1}{\Phi_n(p)}
=\frac{2\sigma^2(q)}{n-1}\int_0^1
\frac{\mathcal{S}(p_t)}{\Phi_n(p_t)^2}\,dt.
\label{eq:integral-identity}
\end{equation}

\subsection*{What would imply the full Stam inequality}
The desired inequality
\[
\frac{1}{\Phi_n(p\boxplus_n q)}\ge
\frac{1}{\Phi_n(p)}+\frac{1}{\Phi_n(q)}
\]
would follow from the lower bound
\begin{equation}
\frac{2\sigma^2(q)}{n-1}\int_0^1
\frac{\mathcal{S}(p_t)}{\Phi_n(p_t)^2}\,dt
\ge \frac{1}{\Phi_n(q)}.
\label{eq:target-bound}
\end{equation}
By symmetry, it suffices to prove the pointwise estimate
\begin{equation}
\frac{\mathcal{S}(r)}{\Phi_n(r)^2}\ge
\frac{n-1}{2}\,\frac{1}{\sigma^2(r)\,\Phi_n(r)}
\qquad\text{for all } r\in\mathcal{P}_n^{\mathbb{R}}\text{ with distinct roots.}
\label{eq:pointwise-goal}
\end{equation}
Then $\sigma^2(p_t)=\sigma^2(p)+t\sigma^2(q)$ and
\eqref{eq:integral-identity} imply \eqref{eq:target-bound} by integration.

\subsection*{Known partial bound (half-Stam)}
Using the Fisher--variance inequality
\begin{equation}
\Phi_n(r)\,\sigma^2(r)\ge\frac{n(n-1)^2}{4},
\label{eq:fisher-variance}
\end{equation}
and \eqref{eq:integral-identity} gives
\[
\frac{2}{\Phi_n(p\boxplus_n q)}\ge
\frac{1}{\Phi_n(p)}+\frac{1}{\Phi_n(q)}.
\]

\subsection*{Where the gap remains}
The missing ingredient is a sharp lower bound relating
$\mathcal{S}(r)$ to $\Phi_n(r)$ and $\sigma^2(r)$ strong enough to
upgrade half-Stam to full Stam. The conjectured pointwise estimate
\eqref{eq:pointwise-goal} matches equality for $n=2$ and is compatible
with the extremizers of \eqref{eq:fisher-variance}, but no proof is
known for $n\ge3$.

\begin{remark}
Any progress toward a functional inequality of the form
$\mathcal{S}(r)\gtrsim \Phi_n(r)^2/\sigma^2(r)$ would strengthen
\eqref{eq:integral-identity} and could bridge the remaining gap.
\end{remark}

\section*{Exploratory inequalities for the missing bound}
Below are natural candidate inequalities that would imply
\eqref{eq:pointwise-goal} or a close variant. These are not proved here.

\subsection*{Spectral-gap heuristic}
Define weights $w_{ij}:=(\lambda_i-\lambda_j)^{-2}$ and the quadratic form
\[\mathcal{E}(f):=\frac{1}{2}\sum_{i\ne j} w_{ij}(f_i-f_j)^2.
\]
Then $\mathcal{S}(p)=\mathcal{E}(V)$ with $V=(V_1,\ldots,V_n)$ and
$\sum_i V_i=0$. A uniform spectral-gap estimate
\begin{equation}
\mathcal{E}(f)\ge \gamma\sum_{i=1}^n f_i^2
\qquad(\sum_i f_i=0)
\label{eq:spectral-gap}
\end{equation}
with $\gamma$ controlled by $\sigma^2(p)$ and $\Phi_n(p)$ would yield
\eqref{eq:pointwise-goal}. The challenge is that the weights $w_{ij}$
become highly inhomogeneous when roots cluster.

\subsection*{Two-parameter inequality}
Since $\Phi_n$ is homogeneous of degree $-2$ under scaling and
$\mathcal{S}$ has degree $-4$, any scale-invariant bound must involve
$\sigma^2$. A natural candidate is
\begin{equation}
\mathcal{S}(p)\ge c_n\,\frac{\Phi_n(p)^2}{\sigma^2(p)}
\label{eq:two-parameter}
\end{equation}
with $c_n=(n-1)/2$ as in \eqref{eq:pointwise-goal}. Even establishing
\eqref{eq:two-parameter} with some uniform $c_n>0$ would improve the
half-Stam inequality.

\subsection*{Pairwise reduction heuristic}
The identity
\[V_i-V_j=(\lambda_i-\lambda_j)\sum_{k\ne i,j}
\frac{1}{(\lambda_i-\lambda_k)(\lambda_j-\lambda_k)}\]
suggests comparisons of $\mathcal{S}(p)$ to weighted sums of local gaps.
For nearly equally spaced roots one expects
$\mathcal{S}(p)\asymp \Phi_n(p)^2/\sigma^2(p)$. The obstruction is the
presence of clustered roots, where denominators dominate.

\subsection*{Concavity route}
Let $F(p):=1/\Phi_n(p)$ and consider $F(p_t)$. If one could prove
concavity of $F(p_t)$ in $t$, then
\[F(p\boxplus_n q)=F(p_1)\ge F(p_0)+F(q_1)-F(q_0)=F(p)+F(q),\]
since $q_0=x^n$ and $F(q_0)=0$. This reduces the problem to a second
derivative bound for $\Phi_n$ along the semigroup flow, which is open for
$n\ge3$.

\end{document}
