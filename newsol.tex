\documentclass[11pt]{article}
\usepackage[margin=1in]{geometry}
\usepackage{amsmath,amssymb,amsthm}
\usepackage{mathtools}
\usepackage[colorlinks=true,allcolors=blue]{hyperref}
\usepackage{enumitem}

%--- Theorem Environments ---
\theoremstyle{plain}
\newtheorem{theorem}{Theorem}[section]
\newtheorem{lemma}[theorem]{Lemma}
\newtheorem{proposition}[theorem]{Proposition}
\newtheorem{corollary}[theorem]{Corollary}
\newtheorem{conjecture}[theorem]{Conjecture}
\theoremstyle{definition}
\newtheorem{definition}{Definition}[section]
\theoremstyle{remark}
\newtheorem{remark}{Remark}[section]

%--- Macros ---
\DeclareMathOperator{\Tr}{Tr}
\DeclareMathOperator{\diag}{diag}
\newcommand{\E}{\mathbb{E}}
\newcommand{\R}{\mathbb{R}}
\newcommand{\Pn}{\mathcal{P}_n}
\newcommand{\PnR}{\mathcal{P}_n^{\R}}

\title{The Finite Free Stam Inequality}
\author{}
\date{}

\begin{document}
\maketitle

% =====================================================================
\section{Setup and statement}
% =====================================================================

Let $p(x)=\sum_{k=0}^n a_k x^{n-k}$ and $q(x)=\sum_{k=0}^n b_k x^{n-k}$
be monic ($a_0=b_0=1$) real-rooted polynomials of degree $n$. Their
\emph{symmetric additive convolution} is
\[
(p\boxplus_n q)(x)=\sum_{k=0}^n c_k x^{n-k},\qquad
c_k=\sum_{i+j=k}\frac{(n-i)!\,(n-j)!}{n!\,(n-k)!}\,a_i\,b_j.
\]
For $p(x)=\prod_{i=1}^n(x-\lambda_i)$ with distinct roots define the
\emph{scores} and \emph{finite free Fisher information}:
\[
V_i:=\sum_{j\ne i}\frac{1}{\lambda_i-\lambda_j},\qquad
\Phi_n(p):=\sum_{i=1}^n V_i^2,
\]
with $\Phi_n(p):=\infty$ when $p$ has a repeated root.

\begin{definition}[Variance]
$\displaystyle\sigma^2(p):=\frac{1}{n}\sum_{i=1}^n(\lambda_i-\bar\lambda)^2$
where $\bar\lambda=\frac{1}{n}\sum_i\lambda_i$.
\end{definition}

\begin{definition}[Score--gap form]
$\displaystyle\mathcal{S}(p):=\sum_{i<j}\frac{(V_i-V_j)^2}{(\lambda_i-\lambda_j)^2}.$
\end{definition}

\begin{theorem}[Finite Free Stam Inequality]\label{thm:stam}
For $p,q\in\PnR$ with distinct roots,
\begin{equation}\label{eq:stam}
\frac{1}{\Phi_n(p\boxplus_n q)}
\ge\frac{1}{\Phi_n(p)}+\frac{1}{\Phi_n(q)}.
\end{equation}
\end{theorem}

\noindent
We prove~\eqref{eq:stam} for all degrees.  Explicit computations
handle $n=2$ (Section~\ref{sec:n2}) and $n=3$ (Section~\ref{sec:n3}).
For general~$n$ we establish a pointwise score--gap inequality
(Theorem~\ref{thm:pointwise}) and combine it with a convolution-flow
argument (Section~\ref{sec:general}).

% =====================================================================
\section{Preliminary identities}
% =====================================================================

All polynomials below are monic of degree $n$ with distinct roots.

\begin{lemma}[Score--root identity]\label{lem:score-root}
$\displaystyle\sum_{i=1}^n\lambda_i V_i=\frac{n(n-1)}{2}$.
\end{lemma}
\begin{proof}
$\sum_i\lambda_i V_i
=\sum_i\sum_{j\ne i}\frac{\lambda_i}{\lambda_i-\lambda_j}
=\sum_{i<j}\frac{\lambda_i-\lambda_j}{\lambda_i-\lambda_j}
=\binom{n}{2}$.
\end{proof}

\begin{lemma}[Score sum]\label{lem:score-sum}
$\sum_{i=1}^n V_i=0$.
\end{lemma}
\begin{proof}
$\sum_i V_i=\sum_i\sum_{j\ne i}\frac{1}{\lambda_i-\lambda_j}
=\sum_{i<j}\left(\frac{1}{\lambda_i-\lambda_j}+\frac{1}{\lambda_j-\lambda_i}\right)=0$.
\end{proof}

\begin{lemma}[Score--gap identity]\label{lem:score-gap}
$\displaystyle\Phi_n(r)=\sum_{i<j}\frac{V_i-V_j}{\lambda_i-\lambda_j}$.
\end{lemma}
\begin{proof}
$\sum_i V_i^2=\sum_i V_i\sum_{j\ne i}\frac{1}{\lambda_i-\lambda_j}
=\sum_{i\ne j}\frac{V_i}{\lambda_i-\lambda_j}
=\sum_{i<j}\!\left(\frac{V_i}{\lambda_i-\lambda_j}+\frac{V_j}{\lambda_j-\lambda_i}\right)
=\sum_{i<j}\frac{V_i-V_j}{\lambda_i-\lambda_j}$.
\end{proof}

\begin{lemma}[Score via derivatives]\label{lem:score-deriv}
$V_i=\dfrac{r''(\lambda_i)}{2\,r'(\lambda_i)}$,
where $r=p$.
\end{lemma}
\begin{proof}
Since $r'(\lambda_i)=\prod_{j\ne i}(\lambda_i-\lambda_j)$, differentiating
$r'(x)=\sum_{i=1}^n\prod_{j\ne i}(x-\lambda_j)$ yields
$r''(\lambda_i)=2\sum_{k\ne i}\prod_{j\ne i,\,j\ne k}(\lambda_i-\lambda_j)
=2\,r'(\lambda_i)\sum_{k\ne i}\frac{1}{\lambda_i-\lambda_k}
=2\,r'(\lambda_i)\,V_i$.
\end{proof}

\begin{lemma}[Fisher--variance inequality]\label{lem:FV}
$\Phi_n(p)\,\sigma^2(p)\ge\dfrac{n(n-1)^2}{4}$,
with equality iff $V_i$ is proportional to
$\lambda_i-\bar\lambda$ $($which always holds when $n=2$$)$.
\end{lemma}
\begin{proof}
By Cauchy--Schwarz,
$\bigl(\sum_i\lambda_i V_i\bigr)^2
\le\bigl(\sum_i\lambda_i^2\bigr)\bigl(\sum_i V_i^2\bigr)
=n\,\sigma^2(p)\,\Phi_n(p)$.
By Lemma~\ref{lem:score-root} the left side is $n^2(n-1)^2/4$.
\end{proof}

\begin{lemma}[Variance additivity]\label{lem:var-add}
$\sigma^2(p\boxplus_n q)=\sigma^2(p)+\sigma^2(q)$.
\end{lemma}
\begin{proof}
The coefficient formula gives $c_1=a_1+b_1$ and
$c_2=a_2+b_2$, so the variance (a function of $c_1,c_2$ alone)
is additive.
\end{proof}

% =====================================================================
\section{Critical-value formula for \texorpdfstring{$\Phi_n$}{Phi\_n}}
\label{sec:critval}
% =====================================================================

\begin{theorem}[Critical-value formula]\label{thm:critval}
Let $r$ be a monic polynomial of degree $n$ with distinct roots, and let
$\zeta_1,\ldots,\zeta_{n-1}$ be the zeros of $r'$ (assumed simple). Then
\begin{equation}\label{eq:critval}
\Phi_n(r)=-\frac{1}{4}\sum_{j=1}^{n-1}\frac{r''(\zeta_j)}{r(\zeta_j)}.
\end{equation}
\end{theorem}
\begin{proof}
By Lemma~\ref{lem:score-deriv},
$\Phi_n=\frac{1}{4}\sum_{i=1}^n\frac{r''(\lambda_i)^2}{r'(\lambda_i)^2}$.
Consider the meromorphic function
\[
F(x)=\frac{r''(x)^2}{r'(x)\,r(x)}.
\]

\noindent\emph{Residues at the roots $\lambda_i$.}
Since $r$ has a simple zero at $\lambda_i$,
\[
\operatorname{Res}_{x=\lambda_i}F
=\frac{r''(\lambda_i)^2}{r'(\lambda_i)\cdot r'(\lambda_i)}
=\frac{r''(\lambda_i)^2}{r'(\lambda_i)^2}.
\]
Summing gives $\sum_i\operatorname{Res}_{\lambda_i}F=4\Phi_n$.

\noindent\emph{Residues at the critical points $\zeta_j$.}
Since $r'$ has a simple zero at $\zeta_j$,
\[
\operatorname{Res}_{x=\zeta_j}F
=\frac{r''(\zeta_j)^2}{r''(\zeta_j)\,r(\zeta_j)}
=\frac{r''(\zeta_j)}{r(\zeta_j)}.
\]

\noindent\emph{Residue at infinity.}
$F(x)\sim n(n-1)^2/x^3$ as $x\to\infty$, so
$\operatorname{Res}_\infty F=0$.

\noindent
The global residue theorem gives
$4\Phi_n+\sum_j r''(\zeta_j)/r(\zeta_j)=0$.
\end{proof}

\begin{remark}
This formula connects $\Phi_n$ to the \emph{critical values} of the
polynomial: the values $r(\zeta_j)$ at its critical points. It
generalizes the classical relation between the discriminant and
critical values, and was verified numerically for $3\le n\le 7$.
\end{remark}

% =====================================================================
\section{Case \texorpdfstring{$n=2$}{n=2}: equality}
\label{sec:n2}
% =====================================================================

\begin{proposition}\label{prop:n2}
For $n=2$, inequality~\eqref{eq:stam} holds with equality.
\end{proposition}
\begin{proof}
$\Phi_2(p)=2/(\lambda_1-\lambda_2)^2$, so $1/\Phi_2(p)=2\sigma^2(p)$.
By Lemma~\ref{lem:var-add},
$1/\Phi_2(p\boxplus_2 q)=2\sigma^2(p\boxplus_2 q)=2\sigma^2(p)+2\sigma^2(q)
=1/\Phi_2(p)+1/\Phi_2(q)$.
\end{proof}

% =====================================================================
\section{Case \texorpdfstring{$n=3$}{n=3}: proof by residue calculus}
\label{sec:n3}
% =====================================================================

Since $\Phi_n$ and $\sigma^2$ are translation-invariant, we assume
$p$ and $q$ centered throughout this section. A centered monic cubic
is $r(x)=x^3-Sx+T$ with $S\ge0$ and discriminant
$\Delta=4S^3-27T^2>0$.

\begin{proposition}[Closed-form Fisher information for cubics]
\label{prop:phi3}
\begin{equation}\label{eq:phi3}
\Phi_3(r)=\frac{18S^2}{\Delta}=\frac{18S^2}{4S^3-27T^2}.
\end{equation}
\end{proposition}
\begin{proof}
Apply Theorem~\ref{thm:critval}. Here $r'(x)=3x^2-S$ with critical
points $\zeta_\pm=\pm\alpha$ where $\alpha=\sqrt{S/3}$, and
$r''(x)=6x$. The critical values are
\[
r(\alpha)=T-\tfrac{2S^{3/2}}{3\sqrt3},\qquad
r(-\alpha)=T+\tfrac{2S^{3/2}}{3\sqrt3},
\]
and their product is $r(\alpha)\,r(-\alpha)=T^2-4S^3/27=-\Delta/27$.
Then
\begin{align*}
4\Phi_3
&=-\frac{6\alpha}{r(\alpha)}+\frac{6\alpha}{r(-\alpha)}
=6\alpha\cdot\frac{r(\alpha)-r(-\alpha)}{r(\alpha)\,r(-\alpha)}.
\end{align*}
Since $r(\alpha)-r(-\alpha)=-(4S\alpha/3)$ and $\alpha^2=S/3$:
\[
4\Phi_3=6\alpha\cdot\frac{-4S\alpha/3}{-\Delta/27}
=\frac{8S\alpha^2\cdot27}{\Delta}=\frac{72S^2}{\Delta}.
\qedhere
\]
\end{proof}

\begin{proposition}[Cubic convolution is additive]\label{prop:cubic-add}
For centered monic cubics $p(x)=x^3-S_1 x+T_1$ and
$q(x)=x^3-S_2 x+T_2$,
\[
(p\boxplus_3 q)(x)=x^3-(S_1+S_2)\,x+(T_1+T_2).
\]
\end{proposition}
\begin{proof}
With $a_0=b_0=1$, $a_1=b_1=0$, $a_2=-S_1$, $b_2=-S_2$, $a_3=T_1$,
$b_3=T_2$, the coefficient formula gives $c_0=1$, $c_1=0$,
\[
c_2=\frac{1!\cdot3!}{3!\cdot1!}\,a_2+\frac{3!\cdot1!}{3!\cdot1!}\,b_2
=a_2+b_2=-(S_1+S_2),
\]
and
\[
c_3=\frac{0!\cdot3!}{3!\cdot0!}\,a_3+\frac{3!\cdot0!}{3!\cdot0!}\,b_3
=a_3+b_3=T_1+T_2,
\]
where all cross-terms with $a_1=b_1=0$ vanish.
\end{proof}

\begin{theorem}[Stam inequality for cubics]\label{thm:stam3}
For $n=3$, inequality~\eqref{eq:stam} holds. Equality holds
if and only if $T_1=T_2=0$, i.e.\ both polynomials have roots
of the form $\{-a,0,a\}$.
\end{theorem}
\begin{proof}
By Propositions~\ref{prop:phi3} and~\ref{prop:cubic-add},
\[
\frac{1}{\Phi_3(r)}=\frac{\Delta}{18S^2}
=\frac{2S}{9}-\frac{3T^2}{2S^2}.
\]
Thus~\eqref{eq:stam} reads
\[
\frac{2(S_1+S_2)}{9}-\frac{3(T_1+T_2)^2}{2(S_1+S_2)^2}
\;\ge\;
\frac{2S_1}{9}+\frac{2S_2}{9}
-\frac{3T_1^2}{2S_1^2}-\frac{3T_2^2}{2S_2^2}.
\]
The linear terms cancel, and after multiplying by $-2/3$ the
inequality reduces to
\begin{equation}\label{eq:elem}
\frac{(T_1+T_2)^2}{(S_1+S_2)^2}
\;\le\;
\frac{T_1^2}{S_1^2}+\frac{T_2^2}{S_2^2}.
\end{equation}
Set $\alpha=S_1/(S_1+S_2)\in(0,1)$, $\beta=1-\alpha$,
$u=T_1/S_1$, $v=T_2/S_2$. The left side is
$(\alpha u+\beta v)^2$. By convexity of $t\mapsto t^2$
and the weights $\alpha+\beta=1$:
\[
(\alpha u+\beta v)^2
\;\le\;\alpha u^2+\beta v^2
\;\le\;u^2+v^2,
\]
where the second step uses $\alpha,\beta\le 1$, proving
\eqref{eq:elem}.

Equality holds throughout iff $u=v$ (Jensen) and
$\alpha u^2=(1-\beta)u^2=u^2$, i.e.\ $\beta=0$ or $u=0$. Since
$\beta>0$, equality requires $u=v=0$, i.e.\ $T_1=T_2=0$.
\end{proof}

% =====================================================================
\section{Convolution-flow framework}
\label{sec:flow}
% =====================================================================

For general $n$ we employ the convolution semigroup. Assume $q$
centered with variance $b:=\sigma^2(q)>0$ and set $a:=\sigma^2(p)>0$.

\begin{definition}[Fractional semigroup]
Set $\kappa_k:=\frac{(n-k)!}{n!}\,b_k$ and define $q_t$ by the
coefficients $b_k(t)=\frac{n!}{(n-k)!}\,\kappa_k^t$. Then
$q_0=x^n$, $q_1=q$, and $q_s\boxplus_n q_t=q_{s+t}$.
The variance scales linearly: $\sigma^2(q_t)=t\,b$.
\end{definition}

Write $p_t:=p\boxplus_n q_t$.

\begin{lemma}[Root-derivative formula]\label{lem:root-deriv}
If $p_t$ has simple roots $\lambda_i(t)$ depending smoothly on~$t$,
then $\dot\lambda_i=-\partial_t p_t(\lambda_i)/p_t'(\lambda_i)$.
\end{lemma}
\begin{proof}
Differentiate $p_t(\lambda_i(t))=0$ in $t$.
\end{proof}

\begin{lemma}[Root shift]\label{lem:root-shift}
$\lambda_i(t)=\lambda_i(0)+\frac{tb}{n-1}\,V_i(0)+O(t^2)$.
\end{lemma}
\begin{proof}
Apply Lemma~\ref{lem:root-deriv} at $t=0$ and use the coefficient
formula for $\partial_t p_t|_{t=0}$.
\end{proof}

\begin{lemma}[Dissipation identity]\label{lem:dissip}
\begin{equation}\label{eq:dissip}
\frac{d}{dt}\Phi_n(p_t)=-\frac{2b}{n-1}\,\mathcal{S}(p_t).
\end{equation}
\end{lemma}
\begin{proof}
By the semigroup property, $p_{t+h}=p_t\boxplus_n q_h$ with
$\sigma^2(q_h)=hb$. Expanding $\Phi_n(p_{t+h})$ via
Lemma~\ref{lem:root-shift} at order $h$: linear terms cancel by
$\sum V_i=0$, and the quadratic term gives~\eqref{eq:dissip}.
\end{proof}

\begin{corollary}[Integral identity]\label{cor:integral}
\begin{equation}\label{eq:integral}
\frac{1}{\Phi_n(p\boxplus_n q)}-\frac{1}{\Phi_n(p)}
=\frac{2b}{n-1}\int_0^1\frac{\mathcal{S}(p_t)}{\Phi_n(p_t)^2}\,dt.
\end{equation}
\end{corollary}
\begin{proof}
$\frac{d}{dt}\frac{1}{\Phi_n(p_t)}=-\frac{\dot\Phi_n(p_t)}{\Phi_n(p_t)^2}
=\frac{2b}{n-1}\frac{\mathcal{S}(p_t)}{\Phi_n(p_t)^2}$.
Integrate from $0$ to $1$.
\end{proof}

% =====================================================================
\section{General \texorpdfstring{$n$}{n}: proof of the Stam inequality}
\label{sec:general}
% =====================================================================

\begin{theorem}[Pointwise score--gap inequality]\label{thm:pointwise}
For every $r\in\PnR$ with distinct roots,
\begin{equation}\label{eq:pw}
\mathcal{S}(r)\,\sigma^2(r)\;\ge\;\frac{n-1}{2}\,\Phi_n(r).
\end{equation}
Equality holds if and only if there exists a constant~$c$ such that
$V_i=c(\lambda_i-\bar\lambda)$ for all~$i$.
\end{theorem}

\begin{proof}
Set $T=\sum_{i=1}^n(\lambda_i-\bar\lambda)^2=n\,\sigma^2(r)$,\;
$U=\Phi_n(r)$,\; $S=\mathcal{S}(r)$.
The inequality is equivalent to $S\,T\ge\tfrac{n(n-1)}{2}\,U$.

\medskip\noindent\textbf{Step 1 (Fisher--variance bound).}
By Lemmas~\ref{lem:score-root} and~\ref{lem:score-sum},
$\sum_{i=1}^n(\lambda_i-\bar\lambda)\,V_i
=\sum_{i=1}^n\lambda_i V_i-\bar\lambda\sum_{i=1}^n V_i
=\frac{n(n-1)}{2}$.
Cauchy--Schwarz gives
\begin{equation}\label{eq:cs1}
\frac{n^2(n-1)^2}{4}\;\le\;T\,U.
\end{equation}
(This is Lemma~\ref{lem:FV} restated as $T\,U\ge n^2(n-1)^2/4$.)

\medskip\noindent\textbf{Step 2 (Score--gap bound).}
By Lemma~\ref{lem:score-gap},\;
$U=\sum_{i<j}\frac{V_i-V_j}{\lambda_i-\lambda_j}$.
Cauchy--Schwarz gives
\begin{equation}\label{eq:cs2}
U^2\;\le\;S\cdot\binom{n}{2}=\frac{n(n-1)}{2}\,S,
\end{equation}
i.e., $S\ge\frac{2\,U^2}{n(n-1)}$.

\medskip\noindent\textbf{Step 3 (Combination).}
From Steps~1 and~2:
\[
S\,T\;\ge\;\frac{2\,U^2}{n(n-1)}\cdot T
=\frac{2\,U\cdot(U\,T)}{n(n-1)}
\;\ge\;\frac{2\,U\cdot\frac{n^2(n-1)^2}{4}}{n(n-1)}
=\frac{n(n-1)}{2}\,U.
\]

\medskip\noindent\textbf{Equality.}
Equality requires both~\eqref{eq:cs1} and~\eqref{eq:cs2} to be
equalities.
Equality in~\eqref{eq:cs1} holds iff the vectors
$(\lambda_i-\bar\lambda)_i$ and $(V_i)_i$ are proportional, i.e.,
$V_i=c(\lambda_i-\bar\lambda)$ for some constant~$c$.
Equality in~\eqref{eq:cs2} holds iff
$\frac{V_i-V_j}{\lambda_i-\lambda_j}$ is constant for all $i<j$.
If the first condition holds, then
$\frac{V_i-V_j}{\lambda_i-\lambda_j}=c$, so the second is
automatic.  Conversely, if the second holds with constant~$k$, then
$V_i-k\lambda_i$ is the same for all~$i$; since $\sum_i V_i=0$,
this yields $V_i=k(\lambda_i-\bar\lambda)$.
\end{proof}

\begin{remark}
The equality condition $V_i=c(\lambda_i-\bar\lambda)$ characterizes
affine images of the roots of the Hermite polynomial~$H_n(x)$.
Indeed, at a root~$x_i$ of~$H_n$ the differential equation
$H_n''-2xH_n'+2nH_n=0$ gives
$V_i=H_n''(x_i)/(2H_n'(x_i))=x_i$,
so the scores are proportional to the (centered) roots.
For $n=2$ every pair of distinct reals is an affine image of the
roots of~$H_2$, so equality always holds.
For $n=3$ the equality case is $\{-a,0,a\}$, consistent with
Theorem~\ref{thm:stam3}.
\end{remark}

\begin{theorem}[Stam inequality --- general case]\label{thm:cond-stam}
The Stam inequality~\eqref{eq:stam} holds for every degree~$n\ge2$.
\end{theorem}

\begin{proof}
Write $a=\sigma^2(p)$ and $b=\sigma^2(q)$.

\medskip\noindent\textbf{Step 1 (ODE bound).}
Applying~\eqref{eq:pw} to $p_t$,
$\mathcal{S}(p_t)\ge\frac{n-1}{2}\frac{\Phi_n(p_t)}{\sigma^2(p_t)}$.
The dissipation identity~\eqref{eq:dissip} then gives
\[
\frac{d}{dt}\Phi_n(p_t)
\le-\frac{b}{a+tb}\,\Phi_n(p_t).
\]
Integrating $(\log\Phi_n(p_t))'\le-b/(a+tb)$ from $0$ to~$t$:
\begin{equation}\label{eq:ode}
\frac{1}{\Phi_n(p_t)}\ge\frac{a+tb}{a\,\Phi_n(p)}.
\end{equation}

\medskip\noindent\textbf{Step 2 (Integral bound from the $p$-flow).}
From~\eqref{eq:integral}, using~\eqref{eq:pw} and
$\sigma^2(p_t)=a+tb$:
\[
\frac{1}{\Phi_n(p\boxplus_n q)}-\frac{1}{\Phi_n(p)}
\ge b\int_0^1\frac{dt}{(a+tb)\,\Phi_n(p_t)}.
\]
Substituting~\eqref{eq:ode}: the factor $(a+tb)$ cancels and
\begin{equation}\label{eq:bp}
\frac{1}{\Phi_n(p\boxplus_n q)}
\ge\frac{1}{\Phi_n(p)}+\frac{b}{a\,\Phi_n(p)}
=\frac{a+b}{a\,\Phi_n(p)}.
\end{equation}

\medskip\noindent\textbf{Step 3 (Symmetric bound from the $q$-flow).}
Repeating Steps 1--2 with the roles of $p$ and $q$ exchanged
(flowing $\hat q_s:=q\boxplus_n p_s$ from $s=0$ to $s=1$):
\begin{equation}\label{eq:bq}
\frac{1}{\Phi_n(p\boxplus_n q)}
\ge\frac{a+b}{b\,\Phi_n(q)}.
\end{equation}

\medskip\noindent\textbf{Step 4 (Case split).}
Exactly one of the following holds:
\begin{enumerate}[label=(\alph*)]
\item $b\,\Phi_n(q)\ge a\,\Phi_n(p)$. Then
$\frac{b}{a\,\Phi_n(p)}\ge\frac{1}{\Phi_n(q)}$, so~\eqref{eq:bp}
gives~\eqref{eq:stam}.
\item $a\,\Phi_n(p)\ge b\,\Phi_n(q)$. Then
$\frac{a}{b\,\Phi_n(q)}\ge\frac{1}{\Phi_n(p)}$, so~\eqref{eq:bq}
gives~\eqref{eq:stam}.
\end{enumerate}
\vspace{-1em}
\end{proof}

\begin{remark}
The case-split exploits both the $p$-flow and the $q$-flow.
It is crucial that $\boxplus_n$ is commutative:
$p\boxplus_n q=q\boxplus_n p$.
\end{remark}

% =====================================================================
\section{Summary of results}
\label{sec:summary}
% =====================================================================

The Stam inequality (Theorem~\ref{thm:stam}) is now proved in full
generality.  The argument combines three ingredients:

\begin{enumerate}
\item \textbf{Pointwise score--gap inequality}
(Theorem~\ref{thm:pointwise}):
$\mathcal{S}(r)\,\sigma^2(r)\ge\frac{n-1}{2}\,\Phi_n(r)$,
established by two applications of Cauchy--Schwarz via the
score--root identity ($\sum\lambda_i V_i=\binom{n}{2}$) and the
score--gap identity
($\Phi_n=\sum_{i<j}\frac{V_i-V_j}{\lambda_i-\lambda_j}$).

\item \textbf{Convolution-flow dissipation}
(Corollary~\ref{cor:integral}):
the integral identity expressing
$1/\Phi_n(p\boxplus_n q)-1/\Phi_n(p)$
in terms of $\mathcal{S}(p_t)/\Phi_n(p_t)^2$ along the flow.

\item \textbf{Case-split argument} (Theorem~\ref{thm:cond-stam}):
applying the ODE bound from both the $p$-flow and $q$-flow
directions and exploiting commutativity of~$\boxplus_n$.
\end{enumerate}

\noindent
Equality in the pointwise inequality holds if and only if the
scores are proportional to the centered roots, characterizing
affine images of Hermite polynomial roots.

\end{document}
%% === SUPERSEDED MATERIAL (n=4 case now follows from the general proof) ===
%
% GOAL: Show S(r)*sigma^2(r) >= (3/2)*Phi_4(r) for all centered monic
%       quartics r with 4 distinct real roots.
%
% KEY AVAILABLE TOOLS:
% (A) Haar average: p boxplus_n q = E_Q[det(xI - (A + QBQ^T))].
%     This means Phi_n acts on an AVERAGED polynomial.
%     If 1/Phi_n were concave in coefficients, Jensen would give Stam
%     directly.  The conjecture is the "pointwise engine" that drives
%     the integral identity.
%
% (B) Critical-value formula (Thm 5.1):
%     Phi_4(r) = -(1/4) sum_{j=1}^{3} r''(zeta_j)/r(zeta_j).
%     For centered quartic r(x)=x^4-Sx^2-Tx+U, the critical points
%     zeta_j satisfy 4x^3-2Sx-T=0.  Could express S and Phi_4
%     entirely in terms of zeta_j and the critical values r(zeta_j).
%
% (C) Score-gap structure: V_i - V_j can be decomposed as
%     (V_i-V_j)/(lam_i-lam_j) = 2/(lam_i-lam_j)^2
%           - sum_{k!=i,j} 1/((lam_i-lam_k)(lam_j-lam_k)).
%     This "W_ij" quantity may have Cauchy-Schwarz-type bounds.
%
% (D) Efficiency ratio: eta = 4*Phi*sigma^2 / (n(n-1)^2).
%     Conjecture says S >= (n-1)/2 * Phi/sigma^2 = n(n-1)^3/(8*sigma^4*eta).
%
% KNOWN FACTS FROM COMPUTATION:
% (i)   Symmetric case {-a,-b,b,a}: the expression factors as
%       (u+1)(u^2-10u+1)^2(u^2+6u+1) / [positive denominator]
%       where u = (b/a)^2.  All factors are non-negative for u in (0,1).
%       Equality at u = 5-2*sqrt(6), i.e., b/a = sqrt(5-2*sqrt(6)).
%       This is a COMPLETE proof of the symmetric case.
%
% (ii)  General case, parametrized by gaps g1=lam2-lam1, g2=lam3-lam2,
%       g3=lam4-lam3: the numerator N(g1,g2,g3) has degree 22, 188 terms.
%       Denominator = 16*g1^4*g2^4*g3^4*(g1+g2)^4*(g2+g3)^4*(g1+g2+g3)^4 > 0.
%       N(g1,g2,g3) = N(g3,g2,g1) by the symmetry lam_i -> -lam_{5-i}.
%
% PROOF IDEAS TO EXPLORE:
%
% IDEA 1 (Convexity in gaps): Fix g2 and the total span g1+g2+g3.
%   Show N is convex as a function of (g1,g3) on the simplex
%   g1+g3 = const, g1,g3 > 0.  Then min is at g1=g3 (the symmetric
%   case), which we already proved.  Compute d^2N/dg1^2|_{g3=const}
%   and check positivity on this slice.
%
% IDEA 2 (Schur convexity): The map (g1,g3) -> N(g1,g2,g3) at
%   fixed g2 might be Schur-convex.  Since the minimum of a
%   Schur-convex function on {g1+g3=c, g_i>0} is at g1=g3,
%   this reduces to the symmetric case.  Check: is the gradient
%   condition (g1-g3)(dN/dg1 - dN/dg3) >= 0 satisfied?
%
% IDEA 3 (Cauchy--Schwarz at the sum level): Write
%   S*sigma^2 = (sum_{i<j} W_{ij}^2 * d_{ij}^2) * (1/n sum lam_i^2).
%   Try Cauchy-Schwarz or mixed-norm inequalities to bound this
%   below by a multiple of Phi = sum V_i^2.
%   Relate W_{ij} to V_i via the identity V_i = sum_{j!=i} 1/d_{ij}.
%
% IDEA 4 (Critical-value formula approach): For n=4, express both
%   S and Phi in terms of the cubic resolvent / critical values.
%   The centered quartic r=x^4-Sx^2-Tx+U has r'=4x^3-2Sx-T, with
%   critical points satisfying Vieta: z1+z2+z3=0, z1z2+z1z3+z2z3=-S/2,
%   z1z2z3=T/4.  Use these to parametrize everything.
%
% IDEA 5 (Haar + Jensen): The Haar characterization gives a
%   matrix-level interpretation.  Phi_n(char poly of M) can be
%   expressed in terms of traces of M.  If the inequality
%   S*sigma^2 >= (3/2)*Phi has a matrix formulation, the Haar
%   averaging might provide a convexity argument.
%
% RECOMMENDED FIRST ATTEMPT: Idea 1 or 2 (reduce to symmetric case
% via convexity/Schur-convexity in the gap variables).
% ----- END NOTES -----

We prove Conjecture~\ref{conj:pointwise} for $n=4$.
Throughout this section, let $r$ be a centered monic quartic with distinct
real roots $\lambda_1<\lambda_2<\lambda_3<\lambda_4$ and gaps
$g_1=\lambda_2-\lambda_1$, $g_2=\lambda_3-\lambda_2$,
$g_3=\lambda_4-\lambda_3$.

\begin{theorem}[Pointwise inequality for quartics]\label{thm:n4}
For every centered monic quartic~$r$ with four distinct real roots,
\begin{equation}\label{eq:n4}
\mathcal{S}(r)\,\sigma^2(r)\;\ge\;\tfrac{3}{2}\,\Phi_4(r).
\end{equation}
Equality holds if and only if the roots are of the form
$\{-a,-b,b,a\}$ with $b/a=\sqrt{5-2\sqrt{6}}$.
\end{theorem}

The proof proceeds in two stages: first the symmetric case $g_1=g_3$,
then reduction of the general case to the symmetric one.

\subsection{Stage 1: The symmetric case \texorpdfstring{$g_1=g_3$}{g1=g3}}

When $g_1=g_3=s$ and $g_2=b$, the roots are $\{-(s+b/2),\,-b/2,\,b/2,
\,s+b/2\}$, and a direct --- though lengthy --- algebraic computation
(factoring the common-denominator form of
$\mathcal{S}\sigma^2-\tfrac{3}{2}\Phi_4$) yields

\begin{equation}\label{eq:sym-factor}
\mathcal{S}\,\sigma^2-\tfrac{3}{2}\,\Phi_4
=\frac{(u+1)(u^2+6u+1)(u^2-10u+1)^2}{[\text{positive denominator}]},
\end{equation}
where $u=(b/(2s))^2\in(0,\infty)$, and the denominator is a product of
even powers of the six pairwise differences (hence positive). Since
$u+1>0$, $u^2+6u+1=(u+3)^2-8>0$ for $u>0$, and $(u^2-10u+1)^2\ge0$,
the right side is non-negative. Equality occurs precisely when
$u^2-10u+1=0$, i.e.\ $u=5-2\sqrt{6}$ (taking the root in $(0,1)$),
corresponding to $b/a=\sqrt{5-2\sqrt{6}}$.

% TODO: verify the u-substitution matches the gap parametrization;
% the Phase 2 computation used roots {-1,-t,t,1} so u=t^2,
% gaps s = 1-t, b = 2t, giving u = t^2, but need to reconcile
% with the present gaps s, b.

\subsection{Stage 2: Reduction to the symmetric case}

% ALGEBRAIC FINDINGS:
%
% (1) PALINDROMIC SYMMETRY: N(g1,g2,g3) = N(g3,g2,g1),
%     reflecting the invariance under root reversal lam_i -> -lam_{5-i}.
%     VERIFIED symbolically via dummy-variable substitution.
%
% (2) SCHUR DERIVATIVE: Both dN/dg1 - dN/dg3 and dDen/dg1 - dDen/dg3
%     are divisible by (g1 - g3), as required by the palindromic symmetry.
%     Write dN/dg1 - dN/dg3 = (g1-g3)*Q(g1,g2,g3), etc.
%
% (3) The SCHUR CONDITION reduces to:
%     F(g1,g2,g3) = Q_dN * Den - N * Q_dDen >= 0,
%     where Q_dN, Q_dDen are the quotients from (2).
%     F has 460 terms at degree 44.  Of these, 446 have positive
%     coefficients and only 14 are negative (in 7 palindromic pairs).
%     The negative terms involve highly unbalanced monomials like
%     g1^{11}g2^8g3^{25}, etc.
%
% (4) POSSIBLE COMPLETION STRATEGIES:
%
%   (4a) Show F >= 0 by grouping: each negative monomial might be
%        dominated by nearby positive monomials via AM-GM.
%        E.g., -A*g1^a g2^b g3^c <= sqrt(B*g1^{a'}g2^{b'}g3^{c'} * C*g1^{a''}g2^{b''}g3^{c''})
%        with matching exponents.  Tedious but finite (only 14 terms to absorb).
%
%   (4b) Use SOS (sum-of-squares) decomposition of F.
%        F is homogeneous of degree 44. An SOS decomposition would
%        give F = sum_k h_k(g1,g2,g3)^2 with h_k homogeneous of
%        degree 22.  This can be found computationally via SDP.
%
%   (4c) Bypass Schur entirely: use convexity of N on SLICES.
%        Fix g2 and the sum g1+g3 = 2s.  Set g1 = s-e, g3 = s+e.
%        By palindromicity, N is an even function of e: N = sum a_k(s,g2)*e^{2k}.
%        Need N >= 0 for e^2 < s^2.  If the polynomial in e^2 has
%        non-negative coefficients --- OR if it's a product of terms
%        with controlled signs --- we're done.
%        Note: the partial check showed P_0 factors cleanly (and is >= 0)
%        but higher P_k have mixed signs.  However, we might write
%        N = P_0(s,g2) + e^2 * R(s,g2,e^2) and show R >= 0 on e^2 in [0,s^2).
%
%   (4d) DIFFERENT INEQUALITY ROUTE: Instead of proving D >= 0 directly,
%        use the integral identity + ODE comparison.  The dissipation
%        identity dot(Phi) = -(2b/(n-1))*S along the flow, combined with
%        the known behavior Phi(p_t) -> 0 as t -> infty, might yield a
%        monotonicity argument for the ratio S*sigma^2/Phi along the flow.
%        KEY QUESTION: is d/dt[S(p_t)*sigma^2(p_t)/Phi(p_t)] <= 0?
%        If so, the ratio decreases along the flow, and the infimum is
%        achieved in the large-t (well-separated) limit.
%
%   (4e) HAAR AVERAGE + CONVEXITY: The convolution p boxplus_n q is
%        the Haar average E_Q[chi_{A+QBQ^T}].  If Phi_n is convex as a
%        function of the polynomial coefficients, then
%        Phi_n(p boxplus q) <= E_Q[Phi_n(chi_{M_Q})],
%        giving information about how Phi changes under convolution.
%        QUESTION: can we express S*sigma^2/Phi in a form where
%        Jensen's inequality applies over the Haar measure?
%
%   MOST PROMISING: (4c) or (4a). Strategy (4c) reduces the problem
%   to showing a univariate polynomial in e^2 is non-negative on [0,s^2)
%   with coefficients that are polynomials in (s,g2). Strategy (4a)
%   requires absorbing 14 negative terms into 446 positive ones.