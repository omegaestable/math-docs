\documentclass[11pt]{article}
\usepackage[margin=1in]{geometry}
\usepackage[T1]{fontenc}
\usepackage{lmodern}
\usepackage{microtype}
\usepackage{amsmath,amssymb,amsthm}
\usepackage{mathtools}
\usepackage[colorlinks=true,allcolors=blue]{hyperref}
\usepackage{enumitem}
\usepackage{booktabs}

%--- Theorem Environments ---
\theoremstyle{plain}
\newtheorem{theorem}{Theorem}[section]
\newtheorem{lemma}[theorem]{Lemma}
\newtheorem{proposition}[theorem]{Proposition}
\newtheorem{corollary}[theorem]{Corollary}
\theoremstyle{definition}
\newtheorem{definition}{Definition}[section]
\theoremstyle{remark}
\newtheorem{remark}{Remark}[section]

%--- Macros ---
\newcommand{\R}{\mathbb{R}}
\newcommand{\Pn}{\mathcal{P}_n}
\newcommand{\PnR}{\mathcal{P}_n^{\R}}

\title{\textbf{The Finite Free Stam Inequality}}
\author{}
\date{}

\begin{document}
\maketitle

\begin{abstract}
The classical Stam inequality asserts the superadditivity of the reciprocal
Fisher information under convolution of independent random variables.
We prove the polynomial analogue in the framework of finite free probability:
for monic, degree-$n$, real-rooted polynomials $p$ and $q$,
\[
  \frac{1}{\Phi_n(p\boxplus_n q)}
  \;\ge\;
  \frac{1}{\Phi_n(p)} + \frac{1}{\Phi_n(q)},
\]
where $\boxplus_n$ is the symmetric additive convolution of Marcus,
Spielman, and Srivastava, and $\Phi_n$ is the finite free Fisher information.
The proof combines an algebraic inequality---the
Score-Gradient Inequality, established via two applications of
Cauchy--Schwarz---with a flow-based argument exploiting
the semigroup structure of~$\boxplus_n$.
We also derive a closed-form expression for~$\Phi_n$ in terms of
the critical values of the polynomial via residue calculus, and use it
to verify the inequality explicitly for cubics.
\end{abstract}

\tableofcontents

%======================================================================
\section{Introduction}\label{sec:intro}
%======================================================================

\subsection{Background and motivation}

In information theory, the Stam inequality~\cite{Stam59} states that if
$X$ and $Y$ are independent random variables with finite Fisher information
$I(X)$ and $I(Y)$, then
\[
  \frac{1}{I(X+Y)} \;\ge\; \frac{1}{I(X)} + \frac{1}{I(Y)}.
\]
This fundamental inequality---equivalent to the entropy power inequality
of Shannon and Stam---captures the principle that convolution of
independent sources strictly increases disorder.

Finite free probability, introduced by Marcus, Spielman, and
Srivastava~\cite{MSS15}, provides a polynomial analogue of free
probability in which random variables are replaced by real-rooted
polynomials and addition by a deterministic convolution
operation~$\boxplus_n$.
Within this framework, the natural question arises:

\begin{quote}
\emph{Does the Stam inequality hold for the finite free additive
convolution?}
\end{quote}

\noindent
The purpose of this paper is to answer this question affirmatively.

\subsection{Statement of the main result}

Let $\Pn$ denote the space of monic polynomials of degree~$n$ with real
coefficients, and $\PnR \subset \Pn$ the subset with all real roots.
For $p \in \PnR$ with distinct roots $\lambda_1 < \cdots < \lambda_n$,
define the \emph{scores}
$V_i = \sum_{j \neq i}(\lambda_i - \lambda_j)^{-1}$
and the \emph{finite free Fisher information}
$\Phi_n(p) = \sum_{i=1}^n V_i^2$.
The \emph{symmetric additive convolution} $p \boxplus_n q$ is recalled
in Section~\ref{sec:prelim}.

\begin{theorem}[Finite Free Stam Inequality]\label{thm:main}
For $p,q \in \PnR$ with positive variance,
\begin{equation}\label{eq:stam}
  \frac{1}{\Phi_n(p \boxplus_n q)}
  \;\ge\;
  \frac{1}{\Phi_n(p)} + \frac{1}{\Phi_n(q)}.
\end{equation}
\end{theorem}

The proof combines three ingredients: the Score-Gradient
Inequality (Theorem~\ref{thm:sgi}), a dissipation identity for the
convolution flow (Lemma~\ref{lem:dissip}), and a case-split
argument exploiting commutativity of~$\boxplus_n$
(Theorem~\ref{thm:general}).
%En route, we obtain a critical-value formula for~$\Phi_n$ via residue
%calculus (Theorem~\ref{thm:critval}) and use it to give an
%explicit verification for $n=3$ (Theorem~\ref{thm:stam3}).

\medskip
\noindent\textbf{Convention.}
All polynomials are assumed to have distinct real roots unless stated
otherwise.
Since such polynomials are dense in~$\PnR$ and all quantities involved
are continuous, inequality~\eqref{eq:stam} extends to all
of~$\PnR$ by a limiting argument.

%======================================================================
\section{Preliminaries}\label{sec:prelim}
%======================================================================

\subsection{Root statistics}

For $p(x) = \prod_{i=1}^n(x-\lambda_i) = \sum_{k=0}^n a_k\,x^{n-k}$
with $a_0=1$, the mean and variance of the root distribution are
\[
  \bar\lambda = \frac{1}{n}\sum_{i=1}^n \lambda_i,
  \qquad
  \sigma^2(p) = \frac{1}{n}\sum_{i=1}^n(\lambda_i - \bar\lambda)^2.
\]

\begin{lemma}\label{lem:var-coeff}
$\sigma^2(p) = \dfrac{(n-1)\,a_1^2}{n^2} - \dfrac{2\,a_2}{n}$.
\end{lemma}

\begin{proof}
By Vieta's formulas, $\sum_i \lambda_i = -a_1$ and
$\sum_{i<j}\lambda_i\lambda_j = a_2$, whence
$\sum_i \lambda_i^2 = a_1^2-2a_2$.
The result follows from
$\sigma^2 = \frac{1}{n}\sum_i\lambda_i^2 - \bar\lambda^2$.
\end{proof}

\subsection{Symmetric additive convolution}

Let $A$ and $B$ be real symmetric matrices with characteristic polynomials
$p$ and~$q$. The finite free additive convolution is defined by averaging
over the orthogonal group:
\[
  (p \boxplus_n q)(x)
  = \int_{O(n)}\det\bigl(xI-(A+QBQ^T)\bigr)\,d\mu_{\mathrm{Haar}}(Q).
\]
By the MSS theorem~\cite{MSS15}, this admits a differential operator
representation: if $q(x)=\sum_{k=0}^n b_k\,x^{n-k}$, then
\begin{equation}\label{eq:mss}
  (p\boxplus_n q)(x) = T_q\,p(x),
  \qquad
  T_q = \sum_{k=0}^n \frac{(n-k)!}{n!}\,b_k\,\partial_x^k.
\end{equation}
The coefficients of $r = p\boxplus_n q$, $r(x)=\sum_k c_k\,x^{n-k}$,
satisfy
\begin{equation}\label{eq:conv-coeff}
  c_k = \sum_{i+j=k}
    \frac{(n-i)!\,(n-j)!}{n!\,(n-k)!}\,a_i\,b_j.
\end{equation}

Two fundamental properties we shall use repeatedly:

\begin{theorem}[{\cite{MSS15}}]\label{thm:preserve}
If $p,q \in \PnR$, then $p\boxplus_n q \in \PnR$.
\end{theorem}

\begin{lemma}[Variance additivity]\label{lem:var-add}
$\sigma^2(p\boxplus_n q) = \sigma^2(p) + \sigma^2(q)$.
\end{lemma}

\begin{proof}
From~\eqref{eq:conv-coeff}, $c_1 = a_1+b_1$ and
$c_2 = a_2+\frac{n-1}{n}\,a_1 b_1+b_2$.
Substituting into Lemma~\ref{lem:var-coeff} and expanding
$(a_1+b_1)^2$, the cross-terms $\frac{2(n-1)a_1 b_1}{n^2}$
and $-\frac{2(n-1)a_1 b_1}{n^2}$ cancel, yielding
$\sigma^2(p\boxplus_n q) = \sigma^2(p)+\sigma^2(q)$.
\end{proof}

\subsection{Scores and Fisher information}

\begin{definition}\label{def:score-fisher}
For $p \in \PnR$ with distinct roots $\lambda_1<\cdots<\lambda_n$, the
\emph{score} at~$\lambda_i$ and the
\emph{finite free Fisher information} are
\[
  V_i = \sum_{j \neq i}\frac{1}{\lambda_i-\lambda_j},
  \qquad
  \Phi_n(p) = \sum_{i=1}^n V_i^2.
\]
The \emph{score-gradient energy} is
$\displaystyle\mathcal{S}(p) = \sum_{i<j}
  \frac{(V_i-V_j)^2}{(\lambda_i-\lambda_j)^2}$.
\end{definition}

\begin{lemma}\label{lem:score-deriv}
$V_i = \dfrac{p''(\lambda_i)}{2\,p'(\lambda_i)}$.
\end{lemma}

\begin{proof}
Since $p'(\lambda_i) = \prod_{j \neq i}(\lambda_i-\lambda_j)$,
differentiating once more yields
$p''(\lambda_i)
= 2\sum_{k \neq i}\prod_{j \neq i,\,j \neq k}(\lambda_i-\lambda_j)
= 2\,p'(\lambda_i)\,V_i$.
\end{proof}

\begin{lemma}[Score identities]\label{lem:score-id}
\begin{enumerate}[label=\textup{(\roman*)},nosep]
  \item $\displaystyle\sum_{i=1}^n V_i = 0$. \label{it:score-sum}
  \item $\displaystyle\sum_{i=1}^n \lambda_i\,V_i
        = \binom{n}{2}$. \label{it:score-root}
  \item $\displaystyle\sum_{i=1}^n (\lambda_i-\bar\lambda)\,V_i
        = \binom{n}{2}$. \label{it:score-centered}
  \item $\displaystyle\Phi_n(p)
        = \sum_{i<j}\frac{V_i-V_j}{\lambda_i-\lambda_j}$.
        \label{it:score-gap}
\end{enumerate}
\end{lemma}

\begin{proof}
\ref{it:score-sum}:
$\sum_i V_i = \sum_{i \neq j}(\lambda_i-\lambda_j)^{-1} = 0$ by
antisymmetry.

\ref{it:score-root}:
$\sum_i \lambda_i V_i
= \sum_{i \neq j}\frac{\lambda_i}{\lambda_i-\lambda_j}
= \sum_{i<j}\bigl(\frac{\lambda_i}{\lambda_i-\lambda_j}
    +\frac{\lambda_j}{\lambda_j-\lambda_i}\bigr)
= \sum_{i<j}1
= \binom{n}{2}$.

\ref{it:score-centered}: Immediate from
\ref{it:score-root} and \ref{it:score-sum}.

\ref{it:score-gap}:
$\sum_i V_i^2
= \sum_{i \neq j}\frac{V_i}{\lambda_i-\lambda_j}
= \sum_{i<j}\frac{V_i-V_j}{\lambda_i-\lambda_j}$.
\end{proof}

\begin{lemma}[Fisher--variance inequality]\label{lem:FV}
$\Phi_n(p)\,\sigma^2(p) \ge \dfrac{n(n-1)^2}{4}$.
\end{lemma}

\begin{proof}
By Cauchy--Schwarz applied to
Lemma~\ref{lem:score-id}\ref{it:score-centered}:
$\frac{n^2(n-1)^2}{4}
\le \bigl(\sum_i(\lambda_i-\bar\lambda)^2\bigr)\bigl(\sum_i V_i^2\bigr)
= n\,\sigma^2(p)\,\Phi_n(p)$.
\end{proof}

% [Commented out: critical-value formula and small-degree cases]
% These sections are not needed for the general proof but are kept
% for reference.
%
%%======================================================================
%\section{A critical-value formula for \texorpdfstring{$\Phi_n$}{Phi\_n}}
%\label{sec:critval}
%%======================================================================
%
%\begin{theorem}\label{thm:critval}
%Let $p \in \PnR$ have distinct roots, and let
%$\zeta_1,\ldots,\zeta_{n-1}$ be the (simple) zeros of $p'$. Then
%\begin{equation}\label{eq:critval}
%  \Phi_n(p)
%  = -\frac{1}{4}\sum_{j=1}^{n-1}\frac{p''(\zeta_j)}{p(\zeta_j)}.
%\end{equation}
%\end{theorem}
%
%\begin{proof}
%By Lemma~\ref{lem:score-deriv},
%$\Phi_n = \frac{1}{4}\sum_{i=1}^n
%  \frac{p''(\lambda_i)^2}{p'(\lambda_i)^2}$.
%Consider the meromorphic function on~$\mathbb{P}^1$
%\[
%  F(x) = \frac{p''(x)^2}{p'(x)\,p(x)}.
%\]
%
%\noindent\emph{Poles at the roots.}
%Since $p$ has a simple zero at~$\lambda_i$ and
%$p'(\lambda_i)\neq 0$,
%$\operatorname{Res}_{x=\lambda_i}F
%= p''(\lambda_i)^2/p'(\lambda_i)^2$.
%Summing: $\sum_i\operatorname{Res}_{\lambda_i}F = 4\Phi_n$.
%
%\noindent\emph{Poles at the critical points.}
%At a simple zero~$\zeta_j$ of~$p'$, we have $p(\zeta_j)\neq 0$
%(by the interlacing of roots and critical points), so
%$\operatorname{Res}_{x=\zeta_j}F = p''(\zeta_j)/p(\zeta_j)$.
%
%\noindent\emph{Pole at infinity.}
%$F(x) = n(n-1)^2/x^3 + O(x^{-4})$ as $x \to \infty$, whence
%$\operatorname{Res}_\infty F = 0$.
%
%\noindent
%The sum of all residues on~$\mathbb{P}^1$ vanishes:
%$4\Phi_n + \sum_j p''(\zeta_j)/p(\zeta_j) = 0$.
%\end{proof}
%
%\begin{remark}\label{rem:critval}
%Formula~\eqref{eq:critval} expresses $\Phi_n$ in terms of the
%critical values $p(\zeta_j)$ and the second derivatives at the
%critical points.
%It generalizes the classical relation between the discriminant and
%critical values, and yields closed-form expressions in low degree.
%\end{remark}
%
%%======================================================================
%\section{Small-degree cases}\label{sec:small}
%%======================================================================
%
%\subsection{The case $n=2$: equality}\label{ssec:n2}
%
%\begin{proposition}\label{prop:n2}
%For $n=2$, inequality~\eqref{eq:stam} holds with equality.
%\end{proposition}
%
%\begin{proof}
%If $p(x)=(x-\lambda_1)(x-\lambda_2)$ with $d=\lambda_1-\lambda_2$,
%then $V_1=1/d$, $V_2=-1/d$, so
%$\Phi_2(p)=2/d^2$ and $\sigma^2(p)=d^2/4$.
%Hence $1/\Phi_2(p) = 2\sigma^2(p)$, and the result follows from
%variance additivity (Lemma~\ref{lem:var-add}).
%\end{proof}
%
%\subsection{The case $n=3$: residue calculus}\label{ssec:n3}
%
%Throughout this subsection all cubics are centered
%($\bar\lambda=0$), which entails no loss of generality since
%$\Phi_n$ and $\sigma^2$ are translation-invariant.
%A centered monic cubic takes the form $r(x) = x^3-Sx+T$ with $S \ge 0$.
%It has three distinct real roots if and only if
%its discriminant $\Delta := 4S^3-27T^2$ is positive;
%we assume $\Delta > 0$ throughout.
%
%\begin{proposition}\label{prop:phi3}
%$\Phi_3(r) = \dfrac{18S^2}{4S^3-27T^2}$.
%\end{proposition}
%
%\begin{proof}
%The critical points of~$r$ are
%$\zeta_\pm = \pm\alpha$ with $\alpha=\sqrt{S/3}$,
%and $r''(x)=6x$. The critical values are
%\[
%  r(\pm\alpha)
%  = T \mp \tfrac{2S^{3/2}}{3\sqrt{3}},
%  \qquad
%  r(\alpha)\,r(-\alpha)
%  = T^2 - \tfrac{4S^3}{27}
%  = -\tfrac{\Delta}{27}.
%\]
%By Theorem~\ref{thm:critval}:
%\[
%  4\Phi_3
%  = -\frac{6\alpha}{r(\alpha)}+\frac{6\alpha}{r(-\alpha)}
%  = 6\alpha\cdot
%    \frac{r(\alpha)-r(-\alpha)}{r(\alpha)\,r(-\alpha)}.
%\]
%Since $r(\alpha)-r(-\alpha) = -4S\alpha/3$ and
%$r(\alpha)\,r(-\alpha) = -\Delta/27$:
%$4\Phi_3 = 6\alpha\cdot\frac{-4S\alpha/3}{-\Delta/27}
%= \frac{216S\alpha^2}{\Delta}
%= \frac{72S^2}{\Delta}$,
%where the last step uses $\alpha^2=S/3$.
%\end{proof}
%
%\begin{proposition}\label{prop:cubic-conv}
%For centered monic cubics $p(x)=x^3-S_1 x+T_1$ and
%$q(x)=x^3-S_2 x+T_2$,
%\[
%  (p\boxplus_3 q)(x) = x^3-(S_1+S_2)\,x+(T_1+T_2).
%\]
%\end{proposition}
%
%\begin{proof}
%Since $a_1=b_1=0$, formula~\eqref{eq:conv-coeff} gives
%$c_1=0$, $c_2=a_2+b_2=-(S_1+S_2)$, and $c_3=a_3+b_3=T_1+T_2$
%(all cross-terms involving $a_1$ or $b_1$ vanish).
%\end{proof}
%
%\begin{theorem}\label{thm:stam3}
%Inequality~\eqref{eq:stam} holds for $n=3$, with equality if and
%only if $T_1=T_2=0$.
%\end{theorem}
%
%\begin{proof}
%From Propositions~\ref{prop:phi3} and~\ref{prop:cubic-conv},
%$1/\Phi_3(r) = \Delta/(18S^2) = 2S/9-3T^2/(2S^2)$.
%Inequality~\eqref{eq:stam} thus reduces, after cancelling the
%linear terms, to
%\begin{equation}\label{eq:cubic-ineq}
%  \frac{(T_1+T_2)^2}{(S_1+S_2)^2}
%  \;\le\;
%  \frac{T_1^2}{S_1^2}+\frac{T_2^2}{S_2^2}.
%\end{equation}
%Set $\alpha=S_1/(S_1+S_2)$, $\beta=1-\alpha$, $u=T_1/S_1$,
%$v=T_2/S_2$.
%The left-hand side becomes $(\alpha u+\beta v)^2$.
%By convexity of $t \mapsto t^2$:
%$(\alpha u+\beta v)^2 \le \alpha u^2+\beta v^2 \le u^2+v^2$,
%establishing~\eqref{eq:cubic-ineq}.
%Equality requires $\alpha u^2+\beta v^2 = u^2+v^2$, i.e.\
%$\beta u^2+\alpha v^2=0$, forcing $u=v=0$.
%\end{proof}

%======================================================================
\section{The Score-Gradient Inequality}\label{sec:sgi}
%======================================================================

The following algebraic inequality is the key input for the general
proof.

\begin{theorem}[Score-Gradient Inequality]\label{thm:sgi}
For $p \in \PnR$ of degree $n \ge 2$ with distinct roots,
\begin{equation}\label{eq:sgi}
  \mathcal{S}(p)\,\sigma^2(p)
  \;\ge\;
  \frac{n-1}{2}\,\Phi_n(p),
\end{equation}
with equality if and only if $V_i = c(\lambda_i-\bar\lambda)$ for
some constant~$c$.
\end{theorem}

\begin{proof}
Write $T = n\,\sigma^2(p)$, $U = \Phi_n(p)$, $S = \mathcal{S}(p)$.
The claim is $S\,T \ge \frac{n(n-1)}{2}\,U$.

\medskip\noindent\textbf{Step~1.}
By Lemma~\ref{lem:score-id}\ref{it:score-centered} and
Cauchy--Schwarz,
\begin{equation}\label{eq:cs1}
  \tfrac{n^2(n-1)^2}{4} \;\le\; T\,U.
\end{equation}

\noindent\textbf{Step~2.}
By Lemma~\ref{lem:score-id}\ref{it:score-gap} and
Cauchy--Schwarz,
\begin{equation}\label{eq:cs2}
  U^2 \;\le\; S\cdot\binom{n}{2}.
\end{equation}

\noindent\textbf{Step~3.}
Combining:
$S\,T
\ge \frac{2U^2}{n(n-1)}\cdot T
= \frac{2U}{n(n-1)}\cdot TU
\ge \frac{2U}{n(n-1)}\cdot\frac{n^2(n-1)^2}{4}
= \frac{n(n-1)}{2}\,U$.

\medskip\noindent\textbf{Equality.}
Equality in~\eqref{eq:cs1} requires $V_i = c(\lambda_i-\bar\lambda)$.
This implies
$\frac{V_i-V_j}{\lambda_i-\lambda_j} = c$ for all $i<j$, which is
precisely the equality condition for~\eqref{eq:cs2}.
Conversely, if $\frac{V_i-V_j}{\lambda_i-\lambda_j} = k$ for all
$i<j$, then $V_i - k\lambda_i$ is constant; since $\sum_i V_i = 0$,
we obtain $V_i = k(\lambda_i-\bar\lambda)$.
\end{proof}

\begin{remark}
The equality condition $V_i = c(\lambda_i-\bar\lambda)$ characterizes,
up to affine transformation, the zeros of the Hermite
polynomial~$H_n$: evaluating the ODE
$H_n''-2xH_n'+2nH_n=0$ at a zero~$x_k$ yields $V_k = x_k$.
For $n=2$ this holds for all distinct root configurations.
%for $n=3$ it reduces to $T=0$, consistent with Theorem~\ref{thm:stam3}.
\end{remark}

%======================================================================
\section{The convolution flow}\label{sec:flow}
%======================================================================

\subsection{The semigroup and the flow}

Fix $p,q \in \PnR$ with $a = \sigma^2(p)>0$ and $b = \sigma^2(q)>0$.

\begin{definition}\label{def:flow}
Introduce the \emph{normalized coefficients}
$\kappa_k(q) = \frac{(n-k)!}{n!}\,b_k$ and the generating polynomial
$K_q(z) = \sum_{k=0}^n \kappa_k(q)\,z^k$.
The convolution formula~\eqref{eq:conv-coeff} is equivalent to
$K_{p\boxplus_n q}(z) = K_p(z)\,K_q(z)$.
Define the \emph{fractional family} by
\[
  K_{q_t}(z) = K_q(z)^{\,t},
  \qquad t \in [0,1],
\]
expanded as a power series and truncated at degree~$n$.
Then $q_0=x^n$, $q_1=q$, $\sigma^2(q_t)=tb$, and
$q_s \boxplus_n q_t = q_{s+t}$.
The \emph{flow polynomial} is $p_t = p \boxplus_n q_t$, satisfying
$\sigma^2(p_t)=a+tb$.
\end{definition}

\begin{remark}\label{rem:qt-status}
For non-integer~$t$, the polynomial $q_t$ need \emph{not} have all
real roots.
For example, taking $n=4$ and
$q=(x{-}100)(x{-}1)(x{+}1)(x{+}100)$, one computes
$\kappa_4(q_{1/2})=\frac{1}{2}\kappa_4-\frac{1}{8}\kappa_2^2<0$,
and $q_{1/2}$ has only two real roots.
This does not affect the proof: only the flow polynomial
$p_t = p\boxplus_n q_t$ needs to be real-rooted, which is established
in Lemma~\ref{lem:flow-real} below.
\end{remark}

\subsection{Perturbation analysis}

\begin{lemma}\label{lem:root-shift}
Let $\lambda_i(t)$ denote the roots of~$p_t$.
Then
$\lambda_i(t{+}h) = \lambda_i(t)+\frac{hb}{n-1}\,V_i(t)+O(h^2)$.
\end{lemma}

\begin{proof}
By the semigroup property, $p_{t+h} = p_t \boxplus_n q_h$ with
$\sigma^2(q_h) = hb$. The coefficients of~$q_h$ satisfy
$b_0=1$, $b_1=0$, $b_2 = -nhb/2+O(h^2)$, so the operator
$T_{q_h}$ acts as
$T_{q_h}\,r(x) = r(x) - \frac{hb}{2(n-1)}\,r''(x)+O(h^2)$.
Setting $\lambda_i(t{+}h)=\lambda_i(t)+\delta_i$ in
$T_{q_h}\,p_t(\lambda_i(t{+}h))=0$ and solving to first order:
$\delta_i
= \frac{hb}{2(n-1)}\cdot\frac{p_t''(\lambda_i)}{p_t'(\lambda_i)}+O(h^2)
= \frac{hb}{n-1}\,V_i(t)+O(h^2)$
by Lemma~\ref{lem:score-deriv}.
\end{proof}

\begin{lemma}\label{lem:phi-change}
$\Phi_n(p_{t+h})
= \Phi_n(p_t) - \dfrac{2hb}{n-1}\,\mathcal{S}(p_t) + O(h^2)$.
\end{lemma}

\begin{proof}
Write $\epsilon = hb/(n-1)$ and suppress the $t$-dependence.
From Lemma~\ref{lem:root-shift}, the perturbed scores are
\[
  V_i^{(h)}
  = \sum_{j \neq i}\frac{1}{(\lambda_i-\lambda_j)
    +\epsilon(V_i-V_j)+O(h^2)}
  = V_i - \epsilon\sum_{j \neq i}
    \frac{V_i-V_j}{(\lambda_i-\lambda_j)^2}+O(h^2).
\]
Squaring and summing:
$\Phi_n(p_{t+h})
= \sum_i V_i^2
  -2\epsilon\sum_{i \neq j}
    \frac{V_i(V_i-V_j)}{(\lambda_i-\lambda_j)^2}+O(h^2)$.
Pairing $(i,j)$ with $(j,i)$:
$\sum_{i \neq j}\frac{V_i(V_i-V_j)}{(\lambda_i-\lambda_j)^2}
= \sum_{i<j}\frac{(V_i-V_j)^2}{(\lambda_i-\lambda_j)^2}
= \mathcal{S}(p_t)$.
\end{proof}

\begin{lemma}\label{lem:flow-real}
For every $t\in[0,1]$, $p_t$ has $n$ simple real roots.
\end{lemma}

\begin{proof}
The coefficients of $p_t$ are smooth in~$t$
(Definition~\ref{def:flow}), so the roots $\lambda_i(t)$ vary
continuously.
Since $p_0=p$ has simple real roots, there is a maximal interval
$[0,T)$ on which $p_t$ has simple real roots; continuity gives
$T>0$.

For $t<T$, the root-shift formula (Lemma~\ref{lem:root-shift})
implies
$\dot\delta_i(t)
= \frac{b}{n-1}\bigl(V_{i+1}(t)-V_i(t)\bigr)$
for consecutive gaps $\delta_i = \lambda_{i+1}-\lambda_i$.
Expanding the score difference:
\[
  V_{i+1}-V_i
  = \frac{2}{\delta_i}
    - \delta_i\!\!\sum_{j \neq i,\,i+1}
      \frac{1}{(\lambda_{i+1}{-}\lambda_j)(\lambda_i{-}\lambda_j)},
\]
so $\dot\delta_i \ge \frac{2b}{(n{-}1)\delta_i}-C$ with $C$
bounded as long as all gaps are positive.
Hence
$\frac{d}{dt}\delta_i^2 = 2\delta_i\,\dot\delta_i
\ge \frac{4b}{n{-}1}-2C\delta_i$.
For $\delta_i < \delta_0 := 2b/(C(n{-}1))$, this is strictly
positive, so no gap can reach zero in finite time.
Since the roots at time~$T$ remain separated, continuity extends
simple real-rootedness past~$T$, forcing $T \ge 1$.
\end{proof}

\subsection{Dissipation and the integral identity}

\begin{lemma}[Dissipation]\label{lem:dissip}
$\dfrac{d}{dt}\Phi_n(p_t)
= -\dfrac{2b}{n-1}\,\mathcal{S}(p_t)$.
\end{lemma}

\begin{proof}
By Lemma~\ref{lem:phi-change},
$\frac{\Phi_n(p_{t+h})-\Phi_n(p_t)}{h}
= -\frac{2b}{n-1}\,\mathcal{S}(p_t)+O(h)$.
Since $p_t$ has simple roots for all $t\in[0,1]$
(Lemma~\ref{lem:flow-real}), the scores $V_i(t)$ and hence
$\mathcal{S}(p_t)$ are continuous in~$t$.
Taking $h\to 0$ yields the result.
\end{proof}

\begin{corollary}[Integral identity]\label{cor:integral}
\begin{equation}\label{eq:integral}
  \frac{1}{\Phi_n(p\boxplus_n q)}-\frac{1}{\Phi_n(p)}
  = \frac{2b}{n-1}\int_0^1
    \frac{\mathcal{S}(p_t)}{\Phi_n(p_t)^2}\,dt.
\end{equation}
\end{corollary}

\begin{proof}
Set $f(t)=1/\Phi_n(p_t)$. By the chain rule and
Lemma~\ref{lem:dissip},
$f'(t) = \frac{2b}{n-1}\cdot\frac{\mathcal{S}(p_t)}{\Phi_n(p_t)^2}
\ge 0$.
By the Fundamental Theorem of Calculus,
$f(1)-f(0) = \int_0^1 f'(t)\,dt$.
Substituting: $f(0) = 1/\Phi_n(p_0) = 1/\Phi_n(p)$ and
$f(1) = 1/\Phi_n(p_1) = 1/\Phi_n(p\boxplus_n q)$
yields~\eqref{eq:integral}.
\end{proof}

%======================================================================
\section{Proof of the main theorem}\label{sec:general}
%======================================================================

\begin{theorem}\label{thm:general}
Inequality~\eqref{eq:stam} holds for every $n \ge 2$.
\end{theorem}

\begin{proof}
Write $a=\sigma^2(p)$ and $b=\sigma^2(q)$.

\medskip
\noindent\textbf{Step~1 (Differential inequality).}
The Score-Gradient Inequality (Theorem~\ref{thm:sgi}) applied to
$p_t$ gives
$\mathcal{S}(p_t) \ge
  \frac{(n-1)\,\Phi_n(p_t)}{2\,\sigma^2(p_t)}$.
Substituting into Lemma~\ref{lem:dissip}:
\[
  \frac{d}{dt}\Phi_n(p_t)
  \;\le\;
  -\frac{b}{a+tb}\,\Phi_n(p_t).
\]
Integrating $\bigl(\log\Phi_n(p_t)\bigr)' \le -b/(a+tb)$
from $0$ to~$t$:
\begin{equation}\label{eq:ode-bound}
  \frac{1}{\Phi_n(p_t)}
  \;\ge\;
  \frac{a+tb}{a\,\Phi_n(p)}.
\end{equation}

\medskip
\noindent\textbf{Step~2 (Forward bound).}
From Corollary~\ref{cor:integral} and the Score-Gradient Inequality:
\[
  \frac{1}{\Phi_n(p\boxplus_n q)}-\frac{1}{\Phi_n(p)}
  \;\ge\;
  b\int_0^1\frac{dt}{(a+tb)\,\Phi_n(p_t)}.
\]
Substituting~\eqref{eq:ode-bound}, the factor $(a+tb)$ cancels:
\begin{equation}\label{eq:fwd}
  \frac{1}{\Phi_n(p\boxplus_n q)}
  \;\ge\;
  \frac{a+b}{a\,\Phi_n(p)}.
\end{equation}

\medskip
\noindent\textbf{Step~3 (Reverse bound).}
Since $p\boxplus_n q = q\boxplus_n p$, repeating Steps~1--2 with
$p$ and $q$ interchanged yields
\begin{equation}\label{eq:rev}
  \frac{1}{\Phi_n(p\boxplus_n q)}
  \;\ge\;
  \frac{a+b}{b\,\Phi_n(q)}.
\end{equation}

\medskip
\noindent\textbf{Step~4 (Conclusion).}
Exactly one of the following holds:
\begin{enumerate}[label=(\alph*),nosep]
  \item $b\,\Phi_n(q) \ge a\,\Phi_n(p)$.
    Then $\frac{b}{a\,\Phi_n(p)} \ge \frac{1}{\Phi_n(q)}$,
    and~\eqref{eq:fwd} gives
    $\frac{1}{\Phi_n(p\boxplus_n q)}
    \ge \frac{1}{\Phi_n(p)}+\frac{b}{a\,\Phi_n(p)}
    \ge \frac{1}{\Phi_n(p)}+\frac{1}{\Phi_n(q)}$.

  \item $a\,\Phi_n(p) \ge b\,\Phi_n(q)$.
    Then $\frac{a}{b\,\Phi_n(q)} \ge \frac{1}{\Phi_n(p)}$,
    and~\eqref{eq:rev} gives
    $\frac{1}{\Phi_n(p\boxplus_n q)}
    \ge \frac{1}{\Phi_n(q)}+\frac{a}{b\,\Phi_n(q)}
    \ge \frac{1}{\Phi_n(q)}+\frac{1}{\Phi_n(p)}$.
    \qedhere
\end{enumerate}
\end{proof}

\begin{remark}
The forward bound~\eqref{eq:fwd} and reverse bound~\eqref{eq:rev}
are each strictly stronger than the Stam inequality in their
respective regimes.
Averaging them yields the \emph{half-Stam inequality}
$\frac{2}{\Phi_n(p\boxplus_n q)}
\ge \frac{1}{\Phi_n(p)}+\frac{1}{\Phi_n(q)}$,
from which the full inequality is recovered via the case split.
\end{remark}

\begin{remark}
Strict inequality holds generically.
Equality in~\eqref{eq:stam} requires that
$V_i(p_t) = c(t)\bigl(\lambda_i(t)-\bar\lambda(t)\bigr)$ for all
$t \in [0,1]$, which forces both $p$ and $q$ to have roots at
affinely rescaled zeros of the Hermite polynomial~$H_n$.
For $n=2$ every polynomial satisfies this.
%for $n=3$ it reduces to $T_1=T_2=0$ (Theorem~\ref{thm:stam3}).
\end{remark}

%======================================================================
\begin{thebibliography}{9}

\bibitem{MSS15}
A.~Marcus, D.~A.~Spielman, and N.~Srivastava,
\emph{Interlacing families {II}: Mixed characteristic polynomials
and the {K}adison--{S}inger problem},
Ann.\ of Math.\ \textbf{182} (2015), 327--350.

\bibitem{Stam59}
A.~J.~Stam,
\emph{Some inequalities satisfied by the quantities of information
of {F}isher and {S}hannon},
Inform.\ Control \textbf{2} (1959), 101--112.

\end{thebibliography}

\end{document}
