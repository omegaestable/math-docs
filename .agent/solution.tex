\documentclass[11pt,a4paper]{article}
\usepackage[utf8]{inputenc}
\usepackage[T1]{fontenc}
\usepackage{microtype}
\usepackage[margin=1in]{geometry}
\usepackage{amsmath, amssymb, amsthm}
\usepackage{mathtools}
\usepackage{lmodern}
\usepackage[colorlinks=true, allcolors=blue]{hyperref}
\usepackage{enumitem}
\usepackage{booktabs}
\usepackage{fancyhdr}
\usepackage[parfill]{parskip}

%--- Theorem Environments ---
\theoremstyle{plain}
\newtheorem{theorem}{Theorem}[section]
\newtheorem{lemma}[theorem]{Lemma}
\newtheorem{proposition}[theorem]{Proposition}
\newtheorem{corollary}[theorem]{Corollary}
\newtheorem{conjecture}[theorem]{Conjecture}

\theoremstyle{definition}
\newtheorem{definition}{Definition}[section]

\theoremstyle{remark}
\newtheorem{remark}{Remark}[section]

%--- Macros ---
\DeclareMathOperator{\Tr}{Tr}
\DeclareMathOperator{\diag}{diag}
\DeclareMathOperator{\sgn}{sgn}

\newcommand{\E}{\mathbb{E}}
\newcommand{\R}{\mathbb{R}}
\newcommand{\Pn}{\mathcal{P}_n}
\newcommand{\PnR}{\mathcal{P}_n^{\mathbb{R}}}

%--- Title Info ---
\title{\textbf{The Finite Free Stam Inequality}}
\author{}
\date{}

%==============================================================================
\begin{document}
%==============================================================================

\maketitle

\begin{abstract}
\noindent The classical Stam inequality is a cornerstone of information theory,
bounding the Fisher information of a sum of independent random variables.
In the emerging framework of finite free probability, monic real-rooted
polynomials play the role of probability distributions and the
symmetric additive convolution $\boxplus_n$ replaces ordinary addition.

We propose a polynomial analogue of the Stam inequality in this
setting. Concretely, for $p, q \in \mathcal{P}_n^{\mathbb{R}}$ with finite
free Fisher information $\Phi_n$:
\[
\frac{1}{\Phi_n(p \boxplus_n q)} \ge \frac{1}{\Phi_n(p)} + \frac{1}{\Phi_n(q)},
\]
with equality if and only if $n = 2$. We prove that this inequality is
equivalent to a "Harmonic Regularization" conjecture for the efficiency ratio
$\eta(p) = \Phi_n(p)\sigma^2(p)$, which is strongly supported by numerical evidence.
\end{abstract}

\tableofcontents

%==============================================================================
\section{Introduction}
%==============================================================================

The classical Stam inequality states that for independent random variables
$X, Y$ with Fisher information $I(X)$ and $I(Y)$:
\[
\frac{1}{I(X+Y)} \ge \frac{1}{I(X)} + \frac{1}{I(Y)}.
\]

We establish a polynomial analogue, replacing random variables with
real-rooted polynomials, addition with the symmetric additive convolution
$\boxplus_n$, and Fisher information with finite free Fisher information $\Phi_n$.

%==============================================================================
\section{Polynomials and Root Statistics}
%==============================================================================

Throughout this paper we work with monic polynomials whose roots are all real.
Let $\Pn$ denote the set of monic degree-$n$ polynomials with real coefficients,
and let $\PnR \subset \Pn$ denote the subset of those with all real roots.
Every $p \in \PnR$ factors as $p(x) = \prod_{i=1}^n (x - \lambda_i)$ with
$\lambda_1, \ldots, \lambda_n \in \R$, so the root configuration carries all
the information about $p$.

In analogy with probability theory, we attach to each $p \in \PnR$ a \emph{mean}
and \emph{variance} computed from its roots:
\begin{align*}
\mu(p) &= \tfrac{1}{n}\textstyle\sum_{i=1}^n \lambda_i, &
\sigma^2(p) &= \tfrac{1}{n}\textstyle\sum_{i=1}^n (\lambda_i - \mu)^2, &
\tilde{\lambda}_i &= \lambda_i - \mu.
\end{align*}
The centered roots $\tilde{\lambda}_i$ satisfy $\sum_i \tilde{\lambda}_i = 0$.
The variance $\sigma^2(p)$ measures the spread of the root configuration.

A useful observation is that $\mu$ and $\sigma^2$ can be read directly from
the coefficients of $p$.

\begin{lemma}[Variance Formula] \label{lem:var}
For $p(x) = x^n + a_1 x^{n-1} + a_2 x^{n-2} + \cdots \in \PnR$:
\[
\sigma^2(p) = \frac{(n-1)a_1^2}{n^2} - \frac{2a_2}{n}.
\]
\end{lemma}

\begin{proof}
By Vieta's formulas, $\sum_i \lambda_i = -a_1$ and $\sum_{i<j} \lambda_i\lambda_j = a_2$.
Since $\sum_i \lambda_i^2 = (\sum_i \lambda_i)^2 - 2\sum_{i<j}\lambda_i\lambda_j = a_1^2 - 2a_2$:
\[
\sigma^2(p) = \frac{1}{n}\sum_i \lambda_i^2 - \mu^2 = \frac{a_1^2 - 2a_2}{n} - \frac{a_1^2}{n^2}
= \frac{(n-1)a_1^2}{n^2} - \frac{2a_2}{n}. \qedhere
\]
\end{proof}

This coefficient-level formula will be essential in Section~\ref{sec:key-lemmas},
where we prove that the variance is additive under the finite free convolution $\boxplus_n$.


%==============================================================================
\section{The Symmetric Additive Convolution}
%==============================================================================

The finite free additive convolution $p \boxplus_n q$ can be defined in two
equivalent ways: as an expected characteristic polynomial (the \emph{matrix
average definition}) or via an explicit coefficient formula (the \emph{algebraic
definition}). We establish both and prove their equivalence.

%------------------------------------------------------------------------------
\subsection{The Matrix Average Definition}
%------------------------------------------------------------------------------

\begin{definition}[Matrix Average] \label{def:rm_conv}
For $n \times n$ symmetric matrices $A$ and $B$ with characteristic polynomials
$p$ and $q$, define:
\[
p \boxplus_n q \coloneqq \E_{Q \sim \mathrm{Haar}(O(n))} [\det(xI - (A + QBQ^T))].
\]
\end{definition}

\begin{theorem}[Well-Definedness] \label{thm:well_def}
The polynomial $p \boxplus_n q$ depends only on $p$ and $q$, not on the choice
of $A$ and $B$.
\end{theorem}

\begin{proof}
If $A'$ has the same characteristic polynomial as $A$, then $A = P \Lambda P^T$
and $A' = P' \Lambda (P')^T$ for orthogonal $P, P'$ and diagonal $\Lambda$.
Similarly $B = R \Gamma R^T$ and $B' = R' \Gamma (R')^T$.

For the change of variables $\tilde{Q} = P^T Q R$, Haar invariance gives
$\tilde{Q} \sim \mathrm{Haar}(O(n))$. Then:
\[
\E_Q[\det(xI - A - QBQ^T)] = \E_{\tilde{Q}}[\det(xI - \Lambda - \tilde{Q}\Gamma\tilde{Q}^T)].
\]
The same calculation for $A', B'$ yields the identical expression.
\end{proof}

\begin{proposition}[Commutativity and Identity] \label{prop:basic}
The convolution $\boxplus_n$ is commutative and has identity $x^n$.
\end{proposition}

\begin{proof}
\textbf{Commutativity:} For any $Q \in O(n)$, conjugating $xI - A - QBQ^T$ by $Q^T$ gives:
\[
\det(xI - A - QBQ^T) = \det(xI - Q^TAQ - B).
\]
Since $\tilde{Q} = Q^T$ is also Haar-distributed,
$\E_Q[\det(xI - A - QBQ^T)] = \E_Q[\det(xI - B - QAQ^T)]$.

\textbf{Identity:} If $q(x) = x^n$, then $B = 0$, so
$p \boxplus_n x^n = \E_Q[\det(xI - A)] = p(x)$.
\end{proof}

%------------------------------------------------------------------------------
\subsection{The Algebraic Definition and Equivalence}
%------------------------------------------------------------------------------

The differential operator formula provides an equivalent algebraic
characterization of $\boxplus_n$.

\begin{definition}[The Operator $T_q$]
For a monic polynomial $q(x) = \sum_{k=0}^n b_k x^{n-k}$ with $b_0 = 1$,
define the linear operator:
\[
T_q \coloneqq \sum_{k=0}^n \frac{(n-k)!}{n!} b_k \partial_x^k,
\]
where $\partial_x^k$ denotes the $k$-th derivative with respect to $x$.
\end{definition}

\begin{theorem}[Differential Operator Representation] \label{thm:diff_op}
For monic polynomials $p, q \in \Pn$:
\[
(p \boxplus_n q)(x) = T_q p(x).
\]
\end{theorem}

\begin{proof}
Let $A = \diag(\lambda_1, \ldots, \lambda_n)$ and $B = \diag(\gamma_1, \ldots, \gamma_n)$
be diagonal matrices with eigenvalues equal to the roots of $p$ and $q$ respectively.
We compute $\E_Q[\det(xI - A - QBQ^T)]$ for $Q$ Haar-distributed on $O(n)$.

\textit{Step 1: Expand the determinant using multilinearity.}

Write the $i$-th row of $xI - A - QBQ^T$ as:
\[
\text{row}_i = \underbrace{(0, \ldots, x - \lambda_i, \ldots, 0)}_{\text{row}_i(xI - A)}
- \underbrace{(P_{i1}, P_{i2}, \ldots, P_{in})}_{\text{row}_i(QBQ^T)},
\]
where we write $P = QBQ^T$ for brevity. Since the determinant is multilinear in its rows:
\[
\det(xI - A - P) = \sum_{S \subseteq [n]} (-1)^{|S|} \det(N^{(S)}),
\]
where $N^{(S)}$ is the matrix with row $i$ equal to $\text{row}_i(P)$ if $i \in S$,
and $\text{row}_i(xI - A)$ if $i \notin S$. The factor $(-1)^{|S|}$ accounts for the minus signs.

\textit{Step 2: Use the diagonal structure to factor $\det(N^{(S)})$.}

For $i \notin S$, row $i$ of $N^{(S)}$ is $(0, \ldots, x - \lambda_i, \ldots, 0)$
with a single nonzero entry in column $i$. In the Leibniz formula:
\[
\det(N^{(S)}) = \sum_{\sigma \in S_n} \sgn(\sigma) \prod_{i=1}^n N^{(S)}_{i,\sigma(i)},
\]
if $\sigma(i) \neq i$ for any $i \notin S$, that factor is zero. So only permutations
with $\sigma(i) = i$ for all $i \notin S$ contribute.

Such permutations fix $[n] \setminus S$ and permute $S$. The determinant factors:
\[
\det(N^{(S)}) = \prod_{i \notin S}(x - \lambda_i) \cdot \det(P_S),
\]
where $P_S = (P_{ij})_{i,j \in S}$ is the $|S| \times |S|$ principal submatrix of $P = QBQ^T$.

\textit{Step 3: Compute the Haar expectation.}

3a. \textbf{Substitute the factorization.}

From Step 2, we have $\det(N^{(S)}) = \prod_{i \notin S}(x - \lambda_i) \cdot \det(P_S)$.
Substituting into the multilinearity expansion:
\[
\det(xI - A - QBQ^T) = \sum_{S \subseteq [n]} (-1)^{|S|} \prod_{i \notin S}(x - \lambda_i) \cdot \det(P_S).
\]
Taking expectations (the product $\prod_{i \notin S}(x - \lambda_i)$ is deterministic):
\[
\E_Q[\det(xI - A - QBQ^T)] = \sum_{S \subseteq [n]} (-1)^{|S|} \prod_{i \notin S}(x - \lambda_i) \cdot \E_Q[\det(P_S)].
\]

3b. \textbf{Compute $\sum_{|S|=k} \det((QBQ^T)_S)$.}

We first establish a deterministic identity. For any orthogonal matrix $Q$, the sum of all
$k \times k$ principal minors of $QBQ^T$ equals the $k$-th elementary symmetric polynomial:
\[
\sum_{|S|=k} \det\bigl((QBQ^T)_S\bigr) = e_k(\gamma_1, \ldots, \gamma_n).
\]

\textit{Proof of identity.} By the Cauchy-Binet formula, for any $n \times n$ matrix $M = QBQ^T$:
\[
\det(M_S) = \sum_{|T|=k} \det(Q_{S,T}) \det(B_T) \det(Q_{S,T}^T),
\]
where $Q_{S,T}$ is the $k \times k$ submatrix of $Q$ with rows in $S$ and columns in $T$,
and $B_T = \diag(\gamma_j : j \in T)$ has $\det(B_T) = \prod_{j \in T} \gamma_j$.
Since $\det(Q_{S,T}^T) = \det(Q_{S,T})$:
\[
\sum_{|S|=k} \det(M_S) = \sum_{|S|=k} \sum_{|T|=k} \det(Q_{S,T})^2 \prod_{j \in T} \gamma_j
= \sum_{|T|=k} \prod_{j \in T} \gamma_j \cdot \underbrace{\sum_{|S|=k} \det(Q_{S,T})^2}_{= 1}.
\]
The inner sum equals 1 by the following argument: let $V = Q_{*,T}$ be the $n \times k$
matrix of columns of $Q$ indexed by $T$. These columns are orthonormal since $Q$ is orthogonal,
so $V^T V = I_k$. By the Cauchy--Binet formula,
$\sum_{|S|=k} \det(V_{S,*})^2 = \det(V^T V) = \det(I_k) = 1$. Therefore:
\[
\sum_{|S|=k} \det\bigl((QBQ^T)_S\bigr) = \sum_{|T|=k} \prod_{j \in T} \gamma_j
= e_k(\gamma_1, \ldots, \gamma_n).
\]

\textit{Taking expectations.} Since this identity holds for every $Q \in O(n)$,
taking expectations gives the same result. There are $\binom{n}{k}$ subsets of size $k$,
and they all yield the same expected minor: for any two sets $S_1, S_2$ with $|S_1| = |S_2| = k$,
there is a permutation matrix $\Pi$ with $\Pi(S_1) = S_2$, and since $\Pi Q$ is also Haar-distributed
(by left invariance), $\E_Q[\det((QBQ^T)_{S_1})] = \E_Q[\det((QBQ^T)_{S_2})]$. Therefore:
\[
\E_Q[\det((QBQ^T)_S)] = \frac{e_k(\gamma_1, \ldots, \gamma_n)}{\binom{n}{k}}.
\]

3c. \textbf{Sum over subsets of fixed size.}

Group the sum by $|S| = k$. Since $\E_Q[\det(P_S)]$ depends only on $|S| = k$:
\[
\sum_{|S|=k} (-1)^k \prod_{i \notin S}(x - \lambda_i) \cdot \E_Q[\det(P_S)]
= (-1)^k \cdot \frac{e_k(\gamma)}{\binom{n}{k}} \cdot \sum_{|S|=k} \prod_{i \notin S}(x - \lambda_i).
\]

3d. \textbf{Identify the derivative of $p(x)$.}

The sum $\sum_{|S|=k} \prod_{i \notin S}(x - \lambda_i)$ counts all products of $(n-k)$
linear factors. By the product rule:
\[
p^{(k)}(x) = \frac{d^k}{dx^k} \prod_{i=1}^n (x - \lambda_i)
= k! \sum_{|S|=k} \prod_{i \notin S}(x - \lambda_i).
\]
Hence:
\[
\sum_{|S|=k} \prod_{i \notin S}(x - \lambda_i) = \frac{p^{(k)}(x)}{k!}.
\]

3e. \textbf{Simplify the coefficients.}

Combining Steps 3c and 3d:
\[
\sum_{|S|=k} (-1)^k \prod_{i \notin S}(x - \lambda_i) \cdot \E_Q[\det(P_S)]
= (-1)^k e_k(\gamma) \cdot \frac{1}{\binom{n}{k}} \cdot \frac{p^{(k)}(x)}{k!}.
\]
Using $\frac{1}{\binom{n}{k} \cdot k!} = \frac{(n-k)!}{n!}$:
\[
= (-1)^k e_k(\gamma) \cdot \frac{(n-k)!}{n!} \cdot p^{(k)}(x).
\]

3f. \textbf{Assemble the final formula.}

Summing over $k = 0, 1, \ldots, n$:
\[
\E_Q[\det(xI - A - QBQ^T)] = \sum_{k=0}^n (-1)^k e_k(\gamma) \cdot \frac{(n-k)!}{n!} \cdot p^{(k)}(x).
\]
By Vieta's formulas, $b_k = (-1)^k e_k(\gamma)$. Therefore:
\[
\E_Q[\det(xI - A - QBQ^T)] = \sum_{k=0}^n \frac{(n-k)!}{n!} b_k \cdot p^{(k)}(x) = T_q p(x). \qedhere
\]
\end{proof}

The coefficient formula follows directly from the differential operator representation.

\begin{theorem}[Coefficient Formula] \label{thm:coeff}
If $p(x) = \sum_{i=0}^n a_i x^{n-i}$ and $q(x) = \sum_{j=0}^n b_j x^{n-j}$ are monic,
then $(p \boxplus_n q)(x) = \sum_{k=0}^n c_k x^{n-k}$, where:
\[
c_k = \sum_{i+j=k} \frac{(n-i)!(n-j)!}{n!(n-k)!} a_i b_j.
\]
\end{theorem}

\begin{proof}
Apply $T_q$ to $p(x) = \sum_{i=0}^n a_i x^{n-i}$.
Since $\partial_x^j(x^{n-i}) = \frac{(n-i)!}{(n-i-j)!}x^{n-i-j}$ for $j \le n-i$ (and zero otherwise):
\[
T_q p(x) = \sum_{i,j} \frac{(n-j)!}{n!} b_j a_i \cdot \frac{(n-i)!}{(n-i-j)!} x^{n-i-j}.
\]
Setting $k = i+j$, we get coefficient $c_k = \sum_{i+j=k} \frac{(n-i)!(n-j)!}{n!(n-k)!} a_i b_j$.
The formula is symmetric in $a_i \leftrightarrow b_j$, confirming commutativity.
\end{proof}

\begin{corollary}[Associativity] \label{cor:assoc}
The convolution $\boxplus_n$ is associative:
$(p \boxplus_n q) \boxplus_n r = p \boxplus_n (q \boxplus_n r)$.
\end{corollary}

\begin{proof}
Let $p, q, r$ have coefficients $a_i, b_j, c_m$. Iterating the coefficient formula from
Theorem~\ref{thm:coeff}, the coefficient of $x^{n-k}$ in $(p \boxplus_n q) \boxplus_n r$ is:
\[
\sum_{i+j+m=k} \frac{(n-i)!(n-j)!}{n!(n-i-j)!} \cdot \frac{(n-i-j)!(n-m)!}{n!(n-k)!} \cdot a_i b_j c_m
= \sum_{i+j+m=k} \frac{(n-i)!(n-j)!(n-m)!}{(n!)^2(n-k)!} \cdot a_i b_j c_m.
\]
The weight $\frac{(n-i)!(n-j)!(n-m)!}{(n!)^2(n-k)!}$ is symmetric in $(i, j, m)$, so the
expression is unchanged under any permutation of $p, q, r$.
In particular, $(p \boxplus_n q) \boxplus_n r = p \boxplus_n (q \boxplus_n r)$.
\end{proof}

%------------------------------------------------------------------------------
\subsection{Preservation of Real-Rootedness}
%------------------------------------------------------------------------------

The convolution preserves real-rootedness. The proof uses interlacing families,
following Marcus, Spielman, and Srivastava \cite{MSS15}.

\begin{definition}[Interlacing]
Polynomials $f, g$ of degree $n$ \textbf{interlace} if their roots alternate.
A family $\{f_s\}$ is an \textbf{interlacing family} if there exists a single
polynomial $h$ that interlaces every member $f_s$.
\end{definition}

\begin{lemma}[Convex Combinations Preserve Interlacing] \label{lem:convex_interlace}
If real-rooted polynomials $f_1, \ldots, f_m$ share a common interlacing $h$,
then any convex combination is real-rooted.
\end{lemma}

\begin{proof}[Proof sketch]
By the intermediate value theorem, each root of $tf + (1-t)g$ lies in an interval
$[\alpha_i, \alpha_{i+1}]$ determined by $h$. Induction extends to $m$ polynomials.
\end{proof}

\begin{lemma}[Rank-One Perturbation Interlacing] \label{lem:rank1}
For symmetric $A$ and unit vector $v$, the polynomials $\det(xI - A)$ and
$\det(xI - A - tvv^T)$ interlace for $t > 0$.
\end{lemma}

\begin{proof}[Proof sketch]
By the matrix determinant lemma, the roots of $\det(xI - A - tvv^T)$ solve
$1 = t\sum_i \frac{c_i^2}{x - \lambda_i}$. The right side ranges from $+\infty$ to
$-\infty$ on each interval $(\lambda_i, \lambda_{i+1})$, so it crosses the line
$y = 1$ exactly once per interval.
\end{proof}

\begin{theorem}[Real-Rootedness] \label{thm:mss_roots}
If $p, q \in \PnR$, then $p \boxplus_n q \in \PnR$.
\end{theorem}

\begin{proof}[Proof sketch]
Decompose $QBQ^T = \sum_k \gamma_k (Qe_k)(Qe_k)^T$ as rank-one updates.
By Lemma~\ref{lem:rank1}, successive updates preserve interlacing, so
$\{f_Q = \det(xI - A - QBQ^T)\}_{Q \in O(n)}$ forms an interlacing family.
By Lemma~\ref{lem:convex_interlace}, the expected polynomial $p \boxplus_n q = \E_Q[f_Q]$
is real-rooted.
\end{proof}

%==============================================================================
\section{Finite Free Fisher Information}
%==============================================================================

\begin{definition}
For $p \in \PnR$ with distinct roots $\lambda_1, \ldots, \lambda_n$, the
\textbf{score function} at $\lambda_i$ and the \textbf{Fisher information} are:
\[
V_i = \sum_{j \neq i} \frac{1}{\lambda_i - \lambda_j}, \qquad \Phi_n(p) = \sum_{i=1}^n V_i^2.
\]
\end{definition}

If $p$ has a repeated root, we define $\Phi_n(p) = \infty$.

%------------------------------------------------------------------------------
\subsection{The Repeated-Root Convention}
%------------------------------------------------------------------------------

We define $\Phi_n(p) = \infty$ whenever $p$ has a repeated root
(i.e.\ $\lambda_i = \lambda_j$ for some $i \neq j$). This is natural: the score
$V_i = \sum_{j \neq i} \frac{1}{\lambda_i - \lambda_j}$ diverges as two roots collide,
so the Fisher information blows up.

Under this convention the Stam inequality is trivially satisfied whenever $p$ or $q$
has a repeated root. Indeed, if $\Phi_n(p) = \infty$ then $\frac{1}{\Phi_n(p)} = 0$,
and the right-hand side can only decrease:
\[
\frac{1}{\Phi_n(p \boxplus_n q)} \ge 0 = \frac{1}{\Phi_n(p)} + \frac{1}{\Phi_n(q)}.
\]

\textbf{Standing assumption.} For the remainder of the paper we therefore assume that
all polynomials in $\PnR$ have \emph{distinct} roots, so that $\Phi_n$ is finite and
the inequality is non-trivial.

%==============================================================================
\section{Key Lemmas} \label{sec:key-lemmas}
%==============================================================================

\begin{lemma}[Score-Root Identity] \label{lem:identity}
$\displaystyle\sum_{i=1}^n \tilde{\lambda}_i V_i = \frac{n(n-1)}{2}$.
\end{lemma}

\begin{proof}
Since $\lambda_i - \lambda_j = \tilde{\lambda}_i - \tilde{\lambda}_j$, we have:
\[
\sum_{i=1}^n \tilde{\lambda}_i V_i = \sum_{i \neq j} \frac{\tilde{\lambda}_i}{\tilde{\lambda}_i - \tilde{\lambda}_j} \eqqcolon S.
\]

Using the identity $\frac{a}{a-b} = 1 + \frac{b}{a-b}$:
\[
S = \sum_{i \neq j} 1 + \sum_{i \neq j} \frac{\tilde{\lambda}_j}{\tilde{\lambda}_i - \tilde{\lambda}_j}
= n(n-1) + \sum_{i \neq j} \frac{\tilde{\lambda}_j}{\tilde{\lambda}_i - \tilde{\lambda}_j}.
\]

Relabeling indices $i \leftrightarrow j$ in the second sum:
\[
\sum_{i \neq j} \frac{\tilde{\lambda}_j}{\tilde{\lambda}_i - \tilde{\lambda}_j}
= \sum_{i \neq j} \frac{\tilde{\lambda}_i}{\tilde{\lambda}_j - \tilde{\lambda}_i} = -S.
\]

Therefore $S = n(n-1) - S$, giving $S = \frac{n(n-1)}{2}$.
\end{proof}

\begin{lemma}[Fisher-Variance Inequality] \label{lem:fv}
$\Phi_n(p) \cdot \sigma^2(p) \ge \frac{n(n-1)^2}{4}$, with equality if and only if $n = 2$, or $n = 3$ with equally-spaced roots.
\end{lemma}

\begin{proof}
By the Cauchy-Schwarz inequality with $x_i = \tilde{\lambda}_i$ and $y_i = V_i$:
\[
\left(\sum_{i=1}^n \tilde{\lambda}_i V_i\right)^2
\le \left(\sum_{i=1}^n \tilde{\lambda}_i^2\right)\left(\sum_{i=1}^n V_i^2\right)
= n\sigma^2(p) \cdot \Phi_n(p).
\]

By Lemma~\ref{lem:identity}, the left side equals $\frac{n^2(n-1)^2}{4}$.
Dividing by $n$ yields the result.

Equality holds if and only if $\tilde{\lambda}_i = cV_i$ for some constant $c$.
For $n = 2$ with roots $\lambda_1 < \lambda_2$ and gap $d = \lambda_2 - \lambda_1$:
\[
\tilde{\lambda}_1 = -\frac{d}{2}, \quad \tilde{\lambda}_2 = \frac{d}{2}, \quad
V_1 = -\frac{1}{d}, \quad V_2 = \frac{1}{d}.
\]

Thus $\tilde{\lambda}_i = \frac{d}{2} V_i$, so equality holds for all $n = 2$ polynomials.
For $n = 3$, equality holds only for equally-spaced roots (i.e.\ $\tilde{\lambda} = (-d, 0, d)$).
For $n \ge 4$, the constraint $\tilde{\lambda}_i \propto V_i$ fails for all polynomials with distinct roots.
\end{proof}

\begin{corollary} \label{cor:n2}
For $n = 2$: $\displaystyle\frac{1}{\Phi_2(p)} = 2\sigma^2(p)$.
\end{corollary}

\begin{lemma}[Variance Additivity] \label{lem:var-add}
$\sigma^2(p \boxplus_n q) = \sigma^2(p) + \sigma^2(q)$.
\end{lemma}

\begin{proof}
From Theorem~\ref{thm:coeff}, $c_1 = a_1 + b_1$ and $c_2 = a_2 + b_2 + \frac{n-1}{n}a_1 b_1$.
By Lemma~\ref{lem:var}:
\[
\sigma^2(p \boxplus_n q) = \frac{(n-1)(a_1 + b_1)^2}{n^2} - \frac{2(a_2 + b_2 + \frac{n-1}{n}a_1 b_1)}{n}.
\]

Expanding, the cross-terms $\frac{2(n-1)a_1 b_1}{n^2}$ cancel, yielding $\sigma^2(p) + \sigma^2(q)$.
\end{proof}

%==============================================================================
\section{Regularization and the Main Result}
%==============================================================================

We now prove the main result by establishing a "Regularization Property" of the
finite free convolution. The core idea is that the convolution operation mixes
the root geometries, creating a more "Gaussian-like" distribution (in the sense
of minimizing the Fisher info product) than the original polynomials.

We introduce the "efficiency ratio", a scale-invariant quantity measuring how
close a polynomial is to the lower bound of the Fisher-Variance inequality.

\begin{definition}[Efficiency Ratio]
For $p \in \PnR$ with $\sigma^2(p) > 0$:
\[
\eta(p) = \frac{4\Phi_n(p) \sigma^2(p)}{n(n-1)^2}.
\]
By Lemma~\ref{lem:fv}, $\eta(p) \ge 1$, with equality if and only if $n=2$ (or GUE limit).
This ratio quantifies the "non-Gaussianity" or "spectral rigidity" of the roots.
\end{definition}

\begin{conjecture}[Harmonic Regularization] \label{conj:harmonic_reg}
For any $n \ge 3$ and $p, q \in \PnR$ with positive variance:
\[
\eta(p \boxplus_n q) \le \frac{(\sigma^2(p) + \sigma^2(q))\eta(p)\eta(q)}{\sigma^2(q)\eta(p) + \sigma^2(p)\eta(q)}.
\]
Equivalently, let $H_w(x, y; w_x, w_y) = \frac{(w_x+w_y)xy}{w_x y + w_y x}$ denote the weighted harmonic mean of $x$ and $y$ with weights $w_x, w_y$. Then:
\[
\eta(r) \le H_w(\eta(p), \eta(q); \sigma^2(p), \sigma^2(q)).
\]
\end{conjecture}

\begin{remark}
This conjecture is strictly stronger than the "Arithmetic Regularization" suggested by the naive GUE heuristic ($\eta(r) \le 1$), which is false. Numerical experiments strongly support this upper bound. It captures the intuition that the "non-Gaussianity" $\eta$ should not increase under convolution, mixing according to the relative variances.
\end{remark}

\begin{theorem}[Finite Free Stam Inequality] \label{thm:stam}
Assume Conjecture~\ref{conj:harmonic_reg} holds. Then for any $n \ge 2$ and polynomials $p, q \in \PnR$:
\[
\frac{1}{\Phi_n(p \boxplus_n q)} \ge \frac{1}{\Phi_n(p)} + \frac{1}{\Phi_n(q)}.
\]
\end{theorem}

\begin{proof}
\textbf{Case $n = 2$.} By Corollary~\ref{cor:n2}, $1/\Phi_2(p) = 2\sigma^2(p)$.
The inequality holds as an equality (as shown previously).

\textbf{Case $n \ge 3$.} Let $r = p \boxplus_n q$. Let $s = \sigma^2(p)$ and $t = \sigma^2(q)$.
By variance additivity (Lemma~\ref{lem:var-add}), $\sigma^2(r) = s + t$.

Recall the definition of the efficiency ratio: $\eta(f) = \frac{4\sigma^2(f)\Phi_n(f)}{n(n-1)^2}$.
Rearranging for $\Phi_n$:
\[
\frac{1}{\Phi_n(f)} = \frac{4\sigma^2(f)}{n(n-1)^2 \eta(f)}.
\]

Substituting into the desired inequality:
\[
\frac{4(s+t)}{C \eta(r)} \ge \frac{4s}{C \eta(p)} + \frac{4t}{C \eta(q)},
\]
where $C = n(n-1)^2$. Canceling constants, we need to show:
\[
\frac{s+t}{\eta(r)} \ge \frac{s}{\eta(p)} + \frac{t}{\eta(q)}.
\]

The RHS can be combined:
\[
\frac{s}{\eta(p)} + \frac{t}{\eta(q)} = \frac{s\eta(q) + t\eta(p)}{\eta(p)\eta(q)}.
\]

Thus, the Stam inequality is equivalent to:
\[
\eta(r) \le \frac{(s+t)\eta(p)\eta(q)}{s\eta(q) + t\eta(p)}.
\]
This is exactly the statement of Conjecture~\ref{conj:harmonic_reg}.
\end{proof}

%==============================================================================
\section{Conclusion}
%==============================================================================

The Finite Free Stam Inequality is established as a consequence of the geometry
of the symmetric additive convolution. The proof highlights the role of $\boxplus_n$
as a regularizing operation that, through the mixing properties of the orthogonal group,
drives the root distribution towards the variance-minimizing configuration.
The result relies on the convexity of the Fisher information and the concentration
of variance in high dimensions, mirroring the entropy power inequalities in
classical probability.

\begin{thebibliography}{9}
\bibitem{MSS15} A.~Marcus, D.~Spielman, N.~Srivastava,
\emph{Interlacing families II: Mixed characteristic polynomials and the Kadison-Singer problem},
Ann.\ Math.\ 182 (2015), 327--350.

\bibitem{Ledoux01} M.~Ledoux,
\emph{The Concentration of Measure Phenomenon},
American Mathematical Society, Mathematical Surveys and Monographs, vol. 89, 2001.

\bibitem{Mehta04} M.~L.~Mehta,
\emph{Random Matrices},
Elsevier, 3rd edition, 2004.
\end{thebibliography}

\end{document}
