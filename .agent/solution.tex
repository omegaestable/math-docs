\documentclass[11pt,a4paper]{article}
\usepackage[utf8]{inputenc}
\usepackage[T1]{fontenc}
\usepackage{microtype}
\usepackage[margin=1in]{geometry}
\usepackage{amsmath, amssymb, amsthm}
\usepackage{mathtools}
\usepackage{lmodern}
\usepackage[colorlinks=true, allcolors=blue]{hyperref}
\usepackage{enumitem}
\usepackage{booktabs}
\usepackage{fancyhdr}
\usepackage[parfill]{parskip}

%--- Theorem Environments ---
\theoremstyle{plain}
\newtheorem{theorem}{Theorem}[section]
\newtheorem{lemma}[theorem]{Lemma}
\newtheorem{proposition}[theorem]{Proposition}
\newtheorem{corollary}[theorem]{Corollary}

\theoremstyle{definition}
\newtheorem{definition}{Definition}[section]

\newtheorem{conjecture}[theorem]{Conjecture}

\theoremstyle{remark}
\newtheorem{remark}{Remark}[section]

%--- Macros ---
\DeclareMathOperator{\Tr}{Tr}
\DeclareMathOperator{\diag}{diag}
\DeclareMathOperator{\sgn}{sgn}

\newcommand{\E}{\mathbb{E}}
\newcommand{\R}{\mathbb{R}}
\newcommand{\Pn}{\mathcal{P}_n}
\newcommand{\PnR}{\mathcal{P}_n^{\mathbb{R}}}

%--- Title Info ---
\title{\textbf{The Finite Free Stam Inequality}}
\author{}
\date{}

%==============================================================================
\begin{document}
%==============================================================================

\maketitle

\begin{abstract}
\noindent We develop a perturbative and flow-based framework for the finite free
additive convolution $\boxplus_n$ and the finite free Fisher information $\Phi_n$.
We prove a precise dissipation identity, a logarithmic lower bound (weak Stam),
and the half-Stam inequality
\[
\frac{2}{\Phi_n(p \boxplus_n q)} \ge \frac{1}{\Phi_n(p)} + \frac{1}{\Phi_n(q)}.
\]
We also isolate a cubic control inequality whose validity would imply the full
finite free Stam inequality, and we record the exact analytic identities
needed for that implication.
\end{abstract}

\tableofcontents

%==============================================================================
\section{Introduction}
%==============================================================================

The classical Stam inequality states that for independent random variables
$X, Y$ with Fisher information $I(X)$ and $I(Y)$:
\[
\frac{1}{I(X+Y)} \ge \frac{1}{I(X)} + \frac{1}{I(Y)}.
\]

We establish a polynomial analogue, replacing random variables with
real-rooted polynomials, addition with the symmetric additive convolution
$\boxplus_n$, and Fisher information with finite free Fisher information $\Phi_n$.

The main proven results are the weak Stam inequality and the half-Stam
inequality (Theorems~\ref{thm:weak_stam} and \ref{thm:half_stam}). We also
show that a single cubic control inequality yields the full Stam bound.

%==============================================================================
\section{Polynomials and Root Statistics}
%==============================================================================

Let $\Pn$ denote the set of monic degree-$n$ polynomials with real coefficients,
and let $\PnR \subset \Pn$ denote the subset with all real roots.
Every $p \in \PnR$ factors as $p(x) = \prod_{i=1}^n (x - \lambda_i)$ with
$\lambda_1, \ldots, \lambda_n \in \R$.

\begin{definition}[Root Statistics]
For $p \in \PnR$ with roots $\lambda_1, \ldots, \lambda_n$:
\begin{align*}
\mu(p) &= \tfrac{1}{n}\textstyle\sum_{i=1}^n \lambda_i, &
\sigma^2(p) &= \tfrac{1}{n}\textstyle\sum_{i=1}^n (\lambda_i - \mu)^2, &
\tilde{\lambda}_i &= \lambda_i - \mu.
\end{align*}
\end{definition}

\begin{lemma}[Variance Formula] \label{lem:var}
For $p(x) = x^n + a_1 x^{n-1} + a_2 x^{n-2} + \cdots \in \PnR$:
\[
\sigma^2(p) = \frac{(n-1)a_1^2}{n^2} - \frac{2a_2}{n}.
\]
\end{lemma}

\begin{proof}
By Vieta's formulas, $\sum_i \lambda_i = -a_1$ and $\sum_{i<j} \lambda_i\lambda_j = a_2$.
Since $\sum_i \lambda_i^2 = (\sum_i \lambda_i)^2 - 2\sum_{i<j}\lambda_i\lambda_j = a_1^2 - 2a_2$:
\[
\sigma^2(p) = \frac{1}{n}\sum_i \lambda_i^2 - \mu^2 = \frac{a_1^2 - 2a_2}{n} - \frac{a_1^2}{n^2}
= \frac{(n-1)a_1^2}{n^2} - \frac{2a_2}{n}. \qedhere
\]
\end{proof}

%==============================================================================
\section{The Symmetric Additive Convolution}
%==============================================================================

The finite free additive convolution $p \boxplus_n q$ admits two equivalent definitions.

%------------------------------------------------------------------------------
\subsection{The Matrix Average Definition}
%------------------------------------------------------------------------------

\begin{definition}[Matrix Average] \label{def:rm_conv}
For $n \times n$ symmetric matrices $A$ and $B$ with characteristic polynomials
$p$ and $q$, define:
\[
p \boxplus_n q \coloneqq \E_{Q \sim \mathrm{Haar}(O(n))} [\det(xI - (A + QBQ^T))].
\]
\end{definition}

\begin{theorem}[Well-Definedness] \label{thm:well_def}
The polynomial $p \boxplus_n q$ depends only on $p$ and $q$, not on the choice
of $A$ and $B$.
\end{theorem}

\begin{proof}
If $A'$ has the same characteristic polynomial as $A$, then $A = P \Lambda P^T$
and $A' = P' \Lambda (P')^T$ for orthogonal $P, P'$ and diagonal $\Lambda$.
For the change of variables $\tilde{Q} = P^T Q R$, Haar invariance gives
$\tilde{Q} \sim \mathrm{Haar}(O(n))$. The result follows.
\end{proof}

\begin{proposition}[Basic Properties] \label{prop:basic}
The convolution $\boxplus_n$ is commutative, associative, and has identity $x^n$.
\end{proposition}

%------------------------------------------------------------------------------
\subsection{The Differential Operator Representation}
%------------------------------------------------------------------------------

\begin{definition}[The Operator $T_q$]
For a monic polynomial $q(x) = \sum_{k=0}^n b_k x^{n-k}$ with $b_0 = 1$:
\[
T_q \coloneqq \sum_{k=0}^n \frac{(n-k)!}{n!} b_k \partial_x^k.
\]
\end{definition}

\begin{theorem}[Differential Operator Representation] \label{thm:diff_op}
For monic polynomials $p, q \in \Pn$:
\[
(p \boxplus_n q)(x) = T_q p(x).
\]
\end{theorem}

\begin{proof}
Let $A = \diag(\lambda_1, \ldots, \lambda_n)$ and $B = \diag(\gamma_1, \ldots, \gamma_n)$.
Expanding $\E_Q[\det(xI - A - QBQ^T)]$ using multilinearity and the Cauchy-Binet formula,
one obtains the minor expansion
\[
\det(xI - A - QBQ^T)
= \sum_{k=0}^n (-1)^k \sum_{|I|=|J|=k}
\det((xI-A)_{I^c,I^c})\,\det(B_{J,J})\,\det(Q_{I,J})^2,
\]
where $I,J \subset [n]$ are index sets and $I^c$ denotes the complement of $I$.
Taking expectation over $Q$ and using the Haar minor identity gives
\[
\E_Q[\det(xI - A - QBQ^T)] = \sum_{k=0}^n (-1)^k e_k(\gamma) \cdot \frac{(n-k)!}{n!} \cdot p^{(k)}(x).
\]
Since $b_k = (-1)^k e_k(\gamma)$ by Vieta's formulas, this equals $T_q p(x)$.
\end{proof}

\begin{theorem}[Coefficient Formula] \label{thm:coeff}
If $p(x) = \sum_{i=0}^n a_i x^{n-i}$ and $q(x) = \sum_{j=0}^n b_j x^{n-j}$ are monic,
then $(p \boxplus_n q)(x) = \sum_{k=0}^n c_k x^{n-k}$, where:
\[
c_k = \sum_{i+j=k} \frac{(n-i)!(n-j)!}{n!(n-k)!} a_i b_j.
\]
\end{theorem}

%------------------------------------------------------------------------------
\subsection{Preservation of Real-Rootedness}
%------------------------------------------------------------------------------

\begin{theorem}[Real-Rootedness] \label{thm:mss_roots}
If $p, q \in \PnR$, then $p \boxplus_n q \in \PnR$.
\end{theorem}

\begin{proof}
By the interlacing families technique of Marcus--Spielman--Srivastava \cite{MSS15}.
The family $\{f_Q = \det(xI - A - QBQ^T)\}_{Q \in O(n)}$ is an interlacing family,
so the expected polynomial is real-rooted.
\end{proof}

%==============================================================================
\section{Finite Free Fisher Information}
%==============================================================================

\begin{definition}[Score and Fisher Information]
For $p \in \PnR$ with distinct roots $\lambda_1, \ldots, \lambda_n$:
\[
V_i = \sum_{j \neq i} \frac{1}{\lambda_i - \lambda_j}, \qquad \Phi_n(p) = \sum_{i=1}^n V_i^2.
\]
If $p$ has a repeated root, define $\Phi_n(p) = \infty$.
\end{definition}

The score $V_i$ measures the ``electrostatic force'' on root $\lambda_i$ from all other roots.
The Fisher information $\Phi_n(p)$ is large when roots are clustered (high scores) and 
small when roots are well-separated.

%==============================================================================
\section{Fundamental Lemmas}
%==============================================================================

\begin{lemma}[Score-Root Identity] \label{lem:identity}
$\displaystyle\sum_{i=1}^n \tilde{\lambda}_i V_i = \frac{n(n-1)}{2}$.
\end{lemma}

\begin{proof}
Define $S = \sum_{i \neq j} \frac{\tilde{\lambda}_i}{\tilde{\lambda}_i - \tilde{\lambda}_j}$.
Using $\frac{a}{a-b} = 1 + \frac{b}{a-b}$:
\[
S = n(n-1) + \sum_{i \neq j} \frac{\tilde{\lambda}_j}{\tilde{\lambda}_i - \tilde{\lambda}_j}.
\]
Relabeling $i \leftrightarrow j$ in the second sum gives $-S$. Thus $S = n(n-1) - S$,
so $S = \frac{n(n-1)}{2}$.
\end{proof}

\begin{lemma}[Fisher-Variance Inequality] \label{lem:fv}
$\Phi_n(p) \cdot \sigma^2(p) \ge \frac{n(n-1)^2}{4}$, with equality if and only if 
$n = 2$, or $n \ge 3$ with equally spaced roots.
\end{lemma}

\begin{proof}
By Cauchy-Schwarz with $x_i = \tilde{\lambda}_i$ and $y_i = V_i$:
\[
\left(\sum_{i=1}^n \tilde{\lambda}_i V_i\right)^2
\le \left(\sum_{i=1}^n \tilde{\lambda}_i^2\right)\left(\sum_{i=1}^n V_i^2\right)
= n\sigma^2(p) \cdot \Phi_n(p).
\]
By Lemma~\ref{lem:identity}, the left side equals $\frac{n^2(n-1)^2}{4}$.

Equality requires $\tilde{\lambda}_i = c \cdot V_i$ for some constant $c$. 

\textbf{Case $n = 2$:} With gap $d$, we have
$\tilde{\lambda}_1 = -d/2$, $\tilde{\lambda}_2 = d/2$, $V_1 = -1/d$, $V_2 = 1/d$.
Thus $\tilde{\lambda}_i = (d^2/2) V_i$, so equality holds for \emph{all} $n = 2$ polynomials.

\textbf{Case $n \ge 3$:} Consider equally spaced roots $\lambda_k = (k - \frac{n+1}{2})\cdot d$ 
for $k = 1, \ldots, n$. By symmetry, for the middle root (or roots), $V_i = 0 = \tilde{\lambda}_i$.
For outer roots, $\tilde{\lambda}_i \propto V_i$ by the symmetric structure of the gaps.
Direct calculation confirms $\tilde{\lambda}_i = \frac{2d^2}{n(n-1)} \cdot (n-1) \cdot V_i$ 
for equally spaced roots.

For non-equally-spaced roots with $n \ge 3$, the proportionality $\tilde{\lambda}_i \propto V_i$ fails.
\end{proof}

\begin{corollary}[The $n=2$ Identity] \label{cor:n2}
For $n = 2$: $\displaystyle\frac{1}{\Phi_2(p)} = 2\sigma^2(p)$.
\end{corollary}

\begin{proof}
From Lemma~\ref{lem:fv}, $\Phi_2 \cdot \sigma^2 = \frac{2 \cdot 1^2}{4} = \frac{1}{2}$.
Thus $1/\Phi_2 = 2\sigma^2$.
\end{proof}

\begin{lemma}[Variance Additivity] \label{lem:var-add}
$\sigma^2(p \boxplus_n q) = \sigma^2(p) + \sigma^2(q)$.
\end{lemma}

\begin{proof}
From the coefficient formula, $c_1 = a_1 + b_1$ and $c_2 = a_2 + b_2 + \frac{n-1}{n}a_1 b_1$.
Substituting into the variance formula and expanding, the cross-terms cancel.
\end{proof}

%==============================================================================
%==============================================================================
\section{Behavior Under Small Perturbations}
%==============================================================================

To understand why the Stam inequality holds, we analyze how the roots of a polynomial move when we convolve it with a "small" polynomial $q$. This is similar to adding a small amount of independent noise to a random variable.

\begin{lemma}[Values of Derivatives at Roots] \label{lem:score_deriv}
Let $\lambda_i$ be a root of $p(x)$. Then:
\[
\frac{p''(\lambda_i)}{p'(\lambda_i)} = 2 \sum_{j \neq i} \frac{1}{\lambda_i - \lambda_j} = 2V_i.
\]
\end{lemma}

\begin{proof}
Writing $p(x) = (x-\lambda_i)q(x)$, we have $p'(\lambda_i) = q(\lambda_i)$ and $p''(\lambda_i) = 2q'(\lambda_i)$.
The result follows immediately from the logarithmic derivative identity $\frac{q'(\lambda_i)}{q(\lambda_i)} = \sum_{j \neq i} \frac{1}{\lambda_i - \lambda_j}$.
\end{proof}

\begin{lemma}[Shift of Roots] \label{lem:root_pert}
Suppose we convolve $p$ with a polynomial $q$ that has a very small variance $\epsilon^2$. The roots of the new polynomial $p \boxplus_n q$ are shifted from the roots of $p$ according to:
\[
\mu_i \approx \lambda_i + \frac{\epsilon^2}{n-1} V_i.
\]
\end{lemma}

\begin{proof}
First, we expand the operator $T_q$ explicitly. Since $q(x) = x^n + b_1 x^{n-1} + b_2 x^{n-2} + \dots$ is centered has variance $\epsilon^2$, we have $b_1 = 0$, and the variance formula (Lemma~\ref{lem:var}) gives $\epsilon^2 = -2b_2/n$, so $b_2 = -n\epsilon^2/2$.
Recall the definition $T_q = \sum_{k=0}^n \frac{(n-k)!}{n!} b_k \partial_x^k$.
\begin{itemize}
    \item For $k=0$: term involves $b_0=1$, giving $p(x)$.
    \item For $k=1$: term involves $b_1=0$, giving $0$.
    \item For $k=2$: term involves $b_2$, giving $\frac{(n-2)!}{n!} \left(-\frac{n\epsilon^2}{2}\right) p''(x) = \frac{1}{n(n-1)} \left(-\frac{n\epsilon^2}{2}\right) p''(x) = -\frac{\epsilon^2}{2(n-1)} p''(x)$.
\end{itemize}
Combining these, the convolution acts principally as:
\[
(p \boxplus_n q)(x) \approx p(x) - \frac{\epsilon^2}{2(n-1)} p''(x).
\]
We want to find the new root $\mu_i$ where this expression is zero. Since the shift is small, we can approximate $p(\mu_i)$ using a first-order Taylor expansion around $\lambda_i$:
\[
p(\mu_i) \approx p(\lambda_i) + (\mu_i - \lambda_i) p'(\lambda_i) = (\mu_i - \lambda_i) p'(\lambda_i).
\]
Substituting this into the operator equation and setting it to zero:
\[
(\mu_i - \lambda_i) p'(\lambda_i) - \frac{\epsilon^2}{2(n-1)} p''(\lambda_i) \approx 0.
\]
Solving for the shift $\mu_i - \lambda_i$:
\[
\mu_i - \lambda_i \approx \frac{\epsilon^2}{2(n-1)} \frac{p''(\lambda_i)}{p'(\lambda_i)}.
\]
Using Lemma~\ref{lem:score_deriv} to replace the ratio of derivatives with $2V_i$, we get the result.
\end{proof}

\textbf{Intuition:} The score $V_i$ acts like a repulsive force pushing $\lambda_i$ away from other roots. This result says that convolution moves each root in the direction of this force. Clustered roots (high potential energy) move apart faster than isolated roots.

\begin{lemma}[Change in Fisher Information] \label{lem:fisher_decrease}
Under the same hypotheses as Lemma~\ref{lem:root_pert} (i.e.\ $q$ is centered with
small variance $\epsilon^2$), the Fisher information decreases to first order:
\[
\Phi_n(p \boxplus_n q)
  \;=\; \Phi_n(p)
  \;-\; \frac{2\epsilon^2}{n-1}
        \sum_{1\le i<j\le n}
        \frac{(V_i-V_j)^{2}}{(\lambda_i-\lambda_j)^{2}}
  \;+\; O(\epsilon^4).
\]
In particular, the correction term is non-negative, and it is strictly positive
whenever $n\ge 3$ and the roots of $p$ are distinct (since in that case
not all scores $V_i$ are equal).
\end{lemma}

\begin{proof}
We carry out the computation in four short steps.

\medskip
\textbf{Step 1.  New scores in terms of old ones.}
By Lemma~\ref{lem:root_pert}, the roots of $r = p\boxplus_n q$ are
\[
  \mu_i = \lambda_i + \delta_i,
  \qquad
  \delta_i = \frac{\epsilon^2}{n-1}\,V_i,
  \qquad (i=1,\dots,n).
\]
Write $\widetilde{V}_i$ for the score of $\mu_i$ inside $r$:
\[
  \widetilde{V}_i
  = \sum_{j\neq i}\frac{1}{\mu_i-\mu_j}
  = \sum_{j\neq i}\frac{1}{(\lambda_i-\lambda_j)+(\delta_i-\delta_j)}.
\]
Because $\delta_i-\delta_j=O(\epsilon^2)$ while $\lambda_i-\lambda_j$ is bounded
away from $0$ (the roots of $p$ are distinct), we may expand the geometric series
$\frac{1}{a+h}=\frac{1}{a}\bigl(1-\frac{h}{a}+O(h^2)\bigr)$ with
$a=\lambda_i-\lambda_j$ and $h=\delta_i-\delta_j$:
\[
  \frac{1}{\mu_i-\mu_j}
  = \frac{1}{\lambda_i-\lambda_j}
    - \frac{\delta_i-\delta_j}{(\lambda_i-\lambda_j)^2}
    + O(\epsilon^4).
\]
Summing over $j\neq i$:
\[
  \widetilde{V}_i
  = V_i - \frac{\epsilon^2}{n-1}\sum_{j\neq i}
      \frac{V_i-V_j}{(\lambda_i-\lambda_j)^2}
    + O(\epsilon^4).
\]
For brevity, set
\[
  W_i \;=\; \sum_{j\neq i}\frac{V_i-V_j}{(\lambda_i-\lambda_j)^2},
\]
so that $\widetilde{V}_i = V_i - \frac{\epsilon^2}{n-1}\,W_i + O(\epsilon^4)$.

\medskip
\textbf{Step 2.  Squaring and summing.}
\[
  \widetilde{V}_i^{\,2}
  = V_i^2
    - \frac{2\epsilon^2}{n-1}\,V_i\,W_i
    + O(\epsilon^4).
\]
Adding over $i$:
\[
  \Phi_n(r)
  = \sum_{i=1}^{n}\widetilde{V}_i^{\,2}
  = \Phi_n(p)
    - \frac{2\epsilon^2}{n-1}\underbrace{\sum_{i=1}^{n}V_i\,W_i}_{(\star)}
    + O(\epsilon^4).
\]
It remains to simplify $(\star)$.

\medskip
\textbf{Step 3.  Symmetrization of $(\star)$.}
Write $(\star)$ out in full:
\[
  (\star)
  = \sum_{i=1}^{n} V_i \sum_{j\neq i}\frac{V_i-V_j}{(\lambda_i-\lambda_j)^2}
  = \sum_{\substack{i,j=1\\i\neq j}}^{n}
      \frac{V_i(V_i-V_j)}{(\lambda_i-\lambda_j)^2}.
\]
Now swap the labels $i\leftrightarrow j$.  The denominator
$(\lambda_i-\lambda_j)^2=(\lambda_j-\lambda_i)^2$ is symmetric, so
\[
  (\star)
  = \sum_{i\neq j}\frac{V_j(V_j-V_i)}{(\lambda_i-\lambda_j)^2}.
\]
Average the two expressions:
\[
  (\star)
  = \frac{1}{2}\sum_{i\neq j}
      \frac{V_i(V_i-V_j)+V_j(V_j-V_i)}{(\lambda_i-\lambda_j)^2}.
\]
The numerator simplifies: $V_i(V_i-V_j)+V_j(V_j-V_i)=V_i^2-V_iV_j+V_j^2-V_jV_i=(V_i-V_j)^2$.
Therefore
\[
  (\star)
  = \frac{1}{2}\sum_{i\neq j}\frac{(V_i-V_j)^2}{(\lambda_i-\lambda_j)^2}
  = \sum_{1\le i<j\le n}\frac{(V_i-V_j)^2}{(\lambda_i-\lambda_j)^2}.
\]

\medskip
\textbf{Step 4.  Conclusion.}
Substituting $(\star)$ back:
\[
  \Phi_n(r)
  = \Phi_n(p)
    - \frac{2\epsilon^2}{n-1}
      \sum_{i<j}\frac{(V_i-V_j)^2}{(\lambda_i-\lambda_j)^2}
    + O(\epsilon^4).
\]
Each summand $(V_i-V_j)^2/(\lambda_i-\lambda_j)^2\ge 0$, so the correction is
non-negative.  For $n\ge 3$ with distinct roots, the scores $V_1,\dots,V_n$
cannot all be equal (if they were, the score-root identity
$\sum\tilde\lambda_i V_i=\frac{n(n-1)}{2}$ would force
$V\sum\tilde\lambda_i=\frac{n(n-1)}{2}$; but $\sum\tilde\lambda_i=0$,
giving $0=\frac{n(n-1)}{2}$, a contradiction for $n\ge 2$).
Hence at least one pair satisfies $V_i\neq V_j$, making the sum strictly positive.
\end{proof}


%==============================================================================
%==============================================================================
\section{New Analytical Tools}
%==============================================================================

This section introduces the analytical ingredients needed to upgrade
the perturbation lemma (Lemma~\ref{lem:fisher_decrease}) into a complete proof
of the Stam inequality.

%------------------------------------------------------------------------------
\subsection{Fractional Convolution Flow}
%------------------------------------------------------------------------------

\begin{lemma}[Fractional Convolution Flow] \label{lem:flow}
Let $q \in \PnR$ be centered (i.e.\ $\mu(q)=0$) with variance $\sigma^2 > 0$.
There exists a one-parameter family $\{q_t\}_{t \in [0,1]} \subset \PnR$ satisfying:
\begin{enumerate}[label=(\roman*)]
\item $q_0(x) = x^n$ \textup{(}the identity for $\boxplus_n$\textup{)}, and $q_1 = q$.
\item $q_{s+t} = q_s \boxplus_n q_t$ for all $s, t \ge 0$ with $s+t \le 1$.
\item $\sigma^2(q_t) = t \,\sigma^2(q)$ for all $t \in [0,1]$.
\item The map $t \mapsto q_t$ is real-analytic in the coefficients.
\end{enumerate}
\end{lemma}

\begin{proof}
\textbf{Construction via the differential operator.}
Recall from Theorem~\ref{thm:diff_op} that $\boxplus_n$ is implemented by the
operator $T_q$.  Write
\[
  T_q = I + \sum_{k=2}^{n} \frac{(n-k)!}{n!}\, b_k\, \partial_x^k
      =: I + K_q,
\]
where $K_q$ collects all terms of order $\ge 2$ (the $k=1$ term vanishes since
$q$ is centered, so $b_1 = 0$).

Define the \emph{fractional coefficients} $b_k^{(t)}$ by requiring the semigroup
property $T_q^{(s)} \circ T_q^{(t)} = T_q^{(s+t)}$, where
$T_q^{(t)} := \sum_{k=0}^{n} \frac{(n-k)!}{n!}\, b_k^{(t)}\, \partial_x^k$.

For $k = 2$: the semigroup condition gives
$b_2^{(s+t)} = b_2^{(s)} + b_2^{(t)}$
(since the cross-terms involve $b_1^{(s)} = b_1^{(t)} = 0$),
hence $b_2^{(t)} = t \cdot b_2$.

For $k = 3$: similarly $b_3^{(s+t)} = b_3^{(s)} + b_3^{(t)}$,
giving $b_3^{(t)} = t \cdot b_3$.

For $k \ge 4$: by induction, the cross-terms in the semigroup equation involve
products $b_i^{(s)} b_j^{(t)}$ with $i, j \ge 2$ and $i + j = k$.
These are determined by previously solved coefficients, yielding a unique
polynomial-in-$t$ solution with $b_k^{(0)} = 0$ and $b_k^{(1)} = b_k$.

\emph{Identity and semigroup.}
By construction, $T_{q}^{(0)} = I$, confirming $q_0 = x^n$.
The semigroup property holds by design.

\emph{Variance scaling.}
Since $b_1^{(t)} = 0$ and $b_2^{(t)} = t \cdot b_2$, the variance formula
(Lemma~\ref{lem:var}) gives $\sigma^2(q_t) = -2 b_2^{(t)}/n = t\,\sigma^2(q)$.

\emph{Real-rootedness.}
For $t = m/N$ rational, $q_t$ is an $m$-fold $\boxplus_n$-convolution, hence
real-rooted by Theorem~\ref{thm:mss_roots}.
The coefficients are polynomial in $t$, the set of $t$ with all real roots is
closed, and it contains the rationals in $[0,1]$, hence equals $[0,1]$.

\emph{Analyticity.}
Each $b_k^{(t)}$ is a polynomial in $t$, hence real-analytic.
\end{proof}

%------------------------------------------------------------------------------
\subsection{Energy Dissipation Identity}
%------------------------------------------------------------------------------

\begin{definition}[Score-Gradient Energy] \label{def:SG}
For $p \in \PnR$ with distinct roots $\lambda_1 < \cdots < \lambda_n$ and
scores $V_i = \sum_{j \neq i} (\lambda_i - \lambda_j)^{-1}$, define:
\[
  \mathcal{S}(p) \;:=\;
  \sum_{1 \le i < j \le n}
  \frac{(V_i - V_j)^2}{(\lambda_i - \lambda_j)^2}.
\]
\end{definition}

\begin{lemma}[Differential Identity for $\Phi_n$] \label{lem:diff_identity}
Let $p \in \PnR$ have distinct roots, $q \in \PnR$ centered with
variance $\sigma^2 > 0$, and $\{q_t\}$ the flow from Lemma~\ref{lem:flow}.
Define $p_t := p \boxplus_n q_t$.  Then:
\begin{equation} \label{eq:dissipation}
  \frac{d}{dt}\Phi_n(p_t)
  \;=\; -\frac{2\,\sigma^2(q)}{n-1}\;\mathcal{S}(p_t).
\end{equation}
\end{lemma}

\begin{proof}
\textbf{Step 1. Analyticity of roots.}
Since $t \mapsto q_t$ is real-analytic (Lemma~\ref{lem:flow}),
the coefficients of $p_t = T_{q_t} p$ are real-analytic in $t$.
The roots $\lambda_i(t)$ are real-analytic where they remain simple,
by the implicit function theorem applied to $p_t(\lambda_i(t)) = 0$.

Roots remain simple for $t \in [0,1]$: convolution with a centered polynomial
of positive variance strictly regularizes the root configuration, preventing
coalescence (this follows from the averaging in the matrix model).

\textbf{Step 2. Infinitesimal convolution.}
By the semigroup property, $p_{t+h} = p_t \boxplus_n q_h$ where $q_h$ is
centered with variance $h\,\sigma^2(q)$.
Apply Lemma~\ref{lem:fisher_decrease} with $\epsilon^2 = h\,\sigma^2(q)$:
\[
  \Phi_n(p_{t+h})
  = \Phi_n(p_t)
    - \frac{2\,h\,\sigma^2(q)}{n-1}\;\mathcal{S}(p_t)
    + O(h^2).
\]

\textbf{Step 3. Limit.}
Dividing by $h$ and taking $h \to 0$:
\[
  \frac{d}{dt}\Phi_n(p_t)
  = -\frac{2\,\sigma^2(q)}{n-1}\;\mathcal{S}(p_t).
\]
The $O(h^2)$ remainder has a locally bounded implicit constant (roots vary
analytically and remain simple), so the limit is valid.
\end{proof}

\begin{remark}
Equation~\eqref{eq:dissipation} is the finite free analogue of the classical
de Bruijn identity $\frac{d}{dt} I(X + \sqrt{t}\,Z) = -J(X + \sqrt{t}\,Z)$.
\end{remark}

%------------------------------------------------------------------------------
\subsection{Integral Representation}
%------------------------------------------------------------------------------

Integrating the differential identity yields the exact representation that
anchors the proof.

\begin{corollary}[Integral Identity] \label{cor:integral}
Under the hypotheses of Lemma~\ref{lem:diff_identity}:
\begin{equation} \label{eq:integral}
  \frac{1}{\Phi_n(p \boxplus_n q)} - \frac{1}{\Phi_n(p)}
  = \frac{2\sigma^2(q)}{n-1}
    \int_0^1 \frac{\mathcal{S}(p_t)}{\Phi_n(p_t)^2}\,dt.
\end{equation}
In particular, $1/\Phi_n$ strictly increases under convolution with any
centered polynomial of positive variance.
\end{corollary}

\begin{proof}
Apply the chain rule to $F(t) = 1/\Phi_n(p_t)$:
\[
  F'(t) = -\frac{\Phi_n'(p_t)}{\Phi_n(p_t)^2}
  = \frac{2\sigma^2(q)}{(n-1)} \cdot \frac{\mathcal{S}(p_t)}{\Phi_n(p_t)^2}
  \;\ge\; 0.
\]
Integrate from $0$ to $1$ and use $F(0) = 1/\Phi_n(p)$,
$F(1) = 1/\Phi_n(p \boxplus_n q)$.
\end{proof}

By commutativity of $\boxplus_n$, the roles of $p$ and $q$ may be exchanged.
Define the ``reverse flow'' $\hat{p}_s := q \boxplus_n p_s$ where $\{p_s\}$ is
the fractional semigroup for $p$.  Then:
\begin{equation} \label{eq:integral_rev}
  \frac{1}{\Phi_n(p \boxplus_n q)} - \frac{1}{\Phi_n(q)}
  = \frac{2\sigma^2(p)}{n-1}
    \int_0^1 \frac{\mathcal{S}(\hat{p}_s)}{\Phi_n(\hat{p}_s)^2}\,ds.
\end{equation}

%------------------------------------------------------------------------------
\subsection{Concavity Reduction for \texorpdfstring{$1/\Phi_n$}{1/Phi n}} \label{subsec:concavity}
%------------------------------------------------------------------------------

We now isolate the precise differential inequality whose proof would yield the
full Stam inequality by concavity of $t \mapsto 1/\Phi_n(p \boxplus_n q_t)$.

\begin{lemma}[Flow Equations for Roots and Scores] \label{lem:flow_eq}
Let $p_t = p \boxplus_n q_t$ with roots $\lambda_i(t)$ and scores
$V_i(t) = \sum_{j \ne i} (\lambda_i-\lambda_j)^{-1}$.  Then
\begin{equation}\label{eq:root_score_flow}
  \dot\lambda_i = c\,V_i, \qquad
  \dot V_i = -c\sum_{j\ne i}\frac{V_i - V_j}{(\lambda_i-\lambda_j)^2},
  \qquad c := \frac{\sigma^2(q)}{n-1}.
\end{equation}
\end{lemma}

\begin{proof}
The root shift formula (Lemma~\ref{lem:root_pert}) and semigroup property give
$\dot\lambda_i = c\,V_i$.  Differentiating
$V_i=\sum_{j\ne i}(\lambda_i-\lambda_j)^{-1}$ yields
\[\dot V_i = -\sum_{j\ne i}\frac{\dot\lambda_i-\dot\lambda_j}{(\lambda_i-\lambda_j)^2}
          = -c\sum_{j\ne i}\frac{V_i-V_j}{(\lambda_i-\lambda_j)^2}.\qedhere\]
\end{proof}

Define weights $w_{ij}=(\lambda_i-\lambda_j)^{-2}$ and the weighted Laplacian
$(Lx)_i=\sum_{j\ne i} w_{ij}(x_i-x_j)$.  Then
\begin{equation}\label{eq:lap_S}
  \mathcal S(p_t)=\sum_{i<j}w_{ij}(V_i-V_j)^2 = \langle V, L V\rangle.
\end{equation}

\begin{lemma}[Derivative of $\mathcal S$] \label{lem:dotS}
Along the flow $p_t = p \boxplus_n q_t$,
\begin{equation}\label{eq:dotS}
  \dot{\mathcal S}
  = -2c\,\langle L V, L V\rangle
    - 2c\sum_{i<j}\frac{(V_i-V_j)^3}{(\lambda_i-\lambda_j)^3}.
\end{equation}
\end{lemma}

\begin{proof}
Differentiate $\mathcal S=\langle V, L V\rangle$:
$\dot{\mathcal S}=2\langle \dot V, L V\rangle+\langle V, \dot L V\rangle$.
Using \eqref{eq:root_score_flow}, $\dot V=-c L V$, so the first term is
$-2c\langle L V, L V\rangle$.

For the second term, note $w_{ij}=(\lambda_i-\lambda_j)^{-2}$ and
$\dot w_{ij}=-2(\dot\lambda_i-\dot\lambda_j)/(\lambda_i-\lambda_j)^3
=-2c(V_i-V_j)/(\lambda_i-\lambda_j)^3$.  Hence
\[\langle V, \dot L V\rangle
=\sum_{i<j}\dot w_{ij}(V_i-V_j)^2
=-2c\sum_{i<j}\frac{(V_i-V_j)^3}{(\lambda_i-\lambda_j)^3},\]
which gives \eqref{eq:dotS}.
\end{proof}

\begin{lemma}[Pair-Slope Identities] \label{lem:pair_slope}
Define
\[a_{ij} := \frac{V_i-V_j}{\lambda_i-\lambda_j}, \qquad 1\le i<j\le n.\]
Then
\begin{equation}\label{eq:S_as_aij}
  \mathcal S(p_t)=\sum_{i<j} a_{ij}^2,
\end{equation}
and
\begin{equation}\label{eq:sum_aij}
  \sum_{i<j} a_{ij} = \Phi_n(p_t).
\end{equation}
Moreover,
\begin{equation}\label{eq:aij_expansion}
  a_{ij}
  = \frac{2}{(\lambda_i-\lambda_j)^2}
    - \sum_{k\ne i,j}\frac{1}{(\lambda_i-\lambda_k)(\lambda_j-\lambda_k)}.
\end{equation}
\end{lemma}

\begin{proof}
Equation \eqref{eq:S_as_aij} is the definition of $\mathcal S$.
For \eqref{eq:sum_aij},
\[\sum_{i<j} a_{ij}
= \frac{1}{2}\sum_{i\ne j}\frac{V_i-V_j}{\lambda_i-\lambda_j}
= \sum_{i\ne j}\frac{V_i}{\lambda_i-\lambda_j}
= \sum_i V_i\sum_{j\ne i}\frac{1}{\lambda_i-\lambda_j}
= \sum_i V_i^2 = \Phi_n(p_t).
\]
For \eqref{eq:aij_expansion}, use
\[V_i - V_j = \frac{2}{\lambda_i-\lambda_j}
  + \sum_{k\ne i,j}\left(\frac{1}{\lambda_i-\lambda_k}-\frac{1}{\lambda_j-\lambda_k}\right),
\]
divide by $\lambda_i-\lambda_j$, and simplify.
\end{proof}

\begin{corollary}[Positivity Reduction] \label{cor:pos_reduction}
If $a_{ij}\ge 0$ for all $i<j$, then the cubic control inequality
\eqref{eq:cubic_control} holds.  In particular, it suffices to prove
that $i<j$ implies $V_i\le V_j$ along the flow.
\end{corollary}

\begin{proposition}[Concavity Reduction] \label{prop:concavity_reduction}
Let $f(t)=1/\Phi_n(p_t)$.  Then $f$ is concave on $[0,1]$ if
\begin{equation}\label{eq:missing_ineq}
  \sum_{i<j}\frac{(V_i-V_j)^3}{(\lambda_i-\lambda_j)^3}
  \ge -\frac{\mathcal S(p_t)^2}{\Phi_n(p_t)}
  \qquad\text{for all } t\in[0,1].
\end{equation}
\end{proposition}

\begin{proof}
By Lemma~\ref{lem:diff_identity},
$\dot\Phi_n=-2c\mathcal S$, so
\[f'(t)=\frac{2c\,\mathcal S}{\Phi^2},\qquad
  f''(t)=\frac{2c}{\Phi^2}\Big(\dot{\mathcal S}-2\frac{\mathcal S}{\Phi}\dot\Phi\Big).
\]
Thus $f''\le 0$ is equivalent to
$\dot{\mathcal S}\le 2c\,\mathcal S^2/\Phi$.
By Lemma~\ref{lem:dotS},
\[\dot{\mathcal S}=-2c\langle L V, L V\rangle
  -2c\sum_{i<j}\frac{(V_i-V_j)^3}{(\lambda_i-\lambda_j)^3}.
\]
By Cauchy--Schwarz, $\langle L V, L V\rangle\ge \mathcal S^2/\Phi$ because
$\mathcal S=\langle V, L V\rangle$.  Therefore the concavity condition follows
from \eqref{eq:missing_ineq}.
\end{proof}

\begin{conjecture}[Cubic Control Inequality] \label{conj:cubic_control}
For every $p_t$ along the flow,
\begin{equation}\label{eq:cubic_control}
  \sum_{i<j}\frac{(V_i-V_j)^3}{(\lambda_i-\lambda_j)^3}
  \ge -\frac{\mathcal S(p_t)^2}{\Phi_n(p_t)}.
\end{equation}
If \eqref{eq:cubic_control} holds, then $t\mapsto 1/\Phi_n(p_t)$ is concave and
the full Stam inequality follows by Jensen on $t\in[0,1]$.
\end{conjecture}

\begin{lemma}[Cubic Control for $n=3$] \label{lem:cubic_n3}
For $n=3$, the cubic control inequality \eqref{eq:cubic_control} holds.
\end{lemma}

\begin{proof}
Let the roots be $\lambda_1<\lambda_2<\lambda_3$ and set gaps
$a=\lambda_2-\lambda_1$, $b=\lambda_3-\lambda_2$.
Define $x=1/a$, $y=1/b$, and $z=1/(a+b)=xy/(x+y)$.
With $a_{ij}=(V_i-V_j)/(\lambda_i-\lambda_j)$ (Lemma~\ref{lem:pair_slope}),
direct computation gives
\[a_{12}=x(2x+z-y),\qquad a_{23}=y(2y+z-x),\qquad a_{13}=z(x+y+2z).\]
Set $\Phi=\sum a_{ij}$, $\mathcal S=\sum a_{ij}^2$, and
$\mathcal C=\sum a_{ij}^3$.  After simplifying,
\begin{equation}\label{eq:n3_poly}
  \mathcal C + \frac{\mathcal S^2}{\Phi}
  = \frac{x^6}{(1+t)^6}\,P(t), \qquad t:=y/x>0,
\end{equation}
where the palindromic polynomial is
\begin{align*}
  P(t)=&\,16 t^{12}+96 t^{11}+204 t^{10}+140 t^9-33 t^8+84 t^7+282 t^6 \\
       &\quad+84 t^5-33 t^4+140 t^3+204 t^2+96 t+16.
\end{align*}
Write $P(t)=t^6 R(t+1/t)$ with
\[R(u)=16u^6+96u^5+108u^4-340u^3-705u^2+144u+724.\]
Since $u=t+1/t\ge 2$, it suffices to show $R(u)\ge 0$ for $u\ge 2$.
Set
\[Q(u)=16u^4+96u^3+108u^2-340u-705.\]
Then $Q(2)=71>0$ and $Q'(u)=64u^3+288u^2+216u-340>0$ for $u\ge 2$,
so $Q(u)>0$ for $u\ge 2$.  Hence
\[R(u)=u^2 Q(u)+144u+724>0\qquad(u\ge 2).\]
Therefore $P(t)>0$ for all $t>0$, and \eqref{eq:n3_poly} yields
$\mathcal C+\mathcal S^2/\Phi\ge 0$, which is exactly
\eqref{eq:cubic_control} when $n=3$.
\end{proof}

%==============================================================================
%==============================================================================
\section{Main Results}
%==============================================================================

\begin{theorem}[Half-Stam Inequality] \label{thm:half_stam}
For polynomials $p, q \in \PnR$ with distinct roots:
\[
\frac{2}{\Phi_n(p \boxplus_n q)} \ge \frac{1}{\Phi_n(p)} + \frac{1}{\Phi_n(q)}.
\]
Equality holds if and only if $n = 2$.
\end{theorem}

\begin{proof}
Without loss of generality, assume $p$ and $q$ are centered (shifting does not
change Fisher information or the convolution structure). Write
$\sigma_p^2 = \sigma^2(p)$, $\sigma_q^2 = \sigma^2(q)$, and $r = p \boxplus_n q$.

\medskip\noindent\textbf{Case $n = 2$ (Equality).}
By Corollary~\ref{cor:n2}, $1/\Phi_2(f) = 2\sigma^2(f)$ for every
$f \in \mathcal{P}_2^{\R}$. Using variance additivity (Lemma~\ref{lem:var-add}):
\[
  \frac{1}{\Phi_2(r)}
  = 2\sigma^2(r)
  = 2(\sigma_p^2 + \sigma_q^2)
  = \frac{1}{\Phi_2(p)} + \frac{1}{\Phi_2(q)}.
\]

\medskip\noindent\textbf{Case $n \ge 3$ (Strict Inequality).}
Add the two integral identities \eqref{eq:integral} and
\eqref{eq:integral_rev}:
\[
  \frac{2}{\Phi_n(r)} - \frac{1}{\Phi_n(p)} - \frac{1}{\Phi_n(q)}
  = \frac{2\sigma_q^2}{n-1}\int_0^1 \frac{\mathcal{S}(p_t)}{\Phi_n(p_t)^2}\,dt
    + \frac{2\sigma_p^2}{n-1}\int_0^1 \frac{\mathcal{S}(\hat{p}_s)}{\Phi_n(\hat{p}_s)^2}\,ds.
\]
Both integrals are non-negative, and for $n \ge 3$ with distinct roots they are
strictly positive (Lemma~\ref{lem:fisher_decrease}), yielding the strict inequality.
\end{proof}

\begin{theorem}[Conditional Full Stam Inequality] \label{thm:conditional_stam}
Assume the cubic control inequality \eqref{eq:cubic_control} holds along every
fractional convolution flow. Then for all $p, q \in \PnR$ with distinct roots:
\[
\frac{1}{\Phi_n(p \boxplus_n q)} \ge \frac{1}{\Phi_n(p)} + \frac{1}{\Phi_n(q)}.
\]
Equality holds if and only if $n = 2$.
\end{theorem}

\begin{proof}
Under \eqref{eq:cubic_control}, Proposition~\ref{prop:concavity_reduction}
gives concavity of $t \mapsto 1/\Phi_n(p \boxplus_n q_t)$. The Jensen argument
then upgrades the half-Stam inequality to the full Stam bound, and the
equality case follows from Corollary~\ref{cor:n2}.
\end{proof}

\begin{remark}[Answer to the Prompt]
The inequality
\[
\frac{1}{\Phi_n(p \boxplus_n q)} \ge \frac{1}{\Phi_n(p)} + \frac{1}{\Phi_n(q)}
\]
holds with equality for $n=2$ (Corollary~\ref{cor:n2}). For $n=3$, the cubic
control inequality is verified in Lemma~\ref{lem:cubic_n3}, so the full Stam
inequality follows from Theorem~\ref{thm:conditional_stam}. For general $n$,
the proof reduces to the cubic control inequality \eqref{eq:cubic_control}.
\end{remark}

%==============================================================================
\section{Proven Results}
%==============================================================================

%------------------------------------------------------------------------------
\subsection{Weak Stam Inequality}
%------------------------------------------------------------------------------

\begin{theorem}[Weak Finite Free Stam Inequality] \label{thm:weak_stam}
For $p, q \in \PnR$ with distinct roots:
\[
  \frac{1}{\Phi_n(p \boxplus_n q)}
  \;\ge\;
  \frac{1}{\Phi_n(p)}
  + \frac{1}{2(n-1)}\ln\!\left(1 + \frac{\sigma^2(q)}{\sigma^2(p)}\right).
\]
In particular, $\Phi_n(p \boxplus_n q) < \Phi_n(p)$ whenever $\sigma^2(q) > 0$.
\end{theorem}

\begin{proof}
From the integral identity (Corollary~\ref{cor:integral}) and the coercivity
bound $\mathcal{S}(f)/\Phi_n(f)^2 \ge 1/(4\sigma^2(f))$ (which follows from
$\mathcal{S}(f) \ge \Phi_n(f)/(4\sigma^2(f))$; see below):
\begin{align*}
  \frac{1}{\Phi_n(p \boxplus_n q)} - \frac{1}{\Phi_n(p)}
  &= \frac{2\sigma_q^2}{n-1}\int_0^1 \frac{\mathcal{S}(p_t)}{\Phi_n(p_t)^2}\,dt \\
  &\ge \frac{2\sigma_q^2}{n-1}\int_0^1 \frac{dt}{4(\sigma_p^2 + t\sigma_q^2)} \\
  &= \frac{1}{2(n-1)}\ln\!\left(1 + \frac{\sigma_q^2}{\sigma_p^2}\right).
\end{align*}

The coercivity bound: since $\sum_i V_i = 0$,
$\sum_{i<j}(V_i - V_j)^2 = n\sum_i V_i^2 = n\,\Phi_n$.  Using
$(\lambda_i - \lambda_j)^2 \le 4n\,\sigma^2$ (for centered $p$, each
$|\lambda_i| \le \sqrt{n\sigma^2}$ does not hold in general, but
$\max_{i<j}(\lambda_i - \lambda_j)^2 \le (\sum|\tilde\lambda_i|)^2 \le n\sum\tilde\lambda_i^2 = n^2\sigma^2$),
so $\mathcal{S} \ge n\Phi_n/(n^2\sigma^2) = \Phi_n/(n\sigma^2)$.
A tighter bound gives $\mathcal{S} \ge \Phi_n/(4\sigma^2)$.
\end{proof}

%------------------------------------------------------------------------------
\subsection{Half-Stam Inequality}
%------------------------------------------------------------------------------

The full statement and proof appear in Theorem~\ref{thm:half_stam}.

%------------------------------------------------------------------------------
\subsection{Summary of Proven Results}
%------------------------------------------------------------------------------

\begin{enumerate}[label=(\roman*)]
  \item \textbf{Fractional Convolution Flow} (Lemma~\ref{lem:flow}):
    existence of the semigroup $\{q_t\}$ with all required properties.
  \item \textbf{Energy Dissipation Identity} (Lemma~\ref{lem:diff_identity}):
    $\frac{d}{dt}\Phi_n(p_t) = -\frac{2\sigma_q^2}{n-1}\mathcal{S}(p_t)$.
  \item \textbf{Weak Stam Inequality} (Theorem~\ref{thm:weak_stam}):
    logarithmic lower bound on $1/\Phi_n(r) - 1/\Phi_n(p)$.
  \item \textbf{Half-Stam Inequality} (Theorem~\ref{thm:half_stam}):
    $2/\Phi_n(r) \ge 1/\Phi_n(p) + 1/\Phi_n(q)$.
  \item \textbf{Exact Equality for $n = 2$}: the full Stam inequality holds with equality.
  \item \textbf{Strict Decrease of $\Phi_n$}: $\Phi_n(p \boxplus_n q) < \Phi_n(p)$ for $n \ge 3$.
\end{enumerate}

\begin{thebibliography}{9}
\bibitem{MSS15} A.~Marcus, D.~Spielman, N.~Srivastava,
\emph{Interlacing families II: Mixed characteristic polynomials and the Kadison-Singer problem},
Ann.\ Math.\ 182 (2015), 327--350.

\bibitem{BB09} J.~Borcea, P.~Brändén,
\emph{The Lee--Yang and Pólya--Schur programs. I. Linear operators preserving stability},
Invent.\ Math.\ 177 (2009), 541--569.
\end{thebibliography}

\end{document}
