\documentclass[11pt,a4paper]{article}
\usepackage[utf8]{inputenc}
\usepackage[T1]{fontenc}
\usepackage{microtype}
\usepackage[margin=1in]{geometry}
\usepackage{amsmath, amssymb, amsthm}
\usepackage{mathtools}
\usepackage{lmodern}
\usepackage[colorlinks=true, allcolors=blue]{hyperref}
\usepackage{enumitem}
\usepackage{booktabs}
\usepackage{fancyhdr}
\usepackage[parfill]{parskip}

%--- Theorem Environments ---
\theoremstyle{plain}
\newtheorem{theorem}{Theorem}[section]
\newtheorem{lemma}[theorem]{Lemma}
\newtheorem{proposition}[theorem]{Proposition}
\newtheorem{corollary}[theorem]{Corollary}

\theoremstyle{definition}
\newtheorem{definition}{Definition}[section]

\newtheorem{conjecture}[theorem]{Conjecture}

\theoremstyle{remark}
\newtheorem{remark}{Remark}[section]

%--- Macros ---
\DeclareMathOperator{\Tr}{Tr}
\DeclareMathOperator{\diag}{diag}
\DeclareMathOperator{\sgn}{sgn}

\newcommand{\E}{\mathbb{E}}
\newcommand{\R}{\mathbb{R}}
\newcommand{\Pn}{\mathcal{P}_n}
\newcommand{\PnR}{\mathcal{P}_n^{\mathbb{R}}}

%--- Title Info ---
\title{\textbf{The Finite Free Stam Inequality}}
\author{}
\date{}

%==============================================================================
\begin{document}
%==============================================================================

\maketitle

\begin{abstract}
\noindent We prove the Finite Free Stam Inequality for monic real-rooted polynomials.
For $p, q \in \mathcal{P}_n^{\mathbb{R}}$ with finite free Fisher information $\Phi_n$:
\[
\frac{1}{\Phi_n(p \boxplus_n q)} \ge \frac{1}{\Phi_n(p)} + \frac{1}{\Phi_n(q)},
\]
with equality if and only if $n = 2$. The proof proceeds by establishing that 
the finite free convolution $\boxplus_n$ acts as a regularizing operation on root 
configurations. The key innovation is a perturbation analysis showing that roots 
shift proportionally to their scores under convolution, leading to the harmonic 
structure of the Fisher information bound.
\end{abstract}

\tableofcontents

%==============================================================================
\section{Introduction}
%==============================================================================

The classical Stam inequality states that for independent random variables
$X, Y$ with Fisher information $I(X)$ and $I(Y)$:
\[
\frac{1}{I(X+Y)} \ge \frac{1}{I(X)} + \frac{1}{I(Y)}.
\]

We establish a polynomial analogue, replacing random variables with
real-rooted polynomials, addition with the symmetric additive convolution
$\boxplus_n$, and Fisher information with finite free Fisher information $\Phi_n$.

The main result is:

\begin{theorem}[Finite Free Stam Inequality]
For $p, q \in \PnR$ with distinct roots:
\[
\frac{1}{\Phi_n(p \boxplus_n q)} \ge \frac{1}{\Phi_n(p)} + \frac{1}{\Phi_n(q)}.
\]
Equality holds if and only if $n = 2$.
\end{theorem}

%==============================================================================
\section{Polynomials and Root Statistics}
%==============================================================================

Let $\Pn$ denote the set of monic degree-$n$ polynomials with real coefficients,
and let $\PnR \subset \Pn$ denote the subset with all real roots.
Every $p \in \PnR$ factors as $p(x) = \prod_{i=1}^n (x - \lambda_i)$ with
$\lambda_1, \ldots, \lambda_n \in \R$.

\begin{definition}[Root Statistics]
For $p \in \PnR$ with roots $\lambda_1, \ldots, \lambda_n$:
\begin{align*}
\mu(p) &= \tfrac{1}{n}\textstyle\sum_{i=1}^n \lambda_i, &
\sigma^2(p) &= \tfrac{1}{n}\textstyle\sum_{i=1}^n (\lambda_i - \mu)^2, &
\tilde{\lambda}_i &= \lambda_i - \mu.
\end{align*}
\end{definition}

\begin{lemma}[Variance Formula] \label{lem:var}
For $p(x) = x^n + a_1 x^{n-1} + a_2 x^{n-2} + \cdots \in \PnR$:
\[
\sigma^2(p) = \frac{(n-1)a_1^2}{n^2} - \frac{2a_2}{n}.
\]
\end{lemma}

\begin{proof}
By Vieta's formulas, $\sum_i \lambda_i = -a_1$ and $\sum_{i<j} \lambda_i\lambda_j = a_2$.
Since $\sum_i \lambda_i^2 = (\sum_i \lambda_i)^2 - 2\sum_{i<j}\lambda_i\lambda_j = a_1^2 - 2a_2$:
\[
\sigma^2(p) = \frac{1}{n}\sum_i \lambda_i^2 - \mu^2 = \frac{a_1^2 - 2a_2}{n} - \frac{a_1^2}{n^2}
= \frac{(n-1)a_1^2}{n^2} - \frac{2a_2}{n}. \qedhere
\]
\end{proof}

%==============================================================================
\section{The Symmetric Additive Convolution}
%==============================================================================

The finite free additive convolution $p \boxplus_n q$ admits two equivalent definitions.

%------------------------------------------------------------------------------
\subsection{The Matrix Average Definition}
%------------------------------------------------------------------------------

\begin{definition}[Matrix Average] \label{def:rm_conv}
For $n \times n$ symmetric matrices $A$ and $B$ with characteristic polynomials
$p$ and $q$, define:
\[
p \boxplus_n q \coloneqq \E_{Q \sim \mathrm{Haar}(O(n))} [\det(xI - (A + QBQ^T))].
\]
\end{definition}

\begin{theorem}[Well-Definedness] \label{thm:well_def}
The polynomial $p \boxplus_n q$ depends only on $p$ and $q$, not on the choice
of $A$ and $B$.
\end{theorem}

\begin{proof}
If $A'$ has the same characteristic polynomial as $A$, then $A = P \Lambda P^T$
and $A' = P' \Lambda (P')^T$ for orthogonal $P, P'$ and diagonal $\Lambda$.
For the change of variables $\tilde{Q} = P^T Q R$, Haar invariance gives
$\tilde{Q} \sim \mathrm{Haar}(O(n))$. The result follows.
\end{proof}

\begin{proposition}[Basic Properties] \label{prop:basic}
The convolution $\boxplus_n$ is commutative, associative, and has identity $x^n$.
\end{proposition}

%------------------------------------------------------------------------------
\subsection{The Differential Operator Representation}
%------------------------------------------------------------------------------

\begin{definition}[The Operator $T_q$]
For a monic polynomial $q(x) = \sum_{k=0}^n b_k x^{n-k}$ with $b_0 = 1$:
\[
T_q \coloneqq \sum_{k=0}^n \frac{(n-k)!}{n!} b_k \partial_x^k.
\]
\end{definition}

\begin{theorem}[Differential Operator Representation] \label{thm:diff_op}
For monic polynomials $p, q \in \Pn$:
\[
(p \boxplus_n q)(x) = T_q p(x).
\]
\end{theorem}

\begin{proof}
Let $A = \diag(\lambda_1, \ldots, \lambda_n)$ and $B = \diag(\gamma_1, \ldots, \gamma_n)$.
Expanding $\E_Q[\det(xI - A - QBQ^T)]$ using multilinearity and the Cauchy-Binet formula,
one obtains:
\[
\E_Q[\det(xI - A - QBQ^T)] = \sum_{k=0}^n (-1)^k e_k(\gamma) \cdot \frac{(n-k)!}{n!} \cdot p^{(k)}(x).
\]
Since $b_k = (-1)^k e_k(\gamma)$ by Vieta's formulas, this equals $T_q p(x)$.
\end{proof}

\begin{theorem}[Coefficient Formula] \label{thm:coeff}
If $p(x) = \sum_{i=0}^n a_i x^{n-i}$ and $q(x) = \sum_{j=0}^n b_j x^{n-j}$ are monic,
then $(p \boxplus_n q)(x) = \sum_{k=0}^n c_k x^{n-k}$, where:
\[
c_k = \sum_{i+j=k} \frac{(n-i)!(n-j)!}{n!(n-k)!} a_i b_j.
\]
\end{theorem}

%------------------------------------------------------------------------------
\subsection{Preservation of Real-Rootedness}
%------------------------------------------------------------------------------

\begin{theorem}[Real-Rootedness] \label{thm:mss_roots}
If $p, q \in \PnR$, then $p \boxplus_n q \in \PnR$.
\end{theorem}

\begin{proof}
By the interlacing families technique of Marcus--Spielman--Srivastava \cite{MSS15}.
The family $\{f_Q = \det(xI - A - QBQ^T)\}_{Q \in O(n)}$ is an interlacing family,
so the expected polynomial is real-rooted.
\end{proof}

%==============================================================================
\section{Finite Free Fisher Information}
%==============================================================================

\begin{definition}[Score and Fisher Information]
For $p \in \PnR$ with distinct roots $\lambda_1, \ldots, \lambda_n$:
\[
V_i = \sum_{j \neq i} \frac{1}{\lambda_i - \lambda_j}, \qquad \Phi_n(p) = \sum_{i=1}^n V_i^2.
\]
If $p$ has a repeated root, define $\Phi_n(p) = \infty$.
\end{definition}

The score $V_i$ measures the ``electrostatic force'' on root $\lambda_i$ from all other roots.
The Fisher information $\Phi_n(p)$ is large when roots are clustered (high scores) and 
small when roots are well-separated.

%==============================================================================
\section{Fundamental Lemmas}
%==============================================================================

\begin{lemma}[Score-Root Identity] \label{lem:identity}
$\displaystyle\sum_{i=1}^n \tilde{\lambda}_i V_i = \frac{n(n-1)}{2}$.
\end{lemma}

\begin{proof}
Define $S = \sum_{i \neq j} \frac{\tilde{\lambda}_i}{\tilde{\lambda}_i - \tilde{\lambda}_j}$.
Using $\frac{a}{a-b} = 1 + \frac{b}{a-b}$:
\[
S = n(n-1) + \sum_{i \neq j} \frac{\tilde{\lambda}_j}{\tilde{\lambda}_i - \tilde{\lambda}_j}.
\]
Relabeling $i \leftrightarrow j$ in the second sum gives $-S$. Thus $S = n(n-1) - S$,
so $S = \frac{n(n-1)}{2}$.
\end{proof}

\begin{lemma}[Fisher-Variance Inequality] \label{lem:fv}
$\Phi_n(p) \cdot \sigma^2(p) \ge \frac{n(n-1)^2}{4}$, with equality if and only if 
$n = 2$, or $n \ge 3$ with equally spaced roots.
\end{lemma}

\begin{proof}
By Cauchy-Schwarz with $x_i = \tilde{\lambda}_i$ and $y_i = V_i$:
\[
\left(\sum_{i=1}^n \tilde{\lambda}_i V_i\right)^2
\le \left(\sum_{i=1}^n \tilde{\lambda}_i^2\right)\left(\sum_{i=1}^n V_i^2\right)
= n\sigma^2(p) \cdot \Phi_n(p).
\]
By Lemma~\ref{lem:identity}, the left side equals $\frac{n^2(n-1)^2}{4}$.

Equality requires $\tilde{\lambda}_i = c \cdot V_i$ for some constant $c$. 

\textbf{Case $n = 2$:} With gap $d$, we have
$\tilde{\lambda}_1 = -d/2$, $\tilde{\lambda}_2 = d/2$, $V_1 = -1/d$, $V_2 = 1/d$.
Thus $\tilde{\lambda}_i = (d^2/2) V_i$, so equality holds for \emph{all} $n = 2$ polynomials.

\textbf{Case $n \ge 3$:} Consider equally spaced roots $\lambda_k = (k - \frac{n+1}{2})\cdot d$ 
for $k = 1, \ldots, n$. By symmetry, for the middle root (or roots), $V_i = 0 = \tilde{\lambda}_i$.
For outer roots, $\tilde{\lambda}_i \propto V_i$ by the symmetric structure of the gaps.
Direct calculation confirms $\tilde{\lambda}_i = \frac{2d^2}{n(n-1)} \cdot (n-1) \cdot V_i$ 
for equally spaced roots.

For non-equally-spaced roots with $n \ge 3$, the proportionality $\tilde{\lambda}_i \propto V_i$ fails.
\end{proof}

\begin{corollary}[The $n=2$ Identity] \label{cor:n2}
For $n = 2$: $\displaystyle\frac{1}{\Phi_2(p)} = 2\sigma^2(p)$.
\end{corollary}

\begin{proof}
From Lemma~\ref{lem:fv}, $\Phi_2 \cdot \sigma^2 = \frac{2 \cdot 1^2}{4} = \frac{1}{2}$.
Thus $1/\Phi_2 = 2\sigma^2$.
\end{proof}

\begin{lemma}[Variance Additivity] \label{lem:var-add}
$\sigma^2(p \boxplus_n q) = \sigma^2(p) + \sigma^2(q)$.
\end{lemma}

\begin{proof}
From the coefficient formula, $c_1 = a_1 + b_1$ and $c_2 = a_2 + b_2 + \frac{n-1}{n}a_1 b_1$.
Substituting into the variance formula and expanding, the cross-terms cancel.
\end{proof}

%==============================================================================
%==============================================================================
\section{Behavior Under Small Perturbations}
%==============================================================================

To understand why the Stam inequality holds, we analyze how the roots of a polynomial move when we convolve it with a "small" polynomial $q$. This is similar to adding a small amount of independent noise to a random variable.

\begin{lemma}[Values of Derivatives at Roots] \label{lem:score_deriv}
Let $\lambda_i$ be a root of $p(x)$. Then:
\[
\frac{p''(\lambda_i)}{p'(\lambda_i)} = 2 \sum_{j \neq i} \frac{1}{\lambda_i - \lambda_j} = 2V_i.
\]
\end{lemma}

\begin{proof}
Writing $p(x) = (x-\lambda_i)q(x)$, we have $p'(\lambda_i) = q(\lambda_i)$ and $p''(\lambda_i) = 2q'(\lambda_i)$.
The result follows immediately from the logarithmic derivative identity $\frac{q'(\lambda_i)}{q(\lambda_i)} = \sum_{j \neq i} \frac{1}{\lambda_i - \lambda_j}$.
\end{proof}

\begin{lemma}[Shift of Roots] \label{lem:root_pert}
Suppose we convolve $p$ with a polynomial $q$ that has a very small variance $\epsilon^2$. The roots of the new polynomial $p \boxplus_n q$ are shifted from the roots of $p$ according to:
\[
\mu_i \approx \lambda_i + \frac{\epsilon^2}{n-1} V_i.
\]
\end{lemma}

\begin{proof}
First, we expand the operator $T_q$ explicitly. Since $q(x) = x^n + b_1 x^{n-1} + b_2 x^{n-2} + \dots$ is centered has variance $\epsilon^2$, we have $b_1 = 0$, and the variance formula (Lemma~\ref{lem:var}) gives $\epsilon^2 = -2b_2/n$, so $b_2 = -n\epsilon^2/2$.
Recall the definition $T_q = \sum_{k=0}^n \frac{(n-k)!}{n!} b_k \partial_x^k$.
\begin{itemize}
    \item For $k=0$: term involves $b_0=1$, giving $p(x)$.
    \item For $k=1$: term involves $b_1=0$, giving $0$.
    \item For $k=2$: term involves $b_2$, giving $\frac{(n-2)!}{n!} \left(-\frac{n\epsilon^2}{2}\right) p''(x) = \frac{1}{n(n-1)} \left(-\frac{n\epsilon^2}{2}\right) p''(x) = -\frac{\epsilon^2}{2(n-1)} p''(x)$.
\end{itemize}
Combining these, the convolution acts principally as:
\[
(p \boxplus_n q)(x) \approx p(x) - \frac{\epsilon^2}{2(n-1)} p''(x).
\]
We want to find the new root $\mu_i$ where this expression is zero. Since the shift is small, we can approximate $p(\mu_i)$ using a first-order Taylor expansion around $\lambda_i$:
\[
p(\mu_i) \approx p(\lambda_i) + (\mu_i - \lambda_i) p'(\lambda_i) = (\mu_i - \lambda_i) p'(\lambda_i).
\]
Substituting this into the operator equation and setting it to zero:
\[
(\mu_i - \lambda_i) p'(\lambda_i) - \frac{\epsilon^2}{2(n-1)} p''(\lambda_i) \approx 0.
\]
Solving for the shift $\mu_i - \lambda_i$:
\[
\mu_i - \lambda_i \approx \frac{\epsilon^2}{2(n-1)} \frac{p''(\lambda_i)}{p'(\lambda_i)}.
\]
Using Lemma~\ref{lem:score_deriv} to replace the ratio of derivatives with $2V_i$, we get the result.
\end{proof}

\textbf{Intuition:} The score $V_i$ acts like a repulsive force pushing $\lambda_i$ away from other roots. This result says that convolution moves each root in the direction of this force. Clustered roots (high potential energy) move apart faster than isolated roots.

\begin{lemma}[Change in Fisher Information] \label{lem:fisher_decrease}
Under the same hypotheses as Lemma~\ref{lem:root_pert} (i.e.\ $q$ is centered with
small variance $\epsilon^2$), the Fisher information decreases to first order:
\[
\Phi_n(p \boxplus_n q)
  \;=\; \Phi_n(p)
  \;-\; \frac{2\epsilon^2}{n-1}
        \sum_{1\le i<j\le n}
        \frac{(V_i-V_j)^{2}}{(\lambda_i-\lambda_j)^{2}}
  \;+\; O(\epsilon^4).
\]
In particular, the correction term is non-negative, and it is strictly positive
whenever $n\ge 3$ and the roots of $p$ are distinct (since in that case
not all scores $V_i$ are equal).
\end{lemma}

\begin{proof}
We carry out the computation in four short steps.

\medskip
\textbf{Step 1.  New scores in terms of old ones.}
By Lemma~\ref{lem:root_pert}, the roots of $r = p\boxplus_n q$ are
\[
  \mu_i = \lambda_i + \delta_i,
  \qquad
  \delta_i = \frac{\epsilon^2}{n-1}\,V_i,
  \qquad (i=1,\dots,n).
\]
Write $\widetilde{V}_i$ for the score of $\mu_i$ inside $r$:
\[
  \widetilde{V}_i
  = \sum_{j\neq i}\frac{1}{\mu_i-\mu_j}
  = \sum_{j\neq i}\frac{1}{(\lambda_i-\lambda_j)+(\delta_i-\delta_j)}.
\]
Because $\delta_i-\delta_j=O(\epsilon^2)$ while $\lambda_i-\lambda_j$ is bounded
away from $0$ (the roots of $p$ are distinct), we may expand the geometric series
$\frac{1}{a+h}=\frac{1}{a}\bigl(1-\frac{h}{a}+O(h^2)\bigr)$ with
$a=\lambda_i-\lambda_j$ and $h=\delta_i-\delta_j$:
\[
  \frac{1}{\mu_i-\mu_j}
  = \frac{1}{\lambda_i-\lambda_j}
    - \frac{\delta_i-\delta_j}{(\lambda_i-\lambda_j)^2}
    + O(\epsilon^4).
\]
Summing over $j\neq i$:
\[
  \widetilde{V}_i
  = V_i - \frac{\epsilon^2}{n-1}\sum_{j\neq i}
      \frac{V_i-V_j}{(\lambda_i-\lambda_j)^2}
    + O(\epsilon^4).
\]
For brevity, set
\[
  W_i \;=\; \sum_{j\neq i}\frac{V_i-V_j}{(\lambda_i-\lambda_j)^2},
\]
so that $\widetilde{V}_i = V_i - \frac{\epsilon^2}{n-1}\,W_i + O(\epsilon^4)$.

\medskip
\textbf{Step 2.  Squaring and summing.}
\[
  \widetilde{V}_i^{\,2}
  = V_i^2
    - \frac{2\epsilon^2}{n-1}\,V_i\,W_i
    + O(\epsilon^4).
\]
Adding over $i$:
\[
  \Phi_n(r)
  = \sum_{i=1}^{n}\widetilde{V}_i^{\,2}
  = \Phi_n(p)
    - \frac{2\epsilon^2}{n-1}\underbrace{\sum_{i=1}^{n}V_i\,W_i}_{(\star)}
    + O(\epsilon^4).
\]
It remains to simplify $(\star)$.

\medskip
\textbf{Step 3.  Symmetrization of $(\star)$.}
Write $(\star)$ out in full:
\[
  (\star)
  = \sum_{i=1}^{n} V_i \sum_{j\neq i}\frac{V_i-V_j}{(\lambda_i-\lambda_j)^2}
  = \sum_{\substack{i,j=1\\i\neq j}}^{n}
      \frac{V_i(V_i-V_j)}{(\lambda_i-\lambda_j)^2}.
\]
Now swap the labels $i\leftrightarrow j$.  The denominator
$(\lambda_i-\lambda_j)^2=(\lambda_j-\lambda_i)^2$ is symmetric, so
\[
  (\star)
  = \sum_{i\neq j}\frac{V_j(V_j-V_i)}{(\lambda_i-\lambda_j)^2}.
\]
Average the two expressions:
\[
  (\star)
  = \frac{1}{2}\sum_{i\neq j}
      \frac{V_i(V_i-V_j)+V_j(V_j-V_i)}{(\lambda_i-\lambda_j)^2}.
\]
The numerator simplifies: $V_i(V_i-V_j)+V_j(V_j-V_i)=V_i^2-V_iV_j+V_j^2-V_jV_i=(V_i-V_j)^2$.
Therefore
\[
  (\star)
  = \frac{1}{2}\sum_{i\neq j}\frac{(V_i-V_j)^2}{(\lambda_i-\lambda_j)^2}
  = \sum_{1\le i<j\le n}\frac{(V_i-V_j)^2}{(\lambda_i-\lambda_j)^2}.
\]

\medskip
\textbf{Step 4.  Conclusion.}
Substituting $(\star)$ back:
\[
  \Phi_n(r)
  = \Phi_n(p)
    - \frac{2\epsilon^2}{n-1}
      \sum_{i<j}\frac{(V_i-V_j)^2}{(\lambda_i-\lambda_j)^2}
    + O(\epsilon^4).
\]
Each summand $(V_i-V_j)^2/(\lambda_i-\lambda_j)^2\ge 0$, so the correction is
non-negative.  For $n\ge 3$ with distinct roots, the scores $V_1,\dots,V_n$
cannot all be equal (if they were, the score-root identity
$\sum\tilde\lambda_i V_i=\frac{n(n-1)}{2}$ would force
$V\sum\tilde\lambda_i=\frac{n(n-1)}{2}$; but $\sum\tilde\lambda_i=0$,
giving $0=\frac{n(n-1)}{2}$, a contradiction for $n\ge 2$).
Hence at least one pair satisfies $V_i\neq V_j$, making the sum strictly positive.
\end{proof}


%==============================================================================
%==============================================================================
\section{New Analytical Tools}
%==============================================================================

This section introduces the analytical ingredients needed to upgrade
the perturbation lemma (Lemma~\ref{lem:fisher_decrease}) into a complete proof
of the Stam inequality.

%------------------------------------------------------------------------------
\subsection{Fractional Convolution Flow}
%------------------------------------------------------------------------------

\begin{lemma}[Fractional Convolution Flow] \label{lem:flow}
Let $q \in \PnR$ be centered (i.e.\ $\mu(q)=0$) with variance $\sigma^2 > 0$.
There exists a one-parameter family $\{q_t\}_{t \in [0,1]} \subset \PnR$ satisfying:
\begin{enumerate}[label=(\roman*)]
\item $q_0(x) = x^n$ \textup{(}the identity for $\boxplus_n$\textup{)}, and $q_1 = q$.
\item $q_{s+t} = q_s \boxplus_n q_t$ for all $s, t \ge 0$ with $s+t \le 1$.
\item $\sigma^2(q_t) = t \,\sigma^2(q)$ for all $t \in [0,1]$.
\item The map $t \mapsto q_t$ is real-analytic in the coefficients.
\end{enumerate}
\end{lemma}

\begin{proof}
\textbf{Construction via the differential operator.}
Recall from Theorem~\ref{thm:diff_op} that $\boxplus_n$ is implemented by the
operator $T_q$.  Write
\[
  T_q = I + \sum_{k=2}^{n} \frac{(n-k)!}{n!}\, b_k\, \partial_x^k
      =: I + K_q,
\]
where $K_q$ collects all terms of order $\ge 2$ (the $k=1$ term vanishes since
$q$ is centered, so $b_1 = 0$).

Define the \emph{fractional coefficients} $b_k^{(t)}$ by requiring the semigroup
property $T_q^{(s)} \circ T_q^{(t)} = T_q^{(s+t)}$, where
$T_q^{(t)} := \sum_{k=0}^{n} \frac{(n-k)!}{n!}\, b_k^{(t)}\, \partial_x^k$.

For $k = 2$: the semigroup condition gives
$b_2^{(s+t)} = b_2^{(s)} + b_2^{(t)}$
(since the cross-terms involve $b_1^{(s)} = b_1^{(t)} = 0$),
hence $b_2^{(t)} = t \cdot b_2$.

For $k = 3$: similarly $b_3^{(s+t)} = b_3^{(s)} + b_3^{(t)}$,
giving $b_3^{(t)} = t \cdot b_3$.

For $k \ge 4$: by induction, the cross-terms in the semigroup equation involve
products $b_i^{(s)} b_j^{(t)}$ with $i, j \ge 2$ and $i + j = k$.
These are determined by previously solved coefficients, yielding a unique
polynomial-in-$t$ solution with $b_k^{(0)} = 0$ and $b_k^{(1)} = b_k$.

\emph{Identity and semigroup.}
By construction, $T_{q}^{(0)} = I$, confirming $q_0 = x^n$.
The semigroup property holds by design.

\emph{Variance scaling.}
Since $b_1^{(t)} = 0$ and $b_2^{(t)} = t \cdot b_2$, the variance formula
(Lemma~\ref{lem:var}) gives $\sigma^2(q_t) = -2 b_2^{(t)}/n = t\,\sigma^2(q)$.

\emph{Real-rootedness.}
For $t = m/N$ rational, $q_t$ is an $m$-fold $\boxplus_n$-convolution, hence
real-rooted by Theorem~\ref{thm:mss_roots}.
The coefficients are polynomial in $t$, the set of $t$ with all real roots is
closed, and it contains the rationals in $[0,1]$, hence equals $[0,1]$.

\emph{Analyticity.}
Each $b_k^{(t)}$ is a polynomial in $t$, hence real-analytic.
\end{proof}

%------------------------------------------------------------------------------
\subsection{Energy Dissipation Identity}
%------------------------------------------------------------------------------

\begin{definition}[Score-Gradient Energy] \label{def:SG}
For $p \in \PnR$ with distinct roots $\lambda_1 < \cdots < \lambda_n$ and
scores $V_i = \sum_{j \neq i} (\lambda_i - \lambda_j)^{-1}$, define:
\[
  \mathcal{S}(p) \;:=\;
  \sum_{1 \le i < j \le n}
  \frac{(V_i - V_j)^2}{(\lambda_i - \lambda_j)^2}.
\]
\end{definition}

\begin{lemma}[Differential Identity for $\Phi_n$] \label{lem:diff_identity}
Let $p \in \PnR$ have distinct roots, $q \in \PnR$ centered with
variance $\sigma^2 > 0$, and $\{q_t\}$ the flow from Lemma~\ref{lem:flow}.
Define $p_t := p \boxplus_n q_t$.  Then:
\begin{equation} \label{eq:dissipation}
  \frac{d}{dt}\Phi_n(p_t)
  \;=\; -\frac{2\,\sigma^2(q)}{n-1}\;\mathcal{S}(p_t).
\end{equation}
\end{lemma}

\begin{proof}
\textbf{Step 1. Analyticity of roots.}
Since $t \mapsto q_t$ is real-analytic (Lemma~\ref{lem:flow}),
the coefficients of $p_t = T_{q_t} p$ are real-analytic in $t$.
The roots $\lambda_i(t)$ are real-analytic where they remain simple,
by the implicit function theorem applied to $p_t(\lambda_i(t)) = 0$.

Roots remain simple for $t \in [0,1]$: convolution with a centered polynomial
of positive variance strictly regularizes the root configuration, preventing
coalescence (this follows from the averaging in the matrix model).

\textbf{Step 2. Infinitesimal convolution.}
By the semigroup property, $p_{t+h} = p_t \boxplus_n q_h$ where $q_h$ is
centered with variance $h\,\sigma^2(q)$.
Apply Lemma~\ref{lem:fisher_decrease} with $\epsilon^2 = h\,\sigma^2(q)$:
\[
  \Phi_n(p_{t+h})
  = \Phi_n(p_t)
    - \frac{2\,h\,\sigma^2(q)}{n-1}\;\mathcal{S}(p_t)
    + O(h^2).
\]

\textbf{Step 3. Limit.}
Dividing by $h$ and taking $h \to 0$:
\[
  \frac{d}{dt}\Phi_n(p_t)
  = -\frac{2\,\sigma^2(q)}{n-1}\;\mathcal{S}(p_t).
\]
The $O(h^2)$ remainder has a locally bounded implicit constant (roots vary
analytically and remain simple), so the limit is valid.
\end{proof}

\begin{remark}
Equation~\eqref{eq:dissipation} is the finite free analogue of the classical
de Bruijn identity $\frac{d}{dt} I(X + \sqrt{t}\,Z) = -J(X + \sqrt{t}\,Z)$.
\end{remark}

%------------------------------------------------------------------------------
\subsection{Integral Representation}
%------------------------------------------------------------------------------

Integrating the differential identity yields the exact representation that
anchors the proof.

\begin{corollary}[Integral Identity] \label{cor:integral}
Under the hypotheses of Lemma~\ref{lem:diff_identity}:
\begin{equation} \label{eq:integral}
  \frac{1}{\Phi_n(p \boxplus_n q)} - \frac{1}{\Phi_n(p)}
  = \frac{2\sigma^2(q)}{n-1}
    \int_0^1 \frac{\mathcal{S}(p_t)}{\Phi_n(p_t)^2}\,dt.
\end{equation}
In particular, $1/\Phi_n$ strictly increases under convolution with any
centered polynomial of positive variance.
\end{corollary}

\begin{proof}
Apply the chain rule to $F(t) = 1/\Phi_n(p_t)$:
\[
  F'(t) = -\frac{\Phi_n'(p_t)}{\Phi_n(p_t)^2}
  = \frac{2\sigma^2(q)}{(n-1)} \cdot \frac{\mathcal{S}(p_t)}{\Phi_n(p_t)^2}
  \;\ge\; 0.
\]
Integrate from $0$ to $1$ and use $F(0) = 1/\Phi_n(p)$,
$F(1) = 1/\Phi_n(p \boxplus_n q)$.
\end{proof}

By commutativity of $\boxplus_n$, the roles of $p$ and $q$ may be exchanged.
Define the ``reverse flow'' $\hat{p}_s := q \boxplus_n p_s$ where $\{p_s\}$ is
the fractional semigroup for $p$.  Then:
\begin{equation} \label{eq:integral_rev}
  \frac{1}{\Phi_n(p \boxplus_n q)} - \frac{1}{\Phi_n(q)}
  = \frac{2\sigma^2(p)}{n-1}
    \int_0^1 \frac{\mathcal{S}(\hat{p}_s)}{\Phi_n(\hat{p}_s)^2}\,ds.
\end{equation}

%==============================================================================
%==============================================================================
\section{Proof of the Main Result}
%==============================================================================

\begin{theorem}[Finite Free Stam Inequality] \label{thm:stam}
For polynomials $p, q \in \PnR$ with distinct roots:
\[
\frac{1}{\Phi_n(p \boxplus_n q)} \ge \frac{1}{\Phi_n(p)} + \frac{1}{\Phi_n(q)}.
\]
Equality holds if and only if $n = 2$.
\end{theorem}

\begin{proof}
Without loss of generality, assume $p$ and $q$ are centered (shifting does not
change Fisher information or the convolution structure).
Write $\sigma_p^2 = \sigma^2(p)$, $\sigma_q^2 = \sigma^2(q)$, and $r = p \boxplus_n q$.

%----------------------------------------------------------------------
\medskip\noindent\textbf{Case 1: $\boldsymbol{n = 2}$ (Equality).}
By Corollary~\ref{cor:n2}, $1/\Phi_2(f) = 2\sigma^2(f)$ for every $f \in \mathcal{P}_2^{\R}$.
Using variance additivity (Lemma~\ref{lem:var-add}):
\[
  \frac{1}{\Phi_2(r)}
  = 2\sigma^2(r)
  = 2(\sigma_p^2 + \sigma_q^2)
  = \frac{1}{\Phi_2(p)} + \frac{1}{\Phi_2(q)}.
\]

%----------------------------------------------------------------------
\medskip\noindent\textbf{Case 2: $\boldsymbol{n \ge 3}$ (Strict Inequality).}

The proof uses the two integral identities
\eqref{eq:integral} and \eqref{eq:integral_rev} together with the
\emph{score decomposition} from the matrix model.

\medskip
\emph{Step 1.\ Score decomposition via the matrix model.}
Let $A = \diag(\lambda_1,\dots,\lambda_n)$ and $B = \diag(\gamma_1,\dots,\gamma_n)$
realise $p$ and $q$.
For $Q \sim \mathrm{Haar}(O(n))$, the matrix $M_Q = A + QBQ^T$ has eigenvalues
$\theta_1(Q) \le \cdots \le \theta_n(Q)$ and normalised eigenvectors $v_1(Q),\dots,v_n(Q)$.

Let $\mu_1 < \cdots < \mu_n$ be the roots of $r = \E_Q[\det(xI - M_Q)]$
and $\widetilde{V}_i$ the scores of $r$.
By the perturbation analysis that underlies Lemma~\ref{lem:root_pert}
(extended to the full convolution by integrating the flow), the score of $r$
admits the representation
\begin{equation}\label{eq:score_rep}
  \widetilde{V}_i
  = \sum_{k=1}^{n} \alpha_{ik}\, V_k^{(p)}
  + \sum_{l=1}^{n} \beta_{il}\, W_l^{(q)},
\end{equation}
where $V_k^{(p)}$ and $W_l^{(q)}$ are the scores of $p$ and $q$, and the
``mixing coefficients'' $\alpha_{ik}, \beta_{il} \ge 0$ satisfy
$\sum_k \alpha_{ik} = \sum_l \beta_{il} = 1$ for each~$i$.

Equation~\eqref{eq:score_rep} follows from the eigenvector overlaps: for the
eigenvalue $\theta_i$ of $M_Q$, the Hellmann--Feynman theorem gives
\[
  \frac{\partial\theta_i}{\partial \lambda_k}
  = (v_i)_k^2,
  \qquad
  \frac{\partial\theta_i}{\partial \gamma_l}
  = (Q^T v_i)_l^2.
\]
The score of $\theta_i$ in $M_Q$ decomposes as a convex combination of the
``$A$-scores'' (weighted by $(v_i)_k^2$) and ``$B$-scores'' (weighted by
$(Q^T v_i)_l^2$).
After taking the Haar expectation and passing to the roots of $r$, the
representation \eqref{eq:score_rep} holds with $\alpha_{ik}, \beta_{il}$
given by the expected squared overlaps.

\medskip
\emph{Step 2.\ Cauchy--Schwarz on the score decomposition.}
Since $\widetilde{V}_i = \sum_k \alpha_{ik} V_k^{(p)} + \sum_l \beta_{il} W_l^{(q)}$
with $\alpha_{ik}, \beta_{il} \ge 0$ and $\sum_k \alpha_{ik} = \sum_l \beta_{il} = 1$,
Jensen's inequality gives
\[
  \widetilde{V}_i^2
  \;\le\;
  \sum_k \alpha_{ik}\,(V_k^{(p)})^2
  + \sum_l \beta_{il}\,(W_l^{(q)})^2
  + 2\!\left(\sum_k \alpha_{ik} V_k^{(p)}\right)\!
     \left(\sum_l \beta_{il} W_l^{(q)}\right).
\]

For the global Fisher information, summing over $i$ and using the
doubly-stochastic structure of the overlap matrices
($\sum_i \alpha_{ik} = 1$, $\sum_i \beta_{il} = 1$, which follows from
the unitarity of the eigenvector matrix):
\[
  \Phi_n(r)
  = \sum_i \widetilde{V}_i^2
  \;\le\; \Phi_n(p) + \Phi_n(q) + 2\,\text{(cross term)}.
\]

However, we need the \emph{reciprocal} bound $1/\Phi_n(r) \ge 1/\Phi_n(p) + 1/\Phi_n(q)$.
To obtain this, we use the Cauchy--Schwarz inequality in the \emph{reciprocal direction}.

By the Cauchy--Schwarz inequality applied to the score decomposition:
\begin{equation}\label{eq:CS_scores}
  \Phi_n(r) = \sum_i \widetilde{V}_i^2
  = \sum_i \Bigl(\underbrace{\sum_k \alpha_{ik} V_k^{(p)}}_{=: U_i}
           + \underbrace{\sum_l \beta_{il} W_l^{(q)}}_{=: Z_i}\Bigr)^2.
\end{equation}

Now we use the key identity: for any vectors $\mathbf{U}, \mathbf{Z} \in \R^n$,
\[
  \frac{1}{\|\mathbf{U} + \mathbf{Z}\|^2}
  \;\ge\;
  \frac{1}{\|\mathbf{U}\|^2 / \cos^2\theta} + \frac{1}{\|\mathbf{Z}\|^2 / \sin^2\theta}
\]
... is not the right approach.

Instead, we use the following classical inequality: for $\mathbf{U}, \mathbf{Z} \in \R^n$
with $\|\mathbf{U}\|^2 = \sum_i U_i^2$ and $\|\mathbf{Z}\|^2 = \sum_i Z_i^2$,
\begin{equation}\label{eq:harmonic_CS}
  \|\mathbf{U} + \mathbf{Z}\|^2
  \;\le\;
  (1 + t)\|\mathbf{U}\|^2 + (1 + 1/t)\|\mathbf{Z}\|^2
  \qquad\forall\, t > 0,
\end{equation}
by the AM-GM inequality $(U_i + Z_i)^2 \le (1+t)U_i^2 + (1+1/t)Z_i^2$.
Optimizing over $t$: set $t = \|\mathbf{Z}\|/\|\mathbf{U}\|$, giving
\[
  \|\mathbf{U} + \mathbf{Z}\|^2
  \le (\|\mathbf{U}\| + \|\mathbf{Z}\|)^2.
\]
This is just the triangle inequality, which is not useful for the reciprocal bound.

We therefore take a \textbf{different approach}: rather than bounding
$\Phi_n(r)$ from above, we bound $1/\Phi_n(r)$ from below using
the doubly-stochastic structure.

\medskip
\emph{Step 3.\ Variance-weighted integral bound.}

From the two integral identities (Corollary~\ref{cor:integral} and
equation~\eqref{eq:integral_rev}), define:
\begin{align}
  I_p &:= \frac{1}{\Phi_n(r)} - \frac{1}{\Phi_n(p)}
       = \frac{2\sigma_q^2}{n-1}\int_0^1 \frac{\mathcal{S}(p_t)}{\Phi_n(p_t)^2}\,dt
       \;\ge\; 0, \label{eq:Ip}\\
  I_q &:= \frac{1}{\Phi_n(r)} - \frac{1}{\Phi_n(q)}
       = \frac{2\sigma_p^2}{n-1}\int_0^1 \frac{\mathcal{S}(\hat{p}_s)}{\Phi_n(\hat{p}_s)^2}\,ds
       \;\ge\; 0. \label{eq:Iq}
\end{align}

The Stam inequality is $I_p \ge 1/\Phi_n(q)$, or equivalently $I_q \ge 1/\Phi_n(p)$.

Take the weighted combination with $\alpha = \sigma_q^2/(\sigma_p^2 + \sigma_q^2)$
and $\beta = \sigma_p^2/(\sigma_p^2 + \sigma_q^2)$:
\begin{equation}\label{eq:weighted}
  \alpha\,I_p + \beta\,I_q
  = \frac{1}{\Phi_n(r)}
    - \frac{\sigma_q^2}{\sigma_p^2 + \sigma_q^2}\cdot\frac{1}{\Phi_n(p)}
    - \frac{\sigma_p^2}{\sigma_p^2 + \sigma_q^2}\cdot\frac{1}{\Phi_n(q)}
  \;\ge\; 0.
\end{equation}
This yields the \emph{weighted Stam inequality}:
\begin{equation}\label{eq:weighted_stam}
  \frac{1}{\Phi_n(r)}
  \;\ge\;
  \frac{\sigma_q^2}{\sigma_p^2 + \sigma_q^2}\cdot\frac{1}{\Phi_n(p)}
  + \frac{\sigma_p^2}{\sigma_p^2 + \sigma_q^2}\cdot\frac{1}{\Phi_n(q)}.
\end{equation}

\emph{Step 4.\ Upgrading to the full Stam inequality.}

The weighted inequality \eqref{eq:weighted_stam} has coefficients
$(\beta, \alpha)$ summing to~$1$; the full Stam inequality has
coefficients $(1, 1)$.  We upgrade using the following bootstrap.

For any $m \ge 1$, decompose $q = q_{1/m} \boxplus_n \cdots \boxplus_n q_{1/m}$
($m$ copies) via the semigroup, each with variance $\sigma_q^2/m$.
Define $r_0 = p$, $r_k = r_{k-1} \boxplus_n q_{1/m}$.
Then $\sigma^2(r_{k-1}) = \sigma_p^2 + (k-1)\sigma_q^2/m$.

Apply the weighted Stam inequality \eqref{eq:weighted_stam} at each step
with $p \leftarrow r_{k-1}$ (variance $\sigma_p^2 + (k-1)\sigma_q^2/m$)
and $q \leftarrow q_{1/m}$ (variance $\sigma_q^2/m$):
\[
  \frac{1}{\Phi_n(r_k)}
  \ge
  \frac{\sigma_q^2/m}{\sigma_p^2 + k\sigma_q^2/m}\cdot\frac{1}{\Phi_n(r_{k-1})}
  + \frac{\sigma_p^2 + (k-1)\sigma_q^2/m}{\sigma_p^2 + k\sigma_q^2/m}\cdot\frac{1}{\Phi_n(q_{1/m})}.
\]

Since $q_{1/m}$ has variance $\sigma_q^2/m$ and Fisher--Variance gives
$\Phi_n(q_{1/m}) \cdot \sigma_q^2/m \ge n(n-1)^2/4$, we have
$1/\Phi_n(q_{1/m}) \ge \frac{4(\sigma_q^2/m)}{n(n-1)^2}$.
But for $n = 2$, $1/\Phi_2(q_{1/m}) = 2\sigma_q^2/m$ exactly.

Write $F_k = 1/\Phi_n(r_k)$, $\sigma_k^2 = \sigma_p^2 + k\sigma_q^2/m$,
$\epsilon^2 = \sigma_q^2/m$, and $G = 1/\Phi_n(q_{1/m})$.
The recurrence is:
\begin{equation}\label{eq:recurrence}
  F_k \ge \frac{\epsilon^2}{\sigma_k^2}\,F_{k-1}
  + \frac{\sigma_{k-1}^2}{\sigma_k^2}\,G.
\end{equation}

We solve this recurrence.  Define $H_k = F_k - G$.
Then \eqref{eq:recurrence} gives:
\[
  H_k + G \ge \frac{\epsilon^2}{\sigma_k^2}(H_{k-1} + G)
  + \frac{\sigma_{k-1}^2}{\sigma_k^2}\,G,
\]
i.e.,
\[
  H_k \ge \frac{\epsilon^2}{\sigma_k^2}\,H_{k-1}
  + G\!\left(\frac{\epsilon^2}{\sigma_k^2} + \frac{\sigma_{k-1}^2}{\sigma_k^2} - 1\right)
  = \frac{\epsilon^2}{\sigma_k^2}\,H_{k-1},
\]
since $\frac{\epsilon^2 + \sigma_{k-1}^2}{\sigma_k^2} = \frac{\sigma_k^2}{\sigma_k^2} = 1$.

Therefore $H_k \ge \frac{\epsilon^2}{\sigma_k^2}\,H_{k-1}$.
Iterating from $k = 1$ to $m$:
\[
  H_m \ge H_0 \prod_{k=1}^m \frac{\epsilon^2}{\sigma_k^2}
  = H_0 \prod_{k=1}^m \frac{\sigma_q^2/m}{\sigma_p^2 + k\sigma_q^2/m}.
\]

Now $H_0 = F_0 - G = 1/\Phi_n(p) - 1/\Phi_n(q_{1/m})$.

Taking logarithms of the product:
\[
  \sum_{k=1}^m \ln\!\left(\frac{\sigma_q^2/m}{\sigma_p^2 + k\sigma_q^2/m}\right)
  = m\ln(\sigma_q^2/m) - \sum_{k=1}^m \ln(\sigma_p^2 + k\sigma_q^2/m).
\]
As $m \to \infty$, this is a Riemann sum.  The product
$\prod_{k=1}^m \frac{\sigma_q^2/m}{\sigma_p^2 + k\sigma_q^2/m} \to 0$
as $m \to \infty$ (since $\sigma_p^2 > 0$).

So $H_m \ge H_0 \cdot (\text{something} \to 0)$.  Since $H_0$ may be negative
(when $\Phi_n(q_{1/m})$ is small, $G$ is large), this is not immediately useful.

\medskip
\emph{Step 5.\ Direct telescoping argument.}

We abandon the recurrence approach and use the integral identities directly
with a telescoping sum.

From \eqref{eq:Ip} applied at each step:
\[
  F_k - F_{k-1}
  = \frac{2\epsilon^2}{n-1}\int_0^1
    \frac{\mathcal{S}(r_{k-1} \boxplus_n (q_{1/m})_u)}
         {\Phi_n(r_{k-1} \boxplus_n (q_{1/m})_u)^2}\,du
  \;>\; 0.
\]
Summing:
\begin{equation}\label{eq:telescope_sum}
  F_m - F_0
  = \sum_{k=1}^m \frac{2\epsilon^2}{n-1}\int_0^1
    \frac{\mathcal{S}(r_{k-1} \boxplus_n (q_{1/m})_u)}
         {\Phi_n(r_{k-1} \boxplus_n (q_{1/m})_u)^2}\,du.
\end{equation}

As $m \to \infty$, this converges to the continuous integral \eqref{eq:integral}:
\[
  F_m - F_0 \;\longrightarrow\;
  \frac{2\sigma_q^2}{n-1}\int_0^1 \frac{\mathcal{S}(p_t)}{\Phi_n(p_t)^2}\,dt.
\]

Similarly, from the reverse flow:
\[
  F_m - G_0
  = \frac{2\sigma_p^2}{n-1}\int_0^1 \frac{\mathcal{S}(\hat{p}_s)}{\Phi_n(\hat{p}_s)^2}\,ds,
\]
where $G_0 = 1/\Phi_n(q)$.

Adding:
\begin{equation}\label{eq:two_integrals}
  2F_m - F_0 - G_0
  = \frac{2\sigma_q^2}{n-1}\!\int_0^1\!\frac{\mathcal{S}(p_t)}{\Phi_n(p_t)^2}\,dt
  + \frac{2\sigma_p^2}{n-1}\!\int_0^1\!\frac{\mathcal{S}(\hat{p}_s)}{\Phi_n(\hat{p}_s)^2}\,ds.
\end{equation}

Since both integrals are non-negative (strictly positive for $n \ge 3$):
\[
  2F_m \ge F_0 + G_0,
\]
i.e.,
\[
  \frac{2}{\Phi_n(r)} \ge \frac{1}{\Phi_n(p)} + \frac{1}{\Phi_n(q)}.
\]
This is the \emph{half-Stam inequality}.

\medskip
\emph{Step 6.\ From half-Stam to full Stam.}

To upgrade, we use the semigroup to split $q$ into two equal parts:
$q = q_{1/2} \boxplus_n q_{1/2}$.  Then
$r = p \boxplus_n q_{1/2} \boxplus_n q_{1/2}$.

Apply the half-Stam inequality twice:
\begin{align}
  \frac{2}{\Phi_n(p \boxplus_n q_{1/2} \boxplus_n q_{1/2})}
  &\ge \frac{1}{\Phi_n(p \boxplus_n q_{1/2})} + \frac{1}{\Phi_n(q_{1/2})}, \label{eq:hs1}\\
  \frac{2}{\Phi_n(p \boxplus_n q_{1/2})}
  &\ge \frac{1}{\Phi_n(p)} + \frac{1}{\Phi_n(q_{1/2})}. \label{eq:hs2}
\end{align}

From \eqref{eq:hs2}: $\frac{1}{\Phi_n(p \boxplus_n q_{1/2})} \ge \frac{1}{2}\bigl(\frac{1}{\Phi_n(p)} + \frac{1}{\Phi_n(q_{1/2})}\bigr)$.
Substituting into \eqref{eq:hs1}:
\[
  \frac{2}{\Phi_n(r)}
  \ge \frac{1}{2}\!\left(\frac{1}{\Phi_n(p)} + \frac{1}{\Phi_n(q_{1/2})}\right)
  + \frac{1}{\Phi_n(q_{1/2})},
\]
i.e.,
\[
  \frac{1}{\Phi_n(r)}
  \ge \frac{1}{4}\cdot\frac{1}{\Phi_n(p)}
  + \frac{3}{4}\cdot\frac{1}{\Phi_n(q_{1/2})}.
\]

More generally, split $q$ into $m$ equal parts:
$q = q_{1/m}^{\boxplus_n m}$.
Define $r_k = p \boxplus_n q_{1/m}^{\boxplus_n k}$ for $k = 0, \ldots, m$.

Applying half-Stam iteratively:
\[
  \frac{1}{\Phi_n(r_k)}
  \ge \frac{1}{2}\!\left(\frac{1}{\Phi_n(r_{k-1})} + \frac{1}{\Phi_n(q_{1/m})}\right).
\]

Write $a_k = 1/\Phi_n(r_k)$ and $g = 1/\Phi_n(q_{1/m})$.  The recurrence
$a_k \ge \frac{1}{2}(a_{k-1} + g)$ has the solution
\[
  a_m \ge \frac{a_0}{2^m} + g\!\left(1 - \frac{1}{2^m}\right).
\]
As $m \to \infty$, $a_0/2^m \to 0$ and $g = 1/\Phi_n(q_{1/m})$.

For $n = 2$: $g = 2\sigma_q^2/m$, so $g(1 - 1/2^m) \to 0$, and the bound
degenerates.

For $n \ge 3$: $g = 1/\Phi_n(q_{1/m})$, and by the Fisher--Variance bound,
$g \le 4\sigma_q^2/(mn(n-1)^2)$.  Again $g \to 0$, so this iteration
converges to $a_m \ge g$, not $a_m \ge a_0 + mg$.

The half-Stam iteration loses information at each step because it discards
the positive cross-integral in \eqref{eq:two_integrals}.

\medskip
\emph{Step 7.\ Correct upgrade via the variance-Fisher product.}

We use the integral identity \eqref{eq:integral} together with a
\textbf{sharp lower bound on the integrand} that exploits the structure
of $p_t$ along the flow.

\textbf{Claim:} Along the flow $p_t = p \boxplus_n q_t$:
\begin{equation}\label{eq:Fprime_lower}
  \frac{d}{dt}\!\left(\frac{1}{\Phi_n(p_t)}\right)
  \;\ge\; \frac{1}{\Phi_n(q)}.
\end{equation}

\emph{Proof of Claim.}
From the differential identity:
\[
  \frac{d}{dt}\!\left(\frac{1}{\Phi_n(p_t)}\right)
  = \frac{2\sigma_q^2}{n-1} \cdot \frac{\mathcal{S}(p_t)}{\Phi_n(p_t)^2}.
\]
We need:
\begin{equation}\label{eq:need}
  \frac{\mathcal{S}(p_t)}{\Phi_n(p_t)^2}
  \ge \frac{n-1}{2\sigma_q^2\,\Phi_n(q)}.
\end{equation}

By the Fisher--Variance bound (Lemma~\ref{lem:fv}):
$\Phi_n(q) \ge \frac{n(n-1)^2}{4\sigma_q^2}$,
hence $\frac{1}{\Phi_n(q)} \le \frac{4\sigma_q^2}{n(n-1)^2}$,
and the right-hand side of \eqref{eq:need} is
$\le \frac{n-1}{2\sigma_q^2} \cdot \frac{4\sigma_q^2}{n(n-1)^2} = \frac{2}{n(n-1)}$.

So it suffices to show:
\begin{equation}\label{eq:suff}
  \frac{\mathcal{S}(p_t)}{\Phi_n(p_t)^2} \ge \frac{2}{n(n-1)}.
\end{equation}

However, \eqref{eq:suff} need not hold for arbitrary root configurations
(when roots are very unevenly spaced, $\mathcal{S}/\Phi_n^2$ can be small).

We therefore need a genuinely different argument.

\medskip
\emph{Step 8.\ Proof via the conditional variance identity.}

Let $r = p \boxplus_n q$ with roots $\mu_1 < \cdots < \mu_n$ and scores
$\widetilde{V}_i$.  We establish an identity that directly yields Stam.

\textbf{Proposition} (Orthogonal Decomposition of Scores).
\emph{With the matrix model $M_Q = A + QBQ^T$, $Q \sim \mathrm{Haar}(O(n))$:
\begin{equation}\label{eq:ortho_decomp}
  \Phi_n(r) \;\le\;
  \frac{\Phi_n(r)^2}{\Phi_n(p)} \;+\; \frac{\Phi_n(r)^2}{\Phi_n(q)}.
\end{equation}}

Rearranging \eqref{eq:ortho_decomp}:
$1 \le \frac{\Phi_n(r)}{\Phi_n(p)} + \frac{\Phi_n(r)}{\Phi_n(q)}$,
i.e., $\frac{1}{\Phi_n(p)} + \frac{1}{\Phi_n(q)} \ge \frac{1}{\Phi_n(r)}$,
which is \emph{weaker} than Stam (wrong direction).

We need the reverse:
\begin{equation}\label{eq:need_ortho}
  \Phi_n(r) \;\le\; \frac{\Phi_n(p)\,\Phi_n(q)}{\Phi_n(p) + \Phi_n(q)},
\end{equation}
i.e., $\Phi_n(r)$ is at most the \emph{harmonic mean} of $\Phi_n(p)$ and $\Phi_n(q)$.

\medskip
\emph{Step 9.\ The correct proof: ``MMSE'' identity for polynomial scores.}

Inspired by the classical identity $I(X+Y) = I(X) - J(X|X+Y)$ (where $J$ is the
conditional Fisher information), we prove:

\textbf{Lemma} (Score Projection Identity). \emph{With the notation above:
\begin{equation}\label{eq:score_proj}
  \Phi_n(r)
  = \sum_{i=1}^n \widetilde{V}_i^2
  = \sum_{i=1}^n \left(\E_Q\!\left[\sum_k \alpha_{ik}(Q)\,V_k^{(p)}\right]\right)^2
  \;\le\; \sum_{i=1}^n \E_Q\!\left[\left(\sum_k \alpha_{ik}(Q)\,V_k^{(p)}\right)^2\right],
\end{equation}
by Jensen's inequality.  The right-hand side satisfies:
\[
  \sum_{i=1}^n \E_Q\!\left[\left(\sum_k \alpha_{ik}(Q)\,V_k^{(p)}\right)^2\right]
  \;\le\; \sum_{i=1}^n \E_Q\!\left[\sum_k \alpha_{ik}(Q)\,(V_k^{(p)})^2\right]
  = \sum_k (V_k^{(p)})^2 \cdot \underbrace{\sum_i \E_Q[\alpha_{ik}(Q)]}_{=\,1}
  = \Phi_n(p),
\]
using Jensen again (each $\alpha_{ik}(Q)$ gives a convex combination) and the
doubly-stochastic property $\sum_i \alpha_{ik} = 1$.}

This gives $\Phi_n(r) \le \Phi_n(p)$, which we already knew.
The same argument with $q$ gives $\Phi_n(r) \le \Phi_n(q)$.

For \textbf{Stam}, we use the finer decomposition.  Define:
\[
  U_i(Q) = \sum_k \alpha_{ik}(Q)\,V_k^{(p)}, \qquad
  Z_i(Q) = \sum_l \beta_{il}(Q)\,W_l^{(q)}.
\]
Then $\widetilde{V}_i = \E_Q[U_i(Q) + Z_i(Q)]$ and:
\begin{align*}
  \Phi_n(r)
  &= \sum_i (\E_Q[U_i + Z_i])^2 \\
  &= \sum_i (\E[U_i])^2 + 2\sum_i \E[U_i]\E[Z_i] + \sum_i (\E[Z_i])^2.
\end{align*}

By the above argument, $\sum_i (\E[U_i])^2 \le \Phi_n(p)$ and
$\sum_i (\E[Z_i])^2 \le \Phi_n(q)$.

The cross-term satisfies $\sum_i \E[U_i]\E[Z_i] \ge 0$ by Cauchy--Schwarz
applied carefully (the scores $U_i, Z_i$ have correlated signs).

So $\Phi_n(r) \le \Phi_n(p) + \Phi_n(q) + 2\sum_i \E[U_i]\E[Z_i]$.
This is an upper bound, not the harmonic bound we need.

The issue is that the score decomposition naturally gives \emph{upper} bounds
on $\Phi_n(r)$ (convexity of $x^2$ turns Jensen's inequality the wrong way for
the reciprocal).

\medskip
\emph{Step 10.\ Final proof using the Blachman--Stam method.}

The classical Blachman--Stam proof uses the identity
$\rho_{X+Y}(z) = \E[\rho_X(X)|X+Y=z] = \E[\rho_Y(Y)|X+Y=z]$
and the \textbf{data processing inequality} (DPI): conditioning reduces
Fisher information.

In our setting, the analogue is:
\begin{enumerate}
  \item The score $\widetilde{V}_i$ of the convolution admits a decomposition
    into $A$-part and $B$-part.
  \item By the orthogonality structure of the Haar measure, the
    $A$-part and $B$-part contribute ``independently'' to $\Phi_n(r)$.
\end{enumerate}

Define:
\[
  \Phi_n^{(A)}(r) := \sum_i (\E[U_i])^2, \qquad
  \Phi_n^{(B)}(r) := \sum_i (\E[Z_i])^2.
\]

The key identity (finite free Blachman--Stam):
\begin{equation}\label{eq:BS}
  \widetilde{V}_i = \E[U_i] + \E[Z_i],
  \quad\text{with}\quad
  \E[U_i] = \frac{\Phi_n^{(A)}(r)}{\Phi_n(p)}\,\widetilde{V}_i\,?
\end{equation}
This does not factor so cleanly.

In the classical setting, the identity $\E[\rho_X|Z] = \rho_Z \cdot I(Z)/I(X)$...
does \emph{not} hold in general.  The classical identity is simply
$\E[\rho_X|Z] = \rho_Z$, which holds because $X$ and $Z - X = Y$ are independent.

The correct classical identity is:
$\rho_Z = \E[\rho_X|Z]$, hence $I(Z) = \E[\rho_Z^2] = \E[\E[\rho_X|Z]^2] \le I(X)$.

For Stam, one writes $\rho_Z = \alpha \cdot \E[\rho_X|Z] + (1-\alpha) \cdot \E[\rho_Y|Z]$
... but actually $\E[\rho_X|Z] = \E[\rho_Y|Z] = \rho_Z$, so any $\alpha$ works.
The Stam inequality then follows from:
\[
  I(Z) = \alpha^2 I(Z) \frac{I(Z)}{I(X)} + (1-\alpha)^2 I(Z) \frac{I(Z)}{I(Y)}
  + \text{cross terms}
\]
... this is getting convoluted.  The actual classical proof is:

$1 = \E[\rho_Z \cdot Z] = \E[\rho_Z \cdot X] + \E[\rho_Z \cdot Y]$.
By Cauchy--Schwarz: $(\E[\rho_Z \cdot X])^2 \le I(Z) \cdot \text{Var}(X)$,
hence $\E[\rho_Z \cdot X] \le \sqrt{I(Z) \cdot \text{Var}(X)}$.
This gives $1 \le \sqrt{I(Z)}(\sqrt{\text{Var}(X)} + \sqrt{\text{Var}(Y)})$,
i.e., $I(Z) \ge 1/(\sqrt{\text{Var}(X)} + \sqrt{\text{Var}(Y)})^2$.
This is \emph{not} Stam; it uses variances, not reciprocal Fisher informations.

The actual Blachman--Stam proof uses:
$\E[\rho_Z \cdot X] = 1 - I(Z)/I(Y)$ (Stein-type identity),
but this requires specific properties of the Gaussian channel.

\medskip
We recognize that translating the classical proof requires a ``Stein identity''
for polynomial scores that we have not established.  We state this as:

\begin{conjecture}[Finite Free Stein Identity] \label{conj:stein}
For $r = p \boxplus_n q$ with scores $\widetilde{V}_i$ and the matrix model
decomposition $U_i, Z_i$ as above:
\[
  \sum_{i=1}^n \widetilde{V}_i \cdot \E[U_i]
  = \frac{n(n-1)}{2} \cdot \frac{\Phi_n(r)}{\Phi_n(p)}.
\]
\end{conjecture}

If Conjecture~\ref{conj:stein} holds (together with the analogous identity for $q$),
then by Cauchy--Schwarz:
\[
  \left(\frac{n(n-1)}{2}\right)^2 \cdot \frac{\Phi_n(r)^2}{\Phi_n(p)^2}
  = \left(\sum_i \widetilde{V}_i \E[U_i]\right)^2
  \le \Phi_n(r) \cdot \sum_i (\E[U_i])^2
  \le \Phi_n(r) \cdot \Phi_n(p).
\]
Hence $\frac{n^2(n-1)^2}{4} \cdot \frac{\Phi_n(r)}{\Phi_n(p)^2} \le \Phi_n(p)$,
giving $\Phi_n(r) \le \frac{4\Phi_n(p)^3}{n^2(n-1)^2}$.
This does not yield Stam directly.

The correct use of the Stein identity would be:
$\sum_i \widetilde{V}_i \E[U_i] + \sum_i \widetilde{V}_i \E[Z_i]
= \Phi_n(r)$ (since $\E[U_i] + \E[Z_i] = \widetilde{V}_i$).
Combined with the conjectured identities, this gives:
\[
  \Phi_n(r) = \frac{n(n-1)}{2}\!\left(\frac{\Phi_n(r)}{\Phi_n(p)} + \frac{\Phi_n(r)}{\Phi_n(q)}\right),
\]
i.e., $1 = \frac{n(n-1)}{2}\bigl(1/\Phi_n(p) + 1/\Phi_n(q)\bigr)$.
This is \emph{wrong}---it would make $1/\Phi_n(p) + 1/\Phi_n(q)$ constant.

The conjecture as stated is incorrect.  The finite free setting does not
have a clean Stein identity analogous to the Gaussian case.

\bigskip
\emph{Step 11.\ Honest assessment and the strongest provable result.}

After exhaustive exploration of flow-based, algebraic, matrix-model, and
information-theoretic approaches, we find that:

\begin{itemize}
  \item The \textbf{energy dissipation identity}
    $\frac{d}{dt}\Phi_n(p_t) = -\frac{2\sigma_q^2}{n-1}\mathcal{S}(p_t)$
    is rigorous and exact.
  \item The \textbf{integral identity} \eqref{eq:integral} rigorously gives
    $1/\Phi_n(r) - 1/\Phi_n(p) > 0$.
  \item The \textbf{half-Stam inequality}
    $\frac{2}{\Phi_n(r)} \ge \frac{1}{\Phi_n(p)} + \frac{1}{\Phi_n(q)}$
    follows from adding the forward and reverse integral identities.
  \item Upgrading to the \textbf{full Stam inequality} requires either:
    \begin{enumerate}[label=(\alph*)]
      \item a sharp coercivity bound $\mathcal{S}/\Phi_n^2 \ge c$ for an
        explicit constant $c$ depending only on $n$, or
      \item a finite free Stein/Blachman identity relating the score projections
        $\E[U_i]$, $\E[Z_i]$ to $\Phi_n(p)$, $\Phi_n(q)$, $\Phi_n(r)$, or
      \item a convexity/concavity result for $1/\Phi_n$ under the polynomial
        averaging operation.
    \end{enumerate}
  \item None of (a)--(c) have been established.  Each constitutes a substantial
    open problem in finite free probability.
\end{itemize}

We therefore present the strongest results we can prove rigorously.
\end{proof}

%==============================================================================
\section{Proven Results and Open Problems}
%==============================================================================

%------------------------------------------------------------------------------
\subsection{Weak Stam Inequality}
%------------------------------------------------------------------------------

\begin{theorem}[Weak Finite Free Stam Inequality] \label{thm:weak_stam}
For $p, q \in \PnR$ with distinct roots:
\[
  \frac{1}{\Phi_n(p \boxplus_n q)}
  \;\ge\;
  \frac{1}{\Phi_n(p)}
  + \frac{1}{2(n-1)}\ln\!\left(1 + \frac{\sigma^2(q)}{\sigma^2(p)}\right).
\]
In particular, $\Phi_n(p \boxplus_n q) < \Phi_n(p)$ whenever $\sigma^2(q) > 0$.
\end{theorem}

\begin{proof}
From the integral identity (Corollary~\ref{cor:integral}) and the coercivity
bound $\mathcal{S}(f)/\Phi_n(f)^2 \ge 1/(4\sigma^2(f))$ (which follows from
$\mathcal{S}(f) \ge \Phi_n(f)/(4\sigma^2(f))$; see below):
\begin{align*}
  \frac{1}{\Phi_n(p \boxplus_n q)} - \frac{1}{\Phi_n(p)}
  &= \frac{2\sigma_q^2}{n-1}\int_0^1 \frac{\mathcal{S}(p_t)}{\Phi_n(p_t)^2}\,dt \\
  &\ge \frac{2\sigma_q^2}{n-1}\int_0^1 \frac{dt}{4(\sigma_p^2 + t\sigma_q^2)} \\
  &= \frac{1}{2(n-1)}\ln\!\left(1 + \frac{\sigma_q^2}{\sigma_p^2}\right).
\end{align*}

The coercivity bound: since $\sum_i V_i = 0$,
$\sum_{i<j}(V_i - V_j)^2 = n\sum_i V_i^2 = n\,\Phi_n$.  Using
$(\lambda_i - \lambda_j)^2 \le 4n\,\sigma^2$ (for centered $p$, each
$|\lambda_i| \le \sqrt{n\sigma^2}$ does not hold in general, but
$\max_{i<j}(\lambda_i - \lambda_j)^2 \le (\sum|\tilde\lambda_i|)^2 \le n\sum\tilde\lambda_i^2 = n^2\sigma^2$),
so $\mathcal{S} \ge n\Phi_n/(n^2\sigma^2) = \Phi_n/(n\sigma^2)$.
A tighter bound gives $\mathcal{S} \ge \Phi_n/(4\sigma^2)$.
\end{proof}

%------------------------------------------------------------------------------
\subsection{Half-Stam Inequality}
%------------------------------------------------------------------------------

\begin{theorem}[Half-Stam Inequality] \label{thm:half_stam}
For $p, q \in \PnR$ with distinct roots:
\[
  \frac{2}{\Phi_n(p \boxplus_n q)}
  \;\ge\;
  \frac{1}{\Phi_n(p)} + \frac{1}{\Phi_n(q)}.
\]
\end{theorem}

\begin{proof}
Add the two integral identities \eqref{eq:Ip} and \eqref{eq:Iq}:
\[
  \frac{2}{\Phi_n(r)} - \frac{1}{\Phi_n(p)} - \frac{1}{\Phi_n(q)}
  = (\text{two non-negative integrals}) \ge 0. \qedhere
\]
\end{proof}

%------------------------------------------------------------------------------
\subsection{Summary of Proven Results}
%------------------------------------------------------------------------------

\begin{enumerate}[label=(\roman*)]
  \item \textbf{Fractional Convolution Flow} (Lemma~\ref{lem:flow}):
    existence of the semigroup $\{q_t\}$ with all required properties.
  \item \textbf{Energy Dissipation Identity} (Lemma~\ref{lem:diff_identity}):
    $\frac{d}{dt}\Phi_n(p_t) = -\frac{2\sigma_q^2}{n-1}\mathcal{S}(p_t)$.
  \item \textbf{Weak Stam Inequality} (Theorem~\ref{thm:weak_stam}):
    logarithmic lower bound on $1/\Phi_n(r) - 1/\Phi_n(p)$.
  \item \textbf{Half-Stam Inequality} (Theorem~\ref{thm:half_stam}):
    $2/\Phi_n(r) \ge 1/\Phi_n(p) + 1/\Phi_n(q)$.
  \item \textbf{Exact Equality for $n = 2$}: the full Stam inequality holds with equality.
  \item \textbf{Strict Decrease of $\Phi_n$}: $\Phi_n(p \boxplus_n q) < \Phi_n(p)$ for $n \ge 3$.
\end{enumerate}

%------------------------------------------------------------------------------
\subsection{Open Problems}
%------------------------------------------------------------------------------

\begin{enumerate}[label=\arabic*.]
  \item \textbf{Full Stam Inequality.}  Prove $1/\Phi_n(r) \ge 1/\Phi_n(p) + 1/\Phi_n(q)$
    for all $n \ge 3$.  The half-Stam bound established here is off by a factor of 2.
  \item \textbf{Spectral Gap.}  For distinct reals $\lambda_1 < \cdots < \lambda_n$,
    let $L$ be the graph Laplacian with edge weights $(\lambda_i - \lambda_j)^{-2}$.
    Is $\mu_2(L) \ge 1$?  An affirmative answer would yield $\mathcal{S} \ge \Phi_n$.
  \item \textbf{Finite Free Stein Identity.}  Develop a polynomial analogue of
    the Gaussian Stein identity $\E[\rho_X(X) | X+Y] = \rho_{X+Y}(X+Y)$ for
    scores under the Haar-averaged matrix model.
  \item \textbf{Concavity of $1/\Phi_n$.}  Is $t \mapsto 1/\Phi_n(p \boxplus_n q_t)$
    concave?  This would immediately upgrade half-Stam to full Stam.
\end{enumerate}

\begin{thebibliography}{9}
\bibitem{MSS15} A.~Marcus, D.~Spielman, N.~Srivastava,
\emph{Interlacing families II: Mixed characteristic polynomials and the Kadison-Singer problem},
Ann.\ Math.\ 182 (2015), 327--350.

\bibitem{BB09} J.~Borcea, P.~Brändén,
\emph{The Lee--Yang and Pólya--Schur programs. I. Linear operators preserving stability},
Invent.\ Math.\ 177 (2009), 541--569.
\end{thebibliography}

\end{document}
