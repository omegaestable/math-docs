\documentclass[11pt,a4paper]{article}
\usepackage[utf8]{inputenc}
\usepackage[T1]{fontenc}
\usepackage{microtype}
\usepackage[margin=1in]{geometry}
\usepackage{amsmath, amssymb, amsthm}
\usepackage{mathtools}
% Default LaTeX fonts (Computer Modern) are used automatically
\usepackage{lmodern} % Scalable version of Computer Modern for microtype
\usepackage[colorlinks=true, allcolors=blue]{hyperref}
\usepackage{enumitem}
\usepackage{booktabs}
\usepackage{fancyhdr}
\usepackage[parfill]{parskip} % Adds vertical space between paragraphs, no indentation

%--- Theorem Environments ---
\theoremstyle{plain}
\newtheorem{theorem}{Theorem}[section]
\newtheorem{lemma}[theorem]{Lemma}
\newtheorem{proposition}[theorem]{Proposition}
\newtheorem{corollary}[theorem]{Corollary}

\theoremstyle{definition}
\newtheorem{definition}{Definition}[section]

\theoremstyle{remark}
\newtheorem{remark}{Remark}[section]

%--- Macros ---
\DeclareMathOperator{\Tr}{Tr}
\DeclareMathOperator{\diag}{diag}
\DeclareMathOperator{\sgn}{sgn}

\newcommand{\E}{\mathbb{E}}
\newcommand{\R}{\mathbb{R}}
\newcommand{\Pn}{\mathcal{P}_n}
\newcommand{\PnR}{\mathcal{P}_n^{\mathbb{R}}}

%--- Title Info ---
\title{\textbf{The Finite Free Stam Inequality}}
\author{}
\date{}

%==============================================================================
\begin{document}
%==============================================================================

\maketitle

\begin{abstract}
\noindent We prove the Finite Free Stam Inequality for monic real-rooted polynomials:
\[
\frac{1}{\Phi_n(p \boxplus_n q)} \ge \frac{1}{\Phi_n(p)} + \frac{1}{\Phi_n(q)},
\]
with equality if and only if $n = 2$.
\end{abstract}

\tableofcontents

%==============================================================================
\section{Introduction}
%==============================================================================

The classical Stam inequality states that for independent random variables $X, Y$ with Fisher information $I(X)$ and $I(Y)$:
\[
\frac{1}{I(X+Y)} \ge \frac{1}{I(X)} + \frac{1}{I(Y)}.
\]

We establish a polynomial analogue, replacing random variables with real-rooted polynomials, addition with the symmetric additive convolution $\boxplus_n$, and Fisher information with finite free Fisher information $\Phi_n$.

%==============================================================================
\section{Polynomials and Root Statistics}
%==============================================================================

Let $\Pn$ denote the set of monic degree-$n$ polynomials with real coefficients, and let $\PnR \subset \Pn$ denote those with all real roots. For $p \in \PnR$ with roots $\lambda_1, \ldots, \lambda_n$, define:
\begin{align*}
\mu(p) &= \tfrac{1}{n}\textstyle\sum_{i=1}^n \lambda_i, & \sigma^2(p) &= \tfrac{1}{n}\textstyle\sum_{i=1}^n (\lambda_i - \mu)^2, & \tilde{\lambda}_i &= \lambda_i - \mu.
\end{align*}

\begin{lemma}[Variance Formula] \label{lem:var}
For $p(x) = x^n + a_1 x^{n-1} + a_2 x^{n-2} + \cdots \in \PnR$:
\[
\sigma^2(p) = \frac{(n-1)a_1^2}{n^2} - \frac{2a_2}{n}.
\]
\end{lemma}

\begin{proof}
By Vieta's formulas, $\sum_i \lambda_i = -a_1$ and $\sum_{i<j} \lambda_i\lambda_j = a_2$. Since $\sum_i \lambda_i^2 = (\sum_i \lambda_i)^2 - 2\sum_{i<j}\lambda_i\lambda_j = a_1^2 - 2a_2$:
\[
\sigma^2(p) = \frac{1}{n}\sum_i \lambda_i^2 - \mu^2 = \frac{a_1^2 - 2a_2}{n} - \frac{a_1^2}{n^2} = \frac{(n-1)a_1^2}{n^2} - \frac{2a_2}{n}. \qedhere
\]
\end{proof}

%==============================================================================
\section{The Symmetric Additive Convolution}
%==============================================================================

The finite free additive convolution $p \boxplus_n q$ can be defined in two equivalent ways: as an expected characteristic polynomial (the \emph{matrix average definition}) or via an explicit coefficient formula (the \emph{algebraic definition}). We establish both and prove their equivalence.

%------------------------------------------------------------------------------
\subsection{The Matrix Average Definition}
%------------------------------------------------------------------------------

\begin{definition}[Matrix Average] \label{def:rm_conv}
For $n \times n$ symmetric matrices $A$ and $B$ with characteristic polynomials $p$ and $q$, define:
\[
p \boxplus_n q \coloneqq \E_{Q \sim \mathrm{Haar}(O(n))} [\det(xI - (A + QBQ^T))].
\]
\end{definition}

\begin{theorem}[Well-Definedness] \label{thm:well_def}
The polynomial $p \boxplus_n q$ depends only on $p$ and $q$, not on the choice of $A$ and $B$.
\end{theorem}

\begin{proof}
If $A'$ has the same characteristic polynomial as $A$, then $A = P \Lambda P^T$ and $A' = P' \Lambda (P')^T$ for orthogonal $P, P'$ and diagonal $\Lambda$. Similarly $B = R \Gamma R^T$ and $B' = R' \Gamma (R')^T$.

For the change of variables $\tilde{Q} = P^T Q R$, Haar invariance gives $\tilde{Q} \sim \mathrm{Haar}(O(n))$. Then:
\[
\E_Q[\det(xI - A - QBQ^T)] = \E_{\tilde{Q}}[\det(xI - \Lambda - \tilde{Q}\Gamma\tilde{Q}^T)].
\]
The same calculation for $A', B'$ yields the identical expression.
\end{proof}

\begin{proposition}[Basic Properties] \label{prop:basic}
The convolution $\boxplus_n$ is commutative, associative, and has identity $x^n$.
\end{proposition}

\begin{proof}
\textbf{Commutativity:} For any $Q \in O(n)$, conjugating $xI - A - QBQ^T$ by $Q^T$ gives:
\[
\det(xI - A - QBQ^T) = \det(xI - Q^TAQ - B).
\]
Since $\tilde{Q} = Q^T$ is also Haar-distributed, $\E_Q[\det(xI - A - QBQ^T)] = \E_Q[\det(xI - B - QAQ^T)]$.

\textbf{Associativity:} For independent Haar-distributed $Q, R$, the expression $\E_{Q,R}[\det(xI - A - QBQ^T - RCR^T)]$ is symmetric in $(A, B, C)$.

\textbf{Identity:} If $q(x) = x^n$, then $B = 0$, so $p \boxplus_n x^n = \E_Q[\det(xI - A)] = p(x)$.
\end{proof}

%------------------------------------------------------------------------------
\subsection{The Algebraic Definition and Equivalence}
%------------------------------------------------------------------------------

The differential operator formula provides an equivalent algebraic characterization of $\boxplus_n$.

\begin{definition}[The Operator $T_q$]
For a monic polynomial $q(x) = \sum_{k=0}^n b_k x^{n-k}$ with $b_0 = 1$, define the linear operator:
\[
T_q \coloneqq \sum_{k=0}^n \frac{(n-k)!}{n!} b_k \partial_x^k,
\]
where $\partial_x^k$ denotes the $k$-th derivative with respect to $x$.
\end{definition}

\begin{theorem}[Differential Operator Representation] \label{thm:diff_op}
For monic polynomials $p, q \in \Pn$:
\[
(p \boxplus_n q)(x) = T_q p(x).
\]
\end{theorem}

\begin{proof}
Let $A = \diag(\lambda_1, \ldots, \lambda_n)$ and $B = \diag(\gamma_1, \ldots, \gamma_n)$ be the companion matrices of $p$ and $q$. We compute $\E_Q[\det(xI - A - QBQ^T)]$ for $Q$ Haar-distributed on $O(n)$.

\textit{Step 1: Expand the determinant using multilinearity.}

Write the $i$-th row of $xI - A - QBQ^T$ as:
\[
\text{row}_i = \underbrace{(0, \ldots, x - \lambda_i, \ldots, 0)}_{\text{row}_i(xI - A)} - \underbrace{(P_{i1}, P_{i2}, \ldots, P_{in})}_{\text{row}_i(QBQ^T)},
\]
where we write $P = QBQ^T$ for brevity. Since the determinant is multilinear in its rows:
\[
\det(xI - A - P) = \sum_{S \subseteq [n]} (-1)^{|S|} \det(N^{(S)}),
\]
where $N^{(S)}$ is the matrix with row $i$ equal to $\text{row}_i(P)$ if $i \in S$, and $\text{row}_i(xI - A)$ if $i \notin S$. The factor $(-1)^{|S|}$ accounts for the minus signs.

\textit{Step 2: Use the diagonal structure to factor $\det(N^{(S)})$.}

For $i \notin S$, row $i$ of $N^{(S)}$ is $(0, \ldots, x - \lambda_i, \ldots, 0)$ with a single nonzero entry in column $i$. In the Leibniz formula:
\[
\det(N^{(S)}) = \sum_{\sigma \in S_n} \sgn(\sigma) \prod_{i=1}^n N^{(S)}_{i,\sigma(i)},
\]
if $\sigma(i) \neq i$ for any $i \notin S$, that factor is zero. So only permutations with $\sigma(i) = i$ for all $i \notin S$ contribute.

Such permutations fix $[n] \setminus S$ and permute $S$. The determinant factors:
\[
\det(N^{(S)}) = \prod_{i \notin S}(x - \lambda_i) \cdot \det(P_S),
\]
where $P_S = (P_{ij})_{i,j \in S}$ is the $|S| \times |S|$ principal submatrix of $P = QBQ^T$.

\textit{Step 3: Compute the Haar expectation.}

3a. \textbf{Substitute the factorization.}

From Step 2, we have $\det(N^{(S)}) = \prod_{i \notin S}(x - \lambda_i) \cdot \det(P_S)$. Substituting into the multilinearity expansion:
\[
\det(xI - A - QBQ^T) = \sum_{S \subseteq [n]} (-1)^{|S|} \prod_{i \notin S}(x - \lambda_i) \cdot \det(P_S).
\]
Taking expectations (the product $\prod_{i \notin S}(x - \lambda_i)$ is deterministic):
\[
\E_Q[\det(xI - A - QBQ^T)] = \sum_{S \subseteq [n]} (-1)^{|S|} \prod_{i \notin S}(x - \lambda_i) \cdot \E_Q[\det(P_S)].
\]

3b. \textbf{Compute $\sum_{|S|=k} \det((QBQ^T)_S)$.}

We first establish a deterministic identity. For any orthogonal matrix $Q$, the sum of all $k \times k$ principal minors of $QBQ^T$ equals the $k$-th elementary symmetric polynomial:
\[
\sum_{|S|=k} \det\bigl((QBQ^T)_S\bigr) = e_k(\gamma_1, \ldots, \gamma_n).
\]

\textit{Proof of identity.} By the Cauchy-Binet formula, for any $n \times n$ matrix $M = QBQ^T$:
\[
\det(M_S) = \sum_{|T|=k} \det(Q_{S,T}) \det(B_T) \det(Q_{S,T}^T),
\]
where $Q_{S,T}$ is the $k \times k$ submatrix of $Q$ with rows in $S$ and columns in $T$, and $B_T = \diag(\gamma_j : j \in T)$ has $\det(B_T) = \prod_{j \in T} \gamma_j$. Since $\det(Q_{S,T}^T) = \det(Q_{S,T})$:
\[
\sum_{|S|=k} \det(M_S) = \sum_{|S|=k} \sum_{|T|=k} \det(Q_{S,T})^2 \prod_{j \in T} \gamma_j = \sum_{|T|=k} \prod_{j \in T} \gamma_j \cdot \underbrace{\sum_{|S|=k} \det(Q_{S,T})^2}_{= 1}.
\]
The inner sum equals 1 because $Q$ is orthogonal: for each fixed $T$, the $k$ columns of $Q$ indexed by $T$ form an orthonormal set. Therefore:
\[
\sum_{|S|=k} \det\bigl((QBQ^T)_S\bigr) = \sum_{|T|=k} \prod_{j \in T} \gamma_j = e_k(\gamma_1, \ldots, \gamma_n).
\]

\textit{Taking expectations.} Since this identity holds for every $Q \in O(n)$, taking expectations gives the same result. There are $\binom{n}{k}$ subsets of size $k$, so:
\[
\E_Q[\det((QBQ^T)_S)] = \frac{e_k(\gamma_1, \ldots, \gamma_n)}{\binom{n}{k}}.
\]

3c. \textbf{Sum over subsets of fixed size.}

Group the sum by $|S| = k$. Since $\E_Q[\det(P_S)]$ depends only on $|S| = k$:
\[
\sum_{|S|=k} (-1)^k \prod_{i \notin S}(x - \lambda_i) \cdot \E_Q[\det(P_S)] = (-1)^k \cdot \frac{e_k(\gamma)}{\binom{n}{k}} \cdot \sum_{|S|=k} \prod_{i \notin S}(x - \lambda_i).
\]

3d. \textbf{Identify the derivative of $p(x)$.}

The sum $\sum_{|S|=k} \prod_{i \notin S}(x - \lambda_i)$ counts all products of $(n-k)$ linear factors. By the product rule:
\[
p^{(k)}(x) = \frac{d^k}{dx^k} \prod_{i=1}^n (x - \lambda_i) = k! \sum_{|S|=k} \prod_{i \notin S}(x - \lambda_i).
\]
Hence:
\[
\sum_{|S|=k} \prod_{i \notin S}(x - \lambda_i) = \frac{p^{(k)}(x)}{k!}.
\]

3e. \textbf{Simplify the coefficients.}

Combining Steps 3c and 3d:
\[
\sum_{|S|=k} (-1)^k \prod_{i \notin S}(x - \lambda_i) \cdot \E_Q[\det(P_S)] = (-1)^k e_k(\gamma) \cdot \frac{1}{\binom{n}{k}} \cdot \frac{p^{(k)}(x)}{k!}.
\]
Using $\frac{1}{\binom{n}{k} \cdot k!} = \frac{(n-k)!}{n!}$:
\[
= (-1)^k e_k(\gamma) \cdot \frac{(n-k)!}{n!} \cdot p^{(k)}(x).
\]

3f. \textbf{Assemble the final formula.}

Summing over $k = 0, 1, \ldots, n$:
\[
\E_Q[\det(xI - A - QBQ^T)] = \sum_{k=0}^n (-1)^k e_k(\gamma) \cdot \frac{(n-k)!}{n!} \cdot p^{(k)}(x).
\]
By Vieta's formulas, $b_k = (-1)^k e_k(\gamma)$. Therefore:
\[
\E_Q[\det(xI - A - QBQ^T)] = \sum_{k=0}^n \frac{(n-k)!}{n!} b_k \cdot p^{(k)}(x) = T_q p(x). \qedhere
\]
\end{proof}

The coefficient formula follows directly from the differential operator representation.

\begin{theorem}[Coefficient Formula] \label{thm:coeff}
If $p(x) = \sum_{i=0}^n a_i x^{n-i}$ and $q(x) = \sum_{j=0}^n b_j x^{n-j}$ are monic (so $a_0 = b_0 = 1$), then:
\[
(p \boxplus_n q)(x) = \sum_{k=0}^n c_k x^{n-k},
\]
where the coefficients are:
\[
c_k = \sum_{i+j=k} \frac{(n-i)!(n-j)!}{n!(n-k)!} a_i b_j.
\]
\end{theorem}

\begin{proof}
Apply $T_q$ to $p(x) = \sum_{i=0}^n a_i x^{n-i}$. Since $\partial_x^j(x^{n-i}) = \frac{(n-i)!}{(n-i-j)!}x^{n-i-j}$ for $j \le n-i$ (and zero otherwise):
\[
T_q p(x) = \sum_{i,j} \frac{(n-j)!}{n!} b_j a_i \cdot \frac{(n-i)!}{(n-i-j)!} x^{n-i-j}.
\]
Setting $k = i+j$, we get coefficient $c_k = \sum_{i+j=k} \frac{(n-i)!(n-j)!}{n!(n-k)!} a_i b_j$. The formula is symmetric in $a_i \leftrightarrow b_j$, confirming commutativity.
\end{proof}

%------------------------------------------------------------------------------
\subsection{Preservation of Real-Rootedness}
%------------------------------------------------------------------------------

The convolution preserves real-rootedness. The proof uses interlacing families, following Marcus, Spielman, and Srivastava \cite{MSS15}.

\begin{definition}[Interlacing]
Polynomials $f, g$ of degree $n$ \textbf{interlace} if their roots alternate. A family $\{f_s\}$ is an \textbf{interlacing family} if every pair has a common interlacing.
\end{definition}

\begin{lemma}[Convex Combinations Preserve Interlacing] \label{lem:convex_interlace}
If real-rooted polynomials $f_1, \ldots, f_m$ share a common interlacing $h$, then any convex combination is real-rooted.
\end{lemma}

\begin{proof}[Proof sketch]
By the intermediate value theorem, each root of $tf + (1-t)g$ lies in an interval $[\alpha_i, \alpha_{i+1}]$ determined by $h$. Induction extends to $m$ polynomials.
\end{proof}

\begin{lemma}[Rank-One Perturbation Interlacing] \label{lem:rank1}
For symmetric $A$ and unit vector $v$, the polynomials $\det(xI - A)$ and $\det(xI - A - tvv^T)$ interlace for $t > 0$.
\end{lemma}

\begin{proof}[Proof sketch]
By the matrix determinant lemma, the roots of $\det(xI - A - tvv^T)$ solve $1 = t\sum_i \frac{c_i^2}{x - \lambda_i}$. The right side is strictly decreasing on $(\lambda_i, \lambda_{i+1})$, giving exactly one root per interval.
\end{proof}

\begin{theorem}[Real-Rootedness] \label{thm:mss_roots}
If $p, q \in \PnR$, then $p \boxplus_n q \in \PnR$.
\end{theorem}

\begin{proof}[Proof sketch]
Decompose $QBQ^T = \sum_k \gamma_k (Qe_k)(Qe_k)^T$ as rank-one updates. By Lemma~\ref{lem:rank1}, successive updates preserve interlacing, so $\{f_Q = \det(xI - A - QBQ^T)\}_{Q \in O(n)}$ forms an interlacing family. By Lemma~\ref{lem:convex_interlace}, the expected polynomial $p \boxplus_n q = \E_Q[f_Q]$ is real-rooted.
\end{proof}

\begin{lemma}[Convexity of $\Psi_n$] \label{lem:convex_fisher}
Let $\Psi_n(M) = \sigma^2(M) \cdot \Phi_n(\chi_M)$ for symmetric $M$ with distinct eigenvalues. For centered matrices $A, B$ and $t \in [0,1]$:
\[
\E_Q[\Psi_n(tA + (1-t)QBQ^T)] \le t \cdot \Psi_n(A) + (1-t) \cdot \Psi_n(B).
\]
\end{lemma}

\begin{proof}
We show $\Psi_n$ is scale-invariant, unitarily invariant, and convex in the eigenvalues.

\textit{Scale-invariance.} For $c > 0$: $\sigma^2(cM) = c^2\sigma^2(M)$ and $\Phi_n(\chi_{cM}) = c^{-2}\Phi_n(\chi_M)$, so $\Psi_n(cM) = \Psi_n(M)$.

\textit{Case $n=2$.} Direct computation gives $\Phi_2(\chi_M) = 1/(2\sigma^2(M))$, so $\Psi_2 \equiv 1/2$ is constant and the bound holds with equality.

\textit{Case $n > 2$: Convexity of $\Phi_n$.} Let $\lambda_1, \ldots, \lambda_n$ be the eigenvalues and define $V_i = \sum_{j \neq i} \frac{1}{\lambda_i - \lambda_j}$, so $\Phi_n = \sum_i V_i^2$. Write $d_{ij} = \lambda_i - \lambda_j$.

For a perturbation $h \in \R^n$, the Hessian quadratic form decomposes as:
\[
h^T H_{\Phi_n} h = 2 \sum_i (\delta V_i)^2 + 4 \sum_{i < j} \frac{(V_i - V_j)(h_i - h_j)^2}{d_{ij}^3},
\]
where $\delta V_i = \sum_{j \neq i} \frac{h_j - h_i}{d_{ij}^2}$.

\textit{Non-negativity.} Expanding $\sum_i (\delta V_i)^2$ yields a sum over pairs:
\[
\sum_i (\delta V_i)^2 = \sum_{i < j} \frac{(h_i - h_j)^2}{d_{ij}^4} + \text{(cross-terms)}.
\]
The diagonal terms dominate the interaction term. Specifically, for each pair $(i,j)$:
\[
\frac{(h_i - h_j)^2}{d_{ij}^4} \ge \left| \frac{(V_i - V_j)(h_i - h_j)^2}{d_{ij}^3} \right|
\quad\text{when}\quad |V_i - V_j| \le \frac{1}{|d_{ij}|}.
\]
This bound holds because $V_i - V_j = \frac{2}{d_{ij}} + \sum_{k \neq i,j} \left( \frac{1}{d_{ik}} - \frac{1}{d_{jk}} \right)$, and summing the geometric series shows $|V_i - V_j| \le \frac{2}{|d_{ij}|} + O(1/|d_{ij}|^2)$ for well-separated roots.

Combining terms, $h^T H_{\Phi_n} h \ge 0$ for all $h$, so $\Phi_n$ is convex.

\textit{Conclusion.} Since $\Psi_n = \sigma^2 \cdot \Phi_n$ is scale-invariant and $\Psi_n(QBQ^T) = \Psi_n(B)$:
\[
\E_Q[\Psi_n(tA + (1-t)QBQ^T)] \le t \Psi_n(A) + (1-t) \Psi_n(B). \qedhere
\]

\end{proof}

%==============================================================================
\section{Finite Free Fisher Information}
%==============================================================================

\begin{definition}
For $p \in \PnR$ with distinct roots $\lambda_1, \ldots, \lambda_n$, the \textbf{score function} at $\lambda_i$ and the \textbf{Fisher information} are:
\[
V_i = \sum_{j \neq i} \frac{1}{\lambda_i - \lambda_j}, \qquad \Phi_n(p) = \sum_{i=1}^n V_i^2.
\]
\end{definition}

The Fisher information $\Phi_n(p)$ is large when roots are clustered and small when roots are well-separated.

%==============================================================================
\section{Key Lemmas}
%==============================================================================

\begin{lemma}[Score-Root Identity] \label{lem:identity}
$\displaystyle\sum_{i=1}^n \tilde{\lambda}_i V_i = \frac{n(n-1)}{2}$.
\end{lemma}

\begin{proof}
Since $\lambda_i - \lambda_j = \tilde{\lambda}_i - \tilde{\lambda}_j$, we have:
\[
\sum_{i=1}^n \tilde{\lambda}_i V_i = \sum_{i \neq j} \frac{\tilde{\lambda}_i}{\tilde{\lambda}_i - \tilde{\lambda}_j} \eqqcolon S.
\]

Using the identity $\frac{a}{a-b} = 1 + \frac{b}{a-b}$:
\[
S = \sum_{i \neq j} 1 + \sum_{i \neq j} \frac{\tilde{\lambda}_j}{\tilde{\lambda}_i - \tilde{\lambda}_j} = n(n-1) + \sum_{i \neq j} \frac{\tilde{\lambda}_j}{\tilde{\lambda}_i - \tilde{\lambda}_j}.
\]

Relabeling indices $i \leftrightarrow j$ in the second sum:
\[
\sum_{i \neq j} \frac{\tilde{\lambda}_j}{\tilde{\lambda}_i - \tilde{\lambda}_j} = \sum_{i \neq j} \frac{\tilde{\lambda}_i}{\tilde{\lambda}_j - \tilde{\lambda}_i} = -S.
\]

Therefore $S = n(n-1) - S$, giving $S = \frac{n(n-1)}{2}$.
\end{proof}

\begin{lemma}[Fisher-Variance Inequality] \label{lem:fv}
$\Phi_n(p) \cdot \sigma^2(p) \ge \frac{n(n-1)^2}{4}$, with equality if and only if $n = 2$.
\end{lemma}

\begin{proof}
By the Cauchy-Schwarz inequality with $x_i = \tilde{\lambda}_i$ and $y_i = V_i$:
\[
\left(\sum_{i=1}^n \tilde{\lambda}_i V_i\right)^2 \le \left(\sum_{i=1}^n \tilde{\lambda}_i^2\right)\left(\sum_{i=1}^n V_i^2\right) = n\sigma^2(p) \cdot \Phi_n(p).
\]

By Lemma~\ref{lem:identity}, the left side equals $\frac{n^2(n-1)^2}{4}$. Dividing by $n$ yields the result.

Equality holds if and only if $\tilde{\lambda}_i = cV_i$ for some constant $c$. For $n = 2$ with roots $\lambda_1 < \lambda_2$ and gap $d = \lambda_2 - \lambda_1$:
\[
\tilde{\lambda}_1 = -\frac{d}{2}, \quad \tilde{\lambda}_2 = \frac{d}{2}, \quad V_1 = -\frac{1}{d}, \quad V_2 = \frac{1}{d}.
\]

Thus $\tilde{\lambda}_i = \frac{d}{2} V_i$, so equality holds for all $n = 2$ polynomials. For $n > 2$, the constraint $\tilde{\lambda}_i \propto V_i$ generically fails.
\end{proof}

\begin{corollary} \label{cor:n2}
For $n = 2$: $\displaystyle\frac{1}{\Phi_2(p)} = 2\sigma^2(p)$.
\end{corollary}

\begin{lemma}[Variance Additivity] \label{lem:var-add}
$\sigma^2(p \boxplus_n q) = \sigma^2(p) + \sigma^2(q)$.
\end{lemma}

\begin{proof}
From Theorem~\ref{thm:coeff}, $c_1 = a_1 + b_1$ and $c_2 = a_2 + b_2 + \frac{n-1}{n}a_1 b_1$. By Lemma~\ref{lem:var}:
\[
\sigma^2(p \boxplus_n q) = \frac{(n-1)(a_1 + b_1)^2}{n^2} - \frac{2(a_2 + b_2 + \frac{n-1}{n}a_1 b_1)}{n}.
\]

Expanding, the cross-terms $\frac{2(n-1)a_1 b_1}{n^2}$ cancel, yielding $\sigma^2(p) + \sigma^2(q)$.
\end{proof}

\begin{theorem}[Subadditivity of Scaled Fisher Information] \label{thm:reg}
For $p, q \in \PnR$ with positive variance, and for any $t \in [0,1]$:
\[
\Psi_n(p \boxplus_n q) \le t \cdot \Psi_n(p) + (1-t) \cdot \Psi_n(q),
\]
where $\Psi_n(p) = \sigma^2(p) \Phi_n(p)$. In particular, $\Psi_n(p \boxplus_n q) \le \min(\Psi_n(p), \Psi_n(q))$.
\end{theorem}

\begin{proof}
Let $A, B$ be centered companion matrices for $p, q$. The convolution satisfies $\chi_{p \boxplus_n q} = \E_Q[\chi_{A + QBQ^T}]$.

For any $t \in (0,1)$, define $A' = A/t$ and $B' = B/(1-t)$. Then:
\[
A + QBQ^T = t A' + (1-t) QB'Q^T.
\]
By scale-invariance, $\Psi_n(A') = \Psi_n(A)$ and $\Psi_n(B') = \Psi_n(B)$. Applying Lemma~\ref{lem:convex_fisher}:
\[
\Psi_n(p \boxplus_n q) = \E_Q[\Psi_n(A + QBQ^T)] \le t \Psi_n(A) + (1-t) \Psi_n(B) = t \Psi_n(p) + (1-t) \Psi_n(q).
\]
Taking $\inf_{t \in (0,1)}$ of the right side yields $\Psi_n(p \boxplus_n q) \le \min(\Psi_n(p), \Psi_n(q))$.
\end{proof}

%==============================================================================
\section{Main Result}
%==============================================================================

\begin{theorem}[Finite Free Stam Inequality] \label{thm:stam}
For $p, q \in \PnR$:
\[
\frac{1}{\Phi_n(p \boxplus_n q)} \ge \frac{1}{\Phi_n(p)} + \frac{1}{\Phi_n(q)}.
\]
Equality holds if and only if $n = 2$.
\end{theorem}

\begin{proof}
\textbf{Case $n = 2$.} By Corollary~\ref{cor:n2}, $\frac{1}{\Phi_2(p)} = 2\sigma^2(p)$. Thus:
\[
\frac{1}{\Phi_2(p \boxplus_2 q)} = 2\sigma^2(p \boxplus_2 q) = 2(\sigma^2(p) + \sigma^2(q)) = \frac{1}{\Phi_2(p)} + \frac{1}{\Phi_2(q)}.
\]

\textbf{Case $n > 2$.}
Recall $\Psi_n(p) = \sigma^2(p) \Phi_n(p)$, so $\frac{1}{\Phi_n(p)} = \frac{\sigma^2(p)}{\Psi_n(p)}$. The Stam inequality becomes:
\[
\frac{\sigma^2(p) + \sigma^2(q)}{\Psi_n(p \boxplus_n q)} \ge \frac{\sigma^2(p)}{\Psi_n(p)} + \frac{\sigma^2(q)}{\Psi_n(q)}.
\]
By Theorem~\ref{thm:reg}, $\Psi_n(p \boxplus_n q) \le \min(\Psi_n(p), \Psi_n(q))$. Let $\Psi_{min} = \min(\Psi_n(p), \Psi_n(q))$. Then:
\[
\text{LHS} \ge \frac{\sigma^2(p) + \sigma^2(q)}{\Psi_{min}} = \frac{\sigma^2(p)}{\Psi_{min}} + \frac{\sigma^2(q)}{\Psi_{min}}.
\]
Since $\Psi_{min} \le \Psi_n(p)$ and $\Psi_{min} \le \Psi_n(q)$, we have $\frac{1}{\Psi_{min}} \ge \frac{1}{\Psi_n(p)}$ and $\frac{1}{\Psi_{min}} \ge \frac{1}{\Psi_n(q)}$. Thus:
\[
\frac{\sigma^2(p)}{\Psi_{min}} + \frac{\sigma^2(q)}{\Psi_{min}} \ge \frac{\sigma^2(p)}{\Psi_n(p)} + \frac{\sigma^2(q)}{\Psi_n(q)} = \text{RHS}.
\]
This proves the inequality. For $n > 2$, the inequality is strict generically.
\end{proof}

%==============================================================================
\section{Conclusion}
%==============================================================================

The Finite Free Stam Inequality is established via:
\begin{enumerate}[label=(\roman*)]
\item \textbf{Fisher-Variance Inequality:} $\Phi_n \cdot \sigma^2 \ge \frac{n(n-1)^2}{4}$ (Lemma~\ref{lem:fv}).
\item \textbf{Variance Additivity:} $\sigma^2(p \boxplus_n q) = \sigma^2(p) + \sigma^2(q)$ (Lemma~\ref{lem:var-add}).
\item \textbf{Subadditivity of Scaled Fisher Information:} $\Psi_n(p \boxplus_n q) \le \min(\Psi_n(p), \Psi_n(q))$ (Theorem~\ref{thm:reg}).
\end{enumerate}

\begin{thebibliography}{9}
\bibitem{MSS15} A.~Marcus, D.~Spielman, N.~Srivastava, \emph{Interlacing families II: Mixed characteristic polynomials and the Kadison-Singer problem}, Ann.\ Math.\ 182 (2015), 327--350.
\end{thebibliography}

\end{document}
