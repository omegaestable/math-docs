\documentclass[11pt, reqno]{amsart}
\usepackage[utf8]{inputenc}


\usepackage{amscd}
\usepackage{float}
\usepackage{amssymb}
\usepackage[english,spanish]{babel}
\usepackage{nomencl}
\usepackage{algorithm}
\usepackage{algpseudocode}
\usepackage{cite}
\usepackage{multirow}

\usepackage{tikz-cd}

\usepackage{graphicx}
\usepackage{rotating}
\usepackage{diagbox}\usepackage{comment}
\usepackage{fullpage} 
\usepackage{fancyvrb}
\usepackage{epsfig}
\usepackage{fancyhdr}
\usepackage{amssymb}
\usepackage{pifont}
\usepackage{amsmath}
\usepackage{amssymb}
\usepackage{enumerate}
\usepackage{mathtools}
\usepackage{ mathrsfs }
\usepackage{bm}
\usepackage{listings}
\usepackage{amsfonts}
\usepackage{mathtools}
\usepackage{epstopdf}
\usepackage{tikz}
\definecolor{mintgreen}{RGB}{152,255,152}
\definecolor{pinksalmon}{RGB}{255,102,102}
\definecolor{hueso}{RGB}{245,245,220}
\definecolor{marfil}{RGB}{255,253,208}
\definecolor{amarillo}{RGB}{255,255,0}
\usetikzlibrary{decorations.markings,arrows}
%\usetikzlibrary{er}
\usetikzlibrary{decorations.pathreplacing}
\DeclareGraphicsRule{.tif}{png}{.png}{`convert #1 `dirname #1`/`basename #1 .tif`.png}

\usepackage[inner=1.0in,outer=1.0in,bottom=1.0in, top=1.0in]{geometry}


\numberwithin{equation}{section}
%\numberwithin{theorem}{section}

\newtheorem{theorem}{Teorema}[section]
%\newtheorem{definition}[theorem]{Definition}
%\newtheorem{example}[theorem]{Example}
\newtheorem{lemma}[theorem]{Lema}
\newtheorem{proposition}[theorem]{Proposición}
\newtheorem{corollary}[theorem]{Corolario}
\newtheorem{conjecture}[theorem]{Conjetura}
\renewenvironment{proof}{\paragraph{\textbf{Prueba: }}}{\hfill$\blacksquare$}
\theoremstyle{definition}
\newtheorem{remark}[theorem]{Observación}
\newtheorem{definition}[theorem]{Definición}
\newtheorem{example}[theorem]{Ejemplo}


\newcommand{\disjoint}{\cdot\!\!\!\!\!\bigcup}

%---------------------------------------
\makeatletter
\def\moverlay{\mathpalette\mov@rlay}
\def\mov@rlay#1#2{\leavevmode\vtop{%
   \baselineskip\z@skip \lineskiplimit-\maxdimen
   \ialign{\hfil$\m@th#1##$\hfil\cr#2\crcr}}}
\newcommand{\charfusion}[3][\mathord]{
    #1{\ifx#1\mathop\vphantom{#2}\fi
        \mathpalette\mov@rlay{#2\cr#3}
      }
    \ifx#1\mathop\expandafter\displaylimits\fi}
\makeatother

\newcommand{\cupdot}{\charfusion[\mathbin]{\cup}{\cdot}}
\newcommand{\bigcupdot}{\charfusion[\mathop]{\bigcup}{\cdot}}

%-------------------------------------
\newcommand{\suchthat}{\;\ifnum\currentgrouptype=16 \middle\fi|\;}
\newcommand{\spec}[1]{\operatorname{Spec}\   #1}
\newcommand{\Z}{\mathbb{Z}}
\newcommand{\C}{\mathbb{C}}
\newcommand{\Q}{\mathbb{Q}}
\newcommand{\R}{\mathbb{R}}
\newcommand{\N}{\mathbb{N}}
\newcommand{\Gal}[1]{\operatorname{Gal}#1}
\newcommand{\op}[1]{\operatorname{#1}}
\newcommand{\cal}[1]{\mathcal{#1}}
\newcommand{\bb}[1]{\mathbb{#1}}
\newcommand{\fr}[1]{\mathfrak{#1}}
\newcommand{\Tr}[1]{\operatorname{Tr}#1}
\newcommand{\Nr}[1]{\operatorname{N}#1}
\newcommand{\e}{\varepsilon}
\newcommand{\CM}{\mathcal{CM}}
\renewcommand{\baselinestretch}{1.1}


\newcommand{\shortmod}{\ensuremath{\negthickspace \negthickspace \negthickspace \pmod}}

\begin{document}

\title{El Teorema de los números primos para progresiones aritméticas. Preliminares}

\author{J. Ignacio Padilla Barrientos
}


\pagestyle{fancy}
\fancyhead[]{}
\lhead{MA-506 Teoría Analítica de Números}
\rhead{Prof. Adrián Barquero Sánchez}
\setlength{\headheight}{13pt}

\address{Escuela de Matem\'atica, Universidad de Costa Rica, San Jos\'e 11501, Costa Rica}

\email{juan.padillabarrientos@ucr.ac.cr}
\begin{abstract}
    El siguiente trabajo corresponde a la primera parte del $2$do examen parcial del curso. El objetivo principal corresponde con la exposición y resumen de los resultados necesarios para la demostración del teorema de los números primos en progresiones aritméticas. Específicamente, se presentan los resultados análogos correspondientes a regiones libres de ceros, cantidad de ceros, y la función $\psi$, los cuales fueron desarrollados en el curso para la función $\zeta$, y se presentan para funciones $L$ en general. Estos resultados corresponden con las secciones $14,16,19$ del libro ``Multiplicative Number Theory'', de Harold Davenport.\\

    \noindent
    \textit{Palabras Clave: caracter, cero no trivial, $L$-función, paridad, primitivo} \\
    \textit{Clasificación:} 11N05
\end{abstract}
\maketitle
\section{Sección 14. Regiones libres de ceros para $L(s,\chi)$}
Esta sección es el análogo directo a la sección 13 de \cite{HD1974}, en donde se demuestra que la función Zeta de Riemann cumple que existe una constante positiva $c$ tal que $\zeta(s)$ no tiene ceros\footnote{A lo largo de la sección denotaremos el argumento de funciones de variable compleja como $s=\sigma + it$.} en la región
$$\sigma \geq 1- \frac{c}{\log t},  \quad  t \geq 2.$$
Ya no consideraremos la función $\zeta(s)$ sino una función $L(s,\chi)$, donde $\chi$ es un caracter de Dirichlet módulo q. El resultado principal de la sección es el siguiente
\begin{theorem}
    Existe una constante positiva $c$ con la siguiente propiedad. Si $\chi$ es un caracter complejo módulo $q$, entonces $L(s,\chi)$ no tiene ceros en la región definida por
    \[
        \sigma \geq
        \begin{cases}
            1-\frac{c}{\log q|t|} \text{ si $t \geq 1$} \\
            1-\frac{c}{\log q} \text{\quad si $t \leq 1$}
        \end{cases},
    \]
    y si $\chi$ es un caracter real (no principal), el único posible cero de $L(s,\chi)$ en esta región es un cero real simple.
\end{theorem}

En nuestro caso, esta vez asumiremos que $t\geq 0$ Esto es suficiente para nuestro estudio de los ceros no triviales, puesto que los ceros de $L(s,\chi)$ con parte imaginaria negativa son precisamente los conjugados de los ceros de $L(s,\bar{\chi})$ con parte imaginaria positiva. El primer resultado análogo, es una expresión para la derivada logarítmica de $L(s,\chi)$, el cual se obtiene derivando la fórmula del producto de Euler correspondiente,
$$-\frac{L'(s,\chi)}{L(s,\chi)} = \sum_{n=1}^{\infty} \Lambda(n)n^{-\sigma}\chi(n)e^{-it\log n},$$
donde, $\Lambda(n)$ es la función de Von Mangoldt, definida unas secciones atrás. La fórmula anterior vale para $\sigma > 1$. Separando las partes reales e imaginarias de esta expresión, y usando desigualdades trigonométricas, llegamos a la desigualdad
\begin{align}
    3\left( - \frac{L'(\sigma,\chi_0)}{L(\sigma,\chi_0}\right) + 4\left(-\Re{\frac{L'(\sigma + it,\chi)}{L(\sigma+it,\chi)}} \right) + \left( -\Re{\frac{L'(\sigma + 2it, \chi^2)}{L(\sigma + 2it, \chi^2)}}\right) \geq 0.
\end{align}
Recuerde que $\chi_0$ es el caracter principal.

Para proseguir con la prueba, vamos a asumir primero que $\chi$ es un caracter complejo primitivo (su periodo minimal es $q$). En este caso la prueba se adapta fácilmente de la prueba para $\zeta(s)$, y se obtiene que
$$ - \frac{L'(\sigma,\chi_0)}{L(\sigma,\chi_0)} = \sum_{1}^{\infty}\chi_0(n)\Lambda(n)n^{-\sigma} \leq -\frac{\zeta'(\sigma)}{\zeta(\sigma)} < \frac{1}{\sigma -1} + C,$$
donde $C$ es alguna constante positiva (y lo continuará siento durante todo el trabajo). Para los otros términos de la desigualdad 1.1, se recurre a propiedades de la función $\gamma$, y se llega a que
$$-\Re{\frac{L'(s,\chi)}{L(s,\chi)}}< C\mathscr{L} - \sum_{\rho} \Re{\frac{1}{s-\rho}},$$
donde $\mathscr{L} = \log q + \log(t+2)$.
Entonces, si se toma $t$ como la parte imaginaria del algún cero no trivial $\rho = \beta + i\gamma$, se obtiene la desigualdad
$$-\Re{\frac{L'(\sigma + it,\chi)}{L(\sigma+it,\chi)}} < C\mathscr{L} - \frac{1}{\sigma - \beta},$$
la cual funciona también para el otro término (el que tiene $2it$). Combinando estos tres estimados, se obtiene
\begin{align}\frac{4}{\sigma-\beta} < \frac{3}{\sigma-1} + C\mathscr{L}, \end{align}
y empleando despejes, se puede demostrar que \textit{existe una constante $c$ positiva tal que si $\chi$ es un caracter complejo modulo $q$, entonces cualquier cero $\rho=\beta + i\gamma$ de $L(s,\chi)$, satisface:}
$$\beta < 1-\frac{c}{\mathscr{L}}$$
En este caso, en la definición de $\mathscr{L}$ se toma $t=|\gamma|$.

El caso cuando $\chi$ es un caracter \textbf{real} y primitivo, requiere un poco más de análisis, pues es necesario relacionar la función $L$ con $\zeta$ de una manera más estrecha. Usando la siguiente desigualdad,
$$\left| \frac{L'(s,\chi_0)}{L(s,\chi_0)} - \frac{\zeta'(s)}{\zeta(s)} \right| \leq \log q,$$
se puede demostrar, usando argumentos parecidos al caso complejo, la desigualdad similar a 1.2, para caracteres reales.

\begin{align}
    \frac{4}{\sigma-\beta} < \frac{3}{\sigma-1} + \Re{\left(\frac{1}{\sigma-1+2it}\right)}+ C\mathscr{L} ,
\end{align}
de la cual, si se toma un cero no trivial $\rho =\beta + i\gamma$, poniendo $t=\gamma $ suficientemente grande y escogiendo $\delta$  tal que $\sigma = 1+ \frac{\delta}{\mathscr{L}}$, se llega a la desigualdad
$$\beta < 1- \frac{4-5c\delta}{16+5c\delta}\frac{\delta}{\mathscr{L}},$$
\pagebreak

y si se asume específicamente que
$|\gamma| \geq \frac{\delta}{\log q}$ se obtiene el resultado para caracteres reales:

\textit{Existe una constante positiva $c$ tal que si $0<\delta < c$, y si $\chi$ es un caracter real (no principal) módulo $q$, entonces cualquier cero no trivial $\rho = \beta+ i\gamma$ de $L(s,\chi)$ que satisfaga $$|\gamma| \geq \frac{\delta}{\log q},$$
    cumplirá que
    \begin{align}\beta < 1- \frac{\delta}{5\mathscr{L}},\end{align}
    donde, $\mathscr{L} = \log q + \log(|\gamma| +2)$, y $\sigma = 1+ \frac{\delta}{\mathscr{L}}$}.

Una pregunta que surge naturalmente es ¿Qué pasa con los demás ceros?, es decir, aquellos cuya parte imaginaria cumple
$$|t| < \frac{\delta}{\log q}.$$
En el libro se demuestra que los ceros de este tipo (si existen), pueden ``desobedecer'' la desigualdad 1.4. Sin embargo, al asumir por contradicción la existencia de más de uno de estos ceros, y luego de estudiar varios posibles casos (ceros reales y complejos), se llega luego de ciertas manipulaciones técnicas (cuyos detalles no importan en este momento), a que este cero, en caso de existir, \textit{es único, simple y real.}

Combinando estos últimos resultados (los resaltados en \textit{itálico}), y modificando las constantes indeterminadas, se concluye el teorema principal.

Posteriormente, Landau determinó que la probabilidad (sobre $q$) de que una función $L$ tenga este tipo de ceros ``intrusos'', es muy baja. Más específicamente, demostró que \textit{si $\chi_1,\chi_2$ son caracteres reales módulo $q_1,q_2$ (respectivamente), y las $L$-funciones correspondientes tienen ceros reales $\beta_1,\beta_2$, entonces
    $$\min{(\beta_1,\beta_2)} < 1 - \frac{c}{\log(q_1 q_2)},$$
    es decir, uno de estos dos ceros no es ``intruso''.}

La prueba de este resultado procede muy parecido a la utilizada al principio de la sección, pero aplicándola al caracter (no primitivo) $\chi_1\chi_2$ modulo $q_1q_2$. El hecho que no sea primitivo fuerza a hacer algunas modificaciones, sin embargo no son tan serias.

Una deducción importante de este resultado,  establece que \textit{de todos los caracteres reales no primitivos módulo $q$, a lo sumo $1$ puede poseer un cero ``intruso''}.

Otro pequeño \textbf{corolario} de este resultado es el hecho de que es posible escoger una sucesión de números $q_1,q_2,\dots$ de enteros tal que sus funciones $L$ asociadas, tengan ceros intrusos. A saber, basta con escoger $q_{j+1} > q_j^2$ para todo $j$. Esto se sigue de inmediato del resultado de Landau, puesto que
$$1-\frac{c'}{\log q_j} < 1-\frac{c}{\log q_jq_{j+1}},$$
para constantes $c$ y $c'$ apropiadas.

En la misma línea de trabajo, Page demostró un resultado ligeramente más fuerte con respecto a estos ceros intrusos. \textit{Dado un número real $z \geq 3$, para alguna constante $c$, existe a lo sumo un caracter real primitivo $\chi$, módulo $q\leq z$, tal que existe un cero de $L(s,\chi)$ que satisface
    $$\beta > 1 - \frac{c}{\log z}.$$}
Esto coincide con el resultado de Landau, que establece que las $L$ series con ceros intrusos son inusuales. La demostración de este resultado es casi directa del de Landau, pues si se asume por contradicción la existencia de $2$ caracteres, se llega a una desigualdad del tipo
$$\beta > 1- \frac{c}{\log z} \geq 1 - \frac{2c}{\log(q_1q_2)}$$
(en este caso la constante $c$ es la misma en ambos lados).
Esto conlleva a una contradicción, pues se da una incompatibilidad de las constantes (debido a como se escogen\footnote{Estrictamente, a lo largo de la sección debería ser necesario numerar todas las constantes, sin embargo esto da paso a $25$ constantes distintas, y por simplicidad todas se denotan como $c$.}).

Finalmente, es posible, utilizando la fórmula de número de clases (la cual no se logró estudiar por razones de tiempo), llegar a una cota un poco más débil para la ubicación de los ceros intrusos, la cual establece que estos ceros están fuera de una franja un poco más angosta, dada por
$$1-\frac{c}{\sqrt{q}\log^2q}.$$
Estos últimos resultados fueron profundizados en gran medida por Siegel, con una pequeña desventaja:  si bien en esta sección se indeterminaron muchas constantes, el valor numérico de todas ellas puede ser estimado\footnote{Se les llama por lo general \textit{constantes efectivas}.}. Sin embargo, en los trabajos de Siegel, no parece haber esperanza alguna de calcular algunas de las constantes que surgen. Los teoremas de Siegel se desarrollan en la sección $21$ del libro.
\section{Sección 15. El número $N(T, \chi)$}
Esta sección corresponde con la sección 15 del libro: El número $N(T)$ y es, naturalmente, el análogo para funciones $L$ de la función contadora de ceros.

El resultado que se demostró en clase dice que  si $N(T)$ representa el número de ceros no triviales de $\zeta(s)$ en el rectángulo $0<t \leq T$, se satisface la fórmula asintótica
$$N(T) = \frac{T}{2 \pi}\log{\frac{T}{2 \pi} - \frac{T}{2\pi} + O(\log T)}.$$
En nuestro caso, vamos a considerar ahora un caracter módulo $q$, su $L$-serie asociada, y vamos a denotar por $N(T, \chi)$, al número de ceros dentro del rectángulo
$$0 < \sigma < 1 , \quad |t| < T. $$
Ya no bastará (como en la sección anterior), estudiar ceros con parte imaginaria positiva, pues no se tiene la propiedad de simetría que poseen los ceros de $\zeta(s)$. El resultado principal de la sección es el siguiente
\begin{theorem}
    En el contexto presente, si $T \geq 2$, entonces
    $$\frac{1}{2}N(T,\chi) = \frac{T}{2\pi}\log{\frac{qT}{2\pi}} - \frac{T}{2\pi} + O(\log T + \log q).$$\end{theorem}
El factor de $1/2$ puede ser fácilmente simplificado, sin embargo el libro lo añade para ilustrar el efecto de duplicar la altura del rectángulo.

Al parecer, la prueba sigue el mismo procedimiento que la de $N(T)$, y debido a esto el libro omite la mayor parte de los detalles. La clave, es, al igual que en la sección anterior, concentrarnos en $\xi(s,\chi)$, en vez de en $L(s,\chi)$, y observar la variación del argumento $\arg{\xi(s,\chi)}$ a lo largo del rectángulo complejo $R$, con vértices
$$\frac{5}{2} - iT, \quad \frac{5}{2} + iT, \quad -\frac{3}{2} + iT, \quad -\frac{3}{2}-iT.$$
El cual se escoge diferentemente a la sección anterior, puesto que existe la posibilidad de un cero en $s=-1$, y necesitamos evitar ceros sobre nuestro contorno. Se sabe además que este rectángulo contiene a lo sumo un cero no trivial de $L(s,\chi)$, ya sea en $s=0$, o en $s=-1$ (dependiendo de la paridad del caracter), y por lo tanto, por el principio del argumento
$$2\pi[N(T,\chi) + 1] = \Delta_R \arg(\xi(s,\chi)).$$
Se demuestra, al igual que en la otra sección, que la contribución en la integral de la mitad izquierda de $R$ es igual a la de la mitad derecha, lo cual se debe a que
$$\arg \xi (\sigma + it, \chi) = \arg\overline{\xi(1-\sigma + it, \chi)}+c,$$
donde $c$ es una constante que no depende de $s$. Esto se sigue de manipulaciones de la ecuación funcional para $L$-series (Sección 9).

Partiendo ahora de la definición de $\xi$,
$$\xi(s,\chi) = \left( \frac{q}{\pi}\right)^{\frac{1}{2}s+\frac{1}{2}\mathfrak{a}}\Gamma \left(\frac{1}{2}s ++\frac{1}{2}\mathfrak{a}  \right)L(s,\chi),$$
donde $\mathfrak{a}$ es $1$ o $0$, dependiendo de la paridad del caracter. Podemos descomponerla en partes y estudiar los argumentos de cada uno de los factores por aparte. Utilizando la teoría de la sección anterior, se puede determinar que
\begin{align}
    \Delta \arg \left( \frac{q}{\pi}\right)^{\frac{1}{2}s+\frac{1}{2}\mathfrak{a}} & = T \log\left( \frac{q}{\pi}\right),            \\
    \Delta \arg \Gamma (\frac{1}{2}s+\frac{1}{2}\mathfrak{a})                      & = T \log \left( \frac{1}{2}T\right) - T + O(1),
\end{align}
y, si se logra demostrar que
\begin{align}\arg L \left( \frac{1}{2} + iT, \chi\right) = O(\log T + \log q),\end{align}
la fórmula asintótica se seguiría de la combinación de esto con 2.1 y 2.2. Para probar esto último, se presenta un modificación a un lema de la sección anterior
\begin{lemma}
    Si $\rho = \beta + i\gamma$ recorre los ceros no triviales de $L(s,\chi)$, donde $\chi$ es un caracter primitivo, entonces para cualquier número real $t$,
    \begin{align}
        \sum_{\rho} \frac{1}{1 + (t-\gamma)^2} = O(\mathscr{L}),
    \end{align}
    donde $\mathscr{L} = \log q(|t| +2 )$
\end{lemma}

\noindent Su prueba es omitida en el libro. De aquí se sigue, de la misma manera que cuando calculamos $N(T)$, que si \textit{$t$ no coincide con la parte imaginaria de un cero, y $-1\leq \sigma \leq 2$,}
$$\frac{L'(s,\chi)}{L(s,\chi)} = \sideset{}{'}\sum_{\rho}\frac{1}{s-\rho} + O(\mathscr{L}),$$
\textit{donde la suma está limitada sobre aquellos $\rho$ que cumplan $|t-\gamma| < 1$.}
Y luego de dominar esta última expresión apropiadamente, e integrar, se obtiene la fórmula 2.3.

Finalmente, para caracteres \textbf{no primitivos} una vez más, se tiene un resultado más débil. Sobre el rectángulo $R$ definido anteriormente, se tiene
$$N_R(T,\chi) = \frac{T}{\pi}\log\left(\frac{T}{2\pi} \right) + O(T \log q) .$$
La fórmula anterior vale para $T \geq 2$.
\section{Sección 19. Una fórmula explícita para $\psi(x, \chi)$}
Esta tercera sección corresponde con la adaptación de la sección 17 para, funciones $L$. De hecho, el enfoque es similar al de las 2 secciones anteriores. Se parte del argumento para $\zeta$, el cual se comienza a ramificar y modificar, de acuerdo con dificultades que surgen, ya sean paridad, primitividad, o existencia de ceros intrusos de las funciones $L$. A manera de referencia, el resultado principal de la sección de $17$ es la fórmula exacta
$$\psi_0(x) = x - \sum_{\rho} \frac{x^{\rho}}{\rho} - \frac{\zeta'(0)}{\zeta(0)} - \frac{1}{2} \log (1-x^{-2}),$$
donde $\psi_0 = \sum_{n\leq x} \Lambda(n)  \Lambda(x)/2$ y la suma se toma sobre los ceros no triviales de $L(s,\chi)$, de forma simétrica (al igual que para $\zeta(s)$). En nuestro caso, vamos a definir el análogo para caracteres de la función $\psi$. Definimos
$$\psi(x,\chi) = \sum_{n \leq x} \chi(n) \Lambda(n),$$
y al igual que en la sección anterior, se modifica ligeramente en caso de que $x$ sea una potencia prima, con la única intención de suavizar las discontinuidades en estos puntos. A esta función ``suavizada'', le llamamos $\psi_0(x,\chi)$ la cual es, esencialmente, $\psi(x,\chi)$. El resultado principal de la sección es, naturalmente, \textit{una fórmula explícita para $\psi_0(x,\chi)$, la cual va a depender únicamente de la paridad del caracter.}
\begin{theorem}
    Sea $\chi$ un caracter primitivo módulo $q$.
    Si $\chi(-1)=-1$, entonces
    $$\psi_0(x,\chi) = -\sum_{\rho}\frac{x^\rho}{\rho} - \frac{L'(0,\chi)}{L(0,\chi)} + \sum_{m=1}^{\infty} \frac{x^{1-2m}}{2m-1}.$$
    Mientras que si $\chi(-1)=1$
    $$\psi_0(x,\chi) = -\sum_{\rho}\frac{x^\rho}{\rho} - \log x - b(\chi)  + \sum_{m=1}^{\infty} \frac{x^{-2m}}{2m},$$
    donde $b(\chi)$ es una constante que depende del caracter en cuestión.
\end{theorem}
La demostración es esencia, muy parecida a la de la sección anterior, incluso se llega a demostrar una fórmula asintótica más fuerte, la cual implica nuestras fórmulas (al igual que se demuestra para $\psi$). La idea de la prueba es similiar: por medio de el uso de integrales, el lema 2.2, y la ecuación funcional
$$L(1-s,\chi)= \varepsilon(\chi) 2^{1-s} \pi^{-s}q^{s-\frac{1}{2}}\cos \left(\frac{1}{2}\pi(s-\mathfrak{a}) \right) \Gamma(s)L(s,\overline{\chi})$$
(donde $|\varepsilon(\chi) | = 1$ y $\mathfrak{a} = 0$ ó $1$), se llega al resultado
\begin{align}
    \psi_0(x,\chi) = -\sum_{|\gamma| < T} \frac{x^\rho}{\rho} - (1-\mathfrak{a})\log x - b(\chi) + \sum_{m=1}^{\infty} \frac{x^{\mathfrak{a}-2m}}{2m-\mathfrak{a}}+ R(x,T),
\end{align}
donde\footnote{Adoptamos la notación del libro, decimos que $f(x) \ll g(x)$ cuando $ f(x) = O(g(x))$ cuando $x$ tienda a algún valor espefícico.}
$$|R(x,T)| \ll \frac{x}{T}\log^2 qxT + (\log x)\min\left( 1, \frac{x}{T \langle x \rangle}\right),$$
y donde $\langle x \rangle$ se define como la distancia entre $x$ y la potencia prima más cercana (distinta de $x$).

\noindent
Esta fórmula asintótica corresponde con una versión finita de las presentadas en el teorema, y para obtener ambas, basa con evaluar el valor de $\mathfrak{a}$, y enviar $T \to \infty$.

Resulta que para nuestro objetivo final (estudiar la distribución de primos en progresiones aritméticas), la fórmula 3.1 es poco útil, pues contiene una constante que depende del caracter ($b(\chi)$), y además, no tenemos un buen control sobre los sumandos $\frac{x^\rho}{\rho}$, pues los ceros no triviales pueden estar muy cerca de $0$ o de $1$ (a diferencia de los de $\zeta(s)$). Además, por la sección anterior, sabemos que cabe la posibilidad de un cero intruso, y es de interés que nuestra fórmula aísle el comportamiento de ese cero. Por lo tanto es necesario trabajar un poco más la fórmula 3.1.

\noindent Después algunas simplificaciones y asumciones adicionales, es posible simplificar la fórmula 3.1 de la siguiente manera\footnote{No haremos más distinción entre $\psi$ y $\psi_0$, pues no es de interés}:
\begin{align}\psi(x,\chi) & = - \sum_{|\gamma |< T} \frac{x^\rho}{\rho} - b(\chi) + R_1(x,T),
             \intertext{donde}
                          & |R_1(x,T)| \ll xT^{-1}\log^2qx.
\end{align}
Entonces, utilizando la expresión de la derivada logarítmica de $L(s,\chi)$, estudiada en la sección 12, y acotando cada uno de los términos, se obtiene que
$$b(\chi) = O(1) - \sum_{\rho} \left(  \frac{1}{\rho} + \frac{1}{2-\rho}\right),$$
Esta serie puede ser estimada usando el lema 2.2 para obtener
$$b(\chi) = O(\log q) - \sum_{|\gamma| < 1}\frac{1}{\rho},$$
para finalmente reescribir 3.2 como
\begin{align}\psi(x,\chi) & = -\sum_{|\gamma| < T} \frac{x^{\rho}}{\rho} + \sum_{|\gamma| < 1}\frac{1}{\rho} + R_2(x,T),
             \intertext{donde una vez más, }
                          & |R_1(x,T)| \ll xT^{-1}\log^2qx.
\end{align}
Seguidamente, asumimos la existencia de un cero ``intruso'', es decir, un cero $\beta+i\gamma$ no trivial de $L(s,\chi)$ que satisfaga
$$|\gamma| < 1 \quad ,  \quad \beta > 1-\frac{c}{\log q}.$$
Sabemos, gracias a la ecuación funcional para funciones $L$, que este cero tiene un reflejo sobre el eje crítico, es decir, tendremos otro cero con parte real $1-\beta$ (este cálculo no es directo de la ecuación funcional), el cual sería también intruso, si reflejamos la franja libre de ceros. Entonces, si denotamos por $\Sigma'$ a la suma sobre todos los ceros no triviales, salvo estos dos ceros mencionados, podemos escribir, después de acotar algunos términos, una nueva expresión más refinada para 3.4.
\begin{align}\psi(x,\chi) = -\frac{x^{\beta}}{\beta} - \sideset{}{'}\sum_{\rho} \frac{x^\rho}{\rho} + R_3(x,T)
    \intertext{donde}
    |R_3(x,T)| \ll xT^{-1}\log^2 (qx) + x^{\frac{1}{4}}\log x.
\end{align}
Es importante recordar que el término $-\frac{x^{\beta}}{\beta}$ sólo puede aparecer si $\chi$ es un caracter real.

Finalmente, la última mejora que se puede hacer a la fórmula 3.6, es remover la hipótesis de que $\chi$ es un caracter primitivo. La fórmula en este caso, no sufre ninguna alteración. Podemos entonces formular el resultado principal en su versión ``final'', la cual utilizaremos en la demostración del teorema de los números primos para progresiones aritméticas (sección $20$).
\begin{theorem}
    Si $\chi$ es un caracter no principal módulo $q$, y $2 \leq T \leq x$, entonces
    $$\psi(x,\chi) = -\frac{x^\beta}{\beta} -\sideset{}{'}\sum_{\rho} \frac{x^\rho}{\rho} +   R_3(x,T),$$
    donde
    $$|R_3(x,T)| \ll xT^{-1}\log^2 (qx) + x^{\frac{1}{4}}\log x.$$
    El término $-\frac{x^{\beta}}{\beta}$ debe ser omitido, a menos que $\chi$ sea un caracter real, para el cual $L(s,\chi)$ tenga un cero no trivial $\beta$ (el cual sabemos que es único y real) que satisfaga
    $$\beta > 1- \frac{c}{\log q},$$
    donde $c$ es alguna constante positiva, y la suma $\Sigma'$ excluye a $\beta$ y a $1-\beta$ (si es que existen). Además, el término en el error $x^\frac{1}{4} \log x$ puede ser omitido si $\beta$ no existe.
\end{theorem}
Este teorema concluye los preliminares que necesitamos para desarrollar la sección 20 del libro, la cual se hará con todo detalle en la parte $2$ del examen.
\bibliographystyle{siam}
\bibliography{Referencias.bib}
\end{document}
